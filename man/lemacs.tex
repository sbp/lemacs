\input texinfo  @c -*-texinfo-*-
@setfilename ../info/lemacs
@comment  node-name,  next,  previous,  up


@ifinfo
This file documents the GNU Emacs editor.

Copyright (C) 1985, 1986, 1988 Richard M. Stallman.
Copyright @copyright{} 1991, 1992 Lucid, Inc.

Permission is granted to make and distribute verbatim copies of
this manual provided the copyright notice and this permission notice
are preserved on all copies.

@ignore
Permission is granted to process this file through Tex and print the
results, provided the printed document carries copying permission
notice identical to this one except for the removal of this paragraph
(this paragraph not being relevant to the printed manual).

@end ignore
Permission is granted to copy and distribute modified versions of this
manual under the conditions for verbatim copying, provided also that the
sections entitled ``The GNU Manifesto'', ``Distribution'' and ``GNU
General Public License'' are included exactly as in the original, and
provided that the entire resulting derived work is distributed under the
terms of a permission notice identical to this one.

Permission is granted to copy and distribute translations of this manual
into another language, under the above conditions for modified versions,
except that the sections entitled ``The GNU Manifesto'',
``Distribution'' and ``GNU General Public License'' may be included in a
translation approved by the author instead of in the original English.
@end ifinfo
@c
@setchapternewpage odd
@settitle Lucid GNU Emacs Manual
@c
@titlepage
@sp 6
@center @titlefont{Lucid GNU Emacs Manual}
@sp 4
@sp 1
@sp 1
@center March 1992
@center (General Public License upgraded, January 1991)
@sp 5
@center Richard Stallman
@sp 1
@center and
@sp 1
@center Lucid, Inc.
@page
@vskip 0pt plus 1filll
Copyright @copyright{} 1985, 1986, 1988 Richard M. Stallman.

Copyright @copyright{} 1991, 1992 Lucid, Inc.

Permission is granted to make and distribute verbatim copies of
this manual provided the copyright notice and this permission notice
are preserved on all copies.

Permission is granted to copy and distribute modified versions of this
manual under the conditions for verbatim copying, provided also that the
sections entitled ``The GNU Manifesto'', ``Distribution'' and ``GNU
General Public License'' are included exactly as in the original, and
provided that the entire resulting derived work is distributed under the
terms of a permission notice identical to this one.

Permission is granted to copy and distribute translations of this manual
into another language, under the above conditions for modified versions,
except that the sections entitled ``The GNU Manifesto'',
``Distribution'' and ``GNU General Public License'' may be included in a
translation approved by the author instead of in the original English.
@end titlepage
@page
@ifinfo
@node Top, Distrib,, (dir)

The Emacs Editor
****************

Emacs is the extensible, customizable, self-documenting real-time
display editor.  This Info file describes how to edit with Emacs
and some of how to customize it, but not how to extend it.

@end ifinfo
@menu
* License::     The GNU General Public License gives you permission
		to redistribute GNU Emacs on certain terms; and also
		explains that there is no warranty.
* Distrib::     How to get Emacs.
* Intro::       An introduction to Emacs concepts.
* Glossary::    The glossary.
* Manifesto::   What's GNU?  Gnu's Not Unix!

Indexes, nodes containing large menus
* Key Index::      An item for each standard Emacs key sequence.
* Command Index::  An item for each command name.
* Variable Index:: An item for each documented variable.
* Concept Index::  An item for each concept.

Important General Concepts
* Screen::      How to interpret what you see on the screen.
* Keystrokes::  Keyboard gestures Emacs recognizes.
* Pull-down Menus:: 
                The Lucid GNU Emacs Pull-down Menus available under X.
* Mouse Selection:: 
                Selecting Text with the Mouse.
* Keystrokes:: 
                Using Keystrokes to represent key sequences.
* Commands::    Commands: named functions run by key sequences to do editing.
* Entering Emacs::    
                Starting Emacs from the shell.
* Command Switches::  
                Hairy startup options.
* Exiting::     Stopping or killing Emacs.
* Basic::       The most basic editing commands.
* Undo::        Undoing recently made changes in the text.
* Minibuffer::  Entering arguments that are prompted for.
* M-x::         Invoking commands by their names.
* Help::        Commands for asking Emacs about its commands.

Important Text-Changing Commands
* Mark::        The mark: how to delimit a ``region'' of text.
* Killing::     Killing text.
* Yanking::     Recovering killed text.  Moving text.
* Using X Selections::
                Using primary selection, cut buffers, and highlighted regions.
* Accumulating Text::
                Other ways of copying text.
* Rectangles::  Operating on the text inside a rectangle on the screen.
* Registers::   Saving a text string or a location in the buffer.
* Display::     Controlling what text is displayed.
* Search::      Finding or replacing occurrences of a string.
* Fixit::       Commands especially useful for fixing typos.

Larger Units of Text
* Files::       All about handling files.
* Buffers::     Multiple buffers; editing several files at once.
* Windows::     Viewing two pieces of text at once.

Advanced Features
* Major Modes:: Text mode vs. Lisp mode vs. C mode ...
* Indentation:: Editing the white space at the beginnings of lines.
* Text::        Commands and modes for editing English.
* Programs::    Commands and modes for editing programs.
* Running::     Compiling, running and debugging programs.
* Abbrevs::     How to define text abbreviations to reduce
                 the number of characters you must type.
* Picture::     Editing pictures made up of characters
                 using the quarter-plane screen model.
* Sending Mail::Sending mail in Emacs.
* Rmail::       Reading mail in Emacs.
* Recursive Edit::
                A command can allow you to do editing
                 "within the command".  This is called a
                 `recursive editing level'.
* Narrowing::   Restricting display and editing to a portion
                 of the buffer.
* Sorting::	Sorting lines, paragraphs or pages within Emacs.
* Shell::       Executing shell commands from Emacs.
* Hardcopy::	Printing buffers or regions.
* Dissociated Press::  Dissociating text for fun.
* CONX::	       A different kind of dissociation.
* Amusements::         Various games and hacks.
* Emulation::	       Emulating some other editors with Emacs.
* Customization::      Modifying the behavior of Emacs.

Recovery from Problems.
* Quitting::    Quitting and aborting.
* Lossage::     What to do if Emacs is hung or malfunctioning.
* Bugs::        How and when to report a bug.

Here are some nodes which are inferiors of the ones already listed. They
are mentioned here so you can get to them in one step:

Subnodes of Screen
* Point::	The place in the text where editing commands operate.
* Echo Area::   Short messages appear at the bottom of the screen.
* Mode Line::	Interpreting the mode line.

Subnodes of Basic
* Blank Lines:: Commands to make or delete blank lines.
* Continuation Lines:: Lines too wide for the screen.
* Position Info::      What page, line, row, or column is point on?
* Arguments::          Giving numeric arguments to commands.

Subnodes of Keystrokes
* Representing Keystrokes::  Using lists of modifiers and keysyms to
                             represent keystrokes.
* Key Sequences::            Combine key strokes into key sequences you can
                             bind to commands.
* String Key Sequences::     Available for upward compatibility.
* Meta Key::                 Using @key{ESC} to represent @key{Meta}
* Super and Hyper Keys::     Adding modifier keys on certain keyboards
* Character Representation:: How characters appear in Emacs buffers.
* Commands::                 How commands are bound to key sequences.
                    
Subnodes of Pull-down Menus
* File Menu::           Items on the File menu
* Edit Menu::           Items on the Edit menu
* Buffers Menu::        Information about the Buffers menu
* Help Menu::           Items on the Help menu
* Menu Customization::  Adding and removing menu items and related
                        operations.

Subnodes of Minibuffer
* Minibuffer File::    Entering file names with the minibuffer.
* Minibuffer Edit::    How to edit in the minibuffer.
* Completion::  An abbreviation facility for minibuffer input.
* Repetition::  Re-executing previous commands that used the minibuffer.

Subnodes of Mark
* Setting Mark::       Commands to set the mark.
* Using Region::       Summary of ways to operate on contents of the region.
* Marking Objects::    Commands to put region around textual units.
* Mark Ring::          Previous mark positions saved so you can go back there.

Subnodes of Yanking
* Kill Ring::          Where killed text is stored.  Basic yanking.
* Appending Kills::    Several kills in a row all yank together.
* Earlier Kills::      Yanking something killed some time ago.

Subnodes of Registers
* RegPos::             Saving positions in registers.
* RegText::            Saving text in registers.
* RegRect::            Saving rectangles in registers.

Subnodes of Display
* Scrolling::	           Moving text up and down in a window.
* Horizontal Scrolling::   Moving text left and right in a window.
* Selective Display::      Hiding lines with lots of indentation.
* Display Vars::           Information on variables for customizing display.

Subnodes of Search
* Incremental Search::     Search happens as you type the string.
* Non-Incremental Search::  Specify entire string and then search.
* Word Search:: 	   Search for sequence of words.
* Regexp Search::	   Search for match for a regexp.
* Regexps::     	   Syntax of regular expressions.
* Search Case::		   To ignore case while searching, or not.
* Replace::     	   Search, and replace some or all matches.
* Unconditional Replace::  Everything about replacement except for querying.
* Query Replace::          How to use querying.
* Other Repeating Search:: Operating on all matches for some regexp.

Subnodes of Fixit
* Kill Errors:: Commands to kill a batch of recently entered text.
* Transpose::   Exchanging two characters, words, lines, lists...
* Fixing Case:: Correcting case of last word entered.

* Spelling::    Apply spelling checker to a word, or a whole file.

Subnodes of Files
* File Names::  How to type and edit file name arguments.
* Visiting::    Visiting a file prepares Emacs to edit the file.
* Saving::      Saving makes your changes permanent.
* Backup::      How Emacs saves the old version of your file.
* Interlocking::How Emacs protects against simultaneous editing
                 of one file by two users.
* Reverting::   Reverting cancels all the changes not saved.
* Auto Save::   Auto Save periodically protects against loss of data.
* ListDir::     Listing the contents of a file directory.
* Dired::       ``Editing'' a directory to delete, rename, etc.
                 the files in it.
* Misc File Ops:: Other things you can do on files.

Subnodes of Buffers
* Select Buffer::   Creating a new buffer or reselecting an old one.
* List Buffers::    Getting a list of buffers that exist.
* Misc Buffer::     Renaming; changing read-only status.
* Kill Buffer::     Killing buffers you no longer need.
* Several Buffers:: How to go through the list of all buffers
                     and operate variously on several of them.

Subnodes of Windows
* Basic Window::    Introduction to Emacs windows.
* Split Window::    New windows are made by splitting existing windows.
* Other Window::    Moving to another window or doing something to it.
* Pop Up Window::   Finding a file or buffer in another window.
* Change Window::   Deleting windows and changing their sizes.

Subnodes of Using X Selections
* X Clipboard Selection::     Storing the primary selection.
* Active Regions::  Using zmacs-style highlighting of the selected region.
* X Cut Buffers::   X cut buffers are available for compatibility.

Subnodes of Indentation
* Indentation Commands:: Various commands and techniques for indentation.
* Tab Stops::   You can set arbitrary "tab stops" and then
                 indent to the next tab stop when you want to.
* Just Spaces:: You can request indentation using just spaces.

Subnodes of Text
* Text Mode::   The major mode for editing text files.
* Nroff Mode::  The major mode for editing input to the formatter nroff.
* TeX Mode::    The major mode for editing input to the formatter TeX.
* Outline Mode::The major mode for editing outlines.
* Words::       Moving over and killing words.
* Sentences::   Moving over and killing sentences.
* Paragraphs::	Moving over paragraphs.
* Pages::	Moving over pages.
* Filling::     Filling or justifying text
* Case::        Changing the case of text

Subnodes of Programs
* Program Modes::       Major modes for editing programs.
* Lists::       Expressions with balanced parentheses.
                 There are editing commands to operate on them.
* Defuns::      Each program is made up of separate functions.
                 There are editing commands to operate on them.
* Grinding::    Adjusting indentation to show the nesting.
* Matching::    Insertion of a close-delimiter flashes matching open.
* Comments::    Inserting, illing and aligning comments.
* Balanced Editing::    Inserting two matching parentheses at once, etc.
* Lisp Completion::     Completion on symbol names in Lisp code.
* Documentation::       Getting documentation of functions you plan to call.
* Change Log::  Maintaining a change history for your program.
* Tags::        Go direct to any function in your program in one
                 command.  Tags remembers which file it is in.
* Fortran::	Fortran mode and its special features.

Subnodes of Running
* Compilation::       Compiling programs in languages other than Lisp
                       (C, Pascal, etc.)
* Lisp Modes::        Various modes for editing Lisp programs, with
                       different facilities for running the Lisp programs.
* Lisp Libraries::    Creating Lisp programs to run in Emacs.
* Lisp Interaction::  Executing Lisp in an Emacs buffer.
* Lisp Eval::         Executing a single Lisp expression in Emacs.
* Lisp Debug::        Debugging Lisp programs running in Emacs.
* External Lisp::     Communicating through Emacs with a separate Lisp.

Subnodes of Abbrevs
* Defining Abbrevs::  Defining an abbrev, so it will expand when typed.
* Expanding Abbrevs:: Controlling expansion: prefixes, canceling expansion.
* Editing Abbrevs::   Viewing or editing the entire list of defined abbrevs.
* Saving Abbrevs::    Saving the entire list of abbrevs for another session.
* Dynamic Abbrevs::   Abbreviations for words already in the buffer.

Subnodes of Picture
* Basic Picture::     Basic concepts and simple commands of Picture Mode.
* Insert in Picture:: Controlling direction of cursor motion
                       after "self-inserting" characters.
* Tabs in Picture::   Various features for tab stops and indentation.
* Rectangles in Picture:: Clearing and superimposing rectangles.

Subnodes of Sending Mail
* Mail Format::       Format of the mail being composed.
* Mail Headers::      Details of allowed mail header fields.
* Mail Mode::         Special commands for editing mail being composed.

Subnodes of Rmail
* Rmail Scrolling::   Scrolling through a message.
* Rmail Motion::      Moving to another message.
* Rmail Deletion::    Deleting and expunging messages.
* Rmail Inbox::       How mail gets into the Rmail file.
* Rmail Files::       Using multiple Rmail files.
* Rmail Output::      Copying message out to files.
* Rmail Labels::      Classifying messages by labeling them.
* Rmail Summary::     Summaries show brief info on many messages.
* Rmail Reply::       Sending replies to messages you are viewing.
* Rmail Editing::     Editing message text and headers in Rmail.
* Rmail Digest::      Extracting the messages from a digest message.

Subnodes of Shell
* Single Shell::      Commands to run one shell command and return.
* Interactive Shell:: Permanent shell taking input via Emacs.
* Shell Mode::        Special Emacs commands used with permanent shell.

Subnodes of Customization
* Minor Modes::       Each minor mode is one feature you can turn on
                       independently of any others.
* Variables::         Many Emacs commands examine Emacs variables
                       to decide what to do; by setting variables,
                       you can control their functioning.
* Examining::         Examining or setting one variable's value.
* Edit Options::      Examining or editing list of all variables' values.
* Locals::            Per-buffer values of variables.
* File Variables::    How files can specify variable values.
* Keyboard Macros::   A keyboard macro records a sequence of keystrokes
                       to be replayed with a single command.
* Key Bindings::      The keymaps say what command each key runs.
                       By changing them, you can "redefine keys".
* Keymaps::           Definition of the keymap data structure.
* Rebinding::         How to redefine one key's meaning conveniently.
* Disabling::         Disabling a command means confirmation is required
                       before it can be executed.  This is done to protect
                       beginners from surprises.
* Syntax::            The syntax table controls how words and expressions
                       are parsed.
* Init File::         How to write common customizations in the @file{.emacs}
                       file.
* Audible Bell::      Changing how Emacs sound the bell.
* Faces::
                      Changing the fonts and colors of a region of text. 

Subnodes of Lossage (and recovery)
* Stuck Recursive::   `[...]' in mode line around the parentheses.
* Screen Garbled::    Garbage on the screen.
* Text Garbled::      Garbage in the text.
* Unasked-for Search::Spontaneous entry to incremental search.
* Emergency Escape::  Emergency escape---
                       What to do if Emacs stops responding.
* Total Frustration:: When you are at your wits' end.
@end menu

@iftex
@unnumbered Preface

  This manual documents the use and simple customization of the Emacs
editor.  The reader is not expected to be a programmer to use this
editor, and simple customizations do not require programming skills either.
Users who are not interested in customizing Emacs can ignore the scattered
customization hints.

  This document is primarily a reference manual, but it can also be used as a
primer.  However, if you are new to Emacs, consider using the on-line,
learn-by-doing tutorial, which you get by running Emacs and typing
@kbd{C-h t}.  With it, you learn Emacs by using Emacs on a specially
designed file which describes commands, tells you when to try them,
and then explains the results you see.  Using the tutorial gives a more vivid
introduction than the printed manual.

  On first reading, just skim chapters one and two, which describe the
notational conventions of the manual and the general appearance of the
Emacs display screen.  Note which questions are answered in these chapters,
so you can refer back later.  After reading chapter four you should
practice the commands there.  The next few chapters describe fundamental
techniques and concepts that are used constantly.  You need to understand
them thoroughly, experimenting with them if necessary.

  To find the documentation on a particular command, look in the index.
Keys (character commands) and command names have separate indexes.  There
is also a glossary, with a cross reference for each term.

@ignore
  If you know vaguely what the command
does, look in the command summary.  The command summary contains a line or
two about each command, and a cross reference to the section of the
manual that describes the command in more detail; related commands
are grouped together.
@end ignore

  This manual comes in two forms: the published form and the Info form.
The Info form is for on-line perusal with the INFO program; it is
distributed along with GNU Emacs.  Both forms contain substantially the
same text and are generated from a common source file, which is also
distributed along with GNU Emacs.

  GNU Emacs is a member of the Emacs editor family.  There are many Emacs
editors, all sharing common principles of organization.  For information on
the underlying philosophy of Emacs and the lessons learned from its
development, write for a copy of AI memo 519a, ``Emacs, the Extensible,
Customizable Self-Documenting Display Editor'', to Publications Department,
Artificial Intelligence Lab, 545 Tech Square, Cambridge, MA 02139, USA.  At
last report they charge $2.25 per copy.  Another useful publication is LCS
TM-165, ``A Cookbook for an Emacs'', by Craig Finseth, available from
Publications Department, Laboratory for Computer Science, 545 Tech Square,
Cambridge, MA 02139, USA.  The price today is $3.

This manual is for GNU Emacs installed on UNIX systems.  GNU
Emacs can also be used on VMS systems, which have different file name
syntax and don't support all GNU Emacs features.  A VMS edition of this
manual may appear in the future.
@end iftex

@comment  node-name,  next,  previous,  up
@node Distrib, License, Top, Top
@unnumbered Distribution

Lucid GNU Emacs is @dfn{free}; this means that everyone is free to use it and
free to redistribute it on a free basis.  Lucid GNU Emacs is not in the public
domain; it is copyrighted and there are restrictions on its
distribution, but these restrictions are designed to permit everything
that a good cooperating citizen would want to do.  What is not allowed
is to try to prevent others from further sharing any version of GNU
Emacs that they might get from you.  The precise conditions are found in
the GNU General Public License that comes with Emacs and also appears
following this section.

The easiest way to get a copy of GNU Emacs is from someone else who has it.
You need not ask for permission to do so, or tell any one else; just copy
it.

If you have access to the Internet, you can send mail to Lucid and we
will send you information on how you can get a copy. Since the
information may change, we are asking you to send mail rather than
giving you instructions on how to get the copy. The email address is: 

@display
emacs-distribution@@lucid.com
@end display

If you do not have access to the Internet, you can send mail to Lucid
and we will send back instructions on how to obtain a copy. The
address is: 

@display 
Emacs Distribution
Lucid, Inc.
707 Laurel Street
Menlo Park, California 94025
@end display

@unnumberedsec Getting Earlier Versions of GNU Emacs

Lucid GNU Emacs, the version of Emacs described in this manual, is not
distributed by the Free Software Foundation. You can, however, still get
other versions of Emacs from there.

If you have access to the Internet, you can get the latest distribution
version of GNU Emacs from host @file{prep.ai.mit.edu} using anonymous
login.  See the file @file{/u2/emacs/GETTING.GNU.SOFTWARE} on that host
to find out about your options for copying and which files to use.

You may also receive GNU Emacs when you buy a computer.  Computer
manufacturers are free to distribute copies on the same terms that apply to
everyone else.  These terms require them to give you the full sources,
including whatever changes they may have made, and to permit you to
redistribute the GNU Emacs received from them under the usual terms of the
General Public License.  In other words, the program must be free for you
when you get it, not just free for the manufacturer.

If you cannot get a copy in any of those ways, you can order one from
the Free Software Foundation.  Though Emacs itself is free, the Free
Software Foundation's distribution service is not.  An order form is
included in the file @file{etc/DISTRIB} in the Emacs distribution.
For further information, write to

@display
Free Software Foundation
675 Mass Ave
Cambridge, MA 02139
USA
@end display

The income from distribution fees goes to support the foundation's
purpose: the development of more free software to distribute just like
GNU Emacs.

If you find GNU Emacs useful, please @b{send a donation} to the Free
Software Foundation.  This will help support development of the rest of the
GNU system, and other useful software beyond that.  Your donation is tax
deductible.

@comment  node-name,  next,  previous,  up
@node License, Intro, Distrib, Top
@unnumbered GNU GENERAL PUBLIC LICENSE
@center Version 1, February 1989
@cindex license to copy Emacs
@cindex General Public License

@display
Copyright @copyright{} 1989 Free Software Foundation, Inc.
675 Mass Ave, Cambridge, MA 02139, USA

Everyone is permitted to copy and distribute verbatim copies
of this license document, but changing it is not allowed.
@end display

@unnumberedsec Preamble

  The license agreements of most software companies try to keep users
at the mercy of those companies.  By contrast, our General Public
License is intended to guarantee your freedom to share and change free
software---to make sure the software is free for all its users.  The
General Public License applies to the Free Software Foundation's
software and to any other program whose authors commit to using it.
You can use it for your programs, too.

  When we speak of free software, we are referring to freedom, not
price.  Specifically, the General Public License is designed to make
sure that you have the freedom to give away or sell copies of free
software, that you receive source code or can get it if you want it,
that you can change the software or use pieces of it in new free
programs; and that you know you can do these things.

  To protect your rights, we need to make restrictions that forbid
anyone to deny you these rights or to ask you to surrender the rights.
These restrictions translate to certain responsibilities for you if you
distribute copies of the software, or if you modify it.

  For example, if you distribute copies of a such a program, whether
gratis or for a fee, you must give the recipients all the rights that
you have.  You must make sure that they, too, receive or can get the
source code.  And you must tell them their rights.

  We protect your rights with two steps: (1) copyright the software, and
(2) offer you this license which gives you legal permission to copy,
distribute and/or modify the software.

  Also, for each author's protection and ours, we want to make certain
that everyone understands that there is no warranty for this free
software.  If the software is modified by someone else and passed on, we
want its recipients to know that what they have is not the original, so
that any problems introduced by others will not reflect on the original
authors' reputations.

  The precise terms and conditions for copying, distribution and
modification follow.

@iftex
@unnumberedsec TERMS AND CONDITIONS
@end iftex
@ifinfo
@center TERMS AND CONDITIONS
@end ifinfo

@enumerate
@item
This License Agreement applies to any program or other work which
contains a notice placed by the copyright holder saying it may be
distributed under the terms of this General Public License.  The
``Program'', below, refers to any such program or work, and a ``work based
on the Program'' means either the Program or any work containing the
Program or a portion of it, either verbatim or with modifications.  Each
licensee is addressed as ``you''.

@item
@cindex Distribution
You may copy and distribute verbatim copies of the Program's source
code as you receive it, in any medium, provided that you conspicuously and
appropriately publish on each copy an appropriate copyright notice and
disclaimer of warranty; keep intact all the notices that refer to this
General Public License and to the absence of any warranty; and give any
other recipients of the Program a copy of this General Public License
along with the Program.  You may charge a fee for the physical act of
transferring a copy.

@item
You may modify your copy or copies of the Program or any portion of
it, and copy and distribute such modifications under the terms of Paragraph
1 above, provided that you also do the following:

@itemize @bullet
@item
cause the modified files to carry prominent notices stating that
you changed the files and the date of any change; and

@item
cause the whole of any work that you distribute or publish, that
in whole or in part contains the Program or any part thereof, either
with or without modifications, to be licensed at no charge to all
third parties under the terms of this General Public License (except
that you may choose to grant warranty protection to some or all
third parties, at your option).

@item
If the modified program normally reads commands interactively when
run, you must cause it, when started running for such interactive use
in the simplest and most usual way, to print or display an
announcement including an appropriate copyright notice and a notice
that there is no warranty (or else, saying that you provide a
warranty) and that users may redistribute the program under these
conditions, and telling the user how to view a copy of this General
Public License.

@item
You may charge a fee for the physical act of transferring a
copy, and you may at your option offer warranty protection in
exchange for a fee.
@end itemize

Mere aggregation of another independent work with the Program (or its
derivative) on a volume of a storage or distribution medium does not bring
the other work under the scope of these terms.

@item
You may copy and distribute the Program (or a portion or derivative of
it, under Paragraph 2) in object code or executable form under the terms of
Paragraphs 1 and 2 above provided that you also do one of the following:

@itemize @bullet
@item
accompany it with the complete corresponding machine-readable
source code, which must be distributed under the terms of
Paragraphs 1 and 2 above; or,

@item
accompany it with a written offer, valid for at least three
years, to give any third party free (except for a nominal charge
for the cost of distribution) a complete machine-readable copy of the
corresponding source code, to be distributed under the terms of
Paragraphs 1 and 2 above; or,

@item
accompany it with the information you received as to where the
corresponding source code may be obtained.  (This alternative is
allowed only for noncommercial distribution and only if you
received the program in object code or executable form alone.)
@end itemize

Source code for a work means the preferred form of the work for making
modifications to it.  For an executable file, complete source code means
all the source code for all modules it contains; but, as a special
exception, it need not include source code for modules which are standard
libraries that accompany the operating system on which the executable
file runs, or for standard header files or definitions files that
accompany that operating system.

@item
You may not copy, modify, sublicense, distribute or transfer the
Program except as expressly provided under this General Public License.
Any attempt otherwise to copy, modify, sublicense, distribute or transfer
the Program is void, and will automatically terminate your rights to use
the Program under this License.  However, parties who have received
copies, or rights to use copies, from you under this General Public
License will not have their licenses terminated so long as such parties
remain in full compliance.

@item
By copying, distributing or modifying the Program (or any work based
on the Program) you indicate your acceptance of this license to do so,
and all its terms and conditions.

@item
Each time you redistribute the Program (or any work based on the
Program), the recipient automatically receives a license from the original
licensor to copy, distribute or modify the Program subject to these
terms and conditions.  You may not impose any further restrictions on the
recipients' exercise of the rights granted herein.

@page
@item
The Free Software Foundation may publish revised and/or new versions
of the General Public License from time to time.  Such new versions will
be similar in spirit to the present version, but may differ in detail to
address new problems or concerns.

Each version is given a distinguishing version number.  If the Program
specifies a version number of the license which applies to it and ``any
later version'', you have the option of following the terms and conditions
either of that version or of any later version published by the Free
Software Foundation.  If the Program does not specify a version number of
the license, you may choose any version ever published by the Free Software
Foundation.

@item
If you wish to incorporate parts of the Program into other free
programs whose distribution conditions are different, write to the author
to ask for permission.  For software which is copyrighted by the Free
Software Foundation, write to the Free Software Foundation; we sometimes
make exceptions for this.  Our decision will be guided by the two goals
of preserving the free status of all derivatives of our free software and
of promoting the sharing and reuse of software generally.

@iftex
@heading NO WARRANTY
@end iftex
@ifinfo
@center NO WARRANTY
@end ifinfo

@item
BECAUSE THE PROGRAM IS LICENSED FREE OF CHARGE, THERE IS NO WARRANTY
FOR THE PROGRAM, TO THE EXTENT PERMITTED BY APPLICABLE LAW.  EXCEPT WHEN
OTHERWISE STATED IN WRITING THE COPYRIGHT HOLDERS AND/OR OTHER PARTIES
PROVIDE THE PROGRAM ``AS IS'' WITHOUT WARRANTY OF ANY KIND, EITHER EXPRESSED
OR IMPLIED, INCLUDING, BUT NOT LIMITED TO, THE IMPLIED WARRANTIES OF
MERCHANTABILITY AND FITNESS FOR A PARTICULAR PURPOSE.  THE ENTIRE RISK AS
TO THE QUALITY AND PERFORMANCE OF THE PROGRAM IS WITH YOU.  SHOULD THE
PROGRAM PROVE DEFECTIVE, YOU ASSUME THE COST OF ALL NECESSARY SERVICING,
REPAIR OR CORRECTION.

@item
IN NO EVENT UNLESS REQUIRED BY APPLICABLE LAW OR AGREED TO IN WRITING WILL
ANY COPYRIGHT HOLDER, OR ANY OTHER PARTY WHO MAY MODIFY AND/OR
REDISTRIBUTE THE PROGRAM AS PERMITTED ABOVE, BE LIABLE TO YOU FOR DAMAGES,
INCLUDING ANY GENERAL, SPECIAL, INCIDENTAL OR CONSEQUENTIAL DAMAGES
ARISING OUT OF THE USE OR INABILITY TO USE THE PROGRAM (INCLUDING BUT NOT
LIMITED TO LOSS OF DATA OR DATA BEING RENDERED INACCURATE OR LOSSES
SUSTAINED BY YOU OR THIRD PARTIES OR A FAILURE OF THE PROGRAM TO OPERATE
WITH ANY OTHER PROGRAMS), EVEN IF SUCH HOLDER OR OTHER PARTY HAS BEEN
ADVISED OF THE POSSIBILITY OF SUCH DAMAGES.
@end enumerate

@iftex
@heading END OF TERMS AND CONDITIONS
@end iftex
@ifinfo
@center END OF TERMS AND CONDITIONS
@end ifinfo

@page
@unnumberedsec Appendix: How to Apply These Terms to Your New Programs

  If you develop a new program, and you want it to be of the greatest
possible use to humanity, the best way to achieve this is to make it
free software which everyone can redistribute and change under these
terms.

  To do so, attach the following notices to the program.  It is safest to
attach them to the start of each source file to most effectively convey
the exclusion of warranty; and each file should have at least the
``copyright'' line and a pointer to where the full notice is found.

@smallexample
@var{one line to give the program's name and a brief idea of what it does.}
Copyright (C) 19@var{yy}  @var{name of author}

This program is free software; you can redistribute it and/or modify
it under the terms of the GNU General Public License as published by
the Free Software Foundation; either version 1, or (at your option)
any later version.

This program is distributed in the hope that it will be useful,
but WITHOUT ANY WARRANTY; without even the implied warranty of
MERCHANTABILITY or FITNESS FOR A PARTICULAR PURPOSE.  See the
GNU General Public License for more details.

You should have received a copy of the GNU General Public License
along with this program; if not, write to the Free Software
Foundation, Inc., 675 Mass Ave, Cambridge, MA 02139, USA.
@end smallexample

Also add information on how to contact you by electronic and paper mail.

If the program is interactive, make it output a short notice like this
when it starts in an interactive mode:

@smallexample
Gnomovision version 69, Copyright (C) 19@var{yy} @var{name of author}
Gnomovision comes with ABSOLUTELY NO WARRANTY; for details type `show w'.
This is free software, and you are welcome to redistribute it
under certain conditions; type `show c' for details.
@end smallexample

The hypothetical commands `show w' and `show c' should show the
appropriate parts of the General Public License.  Of course, the
commands you use may be called something other than `show w' and `show
c'; they could even be mouse-clicks or menu items---whatever suits your
program.

@page
You should also get your employer (if you work as a programmer) or your
school, if any, to sign a ``copyright disclaimer'' for the program, if
necessary.  Here a sample; alter the names:

@example
Yoyodyne, Inc., hereby disclaims all copyright interest in the
program `Gnomovision' (a program to direct compilers to make passes
at assemblers) written by James Hacker.

@var{signature of Ty Coon}, 1 April 1989
Ty Coon, President of Vice
@end example

That's all there is to it!

@node Intro, Glossary, License, Top
@unnumbered Introduction

  You are reading about GNU Emacs, the GNU incarnation of the advanced,
self-documenting, customizable, extensible real-time display editor Emacs.
(The `G' in `GNU' is not silent.)

  We say that Emacs is a @dfn{display} editor because normally the text
being edited is visible on the screen and is updated automatically as you
type.  @xref{Screen,Display}.

  We call Emacs a @dfn{real-time} editor because the display is updated very
frequently, usually after each character or pair of characters you
type.  This minimizes the amount of information you must keep in your
head as you edit.  @xref{Basic,Real-time,Basic Editing}.

  We call Emacs advanced because it provides facilities that go beyond
simple insertion and deletion: filling of text; automatic indentation of
programs; viewing two or more files at once; and dealing in terms of
characters, words, lines, sentences, paragraphs, and pages, as well as
expressions and comments in several different programming languages.  It is
much easier to type one command meaning ``go to the end of the paragraph''
than to find that spot with simple cursor keys.

  @dfn{Self-documenting} means that at any time you can type a special
character, @kbd{Control-h}, to find out what your options are.  You can
also use @kbd{C-h} to find out what a command does, or to find all the
commands relevant to a topic.  @xref{Help}.

  @dfn{Customizable} means you can change the definitions of Emacs
commands.  For example, if you use a programming language in
which comments start with @samp{<**} and end with @samp{**>}, you can tell
the Emacs comment manipulation commands to use those strings
(@pxref{Comments}).  Another sort of customization is rearrangement of the
command set.  For example, you can set up the four basic cursor motion
commands (up, down, left and right) on keys in a diamond pattern on the
keyboard if you prefer.  @xref{Customization}.

  @dfn{Extensible} means you can go beyond simple customization and
write entirely new commands, programs in the Lisp language to be run by
Emacs's own Lisp interpreter.  Emacs is an ``on-line extensible''
system: it is divided into many functions that call each other.  You can
redefine any function in the middle of an editing session and replace
any part of Emacs without making a separate copy of all of Emacs.  Most
of the editing commands of Emacs are written in Lisp; the few
exceptions could have been written in Lisp but are written in C for
efficiency.  Only a programmer can write an extension to Emacs, but anybody
can use it afterward.

@node Screen, Keystrokes, Concept Index, Top
@comment  node-name,  next,  previous,  up
@chapter The Emacs Screen
@cindex screen
@cindex window
@cindex buffer

Since Emacs was developed on environments that only had one window
taking up the whole screen, the terminology used here reflects those
environments: 
@table @asis
@item Screen
In many environments, such as a tty terminal, an Emacs screen
literally takes up the whole screen.  If you are
running Emacs in a multi-window system like the X Window System, the
Emacs screen takes up one X window.  @xref{Emacs under X}, for more
information.@refill

@item Window
No matter what environment you are running in, Emacs allows you to look
at several buffers at the same time by having several windows be part of
the screen.  Often, the whole screen is taken up by just one window, but
you can split the screen into two or more subwindows.  If you are
running Emacs under the X window system, that means you can have several
@dfn{Emacs windows} inside the X window that contains the Emacs screen.
You can even have multiple screens in different X windows, each with
their own set of subwindows. 
@refill
@end table

Each Emacs screen displays a variety of information: 
@itemize @bullet
@item
The biggest area usually displays the text you are editing.  It may
consist of one window or of two windows or more if you need to look at two
buffers a the same time. 
@item
Below each text window's last line is a @dfn{mode line} (@pxref{Mode
Line}), which describes what is going on in that window.  The mode line
is in inverse video if the terminal supports that.  If there are several
Emacs windows in one screen, each window has its own mode line.
@item
At the bottom of each Emacs screen is the @dfn{echo area} or @dfn{minibuffer
window}(@pxref{Echo Area}).  It is used by Emacs to exchange information
with the user.  There is only one echo area per Emacs screen.
@item
If you are running Lucid GNU  Emacs under the X Window System, a
menu bar at the top of the screen makes shortcuts to several of the
commands available (@pxref{Pull-down Menus}).
@end itemize

  You can subdivide the Emacs screen into multiple text windows, and use
each window for a different file (@pxref{Windows}).  Multiple Emacs
windows are tiled vertically on the Emacs screen.  The upper Emacs window
is separated from the lower window by its mode line.

  When there are multiple, tiled Emacs windows on a single Emacs screen,
the Emacs window receiving input from the keyboard has the @dfn{keyboard
focus} and is called the @dfn{selected window}.  The selected window
contains the cursor which indicates the insertion point.  If you are
working in an environment that permits multiple Emacs screens, and you
move the focus from one Emacs screen into another, the
selected window is the one that was last selected in that screen.

  The same text can be displayed simultaneously in several Emacs
windows, which can be in different Emacs screens.  If you alter the text
in an Emacs buffer by editing it in one Emacs window, the changes are
visible in all Emacs windows containing that buffer.


@menu
* Point::	        The place in the text where editing commands operate.  
* Echo Area::           Short messages appear at the bottom of the screen.  
* Mode Line::	        Interpreting the mode line.  
* Emacs under X::       Some information on Using Emacs under the X 
                        Window System. 
* Pull-down Menus::     Menus available in Lucid GNU Emacs
@end menu


@node Emacs under X, ,Screen, Screen
@section Using Emacs under the X Window System
@comment  node-name,  next,  previous,  up


 Lucid GNU Emacs can be used with the X Window System and a window
manager like MWM or TWM.  In that case, the X Window System window
manager opens, closes, and resizes Emacs screens.  You use the window
manager's mouse gestures to perform the operations.  Consult your window
manager guide or reference manual for information on manipulating X
windows.

When you are working under X, each X window (that is, each Emacs screen)
has a menu bar for mouse-controlled operations (@pxref{Pull-down Menus}).

@cindex multi-screen Emacs
@vindex x-screen-count
@findex x-new-screen
Emacs under X is also a multi-screen Emacs.  You can call the function
@code{x-new-screen} or use the @b{New Screen} menu item from the
@b{File} menu to create a new Emacs screen in a new X window from the
same process.  The different screens will share the same buffer list,
but you can look at different buffers in the different screens. You can
control the number of screens associated with the current display, by
setting the variable @code{x-screen-count}.

@findex find-file-new-screen
The function @code{find-file-new-screen} is just like @code{find-file},
but creates a new screen to display the buffer in first.

@findex switch-to-buffer-new-screen
The function @code{switch-to-buffer-new-screen} is just 
like @code{switch-to-buffer}, but creates a new screen to display 
the buffer in first.

@vindex screen-default-alist
You can specify a different default screen size other than the one
provided. Use the variable @code{screen-default-alist}, which is an
alist of default values for screen creation other than the first one.
These may be set in your init file, like this:

@example
  (setq screen-default-alist ((width . 80) (height . 55)))
@end example

@vindex x-screen-defaults
For values specific to the first Emacs screen, you must use X resources.
The variable @code{x-screen-defaults} takes an alist of default screen
creation parameters for X window screens.  These override what is
specified in @file{~/.Xdefaults} but are overridden by the arguments to
the particular call to @code{x-create-screen}.

@vindex create-screen-hook
When you create a new screen, the variable @code{create-screen-hook}
is called with one argument, the screen just created.

If you want to close one or more of the X windows you created using
@b{New Screen}, use the @b{Close} menu item from the @b{File} menu.  

@vindex screen-title-format
@vindex screen-icon-title-format
If you are working with multiple screens, some special information
applies:
@itemize @bullet
@item Two variables, @code{screen-title-format} and
@code{screen-icon-title-format} determine the title of the screen and the
title of the icon that results if you shrink the screen. 

@vindex auto-lower-screen
@vindex auto-raise-screen
@item The variables @code{auto-lower-screen} and
@code{auto-raise-screen} position a screen. If true,
@code{auto-lower-screen} lowers a screen to the bottom when it is no longer
selected. If true, @code{auto-raise-screen} raises a screen to
the top when it is selected. Under X, most ICCCM-compliant window managers
will have options to do this for you, but these variables are provided in
case you are using a broken window manager.

@item There is a new screen/modeline format directive, %S, which expands
to the name of the current screen (a screen's name is distinct from its
title; the name is used for resource lookup, among other things, and the
title is simply what appears above the window.)
@end itemize


@node Echo Area, Mode Line, Point, Screen
@section The Echo Area
@cindex echo area

  The line at the bottom of the screen (below the mode line) is the
@dfn{echo area}.  Emacs uses this area to communicate with the user:

@itemize @bullet
@item
  @dfn{Echoing} means printing out the characters that the user types.  Emacs
never echoes single-character commands.  Multi-character commands are
echoed only if you pause while typing them: As soon as you pause for more
than one second in the middle of a command, all the characters of the command
so far are echoed.  This is intended to @dfn{prompt} you for the rest of
the command.  Once echoing has started, the rest of the command is echoed
immediately as you type it.  This behavior is designed to give confident
users fast response, while giving hesitant users maximum feedback.  You
can change this behavior by setting a variable (@pxref{Display Vars}).
@item
  If you issue a command that cannot be executed, Emacs may print an
@dfn{error message} in the echo area.  Error messages are accompanied by
a beep or by flashing the screen.  Any input you have typed ahead is
thrown away when an error happens.
@item
  Some commands print informative messages in the echo area.  These
messages look similar to error messages, but are not announced with a
beep and do not throw away input.  Sometimes a message tells you what the
command has done, when this is not obvious from looking at the text being
edited.  Sometimes the sole purpose of a command is to print a message
giving you specific information.  For example, the command @kbd{C-x =} is
used to print a message describing the character position of point in the
text and its current column in the window.  Commands that take a long time
often display messages ending in @samp{...} while they are working, and
add @samp{done} at the end when they are finished.
@item
  The echo area is also used to display the @dfn{minibuffer}, a window
that is used for reading arguments to commands, such as the name of a
file to be edited.  When the minibuffer is in use, the echo area displays
with a prompt string that usually ends with a colon.  The cursor
appears after the prompt.  You can always get out of the minibuffer by
typing @kbd{C-g}.  @xref{Minibuffer}.
@end itemize

@node Mode Line,, Echo Area, Screen
@comment  node-name,  next,  previous,  up
@section The Mode Line
@cindex mode line
@cindex top level

  Each text window's last line is a @dfn{mode line} which describes what is
going on in that window.  When there is only one text window, the mode line
appears right above the echo area.  The mode line is in inverse video if
the terminal supports that, starts and ends with dashes, and contains text
like @samp{Emacs:@: @var{something}}.

  If a mode line has something else in place of @samp{Emacs:@:
@var{something}}, the window above it is in a special subsystem
such as Dired.  The mode line then indicates the status of the
subsystem.

  Normally, the mode line has the following appearance:

@example
--@var{ch}-Emacs: @var{buf}      (@var{major} @var{minor})----@var{pos}------
@end example

@noindent
This gives information about the buffer being displayed in the window: the
buffer's name, what major and minor modes are in use, whether the buffer's
text has been changed, and how far down the buffer you are currently
looking.

  @var{ch} contains two stars @samp{**} if the text in the buffer has been
edited (the buffer is ``modified''), or @samp{--} if the buffer has not been
edited.  Exception: for a read-only buffer, it is @samp.

  @var{buf} is the name of the window's chosen @dfn{buffer}.  The chosen
buffer in the selected window (the window that the cursor is in) is also
Emacs's selected buffer, the buffer in which editing takes place.  When
we speak of what some command does to ``the buffer'', we mean the
currently selected buffer.  @xref{Buffers}.

  @var{pos} tells you whether there is additional text above the top of
the screen, or below the bottom.  If your file is small and it is
completely visible on the screen, @var{pos} is @samp{All}.  Otherwise, 
@var{pos} is @samp{Top} if you are looking at the beginning of the file,
@samp{Bot} if you are looking at the end of the file, or
@samp{@var{nn}%}, where @var{nn} is the percentage of the file above the
top of the screen.@refill

  @var{major} is the name of the @dfn{major mode} in effect in the buffer.  At
any time, each buffer is in one and only one major mode.
The available major modes include Fundamental mode (the least specialized),
Text mode, Lisp mode, and C mode.  @xref{Major Modes}, for details
on how the modes differ and how you select one.@refill

  @var{minor} is a list of some of the @dfn{minor modes} that are turned on
in the window's chosen buffer.  For example, @samp{Fill} means that Auto
Fill mode is on.  @code{Abbrev} means that Word Abbrev mode is on.
@code{Ovwrt} means that Overwrite mode is on.  @xref{Minor Modes}, for more
information.  @samp{Narrow} means that the buffer being displayed has
editing restricted to only a portion of its text.  This is not really a
minor mode, but is like one.  @xref{Narrowing}.  @code{Def} means that a
keyboard macro is being defined.  @xref{Keyboard Macros}.

  Some buffers display additional information after the minor modes.  For
example, Rmail buffers display the current message number and the total
number of messages.  Compilation buffers and Shell mode display the status
of the subprocess.

  If Emacs is currently inside a recursive editing level, square
brackets (@samp{[@dots{}]}) appear around the parentheses that surround
the modes.  If Emacs is in one recursive editing level within another,
double square brackets appear, and so on.  Since information on
recursive editing applies to Emacs in general and not to any one buffer,
the square brackets appear in every mode line on the screen or not in
any of them.  @xref{Recursive Edit}.@refill

@findex display-time
  Emacs can optionally display the time and system load in all mode lines.
To enable this feature, type @kbd{M-x display-time}.  The information added
to the mode line usually appears after the file name, before the mode names
and their parentheses.  It looks like this:

@example
@var{hh}:@var{mm}pm @var{l.ll} [@var{d}]
@end example

@noindent
(Some fields may be missing if your operating system cannot support them.)
@var{hh} and @var{mm} are the hour and minute, followed always by @samp{am}
or @samp{pm}.  @var{l.ll} is the average number of running processes in the
whole system recently.  @var{d} is an approximate index of the ratio of
disk activity to cpu activity for all users.

The word @samp{Mail} appears after the load level if there is mail for
you that you have not read yet.

@vindex mode-line-inverse-video
  Customization note: the variable @code{mode-line-inverse-video}
controls whether the mode line is displayed in inverse video (assuming
the terminal supports it); @code{nil} means no inverse video.  The
default is@code{t}.  For X screens, simply set the foreground and
background colors appropriately.
  

@node Point, Echo Area, Screen, Screen
@comment  node-name,  next,  previous,  up
@section Point
@cindex point
@cindex cursor

  When Emacs is running, the cursor shows the location at which editing
commands will take effect.  This location is called @dfn{point}.  You
can use keystrokes or the mouse cursor to move point through the text
and edit the text at different places.

  While the cursor appears to point @var{at} a character, you should
think of point as @var{between} two characters: it points @var{before}
the character on which the cursor appears.  Sometimes people speak
of ``the cursor'' when they mean ``point'', or speak of commands that
move point as ``cursor motion'' commands.

 Each Emacs screen has only one cursor.  When output is in progress, the cursor
must appear where the typing is being done.  This does not mean that
point is moving.  It is only that Emacs has no way to show you the
location of point except when the terminal is idle.

  If you are editing several files in Emacs, each file has its own point
location.  A file that is not being displayed remembers where point is.
Point becomes visible at the correct location when you look at the file again.

  When there are multiple text windows, each window has its own point
location.  The cursor shows the location of point in the selected
window.  The visible cursor also shows you which window is selected.  If
the same buffer appears in more than one window, point can be moved in
each window independently.

  The term `point' comes from the character @samp{.}, which was the
command in TECO (the language in which the original Emacs was written)
for accessing the value now called `point'.

@node Character Representation, Commands, Super and Hyper Keys, Keystrokes
@comment  node-name,  next,  previous,  up
@section Representation of Characters

This section briefly discusses how characters are represented in Emacs
buffers.  @xref{Key Sequences} for information on representing key
sequences to create key bindings. 

  ASCII graphic characters in Emacs buffers are displayed with their
graphics.  @key{LFD} is the same as a newline character; it is displayed
by starting a new line.  @key{TAB} is displayed by moving to the next
tab stop column (usually every 8 spaces).  Other control characters are
displayed as a caret (@samp{^}) followed by the non-control version of
the character; thus, @kbd{C-a} is displayed as @samp{^A}.  Non-ASCII
characters 128 and up are displayed with octal escape sequences; thus,
character code 243 (octal), also called @kbd{M-#} when used as an input
character, is displayed as @samp{\243}.

The variable @code{ctl-arrow} may be used to alter this behavior.
@xref{Display Vars}.

@node Commands, Pull-down Menus, Character Representation, Top
@section Keys and Commands

@cindex binding
@cindex customization
@cindex keymap
@cindex function
@cindex command
  This manual is full of passages that tell you what particular keys do.
But Emacs does not assign meanings to keys directly.  Instead, Emacs
assigns meanings to @dfn{functions}, and then gives keys their meanings
by @dfn{binding} them to functions.

 A function is a Lisp object that can be executed as a program.  Usually
it is a Lisp symbol that has been given a function definition; every
symbol has a name, usually made of a few English words separated by
dashes, such as @code{next-line} or @code{forward-word}.  It also has a
@dfn{definition}, which is a Lisp program.  Only some functions can be the
bindings of keys; these are functions whose definitions use
@code{interactive} to specify how to call them interactively.  Such
functions are called @dfn{commands}, and their names are @dfn{command
names}.  More information on this subject will appear in the @i{GNU
Emacs Lisp Manual}.

  The bindings between keys and functions are recorded in various tables
called @dfn{keymaps}.  @xref{Key Bindings} for more information on key
sequences you can bind commands to.  @xref{Keymaps} for information on
creating keymaps.

  When we say  ``@kbd{C-n} moves down vertically one line'' we are
glossing over a distinction that is irrelevant in ordinary use but is
vital in understanding how to customize Emacs.  The function
@code{next-line} is programmed to move down vertically.  @kbd{C-n}
has this effect @i{because} it is bound to that function.  If you rebind
@kbd{C-n} to the function @code{forward-word} then @kbd{C-n} will move
forward by words instead.  Rebinding keys is a common method of
customization.@refill

   The rest of this manual usually ignores this subtlety to keep
things simple.  To give the customizer the information needed, we often
state the name of the command that really does the work in parentheses
after mentioning the key that runs it.  For example, we will say that
``The command @kbd{C-n} (@code{next-line}) moves point vertically
down,'' meaning that @code{next-line} is a command that moves vertically
down and @kbd{C-n} is a key that is standardly bound to it.

@cindex variables
  While we are on the subject of information for customization only,
it's a good time to tell you about @dfn{variables}.  Often the
description of a command will say, ``To change this, set the variable
@code{mumble-foo}.''  A variable is a name used to remember a value.
Most of the variables documented in this manual exist just to facilitate
customization: some command or other part of Emacs uses the variable
and behaves differently depending on its setting.  Until you are interested in
customizing, you can ignore the information about variables.  When you
are ready to be interested, read the basic information on variables,
then the information on individual variables will make sense.
@xref{Variables}.

@node Pull-down Menus, Mouse Selection, Commands, Top
@comment  node-name,  next,  previous,  up
@section Lucid GNU Emacs Pull-down Menus 

If you are running Lucid GNU Emacs under X, a menu bar on top of the
Emacs screen provides access to pull-down menus of file, edit, and
help-related commands. The menus provide convenient shortcuts and an
easy interface for novice users.  They do not provide additions to the
functionality available via key commands; you can still invoke commands
from the keyboard as in previous versions of Emacs.
        
@table @b
@item File
Perform file and buffer-related operations, such as opening and closing
files, saving and printing buffers, as well as exiting Emacs.

@item Edit
Perform standard editing operations, such as 
cutting, copying, pasting and killing selected text. 

@item Buffers
Present a menu of buffers for selection as well as the option to display
a buffer list.
@cindex Buffers menu

@item Help
Access to Emacs Info.
@end table
@cindex Pull-down Menus
@cindex menus

There are two ways of selecting an item from a pull-down menu:

@page
@itemize @bullet
@item
Select an item in the menu bar by moving the cursor over it and click the
left mouse-button.  Then move the cursor over the item you want to choose
and click left again.
@item
Select an item in the menu bar by moving the cursor over it and click and
hold the left mouse-button.  With the mouse-button depressed, move the
cursor over the item you want, then release it to make your selection. 
@end itemize

If a command in the pull-down menu is not applicable in a given
situation, the command is disabled and its name appears faded.  You
cannot invoke items that are faded.  For example, most commands on the
@b{Edit} menu appear faded until you select text on which they are to
operate; after you select a block of text, edit commands are enabled.
@xref{Mouse Selection} for information on using the mouse to select
text.  @xref{Using X Selections} for related information.

There are also @kbd{M-x} equivalents for each menu item.  To find the
equivalent for any left-button menu item, do the following:

@enumerate
@item
Type @kbd{C-h k} to get the @code{Describe Key} prompt. 
@item
Select the menu item and click. 
@end enumerate

Emacs displays the function associated with the menu item in a separate
window, usually together with some documentation. 

@menu
* File Menu::           Items on the File menu.
* Edit Menu::           Items on the Edit menu. 
* Buffers Menu::        Information about the Buffers menu
* Help Menu::           Items on the Help menu. 
* Menu Customization::  Adding and removing menu items and related
                        operations.
@end menu

@node File Menu, Edit Menu, Pull-down Menus, Pull-down Menus
@subsubsection The File Menu
@comment  node-name,  next,  previous,  up

@cindex File menu

The @b{File} menu bar item contains the items @b{New Screen}, @b{Open
File...}, @b{Save Buffer}, @b{Save Buffer As...}, @b{Revert Buffer},
@b{Print Buffer}, @b{Delete Screen}, @b{Kill Buffer} and @b{Exit Emacs}
on the pull-down menu.  If you select a menu item, Emacs executes the
equivalent command.

@cindex New Screen menu item
@cindex Open File... menu item
@cindex Save Buffer menu item
@cindex Save Buffer As ... menu item
@cindex Revert Buffer menu item
@cindex Print Buffer menu item
@cindex Delete Screen menu item
@cindex Kill Buffer menu item
@cindex Exit Emacs menu item

@table @b
@item New Screen
Creates a new Emacs screen, that is, a new X window running under the
same Emacs process.  You can remove the screen using the @b{Delete
Screen} menu item. When you remove the last screen, you exit Emacs and
are prompted for confirmation.  This is bound to @kbd{C-x 5} by 
default.@refill

@item Open File...
Prompts you for a filename and loads that file into a new buffer. 
@b{Open File...} is equivalent to the Emacs command @code{find-file} (@kbd{C-x
C-f}).@refill 

@page
@item Save Buffer 
Writes and saves the current Emacs buffer as the latest
version of the current visited file.  @b{Save Buffer} is equivalent to the
Emacs command @code{save-buffer} (@kbd{C-x C-s}).@refill

@item Save Buffer As... 
Writes and saves the current Emacs buffer to the filename you specify.
@b{Save Buffer As...} is equivalent to the Emacs command
@code{write-file} (@kbd{C-x C-w}).@refill

@item Revert Buffer
Restores the last saved version of the file to the current buffer.  When
you edit a buffer containing a text file, you must save the buffer
before your changes become effective.  Use @b{Revert Buffer} if you do
not want to keep the changes you have made in the buffer.  @b{Revert
Buffer} is equivalent to the Emacs command @code{revert-file} (@kbd{M-x
revert-buffer}).@refill

@item Print Buffer
Prints a hardcopy of the current buffer.  Equivalent
to the Emacs command @code{print-buffer} (@kbd{M-x print-buffer}).@refill

@item Delete Screen 
Allows you to close all but one of the screens created by @b{New Screen}.
If you created several Emacs screens belonging to the same Emacs
process, you can close all but one of them.  When you attempt to close the
last screen, Emacs informs you that you are attempting to delete the
last screen.  You have to choose @b{Exit Emacs} for that.@refill

@item Kill Buffer...
Prompts you for the name of a buffer to kill. The default is the name of
the currently selected buffer. @b{Kill Buffer} is equivalent to the
Emacs command @code{kill-buffer} (@kbd{C-x k}). @refill

@item Exit Emacs
Shuts down (kills) the Emacs process.  Equivalent to the Emacs command
@code{save-buffers-kill-emacs} (@kbd{C-x C-c}).  Before killing the
Emacs process, the system asks which unsaved buffers to save by going through
the list of all buffers in that Emacs process.@refill
@end table

@node Edit Menu, Buffers Menu, File Menu, Pull-down Menus
@subsubsection The Edit Menu
@comment  node-name,  next,  previous,  up
@cindex Edit menu

The @b{Edit} pull-down menu contains the @b{Undo}, @b{Cut}, @b{Copy},
@b{Paste} and @b{Clear} menu items.  When you select a menu item, Emacs
executes the equivalent command.  Most commands on the @b{Edit} menu
work on a block of text, the X selection.  They appear faded until you
select a block of text (activate a region) with the mouse.  @xref{Using
X Selections}, @pxref{Killing}, and @pxref{Yanking} for more
information.@refill

@c  **** zmacs-regions is on by default these days - jwz
@c
@c Note: By default, you can use the @b{Edit} menu items on the region between
@c point an the mark as well as regions selected with the mouse. To change
@c this behavior, set the variable @code{zmacs-regions} to
@c @code{t}. @xref{Active Regions} for more information.

@cindex Undo menu item
@cindex Cut menu item
@cindex Copy menu item
@cindex Paste menu item
@cindex Clear menu item
@table @b
@item Undo 
Undoes the previous command.  @b{Undo} is equivalent to
the Emacs command @code{undo} (@kbd{C-x u}).@refill

@item Cut
Removes the selected text block from the current buffer, makes it the X
clipboard selection, and places it in the kill ring.  Before executing
this command, you have to select a region using Emacs region selection
commands or with the mouse.@refill 

@item Copy 
Makes a selected text block the X clipboard selection, and places it in
the kill ring.  You can select text using one of the Emacs region
selection commands or select a text region with the mouse.@refill

@item Paste 
Inserts the current value of the X clipboard selection in the current
buffer.  Note that this is not necessarily the same as the Emacs
@code{yank} command because the Emacs kill ring and the X clipboard
selection are not the same thing.  You can paste in text you
have placed in the clipboard using @b{Copy} or @b{Cut}.  You can also
use @b{Paste} to insert text that was pasted into the clipboard from other
applications.

@item Clear
Removes the selected text block from the current buffer but does not
place it in the kill ring or the X clipboard selection. 
@end table

@node Buffers Menu, Help Menu, Edit Menu, Pull-down Menus
@subsubsection The Buffers Menu
@comment  node-name,  next,  previous,  up
@cindex Buffers menu
The @b{Buffers} menu provides a selection of up to ten buffers and the
item @b{List All Buffers}, which provides a Buffer List. @xref{List
Buffers} for more information.  

@node Help Menu, Menu Customization, Buffers Menu, Pull-down Menus
@subsubsection The Help Menu
@comment  node-name,  next,  previous,  up
@cindex Help menu

The Help Menu gives you access to Emacs Info and provides a menu
equivalents for each of the choices you have when using @kbd{C-h}. 
@xref{Help} for more information. 

The Help menu also gives access to UNIX online manual pages via the
@b{UNIX Manual Page} option.  

@comment  node-name,  next,  previous,  up
@node Menu Customization, , Help Menu, Pull-down Menus
@subsection Customizing Lucid GNU Emacs Menus

You can customize any of the pull-down menus by adding or removing menu
items and disabling or enabling existing menu items.
 
The following functions are available: 
@table @kbd
@item add-menu: @var{(menu-path menu-name menu-items &optional before)}
Add a menu to the menubar or one of its submenus.
@item add-menu-item: @var{(menu-path item-name function enabled-p &optional before)}
Add a menu item to a menu, creating the menu first if necessary.
@item delete-menu-item: @var{(path)}
Remove the menu item defined by @var{path} from the menu hierarchy.
@item disable-menu-item: @var{(path)}
Disable the specified menu item.
@item enable-menu-item: @var{(path)}
Enable the specified previously disabled menu item.
@item relabel-menu-item: @var{(path new-name)}
Change the string of the menu item specified by @var{path} to
@var{new-name}.

@end table

@findex add-menu
@cindex adding menus
Use the function @code{add-menu} to add a new menu or submenu.
If a menu or submenu of the given name exists already, it is changed.

@page
@var{menu-path} identifies the menu under which the new menu should be
inserted.  It is a list of strings; for example, @code{("File")} names
the top-level @b{File} menu.  @code{("File" "Foo")} names a hypothetical
submenu of @b{File}.  If @var{menu-path} is @code{nil}, the menu is
added to the menubar itself.

@var{menu-name} is the string naming the menu to be added.  

@var{menu-items} is a list of menu item descriptions.  Each menu item
should be a vector of three elements:

@itemize @bullet
@item 
A string, which is the name of the menu item.
@item 
A symbol naming a command, or a form to evaluate.
@item 
@code{t} or @code{nil} to indicate whether the item is selectable.
@end itemize

The optional argument @var{before} is the name of the menu before which
the new menu or submenu should be added.  If the menu is already
present, it is not moved.

@findex add-menu-item
@cindex adding menu items
The function @code{add-menu-item} adds a menu item to the specified
menu, creating the menu first if necessary.  If the named item already
exists, the menu remains unchanged.

@var{menu-path} identifies the menu into which the new menu item should
be inserted.  It is a list of strings; for example, @code{("File")}
names the top-level @b{File} menu.  @code{("File" "Foo")} names a
hypothetical submenu of @b{File}.

@var{item-name} is the string naming the menu item to add.

@var{function} is the command to invoke when this menu item is selected.
If it is a symbol, it is invoked with @code{call-interactively}, in the
same way that functions bound to keys are invoked.  If it is a list, the
list is simply evaluated.

@var{enabled-p} controls whether the item is selectable or not.

For example, to make the @code{rename-file} command available from the
@b{File} menu, use the following code:

@example
(add-menu-item '("File") "Rename File" 'rename-file t)
@end example

To add a submenu of file management commands using a @b{File Management}
item, use the following code: 

@example
(add-menu-item '("File" "File Management") "Copy File" 'copy-file t)
(add-menu-item '("File" "File Management") "Delete File" 'delete-file t)
(add-menu-item '("File" "File Management") "Rename File" 'rename-file t)
@end example

The optional @var{before} argument is the name of a menu item before
which the new item should be added.  If the item is already present, it
is not moved.

@findex delete-menu-item
@cindex deleting menu items
To remove a specified menu item from the menu hierarchy, use
@code{delete-menu-item}.

@var{path} is a list of strings that identify the position of the menu
item in the menu hierarchy.  @code{("File" "Save")} means the menu item
called @b{Save} under the top level @b{File} menu.  @code{("Menu" "Foo"
"Item")} means the menu item called @b{Item} under the @var{Foo} submenu
of @var{Menu}.

@findex disable-menu-item
@findex enable-menu-item
@cindex enabling menu items
@cindex disabling menu items

To disable a menu item, use @code{disable-menu-item}.  The disabled
menu item is grayed and can no longer be selected.  To make the
item selectable again, use @code{enable-menu-item}.
@code{disable-menu-item} and @code{enable-menu-item} both have the
argument @var{path}.

@findex relabel-menu-item
@cindex changing menu items
To change the string of the specified menu item, use
@code{relabel-menu-item}. This function also takes the argument @var{path}.

@var{new-name} is the string to which the menu item will be changed.

@node Entering Emacs, Exiting, Mouse Selection, Top
@chapter Entering and Exiting Emacs
@cindex entering Emacs
@cindex entering Lucid GNU Emacs

  The usual way to invoke Emacs is to type @kbd{emacs @key{RET}} at
the shell.  To invoke Lucid GNU Emacs, type @kbd{lemacs @key{RET}}.
Emacs clears the screen and then displays an initial advisory message and
copyright notice.  You can begin typing Emacs commands immediately
afterward.

  Some operating systems insist on discarding all type-ahead when Emacs
starts up; they give Emacs no way to prevent this.  Therefore, it is
wise to wait until Emacs clears the screen before typing the first
editing command.

@vindex initial-major-mode
  Before Emacs reads the first command, you have not had a chance to
give a command to specify a file to edit.  Since Emacs must always have a
current buffer for editing, it presents a buffer, by default, a buffer named
@samp{*scratch*}.  The buffer is in Lisp Interaction
mode; you can use it to type Lisp expressions and evaluate them, or you
can ignore that capability and simply doodle.  You can specify a
different major mode for this buffer by setting the variable
@code{initial-major-mode} in your init file.  @xref{Init File}.

  It is possible to give Emacs arguments in the shell command line to
specify files to visit, Lisp files to load, and functions to call.

@node Exiting, Command Switches, Entering Emacs, Top
@section Exiting Emacs
@cindex exiting
@cindex killing Emacs
@cindex suspending
@cindex shrinking Lucid GNU Emacs screen

  There are two commands for exiting Emacs because there are two kinds
of exiting: @dfn{suspending} Emacs and @dfn{killing} Emacs.
@dfn{Suspending} means stopping Emacs temporarily and returning control
to its superior (usually the shell), allowing you to resume editing
later in the same Emacs job, with the same files, same kill ring, same
undo history, and so on.  This is the usual way to exit.  @dfn{Killing}
Emacs means destroying the Emacs job.  You can run Emacs again later,
but you will get a fresh Emacs; there is no way to resume the same
editing session after it has been killed.

@table @kbd
@item C-z
Suspend Emacs (@code{suspend-emacs}).  If used under the X window system,
shrink the X window containing the Emacs screen to an icon.  (see below)
@item C-x C-c
Kill Emacs (@code{save-buffers-kill-emacs}).
@end table

If you use Lucid GNU Emacs under the X window system, @kbd{C-z} shrinks
the X window containing the Emacs screen to an icon.  The Emacs process
is stopped temporarily, and control is returned to the window manager.
If more than one screen is associated with the Emacs process, only the
screen from which you used @kbd{C-z} is retained.  The X windows
containing the other Emacs screens are closed. 

To activate the "suspended" Emacs, use the appropriate window manager
mouse gestures.  Usually left-clicking on the icon reactivates and
reopens the X window containing the Emacs screen, but the window manager
you use determines what exactly happens.  To actually kill the Emacs
process, use @kbd{C-x C-c} or the @b{Exit Emacs} item on the @b{File}
menu.

@kindex C-z
@findex suspend-emacs
  On systems that do not permit programs to be suspended, @kbd{C-z} runs
an inferior shell that communicates directly with the terminal, and
Emacs waits until you exit the subshell.  On these systems, the only way
to return to the shell from which Emacs was started (to log out, for
example) is to kill Emacs.  @kbd{C-d} or @code{exit} are typical
commands to exit a subshell.

@kindex C-x C-c
@findex save-buffers-kill-emacs
  To kill Emacs, type @kbd{C-x C-c} (@code{save-buffers-kill-emacs}).  A
two-character key is used for this to make it harder to type.  In Lucid
GNU Emacs, selecting the @b{Exit Emacs} option of the @b{File} menu is an
alternate way of issuing the command.

Unless a numeric argument is used, this command first offers to save any
modified buffers.  If you do not save all buffers, you are asked for
reconfirmation with @kbd{yes} before killing Emacs, since any changes
not saved will be lost.  If any subprocesses are still running, @kbd{C-x
C-c} asks you to confirm killing them, since killing Emacs kills the
subprocesses simultaneously.

  In most programs running on Unix, certain characters may instantly
suspend or kill the program.  (In Berkeley Unix these characters are
normally @kbd{C-z} and @kbd{C-c}.)  @i{This Unix feature is turned off
while you are in Emacs.} The meanings of @kbd{C-z} and @kbd{C-x C-c} as
keys in Emacs were inspired by the standard Berkeley Unix meanings of
@kbd{C-z} and @kbd{C-c}, but that is their only relationship with Unix.
You could customize these keys to do anything (@pxref{Keymaps}).

@c ??? What about system V here?

@node Command Switches, Basic, Exiting, Top
@section Command Line Switches and Arguments
@cindex command line arguments
@cindex arguments (from shell)

  GNU Emacs supports command line arguments you can use to request
various actions when invoking Emacs.  The commands are for compatibility
with other editors and for sophisticated activities.  If you are using
Lucid GNU Emacs under the X window system, you can also use a number of
standard Xt command line arguments. Command line arguments are not usually
needed for editing with Emacs; new users can skip this section.

Many editors are designed to be started afresh each time you want to
edit.  You start the editor to edit one file; then exit the editor.  The
next time you want to edit either another file or the same one, you
start the editor again.  Under these circumstances, it makes sense to use a
command line argument to say which file to edit.

  The recommended way to use GNU Emacs is to start it only once, just
after you log in, and do all your editing in the same Emacs process.
Each time you want to edit a file, you visit it using the existing
Emacs.  Emacs creates a new buffer for each file and (unless you kill
some of the buffers), Emacs eventually has many files in it ready for
editing.  Usually you do not kill the Emacs process until you are about
to log out.  Since you usually read files by typing commands to Emacs,
command line arguments for specifying a file Emacs is started are seldom
needed.

  Emacs accepts command-line arguments that specify files to visit,
functions to call, and other activities and operating modes.  If you are
running Lucid GNU Emacs under the X window system, a number of standard
Xt command line arguments are available as well. 

The following sections list:
@itemize @bullet
@item 
Command line arguments that you can always use.
@item 
Command line arguments that have to appear at the beginning of the
argument list.
@item
Command line arguments that are only relevant if you are running Lucid
GNU Emacs under X.
@end itemize

 Command line arguments are processed in the order they appear on the
command line; however, certain arguments (the ones in the
second table) must be at the front of the list if they are used.

  Here are the arguments allowed:

@table @samp
@item @var{file}
Visit @var{file} using @code{find-file}.  @xref{Visiting}.

@item +@var{linenum} @var{file}
Visit @var{file} using @code{find-file}, then go to line number
@var{linenum} in it.

@item -l @var{file}
@itemx -load @var{file}
Load a file @var{file} of Lisp code with the function @code{load}.
@xref{Lisp Libraries}.

@item -f @var{function}
@itemx -funcall @var{function}
Call Lisp function @var{function} with no arguments.

@page
@item -i @var{file}
@itemx -insert @var{file}
Insert the contents of @var{file} into the current buffer.  This is like
what @kbd{M-x insert-buffer} does; @xref{Misc File Ops}.

@item -kill
Exit from Emacs without asking for confirmation.

@item -version
Prints version information.  This implies @samp{-batch}.

@example
% emacs -version
GNU Emacs 19.4 Lucid of Mon Dec 28 1992 on thalidomide (berkeley-unix)
@end example

@item -help
Prints a summary of command-line options, and exits.
@end table

  The following arguments are recognized only at the beginning of the
command line.  If more than one of them appears, they must appear in the
order in which they appear in this table.

@table @samp
@item -t @var{file}
Use @var{file} instead of the terminal for input and output. (This
option is currently not valid in Lucid GNU Emacs)

@cindex batch mode
@item -batch
Run Emacs in @dfn{batch mode}, which means that the text being edited is
not displayed and the standard Unix interrupt characters such as
@kbd{C-z} and @kbd{C-c} continue to have their normal effect.  Emacs in
batch mode outputs to @code{stderr} only what would normally be printed
in the echo area under program control.

Batch mode is used for running programs written in Emacs Lisp from shell
scripts, makefiles, and so on.  Normally the @samp{-l} switch or
@samp{-f} switch will be used as well, to invoke a Lisp program to do
the batch processing.

@samp{-batch} implies @samp{-q} (do not load an init file).  It also
causes Emacs to kill itself after all command switches have been
processed.  In addition, auto-saving is not done except in buffers for
which it has been explicitly requested.

@item -q
@itemx -no-init-file
Do not load your Emacs init file @file{~/.emacs}.

@item -u @var{user}
@itemx -user @var{user}
Load @var{user}'s Emacs init file @file{~@var{user}/.emacs} instead of
your own.
@end table

@vindex command-line-args
  Note that the init file can get access to the command line argument
values as the elements of a list in the variable
@code{command-line-args}.  (The arguments in the second table above will
already have been processed and will not be in the list.)  The init file
can override the normal processing of the other arguments by setting
this variable.

  One way to use command switches is to visit many files automatically:

@example
emacs *.c
@end example

@page
@noindent
passes each @code{.c} file as a separate argument to Emacs, so that
Emacs visits each file (@pxref{Visiting}).

  Here is an advanced example that assumes you have a Lisp program file
called @file{hack-c-program.el} which, when loaded, performs some useful
operation on the current buffer, expected to be a C program.

@example
emacs -batch foo.c -l hack-c-program -f save-buffer -kill > log
@end example

@noindent
Here Emacs is told to visit @file{foo.c}, load @file{hack-c-program.el}
(which makes changes in the visited file), save @file{foo.c} (note that
@code{save-buffer} is the function that @kbd{C-x C-s} is bound to), and
then exit to the shell from which the command was executed.  @samp{-batch}
guarantees there will be no problem redirecting output to @file{log},
because Emacs will not assume that it has a display terminal to work
with.

@vindex screen-title-format
@vindex screen-icon-title-format
If you are running Lucid GNU Emacs under X, a number of options are
available to control color, border, and window title and icon name:

@table @samp
@item -T @var{title}
@itemx -wn @var{title}
Use @var{title} as the window title. This sets the
@code{screen-title-format} variable, which controls the title of the X
window corresponding to the selected screen.  This is the same format as
@code{mode-line-format}.

@item -in @var{title}
@itemx -iconname @var{title}
Use @var{title} as the icon name. This sets the
@code{screen-icon-title-format} variable, which controls the title of
the icon corresponding to the selected screen.

@item -mc @var{color}
Use @var{color} as the mouse color.

@item -cr @var{color}
Use @var{color} as the text-cursor foreground color.
@end table

In addition, Lucid GNU Emacs allows you to use a number of standard Xt
command line arguments. 

@table @samp

@item -bg @var{color}
@itemx -background @var{color}
Use @var{color} as the background color.

@item -bd @var{color}
@itemx -bordercolor @var{color}
Use @var{color} as the border color.

@page
@item -bw @var{width}
@itemx -borderwidth @var{width}
Use @var{width} as the border width.

@item -d @var{display}
@itemx -display @var{display}
@item -d @var{display}
When running under the X window system, create the window containing the
Emacs screen on the display named @var{display}.

@item -fg @var{color}
@itemx -foreground @var{color}
Use @var{color} as the foreground color.

@item -fn @var{name}
@itemx -font @var{name}
Use @var{name} as the font.

@item -g @var{spec}
@itemx -geom @var{spec}
@itemx -geometry @var{spec}
Use the geometry specified by @var{spec}.

@item - iconic
Start up iconified.

@item -r
@itemx -reverse
Bring up Emacs in reverse video.

@item -name
Use the resource manager specified by @var{name}.

@item -xrm
Read something into the resource database for this invocation of Emacs. 

@item -title @var{title}
Same as @code{-wn}, sets the window title using @code{window-title-format}.
@end table

@node Basic, Undo, Command Switches, Top
@chapter Basic Editing Commands

@kindex C-h t
@findex help-with-tutorial
  We now give the basics of how to enter text, make corrections, and
save the text in a file.  If this material is new to you, you might
learn it more easily by running the Emacs learn-by-doing tutorial.  To
do this, type @kbd{Control-h t} (@code{help-with-tutorial}).

@section Inserting Text

@cindex insertion
@cindex point
@cindex cursor
@cindex graphic characters
  To insert printing characters into the text you are editing, just type
them.  This inserts the characters into the buffer at the cursor (that
is, at @dfn{point}; @pxref{Point}).  The cursor moves forward.  Any
characters after the cursor move forward too.  If the text in the buffer
is @samp{FOOBAR}, with the cursor before the @samp{B} and you type
@kbd{XX}, the result is @samp{FOOXXBAR}, with the cursor still before the
@samp{B}.

@kindex DEL
@cindex deletion
   To @dfn{delete} text you have just inserted, use @key{DEL}.
@key{DEL} deletes the character @var{before} the cursor (not the one
that the cursor is on top of or under; that is the character @var{after}
the cursor).  The cursor and all characters after it move backwards.
Therefore, if you type a printing character and then type @key{DEL},
they cancel out.

@kindex RET
@cindex newline
   To end a line and start typing a new one, type @key{RET}.  This
inserts a newline character in the buffer.  If point is in the middle of
a line, @key{RET} splits the line.  Typing @key{DEL} when the cursor is
at the beginning of a line rubs out the newline before the line, thus
joining the line with the preceding line.

  Emacs automatically splits lines when they become too long, if you
turn on a special mode called @dfn{Auto Fill} mode.  @xref{Filling}, for
information on using Auto Fill mode.

@findex delete-backward-char
@findex newline
@findex self-insert
  Customization information: @key{DEL}, in most modes, runs the command
@code{delete-backward-char}; @key{RET} runs the command @code{newline},
and self-inserting printing characters run the command
@code{self-insert}, which inserts whatever character was typed to invoke
it.  Some major modes rebind @key{DEL} to other commands.

@cindex quoting
@kindex C-q
@findex quoted-insert
  Direct insertion works for printing characters and @key{SPC}, but
other characters act as editing commands and do not insert themselves.
If you need to insert a control character or a character whose code is
above 200 octal, you must @dfn{quote} it by typing the character
@kbd{control-q} (@code{quoted-insert}) first.  There are two ways to use
@kbd{C-q}:@refill

@itemize @bullet
@item
@kbd{Control-q} followed by any non-graphic character (even @kbd{C-g})
inserts that character.
@item
@kbd{Control-q} followed by three octal digits inserts the character
with the specified character code.
@end itemize

@noindent
A numeric argument to @kbd{C-q} specifies how many copies of the quoted
character should be inserted (@pxref{Arguments}).

  If you prefer to have text characters replace (overwrite) existing
text instead of moving it to the right, you can enable Overwrite mode, a
minor mode.  @xref{Minor Modes}.

@section Changing the Location of Point

  To do more than insert characters, you have to know how to move point
(@pxref{Point}).  Here are a few of the available commands.

@kindex C-a
@kindex C-e
@kindex C-f
@kindex C-b
@kindex C-n
@kindex C-p
@kindex C-l
@kindex C-t
@kindex M->
@kindex M-<
@kindex M-r
@findex beginning-of-line
@findex end-of-line
@findex forward-char
@findex backward-char
@findex next-line
@findex previous-line
@findex recenter
@findex transpose-chars
@findex beginning-of-buffer
@findex end-of-buffer
@findex goto-char
@findex goto-line
@findex move-to-window-line
@table @kbd
@item C-a
Move to the beginning of the line (@code{beginning-of-line}).
@item C-e
Move to the end of the line (@code{end-of-line}).
@item C-f
Move forward one character (@code{forward-char}).
@item C-b
Move backward one character (@code{backward-char}).
@item M-f
Move forward one word (@code{forward-word}).
@item M-b
Move backward one word (@code{backward-word}).
@item C-n
Move down one line, vertically (@code{next-line}).  This command attempts to keep the horizontal position unchanged, so if you start in the middle of one line, you end in the middle of the next.  When on the last line of text, @kbd{C-n} creates a new line and moves onto it.  @item C-p
Move up one line, vertically (@code{previous-line}).
@item C-l
Clear the screen and reprint everything (@code{recenter}).  Text moves
on the screen to bring point to the center of the window.
@item M-r
Move point to left margin on the line halfway down the screen or
window (@code{move-to-window-line}).  Text does not move on the
screen.  A numeric argument says how many screen lines down from the
top of the window (zero for the top).  A negative argument counts from
the bottom (@minus{}1 for the bottom).
@item C-t
Transpose two characters, the ones before and after the cursor
@*(@code{transpose-chars}).
@item M-<
Move to the top of the buffer (@code{beginning-of-buffer}).  With
numeric argument @var{n}, move to @var{n}/10 of the way from the top.
@xref{Arguments}, for more information on numeric arguments.@refill
@item M->
Move to the end of the buffer (@code{end-of-buffer}).
@item M-x goto-char
Read a number @var{n} and move the cursor to character number @var{n}.
Position 1 is the beginning of the buffer.
@item M-x goto-line
Read a number @var{n} and move cursor to line number @var{n}.  Line 1
is the beginning of the buffer.
@item C-x C-n
@findex set-goal-column
Use the current column of point as the @dfn{semi-permanent goal column} for
@kbd{C-n} and @kbd{C-p} (@code{set-goal-column}).  Henceforth, those
commands always move to this column in each line moved into, or as
close as possible given the contents of the line.  This goal column remains
in effect until canceled.
@item C-u C-x C-n
Cancel the goal column.  Henceforth, @kbd{C-n} and @kbd{C-p} once
again try to avoid changing the horizontal position, as usual.
@end table

@vindex track-eol
  If you set the variable @code{track-eol} to a non-@code{nil} value,
@kbd{C-n} and @kbd{C-p} move to the end of the line when at the end of
the starting line.  By default, @code{track-eol} is @code{nil}.

@section Erasing Text

@table @kbd
@item @key{DEL}
Delete the character before the cursor (@code{delete-backward-char}).
@item C-d
Delete the character after the cursor (@code{delete-char}).
@item C-k
Kill to the end of the line (@code{kill-line}).
@item M-d
Kill forward to the end of the next word (@code{kill-word}).
@item M-@key{DEL}
Kill back to the beginning of the previous word
(@code{backward-kill-word}).
@end table

  In contrast to the @key{DEL} key, which deletes the character before
the cursor, @kbd{Control-d} deletes the character after the cursor,
causing the rest of the text on the line to shift left.  If
@kbd{Control-d} is typed at the end of a line, that line and the next
line are joined.

  To erase a larger amount of text, use @kbd{Control-k}, which kills a
line at a time.  If you use @kbd{C-k} at the beginning or in the middle
of a line, it kills all the text up to the end of the line.  If you use
@kbd{C-k} at the end of a line, it joins that line and the next
line.

  @xref{Killing}, for more flexible ways of killing text.

@section Files

@cindex files
  The commands above are sufficient for creating and altering text in an
Emacs buffer.  More advanced Emacs commands just make things easier.  But
to keep any text permanently you must put it in a @dfn{file}.  Files are
named units of text which are stored by the operating system and which
you can retrieve by name.  To look at or use the contents of a file in
any way, including editing the file with Emacs, you must specify the
file name.

  Consider a file named @file{/usr/rms/foo.c}.  To begin editing
this file from Emacs, type

@example
C-x C-f /usr/rms/foo.c @key{RET}
@end example

@noindent
The file name is given as an @dfn{argument} to the command @kbd{C-x
C-f} (@code{find-file}).  The command uses the @dfn{minibuffer} to
read the argument.  You have to type @key{RET} to terminate the argument
(@pxref{Minibuffer}).@refill

  You can also use the @b{Open File...} menu item from the @b{File} menu, then
type the name of the file to the prompt.

  Emacs obeys the command by @dfn{visiting} the file: it creates a
buffer, copies the contents of the file into the buffer, and then
displays the buffer for you to edit.  You can make changes in the
buffer, and then @dfn{save} the file by typing @kbd{C-x C-s}
(@code{save-buffer}) or choosing @b{Save Buffer} from the @b{File} menu.  This
makes the changes permanent by copying the altered contents of the
buffer back into the file @file{/usr/rms/foo.c}.  Until then, the
changes are only inside your Emacs buffer, and the file @file{foo.c} is
not changed.@refill

  To create a file, visit the file with @kbd{C-x C-f} as if it already
existed or choose @b{Open File} from the @b{File} menu and provide the
name for the new file in the minibuffer.  Emacs will create an empty
buffer in which you can insert the text you want to put in the file.
When you save the buffer with @kbd{C-x C-s}, or by choosing @b{Save
Buffer} from the File menu, the file is created.

  To learn more about using files, @pxref{Files}.

@section Help

  If you forget what a key does, you can use the Help character,
@kbd{C-h} to find out: Type @kbd{C-h k} followed by the key you want to know
about.  For example, @kbd{C-h k C-n} tells you what @kbd{C-n}
does.  @kbd{C-h} is a prefix key; @kbd{C-h k} is just one of its
subcommands (the command @code{describe-key}).  The other subcommands of
@kbd{C-h} provide different kinds of help.  Type @kbd{C-h} three times
to get a description of all the help facilities.  @xref{Help}.@refill

@menu
* Blank Lines::        Commands to make or delete blank lines.
* Continuation Lines:: Lines too wide for the screen.
* Position Info::      What page, line, row, or column is point on?
* Arguments::	       Numeric arguments for repeating a command.
@end menu

@node Blank Lines, Continuation Lines, Basic, Basic
@section Blank Lines

  Here are special commands and techniques for entering and removing
blank lines.

@c widecommands
@table @kbd
@item C-o
Insert one or more blank lines after the cursor (@code{open-line}).
@item C-x C-o
Delete all but one of many consecutive blank lines
(@code{delete-blank-lines}).
@end table

@kindex C-o
@kindex C-x C-o
@cindex blank lines
@findex open-line
@findex delete-blank-lines
  When you want to insert a new line of text before an existing line,
you just type the new line of text, followed by @key{RET}.  If you
prefer to first create a blank line and then insert the desired text,
use the key @kbd{C-o} (@code{open-line}), which inserts a newline after
point but leaves point in front of the newline.  Then type
the text into the new line.  @kbd{C-o F O O} has the same effect as
@kbd{F O O @key{RET}}, except for the final location of point.

  To create several blank lines, type @kbd{C-o} several times, or
give @kbd{C-o} an argument indicating how many blank lines to create.
@xref{Arguments}, for more information.

  If you have many blank lines in a row and want to get rid of them, use
@kbd{C-x C-o} (@code{delete-blank-lines}).  If point is on a blank
line which is adjacent to at least one other blank line, @kbd{C-x C-o}
deletes all but one of the blank lines.
If point is on a blank line with no other adjacent blank line, the
sole blank line is deleted.  If point is on a non-blank
line, @kbd{C-x C-o} deletes any blank lines following that non-blank
line.

@node Continuation Lines, Position Info, Blank Lines, Basic
@section Continuation Lines

@cindex continuation line
  If you add too many characters to one line without breaking with a
@key{RET}, the line grows to occupy two (or more) screen lines, with a
curved arrow at the extreme right margin of all but the last line.  The
curved arrow indicates that the following screen line is not really a
distinct line in the text, but just the @dfn{continuation} of a line too
long to fit the screen.  You can use Auto Fill mode, (@pxref{Filling}),
to have Emacs insert newlines automatically when a line gets too long.


@vindex truncate-lines
@cindex truncation
  Instead of continuation, long lines can be displayed by @dfn{truncation}.
This means that all the characters that do not fit in the width of the
screen or window do not appear at all.  They remain in the buffer,
temporarily invisible.  Three diagonal dots in the last column (instead of
the curved arrow inform you that truncation is in effect.

  To turn off continuation for a particular buffer set the
variable @code{truncate-lines} to non-@code{nil} in that buffer.
Truncation instead of continuation also happens whenever horizontal
scrolling is in use, and optionally whenever side-by-side windows are in
use (@pxref{Windows}).  Altering the value of @code{truncate-lines} makes
it local to the current buffer; until that time, the default value is in
effect.  The default is initially @code{nil}.  @xref{Locals}.@refill

@node Position Info, Arguments, Continuation Lines, Basic
@section Cursor Position Information

  If you are accustomed to other display editors, you may be surprised
that Emacs does not always display the page number or line number of
point in the mode line.  In Emacs, this information is only rarely
needed, and a number of commands are available to compute and print it.
Since text is stored in a way that makes it difficult to compute the
information, it is not displayed all the time.

@table @kbd
@item M-x what-page
Print page number of point, and line number within page.
@item M-x what-line
Print line number of point in the buffer.
@item M-=
Print number of lines in the current region (@code{count-lines-region}).
@item C-x =
Print character code of character after point, character position of
point, and column of point (@code{what-cursor-position}).
@end table

@findex what-page
@findex what-line
@cindex line number
@cindex page number
@kindex M-=
@findex count-lines-region

  There are several commands for printing line numbers:
@itemize @bullet
@item
@kbd{M-x what-line} counts lines from the beginning of the file and
prints the line number point is on.  The first line of the file is line
number 1.  You can use these numbers as arguments to @kbd{M-x
goto-line}.
@item
@kbd{M-x what-page} counts pages from the beginning of the file, and
counts lines within the page, printing both of them.  @xref{Pages}, for
the command @kbd{C-x l} which counts the lines in the current page.
@item
@kbd{M-=} (@code{count-lines-region}), prints the number of lines in
the region (@pxref{Mark}).
@end itemize

@kindex C-x =
@findex what-cursor-position
  The command @kbd{C-x =} (@code{what-cursor-position}) provides
information about point and about the column the cursor is in.
It prints a line in the echo area that looks like this:

@example
Char: x (0170)  point=65986 of 563027(12%)  x=44
@end example

@noindent
(In fact, this is the output produced when point is before the @samp{x=44}
in the example.)

  The two values after @samp{Char:} describe the character following point,
first by showing it and second by giving its octal character code.

  @samp{point=} is followed by the position of point expressed as a character
count.  The front of the buffer counts as position 1, one character later
as 2, and so on.  The next, larger number is the total number of characters
in the buffer.  Afterward in parentheses comes the position expressed as a
percentage of the total size.

  @samp{x=} is followed by the horizontal position of point, in columns
from the left edge of the window.

  If the buffer has been narrowed, making some of the text at the
beginning and the end temporarily invisible, @kbd{C-x =} prints
additional text describing the current visible range.  For example, it
might say

@smallexample
Char: x (0170)  point=65986 of 563025(12%) <65102 - 68533>  x=44
@end smallexample

@noindent
where the two extra numbers give the smallest and largest character position
that point is allowed to assume.  The characters between those two positions
are the visible ones.  @xref{Narrowing}.

  If point is at the end of the buffer (or the end of the visible part),
@kbd{C-x =} omits any description of the character after point.
The output looks like

@smallexample
point=563026 of 563025(100%)  x=0
@end smallexample

@node Arguments,, Position Info, Basic
@section Numeric Arguments
@cindex numeric arguments

  Any Emacs command can be given a @dfn{numeric argument}.  Some commands
interpret the argument as a repetition count.  For example, giving an
argument of ten to the key @kbd{C-f} (the command @code{forward-char}, move
forward one character) moves forward ten characters.  With these commands,
no argument is equivalent to an argument of one.  Negative arguments are
allowed.  Often they tell a command to move or act backwards.

@kindex M-1
@kindex M-@t{-}
@findex digit-argument
@findex negative-argument
  If your keyboard has a @key{META} key, the easiest way to
specify a numeric argument is to type digits and/or a minus sign while
holding down the the @key{META} key.  For example,
@example
M-5 C-n
@end example
@noindent
moves down five lines.  The characters @kbd{Meta-1}, @kbd{Meta-2}, and
so on, as well as @kbd{Meta--}, do this because they are keys bound to
commands (@code{digit-argument} and @code{negative-argument}) that are
defined to contribute to an argument for the next command.

@kindex C-u
@findex universal-argument
  Another way of specifying an argument is to use the @kbd{C-u}
(@code{universal-argument}) command followed by the digits of the argument.
With @kbd{C-u}, you can type the argument digits without holding
down shift keys.  To type a negative argument, start with a minus sign.
Just a minus sign normally means @minus{}1.  @kbd{C-u} works on all terminals.

  @kbd{C-u} followed by a character which is neither a digit nor a minus
sign has the special meaning of ``multiply by four''.  It multiplies the
argument for the next command by four.  @kbd{C-u} twice multiplies it by
sixteen.  Thus, @kbd{C-u C-u C-f} moves forward sixteen characters.  This
is a good way to move forward ``fast'', since it moves about 1/5 of a line
in the usual size screen.  Other useful combinations are @kbd{C-u C-n},
@kbd{C-u C-u C-n} (move down a good fraction of a screen), @kbd{C-u C-u
C-o} (make ``a lot'' of blank lines), and @kbd{C-u C-k} (kill four
lines).@refill

  Some commands care only about whether there is an argument, and not about
its value.  For example, the command @kbd{M-q} (@code{fill-paragraph}) with
no argument fills text; with an argument, it justifies the text as well.
(@xref{Filling}, for more information on @kbd{M-q}.)  Just @kbd{C-u} is a
handy way of providing an argument for such commands.

  Some commands use the value of the argument as a repeat count, but do
something peculiar when there is no argument.  For example, the command
@kbd{C-k} (@code{kill-line}) with argument @var{n} kills @var{n} lines,
including their terminating newlines.  But @kbd{C-k} with no argument is
special: it kills the text up to the next newline, or, if point is right at
the end of the line, it kills the newline itself.  Thus, two @kbd{C-k}
commands with no arguments can kill a non-blank line, just like @kbd{C-k}
with an argument of one.  (@xref{Killing}, for more information on
@kbd{C-k}.)@refill

  A few commands treat a plain @kbd{C-u} differently from an ordinary
argument.  A few others may treat an argument of just a minus sign
differently from an argument of @minus{}1.  These unusual cases will be
described when they come up; they are always for reasons of convenience
of use of the individual command.

@c section Autoarg Mode
@ignore
@cindex autoarg mode
  Users of ASCII keyboards may prefer to use Autoarg mode.  Autoarg mode
means that you don't need to type C-U to specify a numeric argument.
Instead, you type just the digits.  Digits followed by an ordinary
inserting character are themselves inserted, but digits followed by an
Escape or Control character serve as an argument to it and are not
inserted.  A minus sign can also be part of an argument, but only at the
beginning.  If you type a minus sign following some digits, both the digits
and the minus sign are inserted.

  To use Autoarg mode, set the variable Autoarg Mode nonzero.
@xref{Variables}.

  Autoargument digits echo at the bottom of the screen; the first nondigit
causes them to be inserted or uses them as an argument.  To insert some
digits and nothing else, you must follow them with a Space and then rub it
out.  C-G cancels the digits, while Delete inserts them all and then rubs
out the last.
@end ignore

@node Undo, Minibuffer, Basic, Top
@chapter Undoing Changes
@cindex undo
@cindex mistakes, correcting

  Emacs allows you to undo all changes you make to the text of a buffer,
up to a certain amount of change (8000 characters).  Each buffer records
changes individually, and the undo command always applies to the
current buffer.  Usually each editing command makes a separate entry
in the undo records, but some commands such as @code{query-replace}
make many entries, and very simple commands such as self-inserting
characters are often grouped to make undoing less tedious.

@table @kbd
@item C-x u
Undo one batch of changes (usually, one command worth) (@code{undo}).
@item C-_
The same.
@end table

@kindex C-x u
@kindex C-_
@findex undo
  The command @kbd{C-x u} or @kbd{C-_} allows you to undo changes.  The
first time you give this command, it undoes the last change.  Point
moves to the text affected by the undo, so you can see what was undone.

  Consecutive repetitions of the @kbd{C-_} or @kbd{C-x u} commands undo
earlier and earlier changes, back to the limit of what has been
recorded.  If all recorded changes have already been undone, the undo
command prints an error message and does nothing.

  Any command other than an undo command breaks the sequence of undo
commands.  Starting at this moment, the previous undo commands are
considered ordinary changes that can themselves be undone.  Thus, you can
redo changes you have undone by typing @kbd{C-f} or any other command
that have no important effect, and then using more undo commands.

  If you notice that a buffer has been modified accidentally, the
easiest way to recover is to type @kbd{C-_} repeatedly until the stars
disappear from the front of the mode line.  When that happens, all the
modifications you made have been cancelled.  If you do not remember
whether you changed the buffer deliberately, type @kbd{C-_} once. When
you see Emacs undo the last change you made, you probably remember why you
made it.  If the change was an accident, leave it undone.  If it was
deliberate, redo the change as described in the preceding paragraph.

  Whenever an undo command makes the stars disappear from the mode line,
the buffer contents is the same as it was when the file was last read in
or saved.

  Not all buffers record undo information.  Buffers whose names start with
spaces don't; these buffers are used internally by Emacs and its extensions
to hold text that users don't normally look at or edit.  Minibuffers,
help buffers and documentation buffers also don't record undo information.

@page
  Emacs can remember at most 8000 or so characters of deleted or
modified text in any one buffer for reinsertion by the undo command.
There is also a limit on the number of individual insert, delete or
change actions that Emacs can remembered.

  There are two keys to run the @code{undo} command, @kbd{C-x u} and
@kbd{C-_}, because on some keyboards, it is not obvious how to type
@kbd{C-_}. @kbd{C-x u} is an alternative you can type in the same
fashion on any terminal.

@node Minibuffer, M-x, Undo, Top
@chapter The Minibuffer
@cindex minibuffer

  Emacs commands use the @dfn{minibuffer} to read arguments more
complicated than a single number.  Minibuffer arguments can be file
names, buffer names, Lisp function names, Emacs command names, Lisp
expressions, and many other things, depending on the command reading the
argument.  To edit the argument in the minibuffer, you can use Emacs
editing commands.


@cindex prompt
  When the minibuffer is in use, it appears in the echo area, and the
cursor moves there.  The beginning of the minibuffer line displays a
@dfn{prompt} indicating what kind of input you should supply and how it
will be used.  The prompt is often derived from the name of the command
the argument is for.  The prompt normally ends with a colon.

@cindex default argument
  Sometimes a @dfn{default argument} appears in parentheses after the
colon; it too is part of the prompt.  The default is used as the
argument value if you enter an empty argument (e.g., just type @key{RET}).
For example, commands that read buffer names always show a default, which
is the name of the buffer that will be used if you type just @key{RET}.

@kindex C-g
  The simplest way to give a minibuffer argument is to type the text you
want, terminated by @key{RET} to exit the minibuffer.  To get out
of the minibuffer and cancel the command that it was for, type
@kbd{C-g}.

  Since the minibuffer uses the screen space of the echo area, it can
conflict with other ways Emacs customarily uses the echo area.  Here is how
Emacs handles such conflicts:

@itemize @bullet
@item
If a command gets an error while you are in the minibuffer, this does
not cancel the minibuffer.  However, the echo area is needed for the
error message and therefore the minibuffer itself is hidden for a
while.  It comes back after a few seconds, or as soon as you type
anything.

@item
If you use a command in the minibuffer whose purpose is to print a
message in the echo area (for example @kbd{C-x =}) the message is
displayed normally, and the minibuffer is hidden for a while.  It comes back
after a few seconds, or as soon as you type anything.

@item
Echoing of keystrokes does not take place while the minibuffer is in
use.
@end itemize

@menu
* File: Minibuffer File.  Entering file names with the minibuffer.
* Edit: Minibuffer Edit.  How to edit in the minibuffer.
* Completion::		  An abbreviation facility for minibuffer input.
* Repetition::		  Re-executing commands that used the minibuffer.
@end menu

@node Minibuffer File, Minibuffer Edit, Minibuffer, Minibuffer
@section Minibuffers for File Names

  Sometimes the minibuffer starts out with text in it.  For example, when
you are supposed to give a file name, the minibuffer starts out containing
the @dfn{default directory}, which ends with a slash.  This informs
you in which directory the file will be found if you do not specify one.
For example, the minibuffer might start out with

@example
Find File: /u2/emacs/src/
@end example

@noindent
where @samp{Find File:@: } is the prompt.  Typing @kbd{buffer.c} specifies
the file 
@*@file{/u2/emacs/src/buffer.c}.  To find files in nearby
directories, use @kbd{..}; thus, if you type @kbd{../lisp/simple.el}, the
file that you visit will be the one named 
@*@file{/u2/emacs/lisp/simple.el}.
Alternatively, you can use  @kbd{M-@key{DEL}} to kill directory names you
don't want (@pxref{Words}).@refill

  You can also type an absolute file name, one starting with a slash or a
tilde, ignoring the default directory.  For example, to find the file
@file{/etc/termcap}, just type the name, giving

@example
Find File: /u2/emacs/src//etc/termcap
@end example

@noindent
Two slashes in a row are not normally meaningful in Unix file names, but
they are allowed in GNU Emacs.  They mean, ``ignore everything before the
second slash in the pair.''  Thus, @samp{/u2/emacs/src/} is ignored, and
you get the file @file{/etc/termcap}.

@vindex insert-default-directory
If you set @code{insert-default-directory} to @code{nil}, the default
directory is not inserted in the minibuffer.  This way, the minibuffer
starts out empty.  But the name you type, if relative, is still
interpreted with respect to the same default directory.

@node Minibuffer Edit, Completion, Minibuffer File, Minibuffer
@section Editing in the Minibuffer

  The minibuffer is an Emacs buffer (albeit a peculiar one), and the usual
Emacs commands are available for editing the text of an argument you are
entering.

  Since @key{RET} in the minibuffer is defined to exit the minibuffer,
you must use @kbd{C-o} or @kbd{C-q @key{LFD}} to insert a newline into
the minibuffer. (Recall that a newline is really the @key{LFD}
character.)

  The minibuffer has its own window which always has space on the screen
but acts as if it were not there when the minibuffer is not in use.
When the minibuffer is in use. The minibuffer window is just like the
others; you can switch to another window with @kbd{C-x o}, edit text in
other windows and perhaps even visit more files, before returning to the
minibuffer to submit the argument.  You can kill text in another window,
return to the minibuffer window, and then yank the text to use it in the
argument.  @xref{Windows}.

  There are, however, some restrictions on the use of the minibuffer window.
You cannot switch buffers in it---the minibuffer and its window are
permanently attached.  You also cannot split or kill the minibuffer
window, but you can make it taller with @kbd{C-x ^}.

@kindex C-M-v
  If you are in the minibuffer and issue a command that displays help
text in another window, that window will be scrolled if you type
@kbd{C-M-v} while in the minibuffer until you exit the minibuffer.  This
feature is helpful if a completing minibuffer gives you a long list of
possible completions.

If the variable @code{minibuffer-confirm-incomplete} is @code{true}, you
are asked for confirmation if there is no known completion for the text
you typed. For example, if you attempted to visit a non-existent file,
the minibuffer might read:
@example
        Find File:chocolate_bar.c [no completions, confirm]
@end example
If you press @kbd{Return} again, that confirms the filename. Otherwise,
you can continue editing it. 

 Emacs supports recursive use of the minibuffer.  However, it is
easy to do this by accident (because of autorepeating keyboards, for
example) and get confused.  Therefore, most Emacs commands that use the
minibuffer refuse to operate if the minibuffer window is selected.  If the
minibuffer is active but you have switched to a different window, recursive
use of the minibuffer is allowed---if you know enough to try to do this,
you probably will not get confused.

@vindex enable-recursive-minibuffers
  If you set the variable @code{enable-recursive-minibuffers} to be
non-@code{nil}, recursive use of the minibuffer is always allowed.

@node Completion, Repetition, Minibuffer Edit, Minibuffer
@section Completion
@cindex completion

  When appropriate, the minibuffer provides a @dfn{completion} facility.
You type the beginning of an argument and one of the completion keys,
and Emacs visibly fills in the rest, depending on what you have already
typed.

  When completion is available, certain keys---@key{TAB}, @key{RET}, and
@key{SPC}---are redefined to complete an abbreviation present in the
minibuffer into a longer string that it stands for, by matching it
against a set of @dfn{completion alternatives} provided by the command
reading the argument.  @kbd{?} is defined to display a list of possible
completions of what you have inserted.

  For example, when the minibuffer is being used by @kbd{Meta-x} to read
the name of a command, it is given a list of all available Emacs command
names to complete against.  The completion keys match the text in the
minibuffer against all the command names, find any additional characters of
the name that are implied by the ones already present in the minibuffer,
and add those characters to the ones you have given.

  Case is normally significant in completion, because it is significant in
most of the names that you can complete (buffer names, file names and
command names).  Thus, @samp{fo} will not complete to @samp{Foo}.  When you
are completing a name in which case does not matter, case may be ignored
for completion's sake if specified by program.

When a completion list is displayed, the completions will highlight as
you move the mouse over them.  Clicking middle on any highlighted completion
will ``select'' it just as if you had typed it in and hit @key{RET}.

@subsection Completion Example

@kindex TAB
@findex minibuffer-complete
  Consider the following example.  If you type @kbd{Meta-x au @key{TAB}},
@key{TAB} looks for alternatives (in this case, command names) that
start with @samp{au}.  There are only two commands: @code{auto-fill-mode} and
@code{auto-save-mode}.  They are the same as far as @code{auto-}, so the
@samp{au} in the minibuffer changes to @samp{auto-}.@refill

  If you type @key{TAB} again immediately, there are multiple possibilities
for the very next character---it could be @samp{s} or @samp{f}---so no more
characters are added; but a list of all possible completions is displayed
in another window.

  If you go on to type @kbd{f @key{TAB}}, this @key{TAB} sees
@samp{auto-f}.  The only command name starting this way is
@code{auto-fill-mode}, so completion inserts the rest of that command.  You
now have @samp{auto-fill-mode} in the minibuffer after typing just @kbd{au
@key{TAB} f @key{TAB}}.  Note that @key{TAB} has this effect because in the
minibuffer it is bound to the function @code{minibuffer-complete} when
completion is supposed to be done.@refill

@subsection Completion Commands

  Here is a list of all the completion commands defined in the minibuffer
when completion is available.

@table @kbd
@item @key{TAB}
Complete the text in the minibuffer as much as possible @*
(@code{minibuffer-complete}).
@item @key{SPC}
Complete the text in the minibuffer but don't add or fill out more
than one word (@code{minibuffer-complete-word}).
@item @key{RET}
Submit the text in the minibuffer as the argument, possibly completing
first as described below (@code{minibuffer-complete-and-exit}).
@item ?
Print a list of all possible completions of the text in the minibuffer
(@code{minibuffer-list-completions}).
@item @key{button2}
Select the highlighted text under the mouse as a minibuffer response.
When the minibuffer is being used to prompt the user for a completion,
any valid completions which are visible on the screen will highlight
when the mouse moves over them.  Clicking @key{button2} will select the
highlighted completion and exit the minibuffer.  
(@code{minibuf-select-highlighted-completion}).
@end table

@kindex SPC
@findex minibuffer-complete-word
@key{SPC} completes similar to @key{TAB}, but never goes beyond the
next hyphen or space.  If you have @samp{auto-f} in the minibuffer and type
@key{SPC}, it finds that the completion is @samp{auto-fill-mode}, but it
stops completing after @samp{fill-}. The result is @samp{auto-fill-}.
Another @key{SPC} at this point completes all the way to
@samp{auto-fill-mode}.  @key{SPC} in the minibuffer runs the function
@code{minibuffer-complete-word} when completion is available.@refill

  There are three different ways that @key{RET} can work in completing
minibuffers, depending on how the argument will be used.

@itemize @bullet
@item
@dfn{Strict} completion is used when it is meaningless to give any
argument except one of the known alternatives.  For example, when
@kbd{C-x k} reads the name of a buffer to kill, it is meaningless to
give anything but the name of an existing buffer.  In strict
completion, @key{RET} refuses to exit if the text in the minibuffer
does not complete to an exact match.

@item
@dfn{Cautious} completion is similar to strict completion, except that
@key{RET} exits only if the text was an exact match already, not
needing completion.  If the text is not an exact match, @key{RET} does
not exit, but it does complete the text.  If it completes to an exact
match, a second @key{RET} will exit.

Cautious completion is used for reading file names for files that must
already exist.

@item
@dfn{Permissive} completion is used when any string is
meaningful, and the list of completion alternatives is just a guide.
For example, when @kbd{C-x C-f} reads the name of a file to visit, any
file name is allowed, in case you want to create a file.  In
permissive completion, @key{RET} takes the text in the minibuffer
exactly as given, without completing it.
@end itemize

  The completion commands display a list of all possible completions in a
window whenever there is more than one possibility for the very next
character.  Typing @kbd{?} explicitly requests such a list.  The
list of completions counts as help text, so @kbd{C-M-v} typed in the
minibuffer scrolls the list.

@vindex completion-ignored-extensions
  When completion is done on file names, certain file names are usually
ignored.  The variable @code{completion-ignored-extensions} contains a list
of strings; a file whose name ends in any of those strings is ignored as a
possible completion.  The standard value of this variable has several
elements including @code{".o"}, @code{".elc"}, @code{".dvi"} and @code{"~"}.
The effect is that, for example, @samp{foo} completes to @samp{foo.c}
even though @samp{foo.o} exists as well.  If the only possible completions
are files that end in ``ignored'' strings, they are not ignored.@refill

@vindex completion-auto-help
  If a completion command finds the next character is undetermined, it
automatically displays a list of all possible completions.  If the variable
@code{completion-auto-help} is set to @code{nil}, this does not happen,
and you must type @kbd{?} to display the possible completions.

@vindex minibuffer-confirm-incomplete
If the variable @code{minibuffer-confirm-incomplete} is set to true,
then in contexts where @code{completing-read} allows answers that are
not valid completions, an extra @key{RET} must be typed to confirm the
response.  This is helpful for catching typos, etc.

@node Repetition,, Completion, Minibuffer
@section Repeating Minibuffer Commands
@cindex command history
@cindex history of commands

  Every command that uses the minibuffer at least once is recorded on a
special history list, together with the values of the minibuffer arguments,
so that you can repeat the command easily.  In particular, every
use of @kbd{Meta-x} is recorded, since @kbd{M-x} uses the minibuffer to
read the command name.

@findex list-command-history
@c widecommands
@table @kbd
@item C-x @key{ESC}
Re-execute a recent minibuffer command @*(@code{repeat-complex-command}).
@item M-p
Within @kbd{C-x @key{ESC}}, move to previous recorded command
(@code{previous-complex-command}).
@item M-n
Within @kbd{C-x @key{ESC}}, move to the next (more recent) recorded
command (@code{next-complex-command}).@refill
@item M-x list-command-history
Display the entire command history, showing all the commands
@kbd{C-x @key{ESC}} can repeat, most recent first.@refill
@end table

@kindex C-x ESC
@findex repeat-complex-command
  @kbd{C-x @key{ESC}} is used to re-execute a recent command that used
the minibuffer. With no argument, it repeats the last command.  A numeric
argument specifies which command to repeat; 1 means the last one, and
larger numbers specify earlier commands.

  @kbd{C-x @key{ESC}} works by turning the previous command into a Lisp
expression and then entering a minibuffer initialized with the text for
that expression.  If you type just @key{RET}, the command is repeated as
before.  You can also change the command by editing the Lisp expression.
The expression you finally submit will be executed.  The repeated
command is added to the front of the command history unless it is
identical to the most recently executed command already there.

  Even if you don't understand Lisp syntax, it will probably be obvious
which command is displayed for repetition.  If you do not change the text,
you can be sure the command will repeat exactly as before.

@kindex M-n
@kindex M-p
@findex next-complex-command
@findex previous-complex-command
  If you are in the minibuffer for @kbd{C-x @key{ESC}} and the command shown
to you is not the one you want to repeat, you can move around the list of
previous commands using @kbd{M-n} and @kbd{M-p}.  @kbd{M-p} replaces the
contents of the minibuffer with the next earlier recorded command, and
@kbd{M-n} replaces it with the next later command.  After finding the
desired previous command, you can edit its expression and then
resubmit it by typing @key{RET}.  Any editing you have done on the
command to be repeated is lost if you use @kbd{M-n} or @kbd{M-p}.

@kbd{M-n} and @kbd{M-p} are specially defined within @kbd{C-x @key{ESC}}
to run the commands @code{previous-complex-command} and
@code{next-complex-command}.

@vindex command-history
  The list of previous commands using the minibuffer is stored as a Lisp
list in the variable @code{command-history}.  Each element of the list
is a Lisp expression which describes one command and its arguments.
Lisp programs can reexecute a command by feeding the corresponding
@code{command-history} element to @code{eval}.

@node M-x, Help, Minibuffer, Top
@chapter Running Commands by Name

  The Emacs commands that are used often or that must be quick to type are
bound to keys---short sequences of characters---for convenient use.  Other
Emacs commands that are used more rarely are not bound to keys; to run
them, you must refer to them by name.

  A command name consists, by convention, of one or more words,
separated by hyphens; for example, @code{auto-fill-mode} or
@code{manual-entry}.  The use of English words makes the command name
easier to remember than a key made up of obscure characters, even though
it results in more characters to type.  You can run any command by name,
even if it can be run by keys as well. 

@kindex M-x
@cindex minibuffer
 To run a command by name, to start with @kbd{M-x}, then type the
command name, and finish with @key{RET}.  @kbd{M-x} uses the minibuffer
to read the command name.  @key{RET} exits the minibuffer and runs the
command.

  Emacs uses the minibuffer for reading input for many different purposes;
on this occasion, the string @samp{M-x} is displayed at the beginning of
the minibuffer as a @dfn{prompt} to remind you that your input should be
the name of a command to be run.  @xref{Minibuffer}, for full information
on the features of the minibuffer.

  You can use completion to enter a command name.  For example, to
invoke the command @code{forward-char}, type:

@example
M-x forward-char @key{RET}
@end example
or
@example
M-x fo @key{TAB} c @key{RET}
@end example

@noindent
Note that @code{forward-char} is the same command that you invoke with
the key @kbd{C-f}.  You can call any command (interactively callable
function) defined in Emacs by its name using @kbd{M-x} whether or not
any keys are bound to it.

  If you type @kbd{C-g} while Emacs reads the command name, you cancel
the @kbd{M-x} command and get out of the minibuffer, ending up at top level.

  To pass a numeric argument to a command you are invoking with
@kbd{M-x}, specify the numeric argument before the @kbd{M-x}.  @kbd{M-x}
passes the argument along to the function that it calls.  The argument
value appears in the prompt while the command name is being read.

@findex interactive
You can use the command @code{M-x interactive} to specify a way of
parsing arguments for interactive use of a function.  For example, write

@example
  (defun foo (arg) "Doc string" (interactive "p") ...use arg...)
@end example

to make @var{arg} be the prefix argument when @code{foo} is called as a
command.  The call to @code{interactive} is actually a declaration
rather than a function; it tells @code{call-interactively} how to read
arguments to pass to the function.  When actually called, @code{interactive}
returns @code{nil}.

The argument of @var{interactive} is usually a string containing a code
letter followed by a prompt.  Some code letters do not use I/O to get
the argument and do not need prompts.  To prompt for multiple arguments,
you must provide a code letter, its prompt, a newline, and another code
letter, etc.  If the argument is not a string, it is evaluated to get a
list of arguments to pass to the function.  If you do not provide an
argument to @code{interactive}, no arguments are passed when calling
interactively.

Available code letters are:

@table @code
@item a
Function name: symbol with a function definition.
@item b
Name of existing buffer.
@item B
Name of buffer, possibly nonexistent.
@item c
Character.
@item C
Command name: symbol with interactive function definition.
@item d
Value of point as number.  Does not do I/O.
@item D
Directory name.
@item e
Last mouse event.
@item f
Existing file name.
@item F
Possibly nonexistent file name.
@item k
Key sequence (string).
@item m
Value of mark as number.  Does not do I/O.
@item n
Number read using minibuffer.
@item N
Prefix arg converted to number, or if none, do like code @code{n}.
@item p
Prefix arg converted to number.  Does not do I/O.
@item P
Prefix arg in raw form.  Does not do I/O.
@item r
Region: point and mark as two numeric arguments, smallest first.  Does
not do I/O.
@item s
Any string.
@item S
Any symbol.
@item v
Variable name: symbol that is @code{user-variable-p}.
@page
@item x
Lisp expression read but not evaluated.
@item X
Lisp expression read and evaluated.
@end table

In addition, if the string begins with @samp{*}, an error is
signaled if the buffer is read-only.  This happens before reading any
arguments.  If the string begins with @samp{@@}, the window the mouse is
over is selected before anything else is done.  You may use both
@samp{@@} and @samp{*}; they are processed in the order that they appear.

Normally, when describing a command that is run by name, we omit the
@key{RET} that is needed to terminate the name.  Thus we may refer to
@kbd{M-x auto-fill-mode} rather than @kbd{M-x auto-fill-mode} @key{RET}.
We mention the @key{RET} only when it it necessary to emphasize its
presence, for example, when describing a sequence of input that contains
a command name and arguments that follow it.

@findex execute-extended-command
  @kbd{M-x} is defined to run the command @code{execute-extended-command},
which is responsible for reading the name of another command and invoking
it.

@node Help, Mark, M-x, Top
@chapter Help
@kindex Help
@cindex help
@cindex self-documentation

  Emacs provides extensive help features which revolve around a single
character, @kbd{C-h}.  @kbd{C-h} is a prefix key that is used only for
documentation-printing commands.  The characters you can type after
@kbd{C-h} are called @dfn{help options}.  One help option is @kbd{C-h};
you use it to ask for help about using @kbd{C-h}.

  @kbd{C-h C-h} prints a list of the possible help options, and then asks
you to type the desired option.  It prompts with a string

@smallexample
A, B, C, F, I, K, L, M, N, S, T, V, W, C-c, C-d, C-n, C-w or C-h for more help:
@end smallexample

@noindent
You should type one of those characters.

  Typing a third @kbd{C-h} displays a description of what the options mean;
Emacs still waits for you to type an option.  To cancel, type @kbd{C-g}.

  Here is a summary of the defined help commands.

@table @kbd
@item C-h a @var{string} @key{RET}
Display list of commands whose names contain @var{string}
(@code{command-@*apropos}).@refill
@item C-h b
Display a table of all key bindings in effect now; local bindings of
the current major mode first, followed by all global bindings
(@code{describe-bindings}).
@item C-h c @var{key}
Print the name of the command that @var{key} runs (@code{describe-key-@*briefly}).
@kbd{c} is for `character'.  For more extensive information on @var{key},
use @kbd{C-h k}.
@item C-h f @var{function} @key{RET}
Display documentation on the Lisp function named @var{function}
(@code{describe-function}).  Note that commands are Lisp functions, so
a command name may be used.
@item C-h i
Run Info, the program for browsing documentation files (@code{info}).
The complete Emacs manual is available on-line in Info.
@item C-h k @var{key}
Display name and documentation of the command @var{key} runs (@code{describe-key}).
@item C-h l
Display a description of the last 100 characters you typed
(@code{view-lossage}).
@item C-h m
Display documentation of the current major mode (@code{describe-mode}).
@item C-h n
Display documentation of Emacs changes, most recent first
(@code{view-emacs-news}).
@item C-h p
Display a table of all mouse bindings in effect now; local bindings
of the current major mode first, followed by all global bindings
(@code{describe-pointer}).
@item C-h s
Display current contents of the syntax table, plus an explanation of
what they mean (@code{describe-syntax}).
@item C-h t
Display the Emacs tutorial (@code{help-with-tutorial}).
@item C-h v @var{var} @key{RET}
Display the documentation of the Lisp variable @var{var}
(@code{describe-@*variable}).
@item C-h w @var{command} @key{RET}
Print which keys run the command named @var{command} (@code{where-is}).
@end table

@section Documentation for a Key

@kindex C-h c
@findex describe-key-briefly
  The most basic @kbd{C-h} options are @kbd{C-h c}
(@code{describe-key-briefly}) and @kbd{C-h k}@*(@code{describe-key}).
@kbd{C-h c @var{key}} prints the name of the command that @var{key} is
bound to in the echo area.  For example, @kbd{C-h c C-f} prints
@samp{forward-char}.  Since command names are chosen to describe what
the command does, using this option is a good way to get a somewhat cryptic
description of what @var{key} does.@refill

@kindex C-h k
@findex describe-key
  @kbd{C-h k @var{key}} is similar to @kbd{C-h c} but gives more
information.  It displays the documentation string of the function
@var{key} is bound to as well as its name.  @var{key} is a string or
vector of events.  When called interactvely, @var{key} may also be a menu
selection.  This information does not usually fit into the echo area, so a
window is used for the display.

@section Help by Command or Variable Name

@kindex C-h f
@findex describe-function
@vindex describe-function-show-arlist
  @kbd{C-h f} (@code{describe-function}) reads the name of a Lisp
function using the minibuffer, then displays that function's
documentation string in a window.  Since commands are Lisp functions,
you can use the argument @var{function} to get the documentation of a
command that you know by name.  For example,

@example
C-h f auto-fill-mode @key{RET}
@end example

@noindent
displays the documentation for @code{auto-fill-mode}. Using @kbd{C-h f}
is the only way to see the documentation of a command that is not bound
to any key, that is, a command you would normally call using @kbd{M-x}.
If the variable @code{describe-function-show-arglist} is @code{t},
@code{describe-function} shows its arglist if the @var{function} is not
an autoload function.

  @kbd{C-h f} is also useful for Lisp functions you are planning to
use in a Lisp program.  For example, if you have just written the code
@code{(make-vector len)} and want to make sure you are using
@code{make-vector} properly, type @kbd{C-h f make-vector @key{RET}}.  Because
@kbd{C-h f} allows all function names, not just command names, you may find
that some of your favorite abbreviations that work in @kbd{M-x} don't work
in @kbd{C-h f}.  An abbreviation may be unique among command names, yet fail
to be unique when other function names are allowed.


If you type @key{RET}, leaving the minibuffer empty, @kbd{C-h f} by
default describes the function called by the innermost Lisp expression
in the buffer around point, @i{provided} that is a valid, defined Lisp
function name.  For example, if point is located following the text
@samp{(make-vector (car x)}, the innermost list containing point is the
one starting with @samp{(make-vector}, so the default is to describe
the function @code{make-vector}.

  @kbd{C-h f} is often useful just to verify that you have the right
spelling for the function name.  If @kbd{C-h f} mentions a default in the
prompt, you have typed the name of a defined Lisp function.  If that is
what you wanted to know, just type @kbd{C-g} to cancel the @kbd{C-h f}
command and continue editing.

@kindex C-h w
@findex where-is
  @kbd{C-h w @var{command} @key{RET}} tells you what keys are bound to
@var{command}.  It prints a list of the keys in the echo area.
Alternatively, it informs you that a command is not bound to any keys, which
implies that you must use @kbd{M-x} to call the command.@refill

@kindex C-h v
@findex describe-variable
  @kbd{C-h v} (@code{describe-variable}) is like @kbd{C-h f} but
describes Lisp variables instead of Lisp functions.  Its default is the
Lisp symbol around or before point, if that is the name of a known Lisp
variable.  @xref{Variables}.@refill

@section Apropos

@kindex C-h a
@findex command-apropos
@cindex apropos
  It is possible to ask a question like, ``What are the commands for
working with files?''  To do this, type @kbd{C-h a file @key{RET}},
which displays a list of all command names that contain @samp{file},
such as @code{copy-file}, @code{find-file}, and so on.  With each
command name a brief description of its use and information on the keys
you can use to invoke it is displayed.  For example, you would be
informed that you can invoke @code{find-file} by typing @kbd{C-x C-f}.
The @kbd{a} in @kbd{C-h a} stands for `Apropos'; @kbd{C-h a} runs the
Lisp function @code{command-apropos}.@refill

  Because @kbd{C-h a} looks only for functions whose names contain the
string you specify, you must use ingenuity in choosing the string.  If
you are looking for commands for killing backwards and @kbd{C-h a
kill-backwards @key{RET}} doesn't reveal any commands, don't give up.
Try just @kbd{kill}, or just @kbd{backwards}, or just @kbd{back}.  Be
persistent.  Pretend you are playing Adventure.  Also note that you can
use a regular expression as the argument (@pxref{Regexps}).

  Here is a set of arguments to give to @kbd{C-h a} that covers many
classes of Emacs commands, since there are strong conventions for naming
standard Emacs commands.  By giving you a feeling for the naming
conventions, this set of arguments can also help you develop a
technique for picking @code{apropos} strings.

@quotation
char, line, word, sentence, paragraph, region, page, sexp, list, defun,
buffer, screen, window, file, dir, register, mode,
beginning, end, forward, backward, next, previous, up, down, search, goto,
kill, delete, mark, insert, yank, fill, indent, case,
change, set, what, list, find, view, describe.
@end quotation

@findex apropos
  To list all Lisp symbols that contain a match for a regexp, not just
the ones that are defined as commands, use the command @kbd{M-x apropos}
instead of @kbd{C-h a}.

@section Other Help Commands

@kindex C-h i
@findex info
  @kbd{C-h i} (@code{info}) runs the Info program, which is used for
browsing through structured documentation files.  The entire Emacs manual
is available within Info.  Eventually all the documentation of the GNU
system will be available.  Type @kbd{h} after entering Info to run
a tutorial on using Info.

@kindex C-h l
@findex view-lossage
  If something surprising happens, and you are not sure what commands you
typed, use @kbd{C-h l} (@code{view-lossage}).  @kbd{C-h l} prints the last
100 command characters you typed.  If you see commands you don't
know, use @kbd{C-h c} to find out what they do.

@kindex C-h m
@findex describe-mode
  Emacs has several major modes. Each mode redefines a few keys and
makes a few other changes in how editing works.  @kbd{C-h m}
(@code{describe-mode}) prints documentation on the current major mode,
which normally describes all the commands that are changed in this mode.

@kindex C-h b
@findex describe-bindings
  @kbd{C-h b} (@code{describe-bindings}) and @kbd{C-h s}
(@code{describe-syntax}) present information about the current Emacs
mode that is not covered by @kbd{C-h m}.  @kbd{C-h b} displays a list of
all key bindings now in effect; the local bindings of the current
major mode first, followed by the global bindings (@pxref{Key
Bindings}).  @kbd{C-h s} displays the contents of the syntax table, with
explanations of each character's syntax (@pxref{Syntax}).@refill

@kindex C-h n
@findex view-emacs-news
@kindex C-h t
@findex help-with-tutorial
@kindex C-h C-c
@findex describe-copying
@kindex C-h C-d
@findex describe-distribution
@kindex C-h C-w
@findex describe-no-warranty
  The other @kbd{C-h} options display various files of useful
information.  @kbd{C-h C-w} displays details on the complete
absence of warranty for GNU Emacs.  @kbd{C-h n} (@code{view-emacs-news})
displays the file @file{emacs/etc/NEWS}, which contains documentation on
Emacs changes arranged chronologically.  @kbd{C-h t}
(@code{help-with-tutorial}) displays the learn-by-doing Emacs tutorial.
@kbd{C-h C-c} (@code{describe-copying}) displays the file
@file{emacs/etc/COPYING}, which tells you the conditions you must obey
in distributing copies of Emacs.  @kbd{C-h C-d}
(@code{describe-distribution}) displays another file named
@file{emacs/etc/DISTRIB}, which tells you how you can order a copy of
the latest version of Emacs.@refill

@node Mark, Killing, Help, Top
@chapter Selecting Text
@cindex mark
@cindex region

  Many Emacs commands operate on an arbitrary contiguous
part of the current buffer. You can select text in two ways:

@itemize @bullet
@item
You use special keys to select text by defining a region between point
and the mark. 
@item
If you are running Lucid GNU Emacs under X, you can also select text
with the mouse. 
@end itemize

@section The Mark and the Region
 To specify the text for a command to operate on, set @dfn{the
mark} at one end of it, and move point to the other end.  The text
between point and the mark is called @dfn{the region}.  You can move
point or the mark to adjust the boundaries of the region.  It doesn't
matter which one is set first chronologically, or which one comes
earlier in the text.
  
  Once the mark has been set, it remains until it is set again at
another place.  The mark remains fixed with respect to the preceding
character if text is inserted or deleted in a buffer.  Each Emacs
buffer has its own mark; when you return to a buffer that had been
selected previously, it has the same mark it had before.

  Many commands that insert text, such as @kbd{C-y} (@code{yank}) and
@kbd{M-x insert-buffer}, position the mark at one end of the inserted
text---the opposite end from where point is positioned, so that the region
contains the text just inserted.

  Aside from delimiting the region, the mark is useful for marking
a spot that you may want to go back to.  To make this feature more useful,
Emacs remembers 16 previous locations of the mark in the @code{mark ring}.

@menu
* Setting Mark::	Commands to set the mark.
* Using Region::	Summary of ways to operate on contents of the region.
* Marking Objects::	Commands to put region around textual units.
* Mark Ring::   	Previous mark positions saved so you can go back there.
* Using X Selections::  Using the mouse to mark a region if running under X.
@end menu

@node Setting Mark, Using Region, Mark, Mark
@subsection Setting the Mark

  Here are some commands for setting the mark:

@c WideCommands
@table @kbd
@item C-@key{SPC}
Set the mark where point is (@code{set-mark-command}).
@item C-@@
The same.
@item C-x C-x
Interchange mark and point (@code{exchange-point-and-mark}).
@page
@item C-<
Pushes a mark at the beginning of the buffer.
@item C->
Pushes a mark at the end of the buffer.
@end table

  For example, to convert part of the buffer to all
upper-case, you can use the @kbd{C-x C-u} (@code{upcase-region})
command, which operates on the text in the region.  First go to the
beginning of the text you want to capitalize and type @kbd{C-@key{SPC}} to
put the mark there, then move to the end, and then type @kbd{C-x C-u} to
capitalize the selected region.  You can also set the mark at the end of the
text, move to the beginning, and then type @kbd{C-x C-u}.  Most commands
that operate on the text in the region have the word @code{region} in
their names.

@kindex C-SPC
@findex set-mark-command
  The most common way to set the mark is with the @kbd{C-@key{SPC}}
command (@code{set-mark-command}).  This command sets the mark where
point is. You can then move point away, leaving the mark behind.  It is
actually incorrect to speak of the character @kbd{C-@key{SPC}}; there is
no such character.  When you type @key{SPC} while holding down
@key{CTRL}, you get the character @kbd{C-@@} on most terminals. This
character is actually bound to @code{set-mark-command}.  But unless you are
unlucky enough to have a terminal where typing @kbd{C-@key{SPC}} does
not produce @kbd{C-@@}, you should think of this character as
@kbd{C-@key{SPC}}.

@kindex C-x C-x
@findex exchange-point-and-mark
  Since terminals have only one cursor, Emacs cannot show you where the
mark is located. Most people use the mark soon after they set it, before
they forget where it is. But you can see where the mark is with the
command @kbd{C-x C-x} (@code{exchange-point-and-mark}) which puts the
mark where point was and point where the mark was.  The extent of the
region is unchanged, but the cursor and point are now at the previous
location of the mark. 

@kindex C-<
@kindex C->
@findex mark-bob
@findex mark-eob
 Another way to set the mark is to push the mark to the beginning of a
buffer while leaving point at its original location. If you supply an
argument to @kbd{C-<} (@code{mark-bob}), the mark is pushed
@var{n}/10 of the way from the true beginning of the buffer. You can
also set the mark at the end of a buffer with @kbd{C->}
(@code{mark-eob}). It pushes the mark to the end of the buffer, leaving
point alone. Supplying an argument to the command, pushes the mark
@var{n}/10 of the way from the true end of the buffer.

If you are using Lucid GNU Emacs under the X window system, you can set
the variable @code{zmacs-regions} to @code{t}. This makes the current
region (defined by point and mark) highlight and makes it available as
the X clipboard selection, which means you can use the menu bar items on
it.  @xref{Active Regions} for more information.
 
  @kbd{C-x C-x} is also useful when you are satisfied with the location of
point but want to move the mark; do @kbd{C-x C-x} to put point there and
then you can move it.  A second use of @kbd{C-x C-x}, if necessary, puts
the mark at the new location with point back at its original location.

@menu
* Active Regions::  Having the region highlight while working under X.
@end menu

@node Using Region, Marking Objects, Setting Mark, Mark
@subsection Operating on the Region

  Once you have created an active region, you can do many things to
the text in it:
@itemize @bullet
@item
Kill it with @kbd{C-w} (@pxref{Killing}).
@item
Save it in a register with @kbd{C-x x} (@pxref{Registers}).
@item
Save it in a buffer or a file (@pxref{Accumulating Text}).
@item
Convert case with @kbd{C-x C-l} or @kbd{C-x C-u} @*(@pxref{Case}).
@item
Evaluate it as Lisp code with @kbd{M-x eval-region} (@pxref{Lisp Eval}).
@item
Fill it as text with @kbd{M-g} (@pxref{Filling}).
@item
Print hardcopy with @kbd{M-x print-region} (@pxref{Hardcopy}).
@item
Indent it with @kbd{C-x @key{TAB}} or @kbd{C-M-\} (@pxref{Indentation}).
@end itemize

@node Marking Objects, Mark Ring, Using Region, Mark
@subsection Commands to Mark Textual Objects

  There are commands for placing point and the mark around a textual
object such as a word, list, paragraph or page.
 
@table @kbd
@item M-@@
Set mark after end of next word (@code{mark-word}).  This command and
the following one do not move point.
@item C-M-@@
Set mark after end of next Lisp expression (@code{mark-sexp}).
@item M-h
Put region around current paragraph (@code{mark-paragraph}).
@item C-M-h
Put region around current Lisp defun (@code{mark-defun}).
@item C-x h
Put region around entire buffer (@code{mark-whole-buffer}).
@item C-x C-p
Put region around current page (@code{mark-page}).
@end table

@kindex M-@@
@kindex C-M-@@
@findex mark-word
@findex mark-sexp
@kbd{M-@@} (@code{mark-word}) puts the mark at the end of the next word,
while @kbd{C-M-@@} (@code{mark-sexp}) puts it at the end of the next Lisp
expression. These characters sometimes save you some typing.

@kindex M-h
@kindex C-M-h
@kindex C-x C-p
@kindex C-x h
@findex mark-paragraph
@findex mark-defun
@findex mark-page
@findex mark-whole-buffer
   A number of commands are available that set both point and mark and
thus delimit an object in the buffer.  @kbd{M-h} (@code{mark-paragraph})
moves point to the beginning of the paragraph that surrounds or follows
point, and puts the mark at the end of that paragraph
(@pxref{Paragraphs}).  You can then indent, case-convert, or kill the
whole paragraph.  In the same fashion, @kbd{C-M-h} (@code{mark-defun})
puts point before and the mark after the current or following defun
(@pxref{Defuns}).  @kbd{C-x C-p} (@code{mark-page}) puts point before
the current page (or the next or previous, depending on the argument),
and mark at the end (@pxref{Pages}).  The mark goes after the
terminating page delimiter (to include it), while point goes after the
preceding page delimiter (to exclude it).  Finally, @kbd{C-x h}
(@code{mark-whole-buffer}) sets up the entire buffer as the region by
putting point at the beginning and the mark at the end.

@node Mark Ring,, Marking Objects, Mark
@subsection The Mark Ring

@kindex C-u C-SPC
@cindex mark ring
@kindex C-u C-@@
  Aside from delimiting the region, the mark is also useful for marking
a spot that you may want to go back to.  To make this feature more
useful, Emacs remembers 16 previous locations of the mark in the
@dfn{mark ring}.  Most commands that set the mark push the old mark onto
this ring.  To return to a marked location, use @kbd{C-u C-@key{SPC}}
(or @kbd{C-u C-@@}); this is the command @code{set-mark-command} given a
numeric argument.  The command moves point to where the mark was, and
restores the mark from the ring of former marks. Repeated use of this
command moves point to all the old marks on the ring, one by one.
The marks you have seen go to the end of the ring, so no marks are lost.

  Each buffer has its own mark ring.  All editing commands use the current
buffer's mark ring.  In particular, @kbd{C-u C-@key{SPC}} always stays in
the same buffer.

  Many commands that can move long distances, such as @kbd{M-<}
(@code{beginning-of-buffer}), start by setting the mark and saving the
old mark on the mark ring.  This makes it easier for you to move back
later.  Searches set the mark, unless they do not actually move point.
When a command sets the mark, @samp{Mark Set} is printed in the
echo area.

@vindex mark-ring-max
  The variable @code{mark-ring-max} is the maximum number of entries to
keep in the mark ring.  If that many entries exist and another entry is
added, the last entry in the list is discarded.  Repeating @kbd{C-u
C-@key{SPC}} circulates through the entries that are currently in the
ring.

@vindex mark-ring
  The variable @code{mark-ring} holds the mark ring itself, as a list of
marker objects in the order most recent first.  This variable is local
in every buffer.
 
@node Mouse Selection, Entering Emacs, Pull-Down Menus, Top
@comment  node-name,  next,  previous,  up
@section Selecting Text with the Mouse
@cindex mouse selection

@cindex cursor shapes
  If you are using Lucid GNU Emacs under X, you can use the mouse cursor
to select text. There are two mouse cursor shapes:
@itemize @bullet
@item
When the mouse cursor is over text, it appears as an I-beam, the same
cursor that @code{xterm} uses. 
@item
When the mouse cursor is not over text, it appears as a plus sign (+).
@end itemize

@vindex x-mode-pointer-shape
@vindex x-nontext-pointer-shape
@vindex x-pointer-shape
You can set the value of the variable @code{x-mode-pointer-shape} to
determine the shape of the mouse pointer when it is over the modeline.  If
the value is @code{nil}, the variable @code{x-nontext-pointer-shape} or
the variable @code{x-pointer-shape} is used.

@vindex x-pointer-background-color
@vindex x-pointer-foreground-color
If you want to get fancy, you can set the foreground and background
colors of the mouse pointer with the variables
@code{x-pointer-background-color} and @code{x-pointer-foreground-color}.

There are two ways to select a region of text with the mouse:

  To select a word in text, double click with the left mouse button
while the mouse cursor is over the word.  The word is highlighted when
selected. On monochrome monitors, a stipled background indicates that a
region of text has been highlighted. On color monitors, a color
background indicates highlighted text. You can triple-click to select
whole lines. 

To select an arbitrary region of text:

@enumerate
@item
Move the mouse cursor over the character at the beginning of the region of
text you want to select.
@item
Press and hold the left mouse button. 
@item
While holding the left mouse button down, drag the cursor to the
character at the end of the region of text you want to select.
@item
Release the left mouse button.
@end enumerate
The selected region of text is highlighted.

  Once a region of text is selected, it becomes the primary X selection
(@pxref{Using X Selections}) as well as the Emacs selected region. You
can paste it into other X applications and use the options from the
@b{Edit} pull-down menu on it.  Since it is also the Emacs region, you
can use Emacs region commands on it.

@comment @node Mouse Selection, Entering Emacs, Pull-Down Menus, Top
@comment  node-name,  next,  previous,  up
@subsection Additional Mouse Operations
@cindex mouse operations

Lucid GNU Emacs also provides the following mouse functions.
Most of these are not bound to mouse gestures by default, but they are
provided for your customization pleasure.  For example, if you
wanted @kbd{shift-left} (that is, holding down the shift key
and clicking the left button) to delete the character at which
you are pointing, then you could do this:

@example
(global-set-key '(shift button1) 'mouse-del-char)
@end example

@findex mouse-del-char
@findex mouse-delete-window
@findex mouse-keep-one-window
@findex mouse-kill-line
@findex mouse-line-length
@findex mouse-scroll
@findex mouse-select
@findex mouse-select-and-split
@findex mouse-set-mark
@findex mouse-set-point
@findex mouse-track
@findex mouse-track-adjust
@findex mouse-track-and-copy-to-cutbuffer
@findex mouse-track-delete-and-insert

@table @kbd
@item mouse-del-char
Delete the character pointed to by the mouse.
@item mouse-delete-window
Delete the Emacs window that the mouse is on.
@item mouse-keep-one-window
Select the Emacs window that the mouse is on, then delete all other
windows on this screen.
@item mouse-kill-line
Kill the line pointed to by the mouse.
@item mouse-line-length
Print the length of the line indicated by the pointer.
@item mouse-scroll
Scroll point to the mouse position.
@item mouse-select
Select the Emacs window the mouse is on.
@item mouse-select-and-split
Select the Emacs window mouse is on, then split it vertically in half.
@item mouse-set-mark
Select the Emacs window the mouse is on and set the mark at the mouse 
position.  Display the cursor at that position for a second.
@item mouse-set-point
Select the Emacs window that the mouse is on and move point to the
mouse position.
@item mouse-track
Make a selection with the mouse.   This is the default binding of 
the left mouse button (@key{button1}).
@item mouse-track-adjust
Extend the existing selection.  This is the default binding of
@key{shift-button1}.
@item mouse-track-and-copy-to-cutbuffer
Make a selection like @code{mouse-track}, but also copy it to the cut buffer.
@item mouse-track-delete-and-insert
Make a selection with the mouse and insert it at point.  This is the
default binding of @key{control-shift-button1}.
@item mouse-track-insert
Make a selection with the mouse and insert it at point.
This is the default binding of @key{control-button1}.
@item mouse-window-to-region
Narrow a window to the region between the cursor and the mouse pointer.
@end table

The @kbd{M-x mouse-track} command should be bound to a mouse button.  If
you click-and-drag, the selection is set to the region between the
point of the initial click and the point at which you release the
button.  These positions do not need to be ordered. 

If you click-and-release without moving the mouse, the point is moved,
and the selection is disowned (there will be no selection owner.)  The
mark will be set to the previous position of point.

If you double-click, the selection will extend by symbols instead of by
characters.  If you triple-click, the selection will extend by lines.

If you drag the mouse off the top or bottom of the window, you can
select pieces of text that are larger than the visible part of the
buffer; the buffer will scroll as necessary.

The selected text becomes the current X selection, and is also copied to
the top of the kill ring.  Point will be left at the position at
which you released the button and the mark will be left at the initial
click position.  Bind a mouse click to @kbd{mouse-track-and-copy-to-cutbuffer} to copy
selections to the cut buffer.

See also the @code{mouse-track-adjust} command, on @kbd{Sh-button1}.

The @kbd{M-x mouse-track-adjust} command should be bound to a mouse
button.  The selection will be enlarged or shrunk so that the point of
the mouse click is one of its endpoints.  This is only meaningful
after the @code{mouse-track} command (@key{button1}) has been executed.

The @kbd{M-x mouse-track-delete-and-insert} command is exactly the same
as the @code{mouse-track} command on @key{button1}, except that point is
not moved; the selected text is immediately inserted after being
selected; and the text of the selection is deleted.

The @kbd{M-x mouse-track-insert} command is exactly the same as the
@code{mouse-track} command on @key{button1}, except that point is not moved;
the selected text is immediately inserted after being selected; and the
selection is immediately disowned afterwards.

@iftex
@chapter Killing and Moving Text

  @dfn{Killing} means erasing text and copying it into the @dfn{kill ring},
from which it can be retrieved by @dfn{yanking} it.  Some other systems
that have recently become popular use the terms ``cutting'' and ``pasting''
for these operations.

  The most common way of moving or copying text with Emacs is to kill it
and later yank it in one or more places.  This is safe because all the
text killed recently is stored in the kill ring, and it is versatile,
because you can use the same commands for killing syntactic units and
for moving those units.  There are other ways of copying text for
special purposes.

  Emacs has only one kill ring, so you can kill text in one buffer and yank
it in another buffer. If you are using Lucid GNU Emacs under X, you can,
also use the X selection mechanism to copy text from one buffer to
another, or between applications. @xref{Using X Selections}.

@end iftex

@node Killing, Yanking, Mark, Top
@section Deletion and Killing
@findex delete-char
@c ??? Should be backward-delete-char
@findex delete-backward-char

@cindex killing
@cindex cutting
@cindex deletion
@kindex C-d
@kindex DEL
  Most commands which erase text from the buffer save it. You can get
the text back if you change your mind, or you can move or copy it to
other parts of the buffer.  Commands which erase text and save it in the
kill ring are known as @dfn{kill} commands.  Some other commands erase
text but do not save it; they are known as @dfn{delete} commands.  (This
distinction is made only for erasing text in the buffer.)

The commands' names and individual descriptions use the words
@samp{kill} and @samp{delete} to indicate what they do.  If you perform
a kill or delete command by mistake, use the @kbd{C-x u} (@code{undo})
command to undo it (@pxref{Undo}). The delete commands include @kbd{C-d}
(@code{delete-char}) and @key{DEL} (@code{delete-backward-char}), which
delete only one character at a time, and those commands that delete only
spaces or newlines.  Commands that can destroy significant amounts of
nontrivial data usually kill.@refill

@subsection Deletion

@table @kbd
@item C-d
Delete next character (@code{delete-char}).
@item @key{DEL}
Delete previous character (@code{delete-backward-char}).
@item M-\
Delete spaces and tabs around point (@code{delete-horizontal-space}).
@item M-@key{SPC}
Delete spaces and tabs around point, leaving one space
(@code{just-one-space}).
@item C-x C-o
Delete blank lines around the current line (@code{delete-blank-lines}).
@item M-^
Join two lines by deleting the intervening newline, and any indentation
following it (@code{delete-indentation}).
@end table

  The most basic delete commands are @kbd{C-d} (@code{delete-char}) and
@key{DEL} (@code{delete-backward-char}).  @kbd{C-d} deletes the
character after point, the one the cursor is ``on top of''.  Point
doesn't move.  @key{DEL} deletes the character before the cursor, and
moves point back.  You can delete newlines like any other characters in
the buffer; deleting a newline joins two lines.  Actually, @kbd{C-d} and
@key{DEL} aren't always delete commands; if you give them an argument,
they kill instead, since they can erase more than one character this
way.

@kindex M-\
@findex delete-horizontal-space
@kindex M-SPC
@findex just-one-space
@kindex C-x C-o
@findex delete-blank-lines
@kindex M-^
@findex delete-indentation
  The other delete commands delete only formatting characters: spaces,
tabs and newlines.  @kbd{M-\} (@code{delete-horizontal-space}) deletes
all spaces and tab characters before and after point.
@kbd{M-@key{SPC}} (@code{just-one-space}) does the same but leaves a
single space after point, regardless of the number of spaces that
existed previously (even zero).

  @kbd{C-x C-o} (@code{delete-blank-lines}) deletes all blank lines after
the current line. If the current line is blank, it deletes all blank lines
preceding the current line as well as leaving one blank line, the current
line.  @kbd{M-^} (@code{delete-indentation}) joins the current line and
the previous line, or the current line and the next line if given an
argument, by deleting a newline and all surrounding spaces, possibly
leaving a single space.  @xref{Indentation,M-^}.

@subsection Killing by Lines

@table @kbd
@item C-k
Kill rest of line or one or more lines (@code{kill-line}).
@end table

@kindex C-k
@findex kill-line
  The simplest kill command is @kbd{C-k}.  If given at the beginning of
a line, it kills all the text on the line, leaving the line blank.  If
given on a blank line, the blank line disappears.  As a consequence, a
line disappears completely if you go to the front of a non-blank line
and type @kbd{C-k} twice.

  More generally, @kbd{C-k} kills from point up to the end of the line,
unless it is at the end of a line.  In that case, it kills the newline
following the line, thus merging the next line into the current one.
Emacs ignores invisible spaces and tabs at the end of the line when deciding
which case applies: if point appears to be at the end of the line, you
can be sure the newline will be killed.

  If you give @kbd{C-k} a positive argument, it kills that many lines
and the newlines that follow them (however, text on the current line
before point is not killed).  With a negative argument, @kbd{C-k} kills
back to a number of line beginnings.  An argument of @minus{}2 means
kill back to the second line beginning.  If point is at the beginning of
a line, that line beginning doesn't count, so @kbd{C-u - 2 C-k} with
point at the front of a line kills the two previous lines.

  @kbd{C-k} with an argument of zero kills all the text before point on the
current line.

@subsection Other Kill Commands
@findex kill-line
@findex kill-region
@findex kill-word
@findex backward-kill-word
@findex kill-sexp
@findex kill-sentence
@findex backward-kill-sentence
@kindex M-d
@kindex M-DEL
@kindex C-M-k
@kindex C-x DEL
@kindex M-k
@kindex C-k
@kindex C-w

@c DoubleWideCommands
@table @kbd
@item C-w
Kill region (from point to the mark) (@code{kill-region}).
@xref{Words}.
@item M-d
Kill word (@code{kill-word}).
@item M-@key{DEL}
Kill word backwards (@code{backward-kill-word}).
@item C-x @key{DEL}
Kill back to beginning of sentence (@code{backward-kill-sentence}).
@xref{Sentences}.
@item M-k
Kill to end of sentence (@code{kill-sentence}).
@item C-M-k
Kill sexp (@code{kill-sexp}).  @xref{Lists}.
@item M-z @var{char}
Kill up to next occurrence of @var{char} (@code{zap-to-char}).
@end table

   @kbd{C-w} (@code{kill-region}) is a very general kill command; it
kills everything between point and the mark. You can use this command to
kill any contiguous sequence of characters by first setting the mark at
one end of a sequence of characters, then going to the other end and
typing @kbd{C-w}.

@kindex M-z
@findex zap-to-char
  A convenient way of killing is combined with searching: @kbd{M-z}
(@code{zap-to-char}) reads a character and kills from point up to (but not
including) the next occurrence of that character in the buffer.  If there
is no next occurrence, killing goes to the end of the buffer.  A numeric
argument acts as a repeat count.  A negative argument means to search
backward and kill text before point.

  Other syntactic units can be killed: words, with @kbd{M-@key{DEL}} and
@kbd{M-d} (@pxref{Words}); sexps, with @kbd{C-M-k} (@pxref{Lists}); and
sentences, with @kbd{C-x @key{DEL}} and @kbd{M-k}
(@pxref{Sentences}).@refill

@node Yanking, Using X Selections, Killing, Top
@section Yanking
@cindex moving text
@cindex copying text
@cindex kill ring
@cindex yanking
@cindex pasting

  @dfn{Yanking} means getting back text which was killed. Some systems
call this ``pasting''.  The usual way to move or copy text is to kill it
and then yank it one or more times.

@page
@table @kbd
@item C-y
Yank last killed text (@code{yank}).
@item M-y
Replace re-inserted killed text with the previously killed text
(@code{yank-pop}).
@item M-w
Save region as last killed text without actually killing it
(@code{copy-region-as-kill}).
@item C-M-w
Append next kill to last batch of killed text (@code{append-next-kill}).
@end table

@menu
* Kill Ring::       Where killed text is stored.  Basic yanking.
* Appending Kills:: Several kills in a row all yank together.
* Earlier Kills::   Yanking something killed some time ago.
@end menu

@node Kill Ring, Appending Kills, Yanking, Yanking
@subsection The Kill Ring

@kindex C-y
@findex Yank
  All killed text is recorded in the @dfn{kill ring}, a list of blocks of
text that have been killed.  There is only one kill ring, used in all
buffers, so you can kill text in one buffer and yank it in another buffer.
This is the usual way to move text from one file to another.
(@xref{Accumulating Text}, for some other ways.)

  If you have two separate Emacs processes, you cannot use the kill ring
to move text. If you are using Lucid GNU Emacs under X, you can,
however, use the X selection mechanism to move text from one to another.

If you are using Lucid GNU Emacs under X and have one Emacs process with
multiple screens, they do share the same kill ring.  You can kill or
copy text in one Emacs screen, then yank it in the other screen
belonging to the same process.

  The command @kbd{C-y} (@code{yank}) reinserts the text of the most recent
kill.  It leaves the cursor at the end of the text and sets the mark at
the beginning of the text.  @xref{Mark}.

  @kbd{C-u C-y} yanks the text, leaves the cursor in front of the text,
and sets the mark after it, if the argument is with just a @kbd{C-u}.
Any other argument, including @kbd{C-u} and digits, has different
results, described below, under ``Yanking Earlier Kills''.

@kindex M-w
@findex copy-region-as-kill
 To copy a block of text, you can also use @kbd{M-w}
(@code{copy-region-as-kill}), which copies the region into the kill ring
without removing it from the buffer. @kbd{M-w} is similar to @kbd{C-w}
followed by @kbd{C-y} but does not mark the buffer as ``modified'' and
does not actually cut anything.

@node Appending Kills, Earlier Kills, Kill Ring, Yanking
@subsection Appending Kills

@cindex television
  Normally, each kill command pushes a new block onto the kill ring.
However, two or more kill commands in a row combine their text into a
single entry, so that a single @kbd{C-y} yanks it all back. This means
you don't have to kill all the text you want to yank in one command; you
can kill line after line, or word after word, until you have killed what
you want, then get it all back at once using @kbd{C-y}. (Thus we join
television in leading people to kill thoughtlessly.)

  Commands that kill forward from point add onto the end of the previous
killed text.  Commands that kill backward from point add onto the
beginning.  This way, any sequence of mixed forward and backward kill
commands puts all the killed text into one entry without rearrangement.
Numeric arguments do not break the sequence of appending kills.  For
example, suppose the buffer contains

@example
This is the first
line of sample text
and here is the third.
@end example

@noindent
with point at the beginning of the second line.  If you type @kbd{C-k C-u 2
M-@key{DEL} C-k}, the first @kbd{C-k} kills the text @samp{line of sample
text}, @kbd{C-u 2 M-@key{DEL}} kills @samp{the first} with the newline that
followed it, and the second @kbd{C-k} kills the newline after the second
line.  The result is that the buffer contains @samp{This is and here is the
third.} and a single kill entry contains @samp{the first@key{RET}line of
sample text@key{RET}}---all the killed text, in its original order.

@kindex C-M-w
@findex append-next-kill
  If a kill command is separated from the last kill command by other
commands (not just numeric arguments), it starts a new entry on the kill
ring.  To force a kill command to append, first type the command @kbd{C-M-w}
(@code{append-next-kill}). @kbd{C-M-w} tells the following command,
if it is a kill command, to append the text it kills to the last killed
text, instead of starting a new entry.  With @kbd{C-M-w}, you can kill
several separated pieces of text and accumulate them to be yanked back
in one place.@refill

@node Earlier Kills,, Appending Kills, Yanking
@subsection Yanking Earlier Kills

@kindex M-y
@findex yank-pop
  To recover killed text that is no longer the most recent kill, you need
the @kbd{Meta-y} (@code{yank-pop}) command.  You can use @kbd{M-y} only
after a @kbd{C-y} or another @kbd{M-y}.  It takes the text previously
yanked and replaces it with the text from an earlier kill.  To recover
the text of the next-to-the-last kill, first use @kbd{C-y} to recover
the last kill, then @kbd{M-y} to replace it with the previous
kill.@refill

  You can think in terms of a ``last yank'' pointer which points at an item
in the kill ring.  Each time you kill, the ``last yank'' pointer moves to
the new item at the front of the ring.  @kbd{C-y} yanks the item
which the ``last yank'' pointer points to.  @kbd{M-y} moves the ``last
yank'' pointer to a different item, and the text in the buffer changes to
match.  Enough @kbd{M-y} commands can move the pointer to any item in the
ring, so you can get any item into the buffer.  Eventually the pointer
reaches the end of the ring; the next @kbd{M-y} moves it to the first item
again.

  Yanking moves the ``last yank'' pointer around the ring, but does not
change the order of the entries in the ring, which always runs from the
most recent kill at the front to the oldest one still remembered.

  Use @kbd{M-y} with a numeric argument, to tell to advance the ``last
yank'' pointer by the specified number of items.  A negative argument
moves the pointer toward the front of the ring; from the front of the
ring, it moves to the last entry and starts moving forward from there.

  Once the text you are looking for is brought into the buffer, you can
stop doing @kbd{M-y} commands and the text will stay there. Since the
text is just a copy of the kill ring item, editing it in the buffer does
not change what's in the ring.  As long you don`t kill additional text,
the ``last yank'' pointer remains at the same place in the kill ring:
repeating @kbd{C-y} will yank another copy of the same old kill.

  If you know how many @kbd{M-y} commands it would take to find the
text you want, you can yank that text in one step using @kbd{C-y} with
a numeric argument.  @kbd{C-y} with an argument greater than one
restores the text the specified number of entries back in the kill
ring.  Thus, @kbd{C-u 2 C-y} gets the next to the last block of killed
text.  It is equivalent to @kbd{C-y M-y}.  @kbd{C-y} with a numeric
argument starts counting from the ``last yank'' pointer, and sets the
``last yank'' pointer to the entry that it yanks.

@vindex kill-ring-max
  The variable @code{kill-ring-max} controls the length of the kill
ring; no more than that many blocks of killed text are saved.

@node Using X Selections, Accumulating Text, Yanking, Top
@section Using X Selections
@comment  node-name,  next,  previous,  up

In the X window system, mouse selections provide a simple mechanism for
text transfer between different applications.  In a typical X
application, you can select text by pressing the left mouse button and
dragging the cursor over the text you want to copy. The text becomes the
primary X selection and is highlighted. The highlighted region is also
the Emacs selected region.

@itemize @bullet
@item
Since the region is the primary X selection, you can go to a
different X application and click the middle mouse button: the text that
you selected in the previous application is pasted into the current
application.
@item
Since the region is the Emacs selected region, you can use all region
commands (@kbd{C-w, M-w} etc.) as well as the options of the @b{Edit}
menu to manipulate the selected text.
@end itemize

@menu
* X Clipboard Selection::     	Storing the primary selection.
* X Selection Commands::	Other operations on the selection.
* X Cut Buffers::       	X cut buffers are available for compatibility.
* Active Regions::      	Using zmacs-style highlighting of the
                        	 selected region.
@end menu

@node X Clipboard Selection, X Selection Commands, Using X Selections, Using X Selections
@comment  node-name,  next,  previous,  up
@subsection The Clipboard Selection
@cindex clipboard selections

There are other kinds of X selection besides the primary selection.  Each
time a region of text is added to the kill ring (for example, with C-k,
C-w, or M-w or with a @b{Cut} or @b{Copy} menu item), that text becomes
the clipboard selection.

Usually, the clipboard selection is not visible. However, if you run
the @file{xclipboard} application, the most recently killed text---the
value of the clipboard selection---is displayed in a window.
Any time Emacs adds text to the kill ring, the @file{xclipboard}
application makes a copy of it and displays it in its window. The
value of the clipboard can survive the lifetime of the running Emacs
process. The @code{xclipboard} man page provides more details.

Warning: If you use the @file{xclipboard} application, remember that it
maintains a list of all things that have been pasted to the clipboard
(that is, killed in Emacs).  If you don't manually delete elements from
this list by clicking on the @b{Delete} button in the @code{xclipboard}
window, the clipboard will eventually consume a lot of memory.

@node X Selection Commands, X Cut Buffers, X Clipboard Selection, Using X Selections
@subsection Miscellaneous X Selection Commands
@comment  node-name,  next,  previous,  up
@cindex cut buffers
@cindex primary selections

@findex x-copy-primary-selection
@findex x-delete-primary-selection
@findex x-insert-selection
@findex x-kill-primary-selection
@findex x-mouse-kill
@findex x-new-screen 
@findex x-own-secondary-selection
@findex x-own-selection
@findex x-set-point-and-insert-selection
@table @kbd
@item M-x x-copy-primary-selection
Copy the primary selection to both the kill ring and the Clipboard.
@item M-x x-insert-selection
Insert the current selection into the buffer at point.
@item M-x x-delete-primary-selection
Deletes the text in the primary selection without copying it to the kill
ring or the Clipboard.
@item M-x x-kill-primary-selection
Deletes the text in the primary selection and copies it to 
both the kill ring and the Clipboard.
@item M-x x-mouse-kill
Kill the text between point and the mouse and copy it to 
the clipboard and to the cut buffer.
@item M-x x-new-screen
Create a new Emacs screen (that is, a new X window).
@item M-x x-own-secondary-selection
Make a secondary X selection of the given argument. 
@item M-x x-own-selection
Make a primary X selection of the given argument.  
@item M-x x-set-point-and-insert-selection
Set point where clicked and insert the primary selection or the
cut buffer.
@end table

@node X Cut Buffers, Active Regions, X Selection Commands, Using X Selections
@subsection X Cut Buffers
@comment  node-name,  next,  previous,  up

X cut buffers are a different, older way of transferring text between
applications.  Lucid GNU Emacs supports cut buffers for compatibility
with older programs, even though selections are now the preferred way of
transferring text.

X has a concept of applications "owning" selections.  When you select
text by clicking and dragging inside an application, the application
tells the X server that it owns the selection.  When another
application asks the X server for the value of the selection, the X
server requests the information from the owner. When you use
selections, the selection data is not actually transferred unless
someone wants it; the act of making a selection doesn't transfer data.
Cut buffers are different: when you "own" a cut buffer, the data is
actually transferred to the X server immediately, and survives the
lifetime of the application.

Any time a region of text becomes the primary selection in Emacs,
Emacs also copies that text to the cut buffer.  This makes it possible
to copy text from a Lucid GNU Emacs buffer and paste it into an older,
non-selection-based application (such as Emacs 18.)

Note: Older versions of Emacs could not access the X selections, only
the X cut buffers.

@node Active Regions, , X Cut Buffers, Using X Selections
@subsection Active Regions
@comment  node-name,  next,  previous,  up
@cindex active regions

  By default, both the text you select in an Emacs buffer using the
click-and-drag mechanism and text you select by setting point and the
mark is highlighted. You can use Emacs region commands as well as the
@b{Cut} and @b{Copy} commands on the highlighted region you selected
with the mouse.

If you prefer, you can make a distinction between text selected with the
mouse and text selected with point and the mark by setting the variable
@code{zmacs-regions} to @code{nil}.  In that case:

@itemize @bullet
@item
The text selected with the mouse becomes both the X selection and the
Emacs selected region. You can use menubar commands as well as Emacs
region commands on it. 
@item
The text selected with point and the mark does not highlight. You can
only use Emacs region commands but not the menu bar items on it. 
@end itemize

  Active regions originally come from Zmacs, the Lisp Machine editor.
The idea behind them is that commands can only operate on a region when
the region is in an "active" state.  Put simply, you can only operate on
a region that is highlighted.

@vindex zmacs-regions
The variable @code{zmacs-regions} checks whether LISPM-style active
regions should be used.  This means that commands that operate on the
region (the area between point and the mark) only work while
the region is in the active state, which is indicated by highlighting.
Most commands causes the region to not be in the active state;
for example, @kbd{C-w} only works immediately after activating the
region.

More specifically:
@itemize @bullet
@item
Commands that operate on the region only work if the region is active.
@item
Only a very small set of commands causes the region to become active---
those commands whose semantics are to mark an area, such as @code{mark-defun}.
@item
The region is deactivated after each command that is executed, except that
motion commands do not change whether the region is active or not.
@end itemize 

@code{set-mark-command} (@kbd{C-SPC}) pushes a mark and activates the
region.  Moving the cursor with normal motion commands (@kbd{C-n},
@kbd{C-p}, etc) will cause the region between point and the
recently-pushed mark to be highlighted.  It will remain highlighted
until some non-motion comand is executed.

@code{exchange-point-and-mark} (@kbd{C-x C-x}) activates the region.
So if you mark a region and execute a command that operates on it, you
can reactivate the same region with @kbd{C-x C-x} (or perhaps @kbd{C-x
C-x C-x C-x}) to operate on it again.

Generally, commands that push marks as a means of navigation, such as
@code{beginning-of-buffer} (@kbd{M-<}) and @code{end-of-buffer}
(@kbd{M->}), do not activate the region.  However, commands that push
marks as a means of marking an area of text, such as @code{mark-defun}
(@kbd{M-C-h}), @code{mark-word} (@kbd{M-@@}), and @code{mark-whole-buffer}
(@kbd{C-x h}), do activate the region.

@page
When @code{zmacs-regions} is @code{t}, there is no distinction between
the primary X selection and the active region selected by point and the
mark.  To see this, set the mark @key{(C-SPC)} and move the cursor
with any cursor-motion command: the region between point and mark is
highlighted, and you can watch it grow and shrink as you move the
cursor.

Any other commands besides cursor-motion commands (such as inserting or
deleting text) will cause the region to no longer be active; it will no
longer be highlighted, and will no longer be the primary selection.
Errors also remove highlighting from a region.

Commands which require a region (such as @kbd{C-w}) signal an error if
the region is not active.  Certain commands cause the region to be in
its active state.  The most common ones are @code{push-mark}
(@key{C-SPC}) and @code{exchange-point-and-mark} (@kbd{C-x C-x}).

@vindex zmacs-region-stays
When @code{zmacs-regions} is @code{t}, programs can be non-intrusive
on the state of the region by setting the variable @code{zmacs-region-stays}
to a non-@code{nil} value.  If you are writing a new emacs command that
is conceptually a ``motion'' command, and should not interfere with the
current highlightedness of the region, then you may set this variable.
It is reset to @code{nil} after each user command is executed.

@findex zmacs-activate-region
When @code{zmacs-regions} is @code{t}, programs can make the region between
point and mark go into the active (highlighted) state by using the
function @code{zmacs-activate-region}. Only a small number of commands
should ever do this. 

@findex zmacs-deactivate-region
When @code{zmacs-regions} is @code{t}, programs can deactivate the region
between point and the mark by using @code{zmacs-deactivate-region}.
Note: you should not have to call this function; the command loop calls
it when appropriate. 

@node Accumulating Text, Rectangles, Using X Selections, Top
@section Accumulating Text
@kindex C-x a
@findex append-to-buffer
@findex prepend-to-buffer
@findex copy-to-buffer
@findex append-to-file
@cindex copying text
@cindex accumulating text

  Usually you copy or move text by killing it and yanking it, but there are
other ways that are useful for copying one block of text in many places, or
for copying many scattered blocks of text into one place.

  If you like, you can accumulate blocks of text from scattered
locations either into a buffer or into a file.  The relevant commands
are described here.  You can also use Emacs registers for storing and
accumulating text.  @xref{Registers}.

@table @kbd
@item C-x a
Append region to contents of specified buffer (@code{append-to-buffer}).
@item M-x prepend-to-buffer
Prepend region to contents of specified buffer.
@item M-x copy-to-buffer
Copy region into specified buffer, deleting that buffer's old contents.
@item M-x insert-buffer
Insert contents of specified buffer into current buffer at point.
@page
@item M-x append-to-file
Append region to contents of specified file, at the end.
@end table

  To accumulate text into a buffer, use the command @kbd{C-x a
@var{buffername}} (@code{append-to-buffer}), which inserts a copy of the
region into the buffer @var{buffername}, at the location of point in
that buffer.  If there is no buffer with the given name, one is created.

  If you append text to a buffer that has been used for editing, the
copied text goes to the place where point is.  Point in that buffer is
left at the end of the copied text, so successive uses of @kbd{C-x a}
accumulate the text in the specified buffer in the same order as they
were copied.  Strictly speaking, @kbd{C-x a} does not always append to
the text already in the buffer; but if @kbd{C-x a} is the only command
used to alter a buffer, it does always append to the existing text
because point is always at the end.

  @kbd{M-x prepend-to-buffer} is similar to @kbd{C-x a} but point in
the other buffer is left before the copied text, so successive prependings
add text in reverse order.  @kbd{M-x copy-to-buffer} is similar except that
any existing text in the other buffer is deleted, so the buffer is left
containing just the text newly copied into it.

  You can retrieve the accumulated text from that buffer with @kbd{M-x
insert-buffer}, which takes @var{buffername} as an argument.  It inserts
a copy of the text in buffer @var{buffername} into the selected buffer.
You could alternatively select the other buffer for editing, perhaps moving
text from it by killing or with @kbd{C-x a}.  @xref{Buffers}, for
background information on buffers.

  Instead of accumulating text within Emacs, in a buffer, you can append
text directly into a file with @kbd{M-x append-to-file}, which takes
@var{file-name} as an argument.  It adds the text of the region to the
end of the specified file.  The file is changed immediately on disk.
This command is normally used with files that are @i{not} being visited
in Emacs.  Using it on a file that Emacs is visiting can produce
confusing results, because the file's text inside Emacs does not change
while the file itself changes.

@node Rectangles, Registers, Accumulating Text, Top
@section Rectangles
@cindex rectangles

  The rectangle commands affect rectangular areas of text: all
characters between a certain pair of columns, in a certain range of lines.
Commands are provided to kill rectangles, yank killed rectangles, clear
them out, or delete them.  Rectangle commands are useful with text in
multicolumnar formats, like code with comments at the right,
or for changing text into or out of such formats.

  To specify the rectangle a command should work on, put the mark at one
corner and point at the opposite corner.  The specified rectangle is
called the @dfn{region-rectangle} because it is controlled about the
same way the region is controlled.  Remember that a given
combination of point and mark values can be interpreted either as
specifying a region or as specifying a rectangle; it is up to the
command that uses them to choose the interpretation.

@table @kbd
@item M-x delete-rectangle
Delete the text of the region-rectangle, moving any following text on
each line leftward to the left edge of the region-rectangle.
@item M-x kill-rectangle
Similar, but also save the contents of the region-rectangle as the
``last killed rectangle''.
@item M-x yank-rectangle
Yank the last killed rectangle with its upper left corner at point.
@item M-x open-rectangle
Insert blank space to fill the space of the region-rectangle.
The previous contents of the region-rectangle are pushed rightward.
@item M-x clear-rectangle
Clear the region-rectangle by replacing its contents with spaces.
@end table

  The rectangle operations fall into two classes: commands deleting and
moving rectangles, and commands for blank rectangles.

@findex delete-rectangle
@findex kill-rectangle
  There are two ways to get rid of the text in a rectangle: you can discard
the text (delete it) or save it as the ``last killed'' rectangle.  The
commands for these two ways are @kbd{M-x delete-rectangle} and @kbd{M-x
kill-rectangle}.  In either case, the portion of each line that falls inside
the rectangle's boundaries is deleted, causing following text (if any) on
the line to move left.

  Note that ``killing'' a rectangle is not killing in the usual sense; the
rectangle is not stored in the kill ring, but in a special place that
only records the most recently killed rectangle (that is, does not
append to a killed rectangle).  Different yank commands
have to be used and only one rectangle is stored because yanking
a rectangle is quite different from yanking linear text and yank-popping
commands are difficult to make sense of.

  Inserting a rectangle is the opposite of deleting one.  You specify
where to put the upper left corner by putting point there.  The
rectangle's first line is inserted at point, the rectangle's second line
is inserted at a point one line vertically down, and so on.  The number
of lines affected is determined by the height of the saved rectangle.

@page
@findex yank-rectangle
  To insert the last killed rectangle, type @kbd{M-x yank-rectangle}.
This can be used to convert single-column lists into double-column
lists; kill the second half of the list as a rectangle and then
yank it beside the first line of the list.

@findex open-rectangle
@findex clear-rectangle
  There are two commands for working with blank rectangles: @kbd{M-x
clear-rectangle} to blank out existing text, and @kbd{M-x open-rectangle}
to insert a blank rectangle.  Clearing a rectangle is equivalent to
deleting it and then inserting a blank rectangle of the same size.

  Rectangles can also be copied into and out of registers.
@xref{RegRect,,Rectangle Registers}.

@node Registers, Display, Rectangles, Top
@chapter Registers
@cindex registers

  Emacs @dfn{registers} are places in which you can save text or
positions for later use.  Text saved in a register can be copied into
the buffer once or many times; a position saved in a register is used by
moving point to that position.  Rectangles can also be copied into and
out of registers (@pxref{Rectangles}).

  Each register has a name, which is a single character.  A register can
store either a piece of text or a position or a rectangle, but only one
thing at any given time.  Whatever you store in a register remains
there until you store something else in that register.

@menu
* RegPos::    Saving positions in registers.
* RegText::   Saving text in registers.
* RegRect::   Saving rectangles in registers.
@end menu

@table @kbd
@item M-x view-register @key{RET} @var{r}
Display a description of what register @var{r} contains.
@end table

@findex view-register
  @kbd{M-x view-register} reads a register name as an argument and then
displays the contents of the specified register.

@node RegPos, RegText, Registers, Registers
@section Saving Positions in Registers

  Saving a position records a spot in a buffer so you can move
back there later.  Moving to a saved position re-selects the buffer
and moves point to the spot.

@table @kbd
@item C-x / @var{r}
Save the location of point in register @var{r} (@code{point-to-register}).
@item C-x j @var{r}
Jump to the location saved in register @var{r} (@code{register-to-point}).
@end table

@kindex C-x /
@findex point-to-register
  To save the current location of point in a register, choose a name
@var{r} and type @kbd{C-x / @var{r}}.  The register @var{r} retains
the location thus saved until you store something else in that
register.@refill

@kindex C-x j
@findex register-to-point
  The command @kbd{C-x j @var{r}} moves point to the location recorded
in register @var{r}.  The register is not affected; it continues to
record the same location.  You can jump to the same position using the
same register as often as you want.

@node RegText, RegRect, RegPos, Registers
@section Saving Text in Registers

  When you want to insert a copy of the same piece of text many times, it
can be impractical to use the kill ring, since each subsequent kill moves
the piece of text further down on the ring.  It becomes hard to keep
track of the argument needed to retrieve the same text with @kbd{C-y}.  An
alternative is to store the text in a register with @kbd{C-x x}
(@code{copy-to-register}) and then retrieve it with @kbd{C-x g}
(@code{insert-register}).

@table @kbd
@item C-x x @var{r}
Copy region into register @var{r} (@code{copy-to-register}).
@item C-x g @var{r}
Insert text contents of register @var{r} (@code{insert-register}).
@end table

@kindex C-x x
@kindex C-x g
@findex copy-to-register
@findex insert-register
  @kbd{C-x x @var{r}} stores a copy of the text of the region into the
register named @var{r}.  Given a numeric argument, @kbd{C-x x} deletes the
text from the buffer as well.

  @kbd{C-x g @var{r}} inserts the text from register @var{r} in the buffer.
By default it leaves point before the text and places the mark after it. 
With a numeric argument, it puts point after the text and the mark
before it.

@node RegRect,, RegText, Registers
@section Saving Rectangles in Registers
@cindex rectangle

  A register can contain a rectangle instead of lines of text.  The rectangle
is represented as a list of strings.  @xref{Rectangles}, for basic
information on rectangles and how to specify rectangles in a buffer.

@table @kbd
@item C-x r @var{r}
Copy the region-rectangle into register @var{r}(@code{copy-region-to-rectangle}).
With a numeric argument, delete it as well.
@item C-x g @var{r}
Insert the rectangle stored in register @var{r} (if it contains a
rectangle) (@code{insert-register}).
@end table

  The @kbd{C-x g} command inserts linear text if the register contains
that, or inserts a rectangle if the register contains one.

@node Display, Search, Registers, Top
@chapter Controlling the Display

  Since only part of a large buffer fits in the window, Emacs tries to show
the part that is likely to be interesting.  The display control commands
allow you to specify which part of the text you want to see.

@table @kbd
@item C-l
Clear screen and redisplay, scrolling the selected window to center
point vertically within it (@code{recenter}).
@item C-v
Scroll forward (a windowful or a specified number of lines) (@code{scroll-up}).
@item M-v
Scroll backward (@code{scroll-down}).
@item @var{arg} C-l
Scroll so point is on line @var{arg} (@code{recenter}).
@item C-x <
Scroll text in current window to the left (@code{scroll-left}).
@item C-x >
Scroll to the right (@code{scroll-right}).
@item C-x $
Make deeply indented lines invisible (@code{set-selective-display}).
@end table

@menu
* Scrolling::	           Moving text up and down in a window.
* Horizontal Scrolling::   Moving text left and right in a window.
* Selective Display::      Hiding lines with lots of indentation.
* Display Vars::           Information on variables for customizing display.
@end menu

@node Scrolling, Horizontal Scrolling, Display, Display
@section Scrolling

  If a buffer contains text that is too large to fit entirely within the
window that is displaying the buffer, Emacs shows a contiguous section of
the text.  The section shown always contains point.

@cindex scrolling
  @dfn{Scrolling} means moving text up or down in the window so that
different parts of the text are visible.  Scrolling forward means that text
moves up, and new text appears at the bottom.  Scrolling backward moves
text down and new text appears at the top.

  Scrolling happens automatically if you move point past the bottom or top
of the window.  You can also explicitly request scrolling with the commands
in this section.

@ifinfo
@table @kbd
@item C-l
Clear screen and redisplay, scrolling the selected window to center
point vertically within it (@code{recenter}).
@item C-v
Scroll forward (a windowful or a specified number of lines) (@code{scroll-up}).
@item M-v
Scroll backward (@code{scroll-down}).
@item @var{arg} C-l
Scroll so point is on line @var{arg} (@code{recenter}).
@end table
@end ifinfo

@kindex C-l
@findex recenter
  The most basic scrolling command is @kbd{C-l} (@code{recenter}) with no
argument.  It clears the entire screen and redisplays all windows.  In
addition, it scrolls the selected window so that point is halfway down
from the top of the window.

@kindex C-v
@kindex M-v
@findex scroll-up
@findex scroll-down
  The scrolling commands @kbd{C-v} and @kbd{M-v} let you move all the text
in the window up or down a few lines.  @kbd{C-v} (@code{scroll-up}) with an
argument shows you that many more lines at the bottom of the window, moving
the text and point up together as @kbd{C-l} might.  @kbd{C-v} with a
negative argument shows you more lines at the top of the window.
@kbd{Meta-v} (@code{scroll-down}) is like @kbd{C-v}, but moves in the
opposite direction.@refill

@vindex next-screen-context-lines
  To read the buffer a windowful at a time, use @kbd{C-v} with no
argument.  @kbd{C-v} takes the last two lines at the bottom of the
window and puts them at the top, followed by nearly a whole windowful of
lines not previously visible.  Point moves to the new top of the window
if it was in the text scrolled off the top.  @kbd{M-v} with no argument
moves backward with similar overlap.  The number of lines of overlap
across a @kbd{C-v} or @kbd{M-v} is controlled by the variable
@code{next-screen-context-lines}; by default, it is two.

  Another way to scroll is using @kbd{C-l} with a numeric argument.
@kbd{C-l} does not clear the screen when given an argument; it only
scrolls the selected window.  With a positive argument @var{n},@kbd{C-l}
repositions text to put point @var{n} lines down from the top.  An
argument of zero puts point on the very top line.  Point does not move
with respect to the text; rather, the text and point move rigidly on the
screen.  @kbd{C-l} with a negative argument puts point that many lines
from the bottom of the window.  For example, @kbd{C-u - 1 C-l} puts
point on the bottom line, and @kbd{C-u - 5 C-l} puts it five lines from
the bottom.  Just @kbd{C-u} as argument, as in @kbd{C-u C-l}, scrolls
point to the center of the screen.

@vindex scroll-step
  Scrolling happens automatically if point has moved out of the visible
portion of the text when it is time to display.  Usually scrolling is
done  to put point vertically centered within the window.  However, if
the variable @code{scroll-step} has a non-zero value, an attempt is made to
scroll the buffer by that many lines; if that is enough to bring point back
into visibility, that is what happens.

@node Horizontal Scrolling,, Scrolling, Display
@section Horizontal Scrolling

@ifinfo
@table @kbd
@item C-x <
Scroll text in current window to the left (@code{scroll-left}).
@item C-x >
Scroll to the right (@code{scroll-right}).
@end table
@end ifinfo

@kindex C-x <
@kindex C-x >
@findex scroll-left
@findex scroll-right
@cindex horizontal scrolling
  The text in a window can also be scrolled horizontally.  This means that
each line of text is shifted sideways in the window, and one or more
characters at the beginning of each line are not displayed at all.  When a
window has been scrolled horizontally in this way, text lines are truncated
rather than continued (@pxref{Continuation Lines}), with a @samp{$} appearing
in the first column when there is text truncated to the left, and in the
last column when there is text truncated to the right.

  The command @kbd{C-x <} (@code{scroll-left}) scrolls the selected
window to the left by @var{n} columns with argument @var{n}.  With no
argument, it scrolls by almost the full width of the window (two columns
less, to be precise).  @kbd{C-x >} (@code{scroll-right}) scrolls
similarly to the right.  The window cannot be scrolled any farther to
the right once it is displaying normally (with each line starting at the
window's left margin); attempting to do so has no effect.

@node Selective Display, Display Vars, Display, Display
@section Selective Display
@findex set-selective-display
@kindex C-x $

  Emacs can hide lines indented more than a certain number
of columns (you specify how many columns).  This allows you  to get an
overview of a part of a program.

  To hide lines, type @kbd{C-x $} (@code{set-selective-display}) with a
numeric argument @var{n}.  (@xref{Arguments}, for information on giving
the argument.)  Lines with at least @var{n} columns of indentation
disappear from the screen.  The only indication of their presence are
three dots (@samp{@dots{}}), which appear at the end of each visible
line that is followed by one or more invisible ones.@refill

  The invisible lines are still present in the buffer, and most editing
commands see them as usual, so it is very easy to put point in the middle
of invisible text.  When this happens, the cursor appears at the end of the
previous line, after the three dots.  If point is at the end of the visible
line, before the newline that ends it, the cursor appears before the three
dots.

  The commands @kbd{C-n} and @kbd{C-p} move across the invisible lines
as if they were not there.

  To make everything visible again, type @kbd{C-x $} with no argument.

@node Display Vars,, Selective Display, Display
@section Variables Controlling Display

  This section contains information for customization only.  Beginning
users should skip it.

@vindex mode-line-inverse-video
  The variable @code{mode-line-inverse-video} controls whether the mode
line is displayed in inverse video (assuming the terminal supports it);
@code{nil} means don't do so.  @xref{Mode Line}.

@vindex inverse-video
  If the variable @code{inverse-video} is non-@code{nil}, Emacs attempts
to invert all the lines of the display from what they normally are.

@vindex no-redraw-on-reenter
  When you reenter Emacs after suspending, Emacs normally clears the
screen and redraws the entire display.  On some terminals with more than
one page of memory, it is possible to arrange the termcap entry so that
the @samp{ti} and @samp{te} strings (output to the terminal when Emacs
is entered and exited, respectively) switch between pages of memory so
as to use one page for Emacs and another page for other output.  In that
case, you might want to set the variable @code{no-redraw-on-reenter}
non-@code{nil} so that Emacs will assume, when resumed, that the screen
page it is using still contains what Emacs last wrote there.

@page
@vindex echo-keystrokes
  The variable @code{echo-keystrokes} controls the echoing of multi-character
keys; its value is the number of seconds of pause required to cause echoing
to start, or zero meaning don't echo at all.  @xref{Echo Area}.

@vindex ctl-arrow
  If the variable @code{ctl-arrow} is @code{nil}, control characters in the
buffer are displayed with octal escape sequences, all except newline and
tab.  If its value is @code{t}, then control characters will be printed 
with an uparrow, for example @kbd{^A}.  

If its value is not @code{t} and not @code{nil}, then characters whose
code is greater than 160 (that is, the space character (32) with its
high bit set) will be assumed to be printable, and will be displayed
without alteration.  This is the default when running under X Windows,
since Lucid Emacs assumes an ISO/8859-1 character set (also known as
``Latin1'').  The @code{ctl-arrow} variable may also be set to an
integer, in which case all characters whose codes are greater than or
equal to that value will be assumed to be printable.

Altering the value of @code{ctl-arrow} makes it local to the current
buffer; until that time, the default value is in effect.  @xref{Locals}.

@vindex tab-width
  Normally, a tab character in the buffer is displayed as whitespace which
extends to the next display tab stop position, and display tab stops come
at intervals equal to eight spaces.  The number of spaces per tab is
controlled by the variable @code{tab-width}, which is made local by
changing it, just like @code{ctl-arrow}.  Note that how the tab character
in the buffer is displayed has nothing to do with the definition of
@key{TAB} as a command.

@vindex selective-display-ellipses
  If you set the variable @code{selective-display-ellipses} to @code{nil},
the three dots at the end of a line that precedes invisible
lines do not appear.  There is no visible indication of the invisible lines.
This variable becomes local automatically when set.

@node Search, Fixit, Display, Top
@chapter Searching and Replacement
@cindex searching

  Like other editors, Emacs has commands for searching for occurrences of
a string.  The principal search command is unusual in that it is
@dfn{incremental}: it begins to search before you have finished typing the
search string.  There are also non-incremental search commands more like
those of other editors.

  Besides the usual @code{replace-string} command that finds all
occurrences of one string and replaces them with another, Emacs has a fancy
replacement command called @code{query-replace} which asks interactively
which occurrences to replace.

@menu
* Incremental Search::     Search happens as you type the string.
* Non-Incremental Search::  Specify entire string and then search.
* Word Search::            Search for sequence of words.
* Regexp Search::          Search for match for a regexp.
* Regexps::                Syntax of regular expressions.
* Search Case::            To ignore case while searching, or not.
* Replace::                Search, and replace some or all matches.
* Other Repeating Search:: Operating on all matches for some regexp.
@end menu

@node Incremental Search, Non-Incremental Search, Search, Search
@section Incremental Search

  An incremental search begins searching as soon as you type the first
character of the search string.  As you type in the search string, Emacs
shows you where the string (as you have typed it so far) is found.
When you have typed enough characters to identify the place you want, you
can stop.  Depending on what you do next, you may or may not need to
terminate the search explicitly with a @key{RET}.

@c WideCommands
@table @kbd
@item C-s
Incremental search forward (@code{isearch-forward}).
@item C-r
Incremental search backward (@code{isearch-backward}).
@end table

@kindex C-s
@kindex C-r
@findex isearch-forward
@findex isearch-backward
  @kbd{C-s} starts an incremental search.  @kbd{C-s} reads characters from
the keyboard and positions the cursor at the first occurrence of the
characters that you have typed.  If you type @kbd{C-s} and then @kbd{F},
the cursor moves right after the first @samp{F}.  Type an @kbd{O}, and see
the cursor move to after the first @samp{FO}.  After another @kbd{O}, the
cursor is after the first @samp{FOO} after the place where you started the
search.  Meanwhile, the search string @samp{FOO} has been echoed in the
echo area.@refill

  The echo area display ends with three dots when actual searching is going
on.  When search is waiting for more input, the three dots are removed.
(On slow terminals, the three dots are not displayed.)

  If you make a mistake in typing the search string, you can erase
characters with @key{DEL}.  Each @key{DEL} cancels the last character of the
search string.  This does not happen until Emacs is ready to read another
input character; first it must either find, or fail to find, the character
you want to erase.  If you do not want to wait for this to happen, use
@kbd{C-g} as described below.@refill

  When you are satisfied with the place you have reached, you can type
@key{RET} (or @key{C-m}), which stops searching, leaving the cursor where 
the search brought it.  Any command not specially meaningful in searches also
stops the search and is then executed.  Thus, typing @kbd{C-a} exits the
search and then moves to the beginning of the line.  @key{RET} is necessary
only if the next command you want to type is a printing character,
@key{DEL}, @key{ESC}, or another control character that is special
within searches (@kbd{C-q}, @kbd{C-w}, @kbd{C-r}, @kbd{C-s} or @kbd{C-y}).

  Sometimes you search for @samp{FOO} and find it, but were actually
looking for a different occurance of it.  To move to the next occurrence
of the search string, type another @kbd{C-s}.  Do this as often as
necessary.  If you overshoot, you can cancel some @kbd{C-s}
characters with @key{DEL}.

  After you exit a search, you can search for the same string again by
typing just @kbd{C-s C-s}: the first @kbd{C-s} is the key that invokes
incremental search, and the second @kbd{C-s} means ``search again''.

  If the specified string is not found at all, the echo area displays
the text @samp{Failing I-Search}.  The cursor is after the place where
Emacs found as much of your string as it could.  Thus, if you search for
@samp{FOOT}, and there is no @samp{FOOT}, the cursor may be after the
@samp{FOO} in @samp{FOOL}.  At this point there are several things you
can do.  If you mistyped the search string, correct it.  If you like the
place you have found, you can type @key{RET} or some other Emacs command
to ``accept what the search offered''.  Or you can type @kbd{C-g}, which
removes from the search string the characters that could not be found
(the @samp{T} in @samp{FOOT}), leaving those that were found (the
@samp{FOO} in @samp{FOOT}).  A second @kbd{C-g} at that point cancels
the search entirely, returning point to where it was when the search
started.

  If a search is failing and you ask to repeat it by typing another
@kbd{C-s}, it starts again from the beginning of the buffer.  Repeating
a failing reverse search with @kbd{C-r} starts again from the end.  This
is called @dfn{wrapping around}.  @samp{Wrapped} appears in the search
prompt once this has happened.

@cindex quitting (in search)
  The @kbd{C-g} ``quit'' character does special things during searches;
just what it does depends on the status of the search.  If the search has
found what you specified and is waiting for input, @kbd{C-g} cancels the
entire search.  The cursor moves back to where you started the search.  If
@kbd{C-g} is typed when there are characters in the search string that have
not been found---because Emacs is still searching for them, or because it
has failed to find them---then the search string characters which have not
been found are discarded from the search string.  The
search is now successful and waiting for more input, so a second @kbd{C-g}
cancels the entire search.

  To search for a control character such as @kbd{C-s} or @key{DEL} or
@key{ESC}, you must quote it by typing @kbd{C-q} first.  This function
of @kbd{C-q} is analogous to its meaning as an Emacs command: it causes
the following character to be treated the way a graphic character would
normally be treated in the same context.

 To search backwards, you can use @kbd{C-r} instead of @kbd{C-s} to
start the search; @kbd{C-r} is the key that runs the command
(@code{isearch-backward}) to search backward.  You can also use
@kbd{C-r} to change from searching forward to searching backwards.  Do
this if a search fails because the place you started was too far down in the
file.  Repeated @kbd{C-r} keeps looking for more occurrences backwards.
@kbd{C-s} starts going forward again.  You can cancel @kbd{C-r} in a
search with @key{DEL}.

  The characters @kbd{C-y} and @kbd{C-w} can be used in incremental search
to grab text from the buffer into the search string.  This makes it
convenient to search for another occurrence of text at point.  @kbd{C-w}
copies the word after point as part of the search string, advancing
point over that word.  Another @kbd{C-s} to repeat the search will then
search for a string including that word.  @kbd{C-y} is similar to @kbd{C-w}
but copies the rest of the current line into the search string.

  The characters @kbd{M-p} and @kbd{M-n} can be used in an incremental
search to recall things which you have searched for in the past.  A
list of the last 16 things you have searched for is retained, and 
@kbd{M-p} and @kbd{M-n} let you cycle through that ring.

The character @kbd{M-@key{TAB}} does completion on the elements in 
the search history ring.  For example, if you know that you have
recently searched for the string @code{POTATOE}, you could type
@kbd{C-s P O M-@key{TAB}}.  If you had searched for other strings
beginning with @code{PO} then you would be shown a list of them, and
would need to type more to select one. 

  You can change any of the special characters in incremental search via
the normal keybinding mechanism: simply add a binding to the 
@code{isearch-mode-map}.  For example, to make the character
@kbd{C-b} mean ``search backwards'' while in isearch-mode, do this:

@example
(define-key isearch-mode-map "\C-b" 'isearch-repeat-backward)
@end example

These are the default bindings of isearch-mode:

@findex isearch-delete-char
@findex isearch-exit
@findex isearch-quote-char
@findex isearch-repeat-forward
@findex isearch-repeat-reverse
@findex isearch-yank-line
@findex isearch-yank-word
@findex isearch-abort
@findex isearch-ring-retreat
@findex isearch-ring-advance
@findex isearch-complete

@kindex DEL (isearch-mode)
@kindex RET (isearch-mode)
@kindex C-q (isearch-mode)
@kindex C-s (isearch-mode)
@kindex C-r (isearch-mode)
@kindex C-y (isearch-mode)
@kindex C-w (isearch-mode)
@kindex C-g (isearch-mode)
@kindex M-p (isearch-mode)
@kindex M-n (isearch-mode)
@kindex M-TAB (isearch-mode)

@table @kbd
@item DEL
(@code{isearch-delete-char})  Delete a character from the incremental
search string.
@item RET
(@code{isearch-exit})  Exit incremental search.
@item C-q
(@code{isearch-quote-char})  Quote special characters for incremental
search.
@item C-s
(@code{isearch-repeat-forward})  Repeat incremental search forward.
@item C-r
(@code{isearch-repeat-reverse})  Repeat incremental search backward.
@item C-y
(@code{isearch-yank-line})  Pull rest of line from buffer into search string.
@item C-w
(@code{isearch-yank-word})  Pull next word from buffer into search
string.
@item C-g
(@code{isearch-abort})  Cancels input back to what has been found
successfully, or aborts the isearch.
@item M-p
(@code{isearch-ring-retreat})  Recall the previous element in the
isearch history ring.
@item M-n
(@code{isearch-ring-advance})  Recall the next element in the
isearch history ring.
@item M-@key{TAB}
(@code{isearch-complete})  Do completion on the elements in the isearch
history ring.

@end table

Any other character which is normally inserted into a buffer when typed
is automatically added to the search string in isearch-mode.

@subsection Slow Terminal Incremental Search

  Incremental search on a slow terminal uses a modified style of display
that is designed to take less time.  Instead of redisplaying the buffer at
each place the search gets to, it creates a new single-line window and uses
that to display the line the search has found.  The single-line window
appears as soon as point gets outside of the text that is already
on the screen.

  When the search is terminated, the single-line window is removed.  Only
at this time the window in which the search was done is redisplayed to show
its new value of point.

  The three dots at the end of the search string, normally used to indicate
that searching is going on, are not displayed in slow style display.

@vindex search-slow-speed
  The slow terminal style of display is used when the terminal baud rate is
less than or equal to the value of the variable @code{search-slow-speed},
initially 1200.

@vindex search-slow-window-lines
  The number of lines to use in slow terminal search display is controlled
by the variable @code{search-slow-window-lines}.  Its normal value is 1.

@node Non-Incremental Search, Word Search, Incremental Search, Search
@section Non-Incremental Search
@cindex non-incremental search

  Emacs also has conventional non-incremental search commands, which require
you type the entire search string before searching begins.

@table @kbd
@item C-s @key{RET} @var{string} @key{RET}
Search for @var{string}.
@item C-r @key{RET} @var{string} @key{RET}
Search backward for @var{string}.
@end table

  To do a non-incremental search, first type @kbd{C-s @key{RET}}
(or @kbd{C-s C-m}.  This enters the minibuffer to read the search string.
Terminate the string with @key{RET} to start the search.  If the string
is not found the search command gets an error.

 By default, @kbd{C-s} invokes incremental search, but if you give it an
empty argument, which would otherwise be useless, it invokes non-incremental
search.  Therefore, @kbd{C-s @key{RET}} invokes non-incremental search. 
@kbd{C-r @key{RET}} also works this way.

@findex search-forward
@findex search-backward
  Forward and backward non-incremental searches are implemented by the
commands @code{search-forward} and @code{search-backward}.  You can bind
these commands to keys.  The reason that incremental
search is programmed to invoke them as well is that @kbd{C-s @key{RET}}
is the traditional sequence of characters used in Emacs to invoke
non-incremental search.

 Non-Incremental searches performed using @kbd{C-s @key{RET}} do
not call @code{search-forward} right away.  They first check
if the next character is @kbd{C-w}, which requests a word search.
@ifinfo
@xref{Word Search}.
@end ifinfo

@node Word Search, Regexp Search, Non-Incremental Search, Search
@section Word Search
@cindex word search

  Word search looks for a sequence of words without regard to how the
words are separated.  More precisely, you type a string of many words,
using single spaces to separate them, and the string is found even if
there are multiple spaces, newlines or other punctuation between the words.

  Word search is useful in editing documents formatted by text formatters.
If you edit while looking at the printed, formatted version, you can't tell
where the line breaks are in the source file.  Word search, allows you
to search  without having to know the line breaks.

@table @kbd
@item C-s @key{RET} C-w @var{words} @key{RET}
Search for @var{words}, ignoring differences in punctuation.
@item C-r @key{RET} C-w @var{words} @key{RET}
Search backward for @var{words}, ignoring differences in punctuation.
@end table

  Word search is a special case of non-incremental search.  It is invoked
with @kbd{C-s @key{RET} C-w} followed by the search string, which
must always be terminated with another @key{RET}.  Being non-incremental, this
search does not start until the argument is terminated.  It works by
constructing a regular expression and searching for that.  @xref{Regexp
Search}.

 You can do a backward word search with @kbd{C-r @key{RET} C-w}.

@findex word-search-forward
@findex word-search-backward
  Forward and backward word searches are implemented by the commands
@code{word-search-forward} and @code{word-search-backward}.  You can
bind these commands to keys.  The reason that incremental
search is programmed to invoke them as well is that @kbd{C-s @key{RET} C-w}
is the traditional Emacs sequence of keys for word search.

@node Regexp Search, Regexps, Word Search, Search
@section Regular Expression Search
@cindex regular expression
@cindex regexp

  A @dfn{regular expression} (@dfn{regexp}, for short) is a pattern that
denotes a set of strings, possibly an infinite set.  Searching for matches
for a regexp is a powerful operation that editors on Unix systems have
traditionally offered.  In GNU Emacs, you can search for the next match for
a regexp either incrementally or not.

@kindex C-M-s
@findex isearch-forward-regexp
@findex isearch-backward-regexp
  Incremental search for a regexp is done by typing @kbd{C-M-s}
(@code{isearch-forward-regexp}).  This command reads a search string
incrementally just like @kbd{C-s}, but it treats the search string as a
regexp rather than looking for an exact match against the text in the
buffer.  Each time you add text to the search string, you make the regexp
longer, and the new regexp is searched for.  A reverse regexp search command
@code{isearch-backward-regexp} also exists but no key runs it.

  All of the control characters that do special things within an ordinary
incremental search have the same functionality in incremental regexp search.
Typing @kbd{C-s} or @kbd{C-r} immediately after starting a search
retrieves the last incremental search regexp used:
incremental regexp and non-regexp searches have independent defaults.

@findex re-search-forward
@findex re-search-backward
  Non-Incremental search for a regexp is done by the functions
@code{re-search-forward} and @code{re-search-backward}.  You can invoke
them with @kbd{M-x} or bind them to keys.  You can also call
@code{re-search-forward} by way of incremental regexp search with
@kbd{C-M-s @key{RET}}.

@node Regexps, Search Case, Regexp Search, Search
@section Syntax of Regular Expressions

Regular expressions have a syntax in which a few characters are special
constructs and the rest are @dfn{ordinary}.  An ordinary character is a
simple regular expression which matches that character and nothing else.
The special characters are @samp{$}, @samp{^}, @samp{.}, @samp{*},
@samp{+}, @samp{?}, @samp{[}, @samp{]} and @samp{\}; no new special
characters will be defined.  Any other character appearing in a regular
expression is ordinary, unless a @samp{\} precedes it.@refill

For example, @samp{f} is not a special character, so it is ordinary, and
therefore @samp{f} is a regular expression that matches the string @samp{f}
and no other string.  (It does @i{not} match the string @samp{ff}.)  Likewise,
@samp{o} is a regular expression that matches only @samp{o}.@refill

Any two regular expressions @var{a} and @var{b} can be concatenated.  The
result is a regular expression which matches a string if @var{a} matches
some amount of the beginning of that string and @var{b} matches the rest of
the string.@refill

As a simple example, you can concatenate the regular expressions @samp{f}
and @samp{o} to get the regular expression @samp{fo}, which matches only
the string @samp{fo}.  To do something nontrivial, you
need to use one of the following special characters:

@table @kbd
@item .@: @r{(Period)}
is a special character that matches any single character except a newline.
Using concatenation, you can make regular expressions like @samp{a.b} which
matches any three-character string which begins with @samp{a} and ends with
@samp{b}.@refill

@item *
is not a construct by itself; it is a suffix, which means the
preceding regular expression is to be repeated as many times as
possible.  In @samp{fo*}, the @samp{*} applies to the @samp{o}, so
@samp{fo*} matches one @samp{f} followed by any number of @samp{o}s.
The case of zero @samp{o}s is allowed: @samp{fo*} does match
@samp{f}.@refill

@samp{*} always applies to the @i{smallest} possible preceding
expression.  Thus, @samp{fo*} has a repeating @samp{o}, not a
repeating @samp{fo}.@refill

The matcher processes a @samp{*} construct by matching, immediately,
as many repetitions as it can find.  Then it continues with the rest
of the pattern.  If that fails, backtracking occurs, discarding some
of the matches of the @samp{*}-modified construct in case that makes
it possible to match the rest of the pattern.  For example, matching
@samp{ca*ar} against the string @samp{caaar}, the @samp{a*} first
tries to match all three @samp{a}s; but the rest of the pattern is
@samp{ar} and there is only @samp{r} left to match, so this try fails.
The next alternative is for @samp{a*} to match only two @samp{a}s.
With this choice, the rest of the regexp matches successfully.@refill

@item +
Is a suffix character similar to @samp{*} except that it requires that
the preceding expression be matched at least once.  For example,
@samp{ca+r} will match the strings @samp{car} and @samp{caaaar}
but not the string @samp{cr}, whereas @samp{ca*r} would match all
three strings.@refill

@item ?
Is a suffix character similar to @samp{*} except that it can match the
preceding expression either once or not at all.  For example,
@samp{ca?r} will match @samp{car} or @samp{cr}; nothing else.

@item [ @dots{} ]
@samp{[} begins a @dfn{character set}, which is terminated by a
@samp{]}.  In the simplest case, the characters between the two form
the set.  Thus, @samp{[ad]} matches either one @samp{a} or one
@samp{d}, and @samp{[ad]*} matches any string composed of just
@samp{a}s and @samp{d}s (including the empty string), from which it
follows that @samp{c[ad]*r} matches @samp{cr}, @samp{car}, @samp{cdr},
@samp{caddaar}, etc.@refill

You can include character ranges in a character set by writing
two characters with a @samp{-} between them.  Thus, @samp{[a-z]}
matches any lower-case letter.  Ranges may be intermixed freely with
individual characters, as in @samp{[a-z$%.]}, which matches any lower
case letter or @samp{$}, @samp{%} or period.@refill

Note that inside a character set the usual special characters are not
special any more.  A completely different set of special characters
exists inside character sets: @samp{]}, @samp{-} and @samp{^}.@refill

To include a @samp{]} in a character set, you must make it the first
character.  For example, @samp{[]a]} matches @samp{]} or @samp{a}.  To
include a @samp{-}, write @samp{---}, which is a range containing only
@samp{-}.  To include @samp{^}, make it other than the first character
in the set.@refill

@item [^ @dots{} ]
@samp{[^} begins a @dfn{complement character set}, which matches any
character except the ones specified.  Thus, @samp{[^a-z0-9A-Z]}
matches all characters @i{except} letters and digits.@refill

@page
@samp{^} is not special in a character set unless it is the first
character.  The character following the @samp{^} is treated as if it
were first (@samp{-} and @samp{]} are not special there).

Note that a complement character set can match a newline, unless
newline is mentioned as one of the characters not to match.

@item ^
is a special character that matches the empty string, but only if at
the beginning of a line in the text being matched.  Otherwise, it fails
to match anything.  Thus, @samp{^foo} matches a @samp{foo} that occurs
at the beginning of a line.

@item $
is similar to @samp{^} but matches only at the end of a line.  Thus,
@samp{xx*$} matches a string of one @samp{x} or more at the end of a line.

@item \
does two things: it quotes the special characters (including
@samp{\}), and it introduces additional special constructs.

Because @samp{\} quotes special characters, @samp{\$} is a regular
expression that matches only @samp{$}, and @samp{\[} is a regular
expression that matches only @samp{[}, and so on.@refill
@end table

Note: for historical compatibility, special characters are treated as
ordinary ones if they are in contexts where their special meanings make no
sense.  For example, @samp{*foo} treats @samp{*} as ordinary since there is
no preceding expression on which the @samp{*} can act.  It is poor practice
to depend on this behavior; better to quote the special character anyway,
regardless of where is appears.@refill

Usually, @samp{\} followed by any character matches only
that character.  However, there are several exceptions: characters
which, when preceded by @samp{\}, are special constructs.  Such
characters are always ordinary when encountered on their own.  Here
is a table of @samp{\} constructs.

@table @kbd
@item \|
specifies an alternative.
Two regular expressions @var{a} and @var{b} with @samp{\|} in
between form an expression that matches anything @var{a} or
@var{b} matches.@refill

Thus, @samp{foo\|bar} matches either @samp{foo} or @samp{bar}
but no other string.@refill

@samp{\|} applies to the largest possible surrounding expressions.  Only a
surrounding @samp{\( @dots{} \)} grouping can limit the grouping power of
@samp{\|}.@refill

Full backtracking capability exists to handle multiple uses of @samp{\|}.

@item \( @dots{} \)
is a grouping construct that serves three purposes:

@enumerate
@item
To enclose a set of @samp{\|} alternatives for other operations.
Thus, @samp{\(foo\|bar\)x} matches either @samp{foox} or @samp{barx}.

@item
To enclose a complicated expression for the postfix @samp{*} to operate on.
Thus, @samp{ba\(na\)*} matches @samp{bananana}, etc., with any (zero or
more) number of @samp{na} strings.@refill

@item
To mark a matched substring for future reference.

@end enumerate

This last application is not a consequence of the idea of a
parenthetical grouping; it is a separate feature which happens to be
assigned as a second meaning to the same @samp{\( @dots{} \)} construct
because in practice there is no conflict between the two meanings.
Here is an explanation:

@item \@var{digit}
after the end of a @samp{\( @dots{} \)} construct, the matcher remembers the
beginning and end of the text matched by that construct.  Then, later on
in the regular expression, you can use @samp{\} followed by @var{digit}
to mean ``match the same text matched the @var{digit}'th time by the
@samp{\( @dots{} \)} construct.''@refill

The strings matching the first nine @samp{\( @dots{} \)} constructs appearing
in a regular expression are assigned numbers 1 through 9 in order that the
open-parentheses appear in the regular expression.  @samp{\1} through
@samp{\9} may be used to refer to the text matched by the corresponding
@samp{\( @dots{} \)} construct.

For example, @samp{\(.*\)\1} matches any newline-free string that is
composed of two identical halves.  The @samp{\(.*\)} matches the first
half, which may be anything, but the @samp{\1} that follows must match
the same exact text.

@item \`
matches the empty string, provided it is at the beginning
of the buffer.

@item \'
matches the empty string, provided it is at the end of
the buffer.

@item \b
matches the empty string, provided it is at the beginning or
end of a word.  Thus, @samp{\bfoo\b} matches any occurrence of
@samp{foo} as a separate word.  @samp{\bballs?\b} matches
@samp{ball} or @samp{balls} as a separate word.@refill

@item \B
matches the empty string, provided it is @i{not} at the beginning or
end of a word.

@item \<
matches the empty string, provided it is at the beginning of a word.

@item \>
matches the empty string, provided it is at the end of a word.

@item \w
matches any word-constituent character.  The editor syntax table
determines which characters these are.

@item \W
matches any character that is not a word-constituent.

@item \s@var{code}
matches any character whose syntax is @var{code}.  @var{code} is a
character which represents a syntax code: thus, @samp{w} for word
constituent, @samp{-} for whitespace, @samp{(} for open-parenthesis,
etc.  @xref{Syntax}.@refill

@item \S@var{code}
matches any character whose syntax is not @var{code}.
@end table

  Here is a complicated regexp, used by Emacs to recognize the end of a
sentence together with any whitespace that follows.  It is given in Lisp
syntax to enable you to distinguish the spaces from the tab characters.  In
Lisp syntax, the string constant begins and ends with a double-quote.
@samp{\"} stands for a double-quote as part of the regexp, @samp{\\} for a
backslash as part of the regexp, @samp{\t} for a tab and @samp{\n} for a
newline.

@example
"[.?!][]\"')]*\\($\\|\t\\|  \\)[ \t\n]*"
@end example

@noindent
This regexp contains four parts: a character set matching
period, @samp{?} or @samp{!}; a character set matching close-brackets,
quotes or parentheses, repeated any number of times; an alternative in
backslash-parentheses that matches end-of-line, a tab or two spaces; and
a character set matching whitespace characters, repeated any number of
times.

@node Search Case, Replace, Regexps, Search
@section Searching and Case

@vindex case-fold-search
  All searches in Emacs normally ignore the case of the text they
are searching through; if you specify searching for @samp{FOO},
@samp{Foo} and @samp{foo} are also considered a match.  Regexps, and in
particular character sets, are included: @samp{[aB]} matches @samp{a}
or @samp{A} or @samp{b} or @samp{B}.@refill

  If you want a case-sensitive search, set the variable
@code{case-fold-search} to @code{nil}.  Then all letters must match
exactly, including case. @code{case-fold-search} is a per-buffer
variable; altering it affects only the current buffer, but
there is a default value which you can change as well.  @xref{Locals}.

@node Replace, Other Repeating Search, Search Case, Search
@section Replacement Commands
@cindex replacement
@cindex string substitution
@cindex global substitution

  Global search-and-replace operations are not needed as often in Emacs as
they are in other editors, but they are available.  In addition to the
simple @code{replace-string} command which is like that found in most
editors, there is a @code{query-replace} command which asks you, for each
occurrence of a pattern, whether to replace it.

  The replace commands all replace one string (or regexp) with one
replacement string.  It is possible to perform several replacements in
parallel using the command @code{expand-region-abbrevs}.  @xref{Expanding
Abbrevs}.

@menu
* Unconditional Replace::  Replacing all matches for a string.
* Regexp Replace::         Replacing all matches for a regexp.
* Replacement and Case::   How replacements preserve case of letters.
* Query Replace::          How to use querying.
@end menu

@node Unconditional Replace, Regexp Replace, Replace, Replace
@subsection Unconditional Replacement
@findex replace-string
@findex replace-regexp

@table @kbd
@item M-x replace-string @key{RET} @var{string} @key{RET} @var{newstring} @key{RET}
Replace every occurrence of @var{string} with @var{newstring}.
@item M-x replace-regexp @key{RET} @var{regexp} @key{RET} @var{newstring} @key{RET}
Replace every match for @var{regexp} with @var{newstring}.
@end table

  To replace every instance of @samp{foo} after point with @samp{bar},
use the command @kbd{M-x replace-string} with the two arguments
@samp{foo} and @samp{bar}.  Replacement occurs only after point: if you
want to cover the whole buffer you must go to the beginning first.  By
default, all occurrences up to the end of the buffer are replaced.  To
limit replacement to part of the buffer, narrow to that part of the
buffer before doing the replacement (@pxref{Narrowing}).

  When @code{replace-string} exits, point is left at the last occurrence
replaced.  The value of point when the @code{replace-string} command was
issued is remembered on the mark ring; @kbd{C-u C-@key{SPC}} moves back
there.

  A numeric argument restricts replacement to matches that are surrounded
by word boundaries.

@node Regexp Replace, Replacement and Case, Unconditional Replace, Replace
@subsection Regexp Replacement

  @code{replace-string} replaces exact matches for a single string.  The
similar command @code{replace-regexp} replaces any match for a specified
pattern.

  In @code{replace-regexp}, the @var{newstring} need not be constant.  It
can refer to all or part of what is matched by the @var{regexp}.  @samp{\&}
in @var{newstring} stands for the entire text being replaced.
@samp{\@var{d}} in @var{newstring}, where @var{d} is a digit, stands for
whatever matched the @var{d}'th parenthesized grouping in @var{regexp}.
For example,@refill

@example
M-x replace-regexp @key{RET} c[ad]+r @key{RET} \&-safe @key{RET}
@end example

@noindent
would replace (for example) @samp{cadr} with @samp{cadr-safe} and @samp{cddr}
with @samp{cddr-safe}.

@example
M-x replace-regexp @key{RET} \(c[ad]+r\)-safe @key{RET} \1 @key{RET}
@end example

@noindent
would perform exactly the opposite replacements.  To include a @samp{\}
in the text to replace with, you must give @samp{\\}.

@node Replacement and Case, Query Replace, Regexp Replace, Replace
@subsection Replace Commands and Case

@vindex case-replace
@vindex case-fold-search
  If the arguments to a replace command are in lower case, the command
preserves case when it makes a replacement.  Thus, the command

@example
M-x replace-string @key{RET} foo @key{RET} bar @key{RET}
@end example

@noindent
replaces a lower case @samp{foo} with a lower case @samp{bar}, @samp{FOO}
with @samp{BAR}, and @samp{Foo} with @samp{Bar}.  If upper case letters are
used in the second argument, they remain upper case every time that
argument is inserted.  If upper case letters are used in the first
argument, the second argument is always substituted exactly as given, with
no case conversion.  Likewise, if the variable @code{case-replace} is set
to @code{nil}, replacement is done without case conversion.  If
@code{case-fold-search} is set to @code{nil}, case is significant in
matching occurrences of @samp{foo} to replace; also, case conversion of the
replacement string is not done.

@node Query Replace,, Replacement and Case, Replace
@subsection Query Replace
@cindex query replace

@table @kbd
@item M-% @var{string} @key{RET} @var{newstring} @key{RET}
@itemx M-x query-replace @key{RET} @var{string} @key{RET} @var{newstring} @key{RET}
Replace some occurrences of @var{string} with @var{newstring}.
@item M-x query-replace-regexp @key{RET} @var{regexp} @key{RET} @var{newstring} @key{RET}
Replace some matches for @var{regexp} with @var{newstring}.
@end table

@kindex M-%
@findex query-replace
  If you want to change only some of the occurrences of @samp{foo} to
@samp{bar}, not all of them, you can (@code{query-replace}) instead of
@kbd{M-%}.  This command finds occurrences of @samp{foo} one by one,
displays each occurrence, and asks you whether to replace it.  A numeric
argument to @code{query-replace} tells it to consider only occurrences
that are bounded by word-delimiter characters.@refill

@findex query-replace-regexp
  Aside from querying, @code{query-replace} works just like
@code{replace-string}, and @code{query-replace-regexp} works
just like @code{replace-regexp}.@refill

  The things you can type when you are shown an occurrence of @var{string}
or a match for @var{regexp} are:

@kindex SPC (query-replace)
@kindex DEL (query-replace)
@kindex , (query-replace)
@kindex ESC (query-replace)
@kindex . (query-replace)
@kindex ! (query-replace)
@kindex ^ (query-replace)
@kindex C-r (query-replace)
@kindex C-w (query-replace)
@kindex C-l (query-replace)

@c WideCommands
@table @kbd
@item @key{SPC}
to replace the occurrence with @var{newstring}.  This preserves case, just
like @code{replace-string}, provided @code{case-replace} is non-@code{nil},
as it normally is.@refill

@item @key{DEL}
to skip to the next occurrence without replacing this one.

@item , @r{(Comma)}
to replace this occurrence and display the result.  You are then
prompted for another input character, however, since the replacement has
already been made, @key{DEL} and @key{SPC} are equivalent.  At this
point, you can type @kbd{C-r} (see below) to alter the replaced text.  To
undo the replacement, you can type @kbd{C-x u}. 
This exits the @code{query-replace}.  If you want to do further
replacement you must use @kbd{C-x ESC} to restart (@pxref{Repetition}).

@item @key{ESC}
to exit without doing any more replacements.

@item .@: @r{(Period)}
to replace this occurrence and then exit.

@item !
to replace all remaining occurrences without asking again.

@item ^
to go back to the location of the previous occurrence (or what used to
be an occurrence), in case you changed it by mistake.  This works by
popping the mark ring.  Only one @kbd{^} in a row is allowed, because
only one previous replacement location is kept during @code{query-replace}.

@item C-r
to enter a recursive editing level, in case the occurrence needs to be
edited rather than just replaced with @var{newstring}.  When you are
done, exit the recursive editing level with @kbd{C-M-c} and the next
occurrence will be displayed.  @xref{Recursive Edit}.

@item C-w
to delete the occurrence, and then enter a recursive editing level as
in @kbd{C-r}.  Use the recursive edit to insert text to replace the
deleted occurrence of @var{string}.  When done, exit the recursive
editing level with @kbd{C-M-c} and the next occurrence will be
displayed.

@item C-l
to redisplay the screen and then give another answer.

@item C-h
to display a message summarizing these options, then give another
answer.
@end table

  If you type any other character, Emacs exits the @code{query-replace}, and
executes the character as a command.  To restart the @code{query-replace},
use @kbd{C-x @key{ESC}}, which repeats the @code{query-replace} because it
used the minibuffer to read its arguments.  @xref{Repetition, C-x ESC}.

@node Other Repeating Search,, Replace, Search
@section Other Search-and-Loop Commands

  Here are some other commands that find matches for a regular expression.
They all operate from point to the end of the buffer.

@findex list-matching-lines
@findex occur
@findex count-matches
@findex delete-non-matching-lines
@findex delete-matching-lines
@c grosscommands
@table @kbd
@item M-x occur
Print each line that follows point and contains a match for the
specified regexp.  A numeric argument specifies the number of context
lines to print before and after each matching line; the default is
none.

@kindex C-c C-c (Occur mode)
The buffer @samp{*Occur*} containing the output serves as a menu for
finding occurrences in their original context.  Find an occurrence
as listed in @samp{*Occur*}, position point there, and type @kbd{C-c
C-c}; this switches to the buffer that was searched and moves point to
the original of the same occurrence.

@item M-x list-matching-lines
Synonym for @kbd{M-x occur}.

@item M-x count-matches
Print the number of matches following point for the specified regexp.

@item M-x delete-non-matching-lines
Delete each line that follows point and does not contain a match for
the specified regexp.

@item M-x delete-matching-lines
Delete each line that follows point and contains a match for the
specified regexp.
@end table

@node Fixit, Files, Search, Top
@chapter Commands for Fixing Typos
@cindex typos
@cindex mistakes, correcting

  This chapter describes commands that are especially useful when you
catch a mistake in your text just after you have made it, or change your
mind while composing text on line.

@menu
* Kill Errors:: Commands to kill a batch of recently entered text.
* Transpose::   Exchanging two characters, words, lines, lists...
* Fixing Case:: Correcting case of last word entered.
* Spelling::    Apply spelling checker to a word, or a whole file.
@end menu

@node Kill Errors, Transpose, Fixit, Fixit
@section Killing Your Mistakes

@table @kbd
@item @key{DEL}
Delete last character (@code{delete-backward-char}).
@item M-@key{DEL}
Kill last word (@code{backward-kill-word}).
@item C-x @key{DEL}
Kill to beginning of sentence (@code{backward-kill-sentence}).
@end table

@kindex DEL
@findex delete-backward-char
  The @key{DEL} character (@code{delete-backward-char}) is the most
important correction command.  When used among graphic (self-inserting)
characters, it can be thought of as canceling the last character typed.

@kindex M-DEL
@kindex C-x DEL
@findex backward-kill-word
@findex backward-kill-sentence
  When your mistake is longer than a couple of characters, it might be more
convenient to use @kbd{M-@key{DEL}} or @kbd{C-x @key{DEL}}.
@kbd{M-@key{DEL}} kills back to the start of the last word, and @kbd{C-x
@key{DEL}} kills back to the start of the last sentence.  @kbd{C-x
@key{DEL}} is particularly useful when you are thinking of what to write as
you type it, in case you change your mind about phrasing.
@kbd{M-@key{DEL}} and @kbd{C-x @key{DEL}} save the killed text for
@kbd{C-y} and @kbd{M-y} to retrieve.  @xref{Yanking}.@refill

  @kbd{M-@key{DEL}} is often useful even when you have typed only a few
characters wrong, if you know you are confused in your typing and aren't
sure exactly what you typed.  At such a time, you cannot correct with
@key{DEL} except by looking at the screen to see what you did.  It requires
less thought to kill the whole word and start over.

@node Transpose, Fixing Case, Kill Errors, Fixit
@section Transposing Text

@table @kbd
@item C-t
Transpose two characters (@code{transpose-chars}).
@item M-t
Transpose two words (@code{transpose-words}).
@item C-M-t
Transpose two balanced expressions (@code{transpose-sexps}).
@item C-x C-t
Transpose two lines (@code{transpose-lines}).
@end table

@cindex transposition
@kindex C-t
@findex transpose-chars
  The common error of transposing two adjacent characters can be fixed
with the @kbd{C-t} command (@code{transpose-chars}).  Normally,
@kbd{C-t} transposes the two characters on either side of point.  When
given at the end of a line, @kbd{C-t} transposes the last two characters
on the line, rather than transposing the last character of the line with
the newline, which would be useless.  If you catch a
transposition error right away, you can fix it with just @kbd{C-t}.
If you catch the error later,  move the cursor back to between
the two transposed characters.  If you transposed a space with the last
character of the word before it, the word motion commands are a good way
of getting there.  Otherwise, a reverse search (@kbd{C-r}) is often the
best way.  @xref{Search}.

@kindex C-x C-t
@findex transpose-lines
@kindex M-t
@findex transpose-words
@kindex C-M-t
@findex transpose-sexps
  @kbd{Meta-t} (@code{transpose-words}) transposes the word before point
with the word after point.  It moves point forward over a word, dragging
the word preceding or containing point forward as well.  The punctuation
characters between the words do not move.  For example, @w{@samp{FOO, BAR}}
transposes into @w{@samp{BAR, FOO}} rather than @samp{@w{BAR FOO,}}.

  @kbd{C-M-t} (@code{transpose-sexps}) is a similar command for transposing
two expressions (@pxref{Lists}), and @kbd{C-x C-t} (@code{transpose-lines})
exchanges lines.  It works like @kbd{M-t} but in determins the
division of the text into syntactic units differently.

  A numeric argument to a transpose command serves as a repeat count: it
tells the transpose command to move the character (word, sexp, line) before
or containing point across several other characters (words, sexps, lines).
For example, @kbd{C-u 3 C-t} moves the character before point forward
across three other characters.  This is equivalent to repeating @kbd{C-t}
three times.  @kbd{C-u - 4 M-t} moves the word before point backward across
four words.  @kbd{C-u - C-M-t} would cancel the effect of plain
@kbd{C-M-t}.@refill

  A numeric argument of zero transposes the character (word, sexp, line)
ending after point with the one ending after the mark (otherwise a
command with a repeat count of zero would do nothing).

@node Fixing Case, Spelling, Transpose, Fixit
@section Case Conversion

@table @kbd
@item M-- M-l
Convert last word to lower case.  Note @kbd{Meta--} is Meta-minus.
@item M-- M-u
Convert last word to all upper case.
@item M-- M-c
Convert last word to lower case with capital initial.
@end table

@findex downcase-word
@findex upcase-word
@findex capitalize-word
@kindex M-@t{-} M-l
@kindex M-@t{-} M-u
@kindex M-@t{-} M-c
@cindex case conversion
@cindex words
  A  common error is to type words in the wrong case.  Because of this,
the word case-conversion commands @kbd{M-l}, @kbd{M-u} and @kbd{M-c} do
not move the cursor when used with a negative argument.
As soon as you see you have mistyped the last word, you can simply
case-convert it and continue typing.  @xref{Case}.@refill

@page
@node Spelling,, Fixing Case, Fixit
@section Checking and Correcting Spelling
@cindex spelling

@c doublewidecommands
@table @kbd
@item M-$
Check and correct spelling of word (@code{spell-word}).
@item M-x spell-buffer
Check and correct spelling of each word in the buffer.
@item M-x spell-region
Check and correct spelling of each word in the region.
@item M-x spell-string
Check spelling of specified word.
@end table

@kindex M-$
@findex spell-word
  To check the spelling of the word before point, and optionally correct
it, use the command @kbd{M-$} (@code{spell-word}).  This command runs an
inferior process containing the @code{spell} program to see whether the
word is correct English.  If it is not, it asks you to edit the word (in
the minibuffer) into a corrected spelling, and then performs a
@code{query-replace} to substitute the corrected spelling for the old
one throughout the buffer.

  If you exit the minibuffer without altering the original spelling, it
means you do not want to do anything to that word.  In that case, the
@code{query-replace} is not done.

@findex spell-buffer
  @kbd{M-x spell-buffer} checks each word in the buffer the same way that
@code{spell-word} does, doing a @code{query-replace} for
every incorrect word if appropriate.@refill

@findex spell-region
  @kbd{M-x spell-region} is similar to @code{spell-buffer} but operates
only on the region, not the entire buffer.

@findex spell-string
  @kbd{M-x spell-string} reads a string as an argument and checks
whether that is a correctly spelled English word.  It prints a message
giving the answer in the echo area.

@node Files, Buffers, Fixit, Top
@chapter File Handling
@cindex files

  The basic unit of stored data in Unix is the @dfn{file}.  To edit a file,
you must tell Emacs to examine the file and prepare a buffer containing a
copy of the file's text.  This is called @dfn{visiting} the file.  Editing
commands apply directly to text in the buffer; that is, to the copy inside
Emacs.  Your changes appear in the file itself only when you @dfn{save} the
buffer back into the file.

  In addition to visiting and saving files, Emacs can delete, copy, rename,
and append to files, and operate on file directories.

@menu
* File Names::   How to type and edit file name arguments.
* Visiting::     Visiting a file prepares Emacs to edit the file.
* Saving::       Saving makes your changes permanent.
* Reverting::    Reverting cancels all the changes not saved.
* Auto Save::    Auto Save periodically protects against loss of data.
* ListDir::      Listing the contents of a file directory.
* Dired::        ``Editing'' a directory to delete, rename, etc.
                  the files in it.
* Misc File Ops:: Other things you can do on files.
@end menu

@node File Names, Visiting, Files, Files
@section File Names
@cindex file names

  Most Emacs commands that operate on a file require you to specify the
file name.  (Saving and reverting are exceptions; the buffer knows which
file name to use for them.)  File names are specified in the minibuffer
(@pxref{Minibuffer}).  @dfn{Completion} is available, to make it easier to
specify long file names.  @xref{Completion}.

  There is always a @dfn{default file name} which is used if you
enter an empty argument by typing just @key{RET}.  Normally the default
file name is the name of the file visited in the current buffer; this
makes it easy to operate on that file with any of the Emacs file
commands.

@vindex default-directory
  Each buffer has a default directory, normally the same as the
directory of the file visited in that buffer.  When Emacs reads a file
name, the default directory is used if you do not specify a directory.
If you specify a directory in a relative fashion, with a name that does
not start with a slash, it is interpreted with respect to the default
directory.  The default directory of the current buffer is kept in the
variable @code{default-directory}, which has a separate value in every
buffer.  The value of the variable should end with a slash.

  For example, if the default file name is @file{/u/rms/gnu/gnu.tasks} then
the default directory is @file{/u/rms/gnu/}.  If you type just @samp{foo},
which does not specify a directory, it is short for @file{/u/rms/gnu/foo}.
@samp{../.login} would stand for @file{/u/rms/.login}.  @samp{new/foo}
would stand for the filename @file{/u/rms/gnu/new/foo}.

@vindex default-directory-alist
The variable @code{default-directory-alist} takes an alist of major
modes and their opinions on @code{default-directory} as a Lisp
expression to evaluate.  A resulting value of @code{nil} is ignored in
favor of @code{default-directory}.

@findex make-directory
@findex remove-directory
@cindex creating directories
@cindex removing directories
You can create a new directory with the function @code{make-directory},
which takes as an argument a file name string. The current directory is
displayed in the minibuffer when the function is called; you can delete
the old directory name and supply a new directory name. For example, if
the current directory is @file{/u/rms/gnu}, you can delete @file{gnu}
and type @file{oryx} and @key{RET} to create @file{/u/rms/oryx}.
Removing a directory is similar to creating one.  To remove a directory,
use @code{remove-directory}; it takes one argument a file name string.

  The command @kbd{M-x pwd} prints the current buffer's default directory,
and the command @kbd{M-x cd} sets it (to a value read using the
minibuffer).  A buffer's default directory changes only when the @code{cd}
command is used.  A file-visiting buffer's default directory is initialized
to the directory of the file that is visited there.  If a buffer is created
with @kbd{C-x b}, its default directory is copied from that of the
buffer that was current at the time.

@vindex insert-default-directory
  The default directory name actually appears in the minibuffer when the
minibuffer becomes active to read a file name.  This serves two
purposes: it shows you what the default is, so that you can type a
relative file name and know with certainty what it will mean, and it
allows you to edit the default to specify a different directory.  To
inhibit the insertion of the default directory, set the variable
@code{insert-default-directory} to @code{nil}.

  Note that it is legitimate to type an absolute file name after you
enter the minibuffer, ignoring the presence of the default directory
name.  The final minibuffer contents may look invalid, but that is not
so.  @xref{Minibuffer File}.

  @samp{$} in a file name is used to substitute environment variables.  For
example, if you have used the shell command @samp{setenv FOO rms/hacks} to
set up an environment variable named @samp{FOO}, then you can use
@file{/u/$FOO/test.c} or @file{/u/$@{FOO@}/test.c} as an abbreviation for
@file{/u/rms/hacks/test.c}.  The environment variable name consists of all
the alphanumeric characters after the @samp{$}; alternatively, it may be
enclosed in braces after the @samp{$}.  Note that the @samp{setenv} command
affects Emacs only if done before Emacs is started.

  To access a file with @samp{$} in its name, type @samp{$$}.  This pair
is converted to a single @samp{$} at the same time variable substitution
is performed for single @samp{$}.  The Lisp function that performs the
substitution is called @code{substitute-in-file-name}.  The substitution
is performed only on filenames read as such using the minibuffer.

@node Visiting, Saving, File Names, Files
@section Visiting Files
@cindex visiting files

@c WideCommands
@table @kbd
@item C-x C-f
Visit a file (@code{find-file}).
@item C-x C-v
Visit a different file instead of the one visited last
(@code{find-alternate-file}).
@item C-x 4 C-f
Visit a file, in another window (@code{find-file-other-window}).  Don't
change this window.
@end table

@cindex files
@cindex visiting
@cindex saving
  @dfn{Visiting} a file means copying its contents into an Emacs buffer
so you can edit it.  Emacs creates a new buffer for each file you
visit.  We say that the buffer is visiting the file that it was created
to hold.  Emacs constructs the buffer name from the file name by
throwing away the directory, keeping just the file name.  For example,
a file named @file{/usr/rms/emacs.tex} is displayed in a buffer named
@samp{emacs.tex}.  If a buffer with that name exists, a unique
name is constructed by appending @samp{<2>}, @samp{<3>},and so on, using
the lowest number that makes a name that is not already in use.

  Each window's mode line shows the name of the buffer that is being displayed
in that window, so you can always tell what buffer you are editing.

  The changes you make with Emacs are made in the Emacs buffer.  They do
not take effect in the file that you visit, or any other permanent
place, until you @dfn{save} the buffer.  Saving the buffer means that
Emacs writes the current contents of the buffer into its visited file.
@xref{Saving}.

@cindex modified (buffer)
  If a buffer contains changes that have not been saved, the buffer is said
to be @dfn{modified}.  This is important because it implies that some
changes will be lost if the buffer is not saved.  The mode line displays
two stars near the left margin if the buffer is modified.

@kindex C-x C-f
@findex find-file
@findex find-file-new-screen
  To visit a file, use the command @kbd{C-x C-f} (@code{find-file}).  Follow
the command with the name of the file you wish to visit, terminated by a
@key{RET}.  If you are using Lucid GNU Emacs under X, you can also use the
@b{Open File...} command from the @b{File} menu bar item. 

  The file name is read using the minibuffer (@pxref{Minibuffer}), with
defaulting and completion in the standard manner (@pxref{File Names}).
While in the minibuffer, you can abort @kbd{C-x C-f} by typing @kbd{C-g}.

  @kbd{C-x C-f} has completed successfully when text appears on the
screen and a new buffer name appears in the mode line.  If the specified
file does not exist and could not be created or cannot be read, an error
results.  The error message is printed in the echo area, and includes
the name of the file that Emacs was trying to visit.

  If you visit a file that is already in Emacs, @kbd{C-x C-f} does not make
another copy.  It selects the existing buffer containing that file.
However, before doing so, it checks that the file itself has not changed
since you visited or saved it last.  If the file has changed, Emacs
prints a warning message.  @xref{Interlocking,,Simultaneous Editing}.

@findex find-this-file
You can switch to a specific file called out in the current buffer by
calling the function @code{find-this-file}. By providing a prefix
argument, this function calls @code{filename-at-point} and switches to a
buffer visiting the file @var{filename}. It creates one if none already
exists. You can use this function to edit the file mentioned in the
buffer you are working in or to test if the file exists. You can do that
by using the minibuffer completion after snatching the all or part of
the filename.

 You can display a file in a new screen by using the function
@code{find-file-new-screen}. This function is similar to
@code{find-file} except that it creates a new screen in which the file
is displayed.

@vindex find-file-use-truenames
@vindex buffer-file-name
If the variable @code{find-file-use-truenames}'s value is
non-@code{nil}, a buffer's visited filename will always be traced back
to the real file. The filename will never be a symbolic link, and there
will never be a symbolic link anywhere in its directory path. In other
words, the @code{buffer-file-name} and @code{buffer-file-truename} will
be equal.

@vindex find-file-compare-truenames
@vindex buffer-file-truename
If the variable @code{find-file-compare-truenames} value is
non-@code{nil}, the @code{find-file} command will check the
@code{buffer-file-truename} of all visited files when deciding whether a
given file is already in a buffer, instead of just
@code{buffer-file-name}.  If you attempt to visit another file which is
a hard-link or symbolic-link to a file that is already in a buffer, the
existing buffer will be found instead of a newly created one.

@cindex creating files
   If you want to create a file, just visit it.  Emacs prints
@samp{(New File)} in the echo area, but in other respects behaves as if you
had visited an existing empty file.  If you make any changes and save them,
the file is created.

@kindex C-x C-v
@findex find-alternate-file
  If you visit a nonexistent file unintentionally (because you typed the
wrong file name), use the @kbd{C-x C-v} (@code{find-alternate-file})
command to visit the file you wanted.  @kbd{C-x C-v} is similar to @kbd{C-x
C-f}, but it kills the current buffer (after first offering to save it if
it is modified).  @kbd{C-x C-v} is allowed even if the current buffer
is not visiting a file.

@vindex find-file-run-dired
  If the file you specify is actually a directory, Dired is called on
that directory (@pxref{Dired}).  To inhibit this, set the variable
@code{find-file-run-dired} to @code{nil}; then it is an error to try to
visit a directory.

@kindex C-x 4 f
@findex find-file-other-window
  @kbd{C-x 4 f} (@code{find-file-other-window}) is like @kbd{C-x C-f}
except that the buffer containing the specified file is selected in another
window.  The window that was selected before @kbd{C-x 4 f} continues to
show the same buffer it was already showing.  If you use this command when
only one window is being displayed, that window is split in two, with one
window showing the same buffer as before, and the other one showing the
newly requested file.  @xref{Windows}.

@findex find-this-file-other-window
 Use the function @code{find-this-file-other-window} to edit a file
mentioned in the buffer yuou are editing or to test if that file exists.
To do do this, use the minibuffer completion after snatching the part or
all of the filename. By providing a prefix argument, the function calls
@code{filename-at-point} abd switches you to a buffer visiting the file
@var{filename} in another window. The function creates a buffer if none
already exists. This function is similar to @code{find-file-other-window}.

@vindex find-file-hooks
@vindex find-file-not-found-hooks
  There are two hook variables that allow extensions to modify the
operation of visiting files.  Visiting a file that does not exist runs the
functions in the list @code{find-file-not-found-hooks}; the value of this
variable is expected to be a list of functions which are
called one by one until one of them returns non-@code{nil}.  Any visiting
of a file, whether extant or not, expects @code{find-file-hooks} to
contain list of functions and calls them all, one by one.  In both cases
the functions receive no arguments.  Visiting a nonexistent file
runs the @code{find-file-not-found-hooks} first.

@node Saving, Reverting, Visiting, Files
@section Saving Files

  @dfn{Saving} a buffer in Emacs means writing its contents back into the file
that was visited in the buffer.

@table @kbd
@item C-x C-s
Save the current buffer in its visited file (@code{save-buffer}).
@item C-x s
Save any or all buffers in their visited files (@code{save-some-buffers}).
@item M-~
Forget that the current buffer has been changed (@code{not-modified}).
@item C-x C-w
Save the current buffer in a specified file, and record that file as
the one visited in the buffer (@code{write-file}).
@item M-x set-visited-file-name
Change file the name under which the current buffer will be saved.
@end table

@kindex C-x C-s
@findex save-buffer
  To save a file and make your changes permanent, type
@kbd{C-x C-s} (@code{save-buffer}).  After saving is finished, @kbd{C-x C-s}
prints a message like

@example
Wrote /u/rms/gnu/gnu.tasks
@end example

@noindent
If the selected buffer is not modified (no changes have been made in it
since the buffer was created or last saved), Emacs does not save it
because it would have no effect.  Instead, @kbd{C-x C-s} prints a message
in the echo area saying

@example
(No changes need to be written)
@end example

@kindex C-x s
@findex save-some-buffers
  The command @kbd{C-x s} (@code{save-some-buffers}) can save any or all
modified buffers.  First it asks, for each modified buffer, whether to
save it.  The questions should be answered with @kbd{y} or @kbd{n}.
@kbd{C-x C-c}, the key that kills Emacs, invokes
@code{save-some-buffers} and therefore asks the same questions.

@kindex M-~
@findex not-modified
  If you have changed a buffer and do not want the changes to be saved,
you should take some action to prevent it.  Otherwise, you are liable to
save it by mistake each time you use @code{save-some-buffers} or a
related command.  One thing you can do is type @kbd{M-~}
(@code{not-modified}), which removes the indication that the buffer
is modified.  If you do this, none of the save commands will believe
that the buffer needs to be saved.  (@samp{~} is often used as a
mathematical symbol for `not'; thus @kbd{Meta-~} is `not', metafied.)
You could also use @code{set-visited-file-name} (see below) to mark the
buffer as visiting a different file name, not in use for
anything important. 

You can also undo all the changes made since the file was visited or
saved, by reading the text from the file again.  This is called
@dfn{reverting}.  @xref{Reverting}.  Alternatively, you can undo all the
changes by repeating the undo command @kbd{C-x u}; but this only works
if you have not made more changes than the undo mechanism can remember.

@findex set-visited-file-name
  @kbd{M-x set-visited-file-name} alters the name of the file that the
current buffer is visiting.  It prompts you for the new file name in the
minibuffer.  You can also use @code{set-visited-file-name} on a buffer
that is not visiting a file.  The buffer's name is changed to correspond
to the file it is now visiting unless the new name is already used by a
different buffer; in that case, the buffer name is not changed.
@code{set-visited-file-name} does not save the buffer in the newly
visited file; it just alters the records inside Emacs so that it will
save the buffer in that file.  It also marks the buffer as ``modified''
so that @kbd{C-x C-s} @i{will} save.

@kindex C-x C-w
@findex write-file
  If you wish to mark a buffer as visiting a different file and save it
right away, use @kbd{C-x C-w} (@code{write-file}).  It is precisely
equivalent to @code{set-visited-file-name} followed by @kbd{C-x C-s}.
@kbd{C-x C-s} used on a buffer that is not visiting  a file has the
same effect as @kbd{C-x C-w}; that is, it reads a file name, marks the
buffer as visiting that file, and saves it there.  The default file name in
a buffer that is not visiting a file is made by combining the buffer name
with the buffer's default directory.

  If Emacs is about to save a file and sees that the date of the latest
version on disk does not match what Emacs last read or wrote, Emacs
notifies you of this fact, because it probably indicates a problem caused
by simultaneous editing and requires your immediate attention.
@xref{Interlocking,, Simultaneous Editing}.

@vindex require-final-newline
  If the variable @code{require-final-newline} is non-@code{nil}, Emacs
puts a newline at the end of any file that doesn't already end in one,
every time a file is saved or written.

@vindex write-file-hooks
@vindex after-write-file-hooks
  Use the hook variable @code{write-file-hooks} to implement other ways
to write files, and specify things to be done before files are written.  The
value of this variable should be a list of Lisp functions.  When a file
is to be written, the functions in the list are called, one by one, with
no arguments.  If one of them returns a non-@code{nil} value, Emacs
takes this to mean that the file has been written in some suitable
fashion; the rest of the functions are not called, and normal writing is
not done. Use the hook variable @code{after-write-file-hooks} to list
all the functions to be called after writing out a buffer to a file.

@menu
* Backup::       How Emacs saves the old version of your file.
* Interlocking:: How Emacs protects against simultaneous editing
                  of one file by two users.
@end menu

@node Backup, Interlocking, Saving, Saving
@subsection Backup Files
@cindex backup file
@vindex make-backup-files

  Because Unix does not provide version numbers in file names, rewriting a
file in Unix automatically destroys all record of what the file used to
contain.  Thus, saving a file from Emacs throws away the old contents of
the file---or it would, except that Emacs carefully copies the old contents
to another file, called the @dfn{backup} file, before actually saving.
(Provided the variable @code{make-backup-files} is non-@code{nil}.
Backup files are not written if this variable is @code{nil}).

  At your option, Emacs can keep either a single backup file or a series of
numbered backup files for each file you edit.

  Emacs makes a backup for a file only the first time a file is saved
from one buffer.  No matter how many times you save a file, its backup file
continues to contain the contents from before the file was visited.
Normally this means that the backup file contains the contents from before
the current editing session; however, if you kill the buffer and then visit
the file again, a new backup file is made by the next save.

@menu
* Names: Backup Names.		How backup files are named;
				Choosing single or numbered backup files.
* Deletion: Backup Deletion.	Emacs deletes excess numbered backups.
* Copying: Backup Copying.	Backups can be made by copying or renaming.
@end menu

@node Backup Names, Backup Deletion, Backup, Backup
@subsubsection Single or Numbered Backups

  If you choose to have a single backup file (the default),
the backup file's name is constructed by appending @samp{~} to the
file name being edited; thus, the backup file for @file{eval.c} is
@file{eval.c~}.

  If you choose to have a series of numbered backup files, backup file
names are made by appending @samp{.~}, the number, and another @samp{~} to
the original file name.  Thus, the backup files of @file{eval.c} would be
called @file{eval.c.~1~}, @file{eval.c.~2~}, and so on, through names
like @file{eval.c.~259~} and beyond.

  If protection stops you from writing backup files under the usual names,
the backup file is written as @file{%backup%~} in your home directory.
Only one such file can exist, so only the most recently made backup is
available.

@vindex version-control
  The choice of single backup or numbered backups is controlled by the
variable @code{version-control}.  Its possible values are

@table @code
@item t
Make numbered backups.
@item nil
Make numbered backups for files that have numbered backups already.
Otherwise, make single backups.
@item never
Never make numbered backups; always make single backups.
@end table

@noindent
@code{version-control} may be set locally in an individual buffer to
control the making of backups for that buffer's file.  For example,
Rmail mode locally sets @code{version-control} to @code{never} to make sure
that there is only one backup for an Rmail file.  @xref{Locals}.

@node Backup Deletion, Backup Copying, Backup Names, Backup
@subsubsection Automatic Deletion of Backups

@vindex kept-old-versions
@vindex kept-new-versions
  To prevent unlimited consumption of disk space, Emacs can delete numbered
backup versions automatically.  Generally Emacs keeps the first few backups
and the latest few backups, deleting any in between.  This happens every
time a new backup is made.  The two variables that control the deletion are
@code{kept-old-versions} and @code{kept-new-versions}.  Their values are, respectively
the number of oldest (lowest-numbered) backups to keep and the number of
newest (highest-numbered) ones to keep, each time a new backup is made.
The values are used just after a new backup version is made;
that newly made backup is included in the count in @code{kept-new-versions}.
By default, both variables are 2.

@vindex trim-versions-without-asking
  If @code{trim-versions-without-asking} is non-@code{nil},  excess
middle versions are deleted without notification.  If it is @code{nil}, the
default, you are asked whether the excess middle versions should
really be deleted.

  You can also use Dired's @kbd{.} (Period) command to delete old versions.
@xref{Dired}.

@node Backup Copying,, Backup Deletion, Backup
@subsubsection Copying vs.@: Renaming

  You can make backup files by copying the old file or by renaming it.
This makes a difference when the old file has multiple names.  If you
rename the old file into the backup file, the alternate names
become names for the backup file.  If you copy the old file instead,
the alternate names remain names for the file that you are editing,
and the contents accessed by those names will be the new contents.

  How you make a backup file may also affect the file's owner
and group.  If you use copying, they do not change.  If renaming is used,
you become the file's owner, and the file's group becomes the default
(different operating systems have different defaults for the group).

  Having the owner change is usually a good idea, because then the owner
always shows who last edited the file.  The owners of the backups show
who produced them.  Occasionally there is a file whose owner should not
change.  Since most files` owner should change, it is a good idea to use
local variable lists to set @code{backup-by-copying-when-mismatch} for
the special cases where the owner should change.(@pxref{File
Variables}).

@vindex backup-by-copying
@vindex backup-by-copying-when-linked
@vindex backup-by-copying-when-mismatch
  Three variables control the choice of renaming or copying.
Normally, renaming is done.  If the variable @code{backup-by-copying} is
non-@code{nil}, copying is used.  Otherwise, if the variable
@code{backup-by-copying-when-linked} is non-@code{nil}, copying is
done for files that have multiple names, but renaming may still be done when
the file being edited has only one name.  If the variable
@code{backup-by-copying-when-mismatch} is non-@code{nil}, copying is
done if renaming would cause the file's owner or group to change.  @refill

@node Interlocking,,Backup,Saving
@subsection Protection against Simultaneous Editing

@cindex file dates
@cindex simultaneous editing
  Simultaneous editing occurs when two users visit the same file, both
make changes, and both save their changes.  If no one was informed that
this was happening, and you saved first, you would later find that your
changes were lost.  On some systems, Emacs notices immediately when the
second user starts to change a file already being edited, and issues a
warning.  When this is not possible, or if the second user has started
to change the file despite the warning, Emacs checks when the file is
saved, and issues a second warning when a user is about to overwrite a
file containing another user's changes.  If you are the user editing the
file, you can take corrective action at this point and prevent actual
loss of work.

@findex ask-user-about-lock
  When you make the first modification in an Emacs buffer that is visiting
a file, Emacs records that you have locked the file.  (It does this by
writing another file in a directory reserved for this purpose.)  The lock
is removed when you save the changes.  The idea is that the file is locked
whenever the buffer is modified.  If you begin to modify the buffer while
the visited file is locked by someone else, this constitutes a collision,
and Emacs asks you what to do.  It does this by calling the Lisp function
@code{ask-user-about-lock}, which you can redefine to customize what it
does.  The standard definition of this function asks you a
question and accepts three possible answers:

@table @kbd
@item s
Steal the lock.  Whoever was already changing the file loses the lock,
and you get the lock.
@item p
Proceed.  Go ahead and edit the file despite its being locked by someone else.
@item q
Quit.  This causes an error (@code{file-locked}) and the modification you
were trying to make in the buffer does not actually take place.
@end table

  Note that locking works on the basis of a file name; if a file has
multiple names, Emacs does not realize that the two names are the same file
and cannot prevent two users from editing it simultaneously under different
names.  However, basing locking on names means that Emacs can interlock the
editing of new files that do not really exist until they are saved.

  Some systems are not configured to allow Emacs to make locks.  On
these systems, Emacs cannot detect trouble in advance, but it can still
detect it in time to prevent you from overwriting someone else's changes.

  Every time Emacs saves a buffer, it first checks the last-modification
date of the existing file on disk to see that it has not changed since the
file was last visited or saved.  If the date does not match, it implies
that changes were made in the file in some other way, and these changes are
about to be lost if Emacs actually does save.  To prevent this, Emacs
prints a warning message and asks for confirmation before saving.
Occasionally you will know why the file was changed and know that it does
not matter; then you can answer @kbd{yes} and proceed.  Otherwise, you should
cancel the save with @kbd{C-g} and investigate the situation.

  The first thing you should do when notified that simultaneous editing
has already taken place is to list the directory with @kbd{C-u C-x C-d}
(@pxref{ListDir,,Directory Listing}).  This will show the file's current
author.  You should attempt to contact that person and ask her not to
continue editing.  Often the next step is to save the contents of your
Emacs buffer under a different name, and use @code{diff} to compare the
two files.@refill

  Simultaneous editing checks are also made when you visit a file that
is already visited with @kbd{C-x C-f} and when you start to modify a
file.  This is not strictly necessary, but it is useful to find out
about such a problem as early as possible, when corrective action takes
less work.

@findex set-default-file-mode
@cindex file protection
Another way to protect your file is to set the read, write, and
executable permissions for the file. Use the function
@code{set-default-file-mode} to set the UNIX @code{umask} value to the
@var{nmask} argument. The @code{umask} value is the default protection
mode for new files.

@node Reverting, Auto Save, Saving, Files
@section Reverting a Buffer
@findex revert-buffer
@cindex drastic changes

  If you have made extensive changes to a file and then change your mind
about them, you can get rid of all changes by reading in the previous
version of the file.  To do this, use @kbd{M-x revert-buffer}, which
operates on the current buffer.  Since reverting a buffer can result in
very extensive changes, you must confirm it with @kbd{yes}.

  If the current buffer has been auto-saved more recently than it has been
saved explicitly, @code{revert-buffer} offers to read the auto save file
instead of the visited file (@pxref{Auto Save}).  Emacs asks you about
the auto save file before the request for confirmation of the
@kbd{revert-buffer} operation, and demands @kbd{y} or @kbd{n}
as an answer.  If you have started to type @kbd{yes} for confirmation
without realizing that the auto-save question was going to be asked, the
@kbd{y} will answer that question, but the @kbd{es} will not be valid
confirmation.  This gives you a chance to cancel the operation with
@kbd{C-g} and try again with the answers you really intend.

  @code{revert-buffer} keeps point at the same distance (measured in
characters) from the beginning of the file.  If the file was edited only
slightly, you will be at approximately the same piece of text after
reverting as before.  If you have made more extensive changes, the value of
point in the old file may bring you to a totally different piece of text
than your last editing point.

A buffer reverted from its visited file is marked ``not modified'' until
you make a change.

  Some kinds of buffers whose contents reflect data bases other than files,
such as Dired buffers, can also be reverted.  For them, reverting means
recalculating their contents from the appropriate data.  Buffers
created randomly with @kbd{C-x b} cannot be reverted; @code{revert-buffer}
reports an error when asked to do so.

@node Auto Save, ListDir, Reverting, Files
@section Auto-Saving: Protection Against Disasters
@cindex Auto-Save mode
@cindex crashes

  Emacs saves all the visited files from time to time (based on counting
your keystrokes) without being asked.  This is called @dfn{auto-saving}.
It prevents you from losing more than a limited amount of work if the
system crashes.

  When Emacs determines it is time for auto-saving, each buffer is
considered, and is auto-saved if auto-saving is turned on for it and it has
changed since the last time it was auto-saved.  If any auto-saving is
done, the message @samp{Auto-saving...} is displayed in the echo area until
auto-saving is finished.  Errors occurring during auto-saving are caught
so that they do not interfere with the execution of commands you have been
typing.

@menu
* Files: Auto Save Files.
* Control: Auto Save Control.
* Recover::		Recovering text from auto-save files.
@end menu

@page
@node Auto Save Files, Auto Save Control, Auto Save, Auto Save
@subsection Auto-Save Files

  Auto-saving does not normally write to the files you visited, because
it can be undesirable to save a program that is in an inconsistent
state when you have made only half of a planned change.  Instead, auto-saving
is done in a different file called the @dfn{auto-save file}, and the
visited file is changed only when you save explicitly, for example, 
with @kbd{C-x C-s}.

  Normally, the name of the auto-save file is generated by appending
@samp{#} to the front and back of the visited file name.  Thus, a buffer
visiting file @file{foo.c} would be auto-saved in a file @file{#foo.c#}.
Most buffers that are not visiting files are auto-saved only if you
request it explicitly; when they are auto-saved, the auto-save file name
is generated by appending @samp{#%} to the front and @samp{#} to the
back of buffer name.  For example, the @samp{*mail*} buffer in which you
compose messages to be sent is auto-saved in a file named
@file{#%*mail*#}.  Names of auto-save files are generated this way
unless you customize the functions @code{make-auto-save-file-name} and
@code{auto-save-file-name-p} to do something different.  The file name
to be used for auto-saving a buffer is calculated at the time auto-saving is
turned on in that buffer.

@vindex auto-save-visited-file-name
  If you want auto-saving to be done in the visited file, set the variable
@code{auto-save-visited-file-name} to be non-@code{nil}.  In this mode,
there is really no difference between auto-saving and explicit saving.

@vindex delete-auto-save-files
  Emacs deletes a buffer's auto-save file when you explicitly save the
buffer.  To inhibit the deletion, set the variable
@code{delete-auto-save-files} to @code{nil}.  Changing the visited file
name with @kbd{C-x C-w} or @code{set-visited-file-name} renames any
auto-save file to correspond to the new visited name.

@node Auto Save Control, Recover, Auto Save Files, Auto Save
@subsection Controlling Auto-Saving

@vindex auto-save-default
@findex auto-save-mode
  Each time you visit a file, auto-saving is turned on for that file's
buffer if the variable @code{auto-save-default} is non-@code{nil} (but
not in batch mode; @pxref{Entering Emacs}).  The default for this
variable is @code{t}, so Emacs auto-saves buffers that visit files by
default.  You can use the command @kbd{M-x auto-save-mode} to turn
auto-saving for a buffer on or off.  Like other minor mode commands,
@kbd{M-x auto-save-mode} turns auto-saving on with a positive argument,
off with a zero or negative argument; with no argument, it toggles.

@vindex auto-save-interval
@findex do-auto-save
  Emacs performs auto-saving periodically based on counting how many
characters you have typed since the last time auto-saving happened.  The
variable @code{auto-save-interval} specifies the number of characters
between auto-saves.  By default, it is 300.  Emacs also auto-saves
whenever you call the function @code{do-auto-save}.

  Emacs also does auto-saving whenever it gets a fatal error.  This
includes killing the Emacs job with a shell command such as @code{kill
-emacs}, or disconnecting a phone line or network connection.

@vindex auto-save-timeout
You can set the number of seconds of idle time before an auto-save is
done.  A value of zero or @code{nil} means disable auto-saving due to
idleness.

The actual amount of idle time between auto-saves is logarithmically
related to the size of the current buffer.  This variable is the number
of seconds after which an auto-save will happen when the current buffer
is 50k or less; the timeout will be 2 1/4 times this in a 200k buffer, 3
3/4 times this in a 1000k buffer, and 4 1/2 times this in a 2000k
buffer.

For this variable to have any effect, you must do @code{(require 'timer)}.

@node Recover,, Auto Save Control, Auto Save
@subsection Recovering Data from Auto-Saves

@findex recover-file
  If you want to use the contents of an auto-save file to recover from a
loss of data, use the command @kbd{M-x recover-file @key{RET} @var{file}
@key{RET}}.  Emacs visits @var{file} and then (after your confirmation)
restores the contents from the auto-save file @file{#@var{file}#}.  You
can then save the file with @kbd{C-x C-s} to put the recovered text into
@var{file} itself.  For example, to recover file @file{foo.c} from its
auto-save file @file{#foo.c#}, do:@refill

@example
M-x recover-file @key{RET} foo.c @key{RET}
C-x C-s
@end example

  Before asking for confirmation, @kbd{M-x recover-file} displays a
directory listing describing the specified file and the auto-save file,
so you can compare their sizes and dates.  If the auto-save file
is older, @kbd{M-x recover-file} does not offer to read it.

  Auto-saving is disabled by @kbd{M-x recover-file} because using
this command implies that the auto-save file contains valuable data
from a past session.  If you save the data in the visited file and
then go on to make new changes, turn auto-saving back on
with @kbd{M-x auto-save-mode}.

@node ListDir, Dired, Auto Save, Files
@section Listing a File Directory

@cindex file directory
@cindex directory listing
  Files are classified by Unix into @dfn{directories}.  A @dfn{directory
listing} is a list of all the files in a directory.  Emacs provides
directory listings in brief format (file names only) and verbose format
(sizes, dates, and authors included).

@page
@table @kbd
@item C-x C-d @var{dir-or-pattern}
Print a brief directory listing (@code{list-directory}).
@item C-u C-x C-d @var{dir-or-pattern}
Print a verbose directory listing.
@end table

@findex list-directory
@kindex C-x C-d
  To print a directory listing, use @kbd{C-x C-d}
(@code{list-directory}).  This command prompts in the minibuffer for a
file name which is either a  directory to be listed or pattern
containing wildcards for the files to be listed.  For example,

@example
C-x C-d /u2/emacs/etc @key{RET}
@end example

@noindent
lists all the files in directory @file{/u2/emacs/etc}.  An example of
specifying a file name pattern is

@example
C-x C-d /u2/emacs/src/*.c @key{RET}
@end example

  Normally, @kbd{C-x C-d} prints a brief directory listing containing just
file names.  A numeric argument (regardless of value) tells it to print a
verbose listing (like @code{ls -l}).

@vindex list-directory-brief-switches
@vindex list-directory-verbose-switches
  Emacs obtains the text of a directory listing by running @code{ls} in
an inferior process.  Two Emacs variables control the switches passed to
@code{ls}: @code{list-directory-brief-switches} is a string giving the
switches to use in brief listings (@code{"-CF"} by default).
@code{list-directory-verbose-switches} is a string giving the switches
to use in a verbose listing (@code{"-l"} by default).

The variable @code{diretory-abbrev-alist} is an alist of abbreviations
for file directories.  The list consists of elements of the form
@code{(FROM .  TO)}, each meaning to replace @code{FROM} with @code{TO}
when it appears in a directory name.  This replacement is done when
setting up the default directory of a newly visited file.  Every @code{FROM}
string should start with `@samp{^}'.

Use this feature when you have directories which you normally refer to
via absolute symbolic links.  Make @code{TO} the name of the link, and
@code{FROM} the name it is linked to.

@node Dired, Misc File Ops, ListDir, Files
@section Dired, the Directory Editor
@cindex Dired
@cindex deletion (of files)

  Dired makes it easy to delete or visit many of the files in a single
directory at once.  It makes an Emacs buffer containing a listing of the
directory.  You can use the normal Emacs commands to move around in this
buffer, and special Dired commands to operate on the files.

@menu
* Enter: Dired Enter.         How to invoke Dired.
* Edit: Dired Edit.           Editing the Dired buffer.
* Deletion: Dired Deletion.   Deleting files with Dired.
* Immed: Dired Immed.         Other file operations through Dired.
@end menu

@node Dired Enter, Dired Edit, Dired, Dired
@subsection Entering Dired

@findex dired
@kindex C-x d
@vindex dired-listing-switches
  To invoke dired, type @kbd{C-x d} or @kbd{M-x dired}.  The command reads a
directory name or wildcard file name pattern as a minibuffer argument just
like the @code{list-directory} command, @kbd{C-x C-d}.  Where @code{dired}
differs from @code{list-directory} is in naming the buffer after the
directory name or the wildcard pattern used for the listing, and putting
the buffer into Dired mode so that the special commands of Dired are
available in it.  The variable @code{dired-listing-switches} is a string
used as an argument to @code{ls} in making the directory; this string
@i{must} contain @samp{-l}.

@findex dired-other-window
@kindex C-x 4 d
  To display the Dired buffer in another window rather than in the selected
window, use @kbd{C-x 4 d} (@code{dired-other-window)} instead of @kbd{C-x d}.

@node Dired Edit, Dired Deletion, Dired Enter, Dired
@subsection Editing in Dired

  Once the Dired buffer exists, you can switch freely between it and other
Emacs buffers.  Whenever the Dired buffer is selected, certain special
commands are provided that operate on files that are listed.  The Dired
buffer is ``read-only'', and inserting text in it is not useful, so
ordinary printing characters such as @kbd{d} and @kbd{x} are used for Dired
commands.  Most Dired commands operate on the file described by the line
that point is on.  Some commands perform operations immediately; others
``flag'' the file to be operated on later.

  Most Dired commands that operate on the current line's file also treat a
numeric argument as a repeat count, meaning to act on the files of the
next few lines.  A negative argument means to operate on the files of the
preceding lines, and leave point on the first of those lines.

  All the usual Emacs cursor motion commands are available in Dired
buffers.  Some special purpose commands are also provided.  The keys
@kbd{C-n} and @kbd{C-p} are redefined so that they try to position
the cursor at the beginning of the filename on the line, rather than
at the beginning of the line.

  For extra convenience, @key{SPC} and @kbd{n} in Dired are equivalent to
@kbd{C-n}.  @kbd{p} is equivalent to @kbd{C-p}.  Moving by lines is done so
often in Dired that it deserves to be easy to type.  @key{DEL} (move up and
unflag) is often useful simply for moving up.@refill

  The @kbd{g} command in Dired runs @code{revert-buffer} to reinitialize
the buffer from the actual disk directory and show any changes made in the
directory by programs other than Dired.  All deletion flags in the Dired
buffer are lost when this is done.

@node Dired Deletion, Dired Immed, Dired Edit, Dired
@subsection Deleting Files with Dired

  The primary use of Dired is to flag files for deletion and then delete
them.

@table @kbd
@item d
Flag this file for deletion.
@item u
Remove deletion-flag on this line.
@item @key{DEL}
Remove deletion-flag on previous line, moving point to that line.
@item x
Delete the files that are flagged for deletion.
@item #
Flag all auto-save files (files whose names start and end with @samp{#})
for deletion (@pxref{Auto Save}).
@item ~
Flag all backup files (files whose names end with @samp{~}) for deletion
(@pxref{Backup}).
@item .@: @r{(Period)}
Flag excess numeric backup files for deletion.  The oldest and newest
few backup files of any one file are exempt; the middle ones are flagged.
@end table

  You can flag a file for deletion by moving to the line describing the
file and typing @kbd{d} or @kbd{C-d}.  The deletion flag is visible as a
@samp{D} at the beginning of the line.  Point is moved to the beginning of
the next line, so that repeated @kbd{d} commands flag successive files.

  The files are flagged for deletion rather than deleted immediately to
avoid the danger of deleting a file accidentally.  Until you direct Dired
to delete the flagged files, you can remove deletion flags using the
commands @kbd{u} and @key{DEL}.  @kbd{u} works just like @kbd{d}, but
removes flags rather than making flags.  @key{DEL} moves upward, removing
flags; it is like @kbd{u} with numeric argument automatically negated.

  To delete the flagged files, type @kbd{x}.  This command first displays a
list of all the file names flagged for deletion, and requests confirmation
with @kbd{yes}.  Once you confirm, all the flagged files are deleted, and their
lines are deleted from the text of the Dired buffer.  The shortened Dired
buffer remains selected.  If you answer @kbd{no} or quit with @kbd{C-g}, you
return immediately to Dired, with the deletion flags still present and no
files actually deleted.

  The @kbd{#}, @kbd{~} and @kbd{.} commands flag many files for
deletion, based on their names.  These commands are useful precisely
because they do not actually delete any files; you can remove the
deletion flags from any flagged files that you really wish to keep.@refill

  @kbd{#} flags for deletion all files that appear to have been made by
auto-saving (that is, files whose names begin and end with @samp{#}).
@kbd{~} flags for deletion all files that appear to have been made as
backups for files that were edited (that is, files whose names end with
@samp{~}).

@page
@vindex dired-kept-versions
  @kbd{.} (Period) flags just some of the backup files for deletion: only
numeric backups that are not among the oldest few nor the newest few
backups of any one file.  Normally @code{dired-kept-versions} (not
@code{kept-new-versions}; that applies only when saving) specifies the
number of newest versions of each file to keep, and
@code{kept-old-versions} specifies the number of oldest versions to keep.
Period with a positive numeric argument, as in @kbd{C-u 3 .}, specifies the
number of newest versions to keep, overriding @code{dired-kept-versions}.
A negative numeric argument overrides @code{kept-old-versions}, using minus
the value of the argument to specify the number of oldest versions of each
file to keep.@refill

@node Dired Immed,, Dired Deletion, Dired
@subsection Immediate File Operations in Dired

  Some file operations in Dired take place immediately when they are
requested.

@table @kbd
@item c
Copies the file described on the current line.  You must supply a file name
to copy to, using the minibuffer.
@item f
Visits the file described on the current line.  It is just like typing
@kbd{C-x C-f} and supplying that file name.  If the file on this line is a
subdirectory, @kbd{f} actually causes Dired to be invoked on that
subdirectory.  @xref{Visiting}.
@item o
Like @kbd{f}, but uses another window to display the file's buffer.  The
Dired buffer remains visible in the first window.  This is like using
@kbd{C-x 4 C-f} to visit the file.  @xref{Windows}.
@item r
Renames the file described on the current line.  You must supply a file
name to rename to, using the minibuffer.
@item v
Views the file described on this line using @kbd{M-x view-file}.  Viewing a
file is like visiting it, but is slanted toward moving around in the file
conveniently and does not allow changing the file.  @xref{Misc File
Ops,View File}.  Viewing a file that is a directory runs Dired on that
directory.@refill
@end table

@node Misc File Ops,, Dired, Files
@section Miscellaneous File Operations

  Emacs has commands for performing many other operations on files.
All operate on one file; they do not accept wildcard file names.

@findex add-name-to-file
  You can use the command @kbd{M-x add-name-to-file} to add a name to an
existing file without removing the old name.  The new name must belong
on the file system that the file is on.

@page
@findex append-to-file
  @kbd{M-x append-to-file} adds the text of the region to the end of the
specified file.

@findex copy-file
@cindex copying files
  @kbd{M-x copy-file} reads the file @var{old} and writes a new file
named @var{new} with the same contents.  Confirmation is required if a
file named @var{new} already exists, because copying overwrites the old
contents of the file @var{new}.

@findex delete-file
@cindex deletion (of files)
  @kbd{M-x delete-file} deletes a specified file, like the @code{rm}
command in the shell.  If you are deleting many files in one directory, it
may be more convenient to use Dired (@pxref{Dired}).

@findex insert-file
  @kbd{M-x insert-file} inserts a copy of the contents of a specified
file into the current buffer at point, leaving point unchanged before the
contents and the mark after them.  @xref{Mark}.

@findex make-symbolic-link
  @kbd{M-x make-symbolic-link} reads two file names @var{old} and
@var{linkname}, and then creates a symbolic link named @var{linkname}
and pointing at @var{old}.  The future attempts to open file
@var{linkname} will then refer to the file named @var{old} at the time
the opening is done, or will result in an error if the name @var{old} is
not in use at that time.  Confirmation is required if you create the
link while @var{linkname} is in use.  Note that not all systems support
symbolic links.

@findex rename-file
  @kbd{M-x rename-file} reads two file names @var{old} and @var{new} using
the minibuffer, then renames file @var{old} as @var{new}.  If a file named
@var{new} already exists, you must confirm with @kbd{yes} or renaming is not
done; this is because renaming causes the old meaning of the name @var{new}
to be lost.  If @var{old} and @var{new} are on different file systems, the
file @var{old} is copied and deleted.

@findex view-file
@cindex viewing
  @kbd{M-x view-file} allows you to scan or read a file by sequential
screenfuls.  It reads a file name argument using the minibuffer.  After
reading the file into an Emacs buffer, @code{view-file} reads and displays
one windowful.  You can then type @key{SPC} to scroll forward one window,
or @key{DEL} to scroll backward.  Various other commands are provided for
moving around in the file, but none for changing it; type @kbd{C-h} while
viewing a file for a list of them.  Most commands are the default Emacs
cursor motion commands.  To exit from viewing, type @kbd{C-c}.

@node Buffers, Windows, Files, Top
@chapter Using Multiple Buffers

@cindex buffers
  Text you are editing in Emacs resides in an object called a
@dfn{buffer}.  Each time you visit a file, Emacs creates a buffer to
hold the file's text.  Each time you invoke Dired, Emacs creates a buffer
to hold the directory listing.  If you send a message with @kbd{C-x m},
a buffer named @samp{*mail*} is used to hold the text of the message.
When you ask for a command's documentation, it appears in a buffer
called @samp{*Help*}.

@cindex selected buffer
@cindex current buffer
  At any time, one and only one buffer is @dfn{selected}.  It is also
called the @dfn{current buffer}.  Saying a command operates on ``the
buffer'' really means that the command operates on the selected
buffer, as most commands do.

  When Emacs creates multiple windows, each window has a chosen buffer which
is displayed there, but at any time only one of the windows is selected and
its chosen buffer is the selected buffer.  Each window's mode line displays
the name of the buffer the window is displaying (@pxref{Windows}).

  Each buffer has a name which can be of any length but is
case-sensitive.  You can select a buffer using its name.  Most
buffers are created when you visit files; their names are derived from
the files' names.  You can also create an empty buffer with any name you
want.  A newly started Emacs has a buffer named @samp{*scratch*} which
you can use for evaluating Lisp expressions in Emacs.

  Each buffer records what file it is visiting, whether it is
modified, and what major mode and minor modes are in effect in it
(@pxref{Major Modes}).  Any Emacs variable can be made @dfn{local to} a
particular buffer, meaning its value in that buffer can be different from
the value in other buffers.  @xref{Locals}.

@menu
* Select Buffer::   Creating a new buffer or reselecting an old one.
* List Buffers::    Getting a list of buffers that exist.
* Misc Buffer::     Renaming; changing read-onliness; copying text.
* Kill Buffer::     Killing buffers you no longer need.
* Several Buffers:: How to go through the list of all buffers
                     and operate variously on several of them.
@end menu

@node Select Buffer, List Buffers, Buffers, Buffers
@section Creating and Selecting Buffers
@cindex changing buffers
@cindex switching buffers

@table @kbd
@item C-x b @var{buffer} @key{RET}
Select or create a buffer named @var{buffer} (@code{switch-to-buffer}).
@item C-x 4 b @var{buffer} @key{RET}
Similar but select a buffer named @var{buffer} in another window
(@code{switch-to-buffer-other-window}).
@item M-x switch-to-other-buffer @var{n}
Switch to the previous buffer.
@end table

@kindex C-x 4 b
@findex switch-to-buffer-other-window
@kindex C-x b
@findex switch-to-buffer
@findex switch-to-buffer-new-screen
  To select a buffer named @var{bufname}, type @kbd{C-x b @var{bufname}
@key{RET}}.  This is the command @code{switch-to-buffer} with argument
@var{bufname}.  You can use completion on an abbreviation for the buffer
name you want (@pxref{Completion}).  An empty argument to @kbd{C-x b}
specifies the most recently selected buffer that is not displayed in any
window.@refill

  Most buffers are created when you visit files, or use Emacs commands
that display text.  You can also create a buffer explicitly by typing
@kbd{C-x b @var{bufname} @key{RET}}, which creates a new, empty buffer
that is not visiting any file, and selects it for editing.  The new
buffer's major mode is determined by the value of
@code{default-major-mode} (@pxref{Major Modes}).  Buffers not visiting
files are usually used for making notes to yourself.  If you try to save
one, you are asked for the file name to use.

 The function @code{switch-to-bufer-new-screen} is similar to
@code{switch-to-buffer} except that it creates a new screen in which to
display the selected buffer.

@findex switch-to-other-buffer
Use @kbd{M-x switch-to-other-buffer} to visit the previous buffer. If
you supply a positive integer @var{n}, the @var{n}th most recent buffer
is displayed. If you supply an argument of 0, the current buffer is
moved to the bottom of the buffer stack.

  Note that you can also use @kbd{C-x C-f} and any other command for
visiting a file to switch buffers.  @xref{Visiting}.

@node List Buffers, Misc Buffer, Select Buffer, Buffers
@section Listing Existing Buffers

@table @kbd
@item C-x C-b
List the existing buffers (@code{list-buffers}).
@end table

@kindex C-x C-b
@findex list-buffers
  To print a list of all existing buffers, type @kbd{C-x C-b}.  Each
line in the list shows one buffer's name, major mode and visited file.
@samp{*} at the beginning of a line indicates the buffer has been
``modified''.  If several buffers are modified, it may be time to save
some with @kbd{C-x s} (@pxref{Saving}).  @samp{%} indicates a read-only
buffer.  @samp{.} marks the selected buffer.  Here is an example of a
buffer list:@refill

@smallexample
 MR Buffer         Size  Mode           File
 -- ------         ----  ----           ----
.*  emacs.tex      383402 Texinfo       /u2/emacs/man/emacs.tex
    *Help*         1287  Fundamental	
    files.el       23076 Emacs-Lisp     /u2/emacs/lisp/files.el
  % RMAIL          64042 RMAIL          /u/rms/RMAIL
 *% man            747   Dired		/u2/emacs/man/
    net.emacs      343885 Fundamental   /u/rms/net.emacs
    fileio.c       27691 C              /u2/emacs/src/fileio.c
    NEWS           67340 Text           /u2/emacs/etc/NEWS
    *scratch*	   0	 Lisp Interaction
@end smallexample

@noindent
Note that the buffer @samp{*Help*} was made by a help request; it is not
visiting any file.  The buffer @code{man} was made by Dired on the
directory @file{/u2/emacs/man/}.

As you move the mouse over the @samp{*Buffer List*} buffer, the lines
are highlighted.  This visual cue indicates that clicking the right
mouse button (@code{button3}) will pop up a menu of commands on the
buffer represented by this line.  This menu duplicates most of those
commands which are bound to keys in the @samp{*Buffer List*} buffer.

@node Misc Buffer, Kill Buffer, List Buffers, Buffers
@section Miscellaneous Buffer Operations

@table @kbd
@item C-x C-q
Toggle read-only status of buffer (@code{toggle-read-only}).
@item M-x rename-buffer
Change the name of the current buffer.
@item M-x view-buffer
Scroll through a buffer.
@end table

@cindex read-only buffer
@kindex C-x C-q
@findex toggle-read-only
@vindex buffer-read-only
  A buffer can be @dfn{read-only}, which means that commands to change
its text are not allowed.  Normally, read-only buffers are created by
subsystems such as Dired and Rmail that have special commands to operate
on the text.  Emacs also creates a read-only buffer if you
visit a file that is protected.  To make changes in a read-only buffer,
use the command @kbd{C-x C-q} (@code{toggle-read-only}).  It makes a
read-only buffer writable, and makes a writable buffer read-only.  This
works by setting the variable @code{buffer-read-only}, which has a local
value in each buffer and makes a buffer read-only if its value is
non-@code{nil}.

@findex rename-buffer
  @kbd{M-x rename-buffer} changes the name of the current buffer,
prompting for the new name in the minibuffer.  There is no default.  If you
specify a name that is used by a different buffer, an error is signalled and
renaming is not done.

@findex view-buffer
  @kbd{M-x view-buffer} is similar to @kbd{M-x view-file} (@pxref{Misc
File Ops}) but examines an already existing Emacs buffer.  View mode
provides commands for scrolling through the buffer conveniently but not
for changing it.  When you exit View mode, the resulting value of point
remains in effect.

To copy text from one buffer to another, use the commands @kbd{C-x a}
(@code{append-to-buffer}) and @kbd{M-x insert-buffer}.  @xref{Accumulating
Text}.@refill

@node Kill Buffer, Several Buffers, Misc Buffer, Buffers
@section Killing Buffers

  After using Emacs for a while, you may accumulate a large number of
buffers, and may want to eliminate the ones you no
longer need.  There are several commands for doing this.

@page
@c WideCommands
@table @kbd
@item C-x k
Kill a buffer, specified by name (@code{kill-buffer}).
@item M-x kill-some-buffers
Offer to kill each buffer, one by one.
@end table

@findex kill-buffer
@findex kill-some-buffers
@kindex C-x k
 
  @kbd{C-x k} (@code{kill-buffer}) kills one buffer, whose name you
specify in the minibuffer.  If you type just @key{RET} in the
minibuffer, the default, killing the current buffer, is used.  If the
current buffer is killed, the buffer that has been selected recently but
does not appear in any window now is selected.  If the buffer being
killed contains unsaved changes you are asked to confirm with @kbd{yes}
before the buffer is killed.

  The command @kbd{M-x kill-some-buffers} asks about each buffer, one by
one.  An answer of @kbd{y} means to kill the buffer.  Killing the current
buffer or a buffer containing unsaved changes selects a new buffer or asks
for confirmation just like @code{kill-buffer}.

@node Several Buffers,, Kill Buffer, Buffers
@section Operating on Several Buffers
@cindex buffer menu

  The @dfn{buffer-menu} facility is like a ``Dired for buffers''; it allows
you to request operations on various Emacs buffers by editing a
buffer containing a list of them.  You can save buffers, kill them
(here called @dfn{deleting} them, for consistency with Dired), or display
them.

@table @kbd
@item M-x buffer-menu
Begin editing a buffer listing all Emacs buffers.
@end table

@findex buffer-menu
  The command @code{buffer-menu} writes a list of all Emacs buffers into
the buffer @samp{*Buffer List*}, and selects that buffer in Buffer Menu
mode.  The buffer is read-only.  You can only change it using the special
commands described in this section.  Most of the commands are graphic
characters.  You can use  Emacs cursor motion commands in the
@samp{*Buffer List*} buffer.  If the cursor is on a line describing a
buffer, the following  special commands apply to that buffer:

@table @kbd
@item d
Request to delete (kill) the buffer, then move down.  A @samp{D} before
the buffer name on a line indicates a deletion request.  Requested
deletions actually take place when you use the @kbd{x} command.
@item k
Synonym for @kbd{d}.
@item C-d
Like @kbd{d} but move up afterwards instead of down.
@item s
Request to save the buffer.  An @samp{S} befor the buffer name on a line
indicates the request.  Requested saves actually take place when you use
the @kbd{x} command.  You can request both saving and deletion for the
same buffer.
@item ~
Mark buffer ``unmodified''.  The command @kbd{~} does this
immediately when typed.
@item x
Perform previously requested deletions and saves.
@item u
Remove any request made for the current line, and move down.
@item @key{DEL}
Move to previous line and remove any request made for that line.
@end table

  All commands that add or remove flags to request later operations
also move down a line.  They accept a numeric argument as a repeat count,
unless otherwise specified.

  There are also special commands to use the buffer list to select another
buffer, and to specify one or more other buffers for display in additional
windows.

@table @kbd
@item 1
Select the buffer in a full-screen window.  This command takes effect
immediately.
@item 2
Immediately set up two windows, with this buffer in one, and the
buffer selected before @samp{*Buffer List*} in the other.
@item f
Immediately select the buffer in place of the @samp{*Buffer List*} buffer.
@item o
Immediately select the buffer in another window as if by @kbd{C-x 4 b},
leaving @samp{*Buffer List*} visible.
@item q
Immediately select this buffer, and display any buffers previously
flagged with the @kbd{m} command in other windows.  If there are no 
buffers flagged with @kbd{m}, this command is equivalent to @kbd{1}.
@item m
Flag this buffer to be displayed in another window if the @kbd{q}
command is used.  The request shows as a @samp{>} at the beginning of
the line.  The same buffer may not have both a delete request and a
display request.
@end table

 Going back between a @code{buffer-menu} buffer and other Emacs buffers is
easy.  You can, for example, switch from the @samp{*Buffer List*}
buffer to another Emacs buffer, and edit there.  You can then reselect the
@code{buffer-menu} buffer, and perform operations already
requested, or you can kill that buffer, or pay no further attention to it.
 All that @code{buffer-menu} does directly is create and select a
suitable buffer, and turn on Buffer Menu mode.  All the other
capabilities of the buffer menu are implemented by special commands
provided in Buffer Menu mode.  

  The only difference between @code{buffer-menu} and @code{list-buffers} is
that @code{buffer-menu} selects the @samp{*Buffer List*} buffer and
@code{list-buffers} does not.  If you run @code{list-buffers} (that is,
type @kbd{C-x C-b}) and select the buffer list manually, you can use all
the commands described here.

@node Windows, Major Modes, Buffers, Top
@chapter Multiple Windows
@cindex windows

  Emacs can split the screen into two or many windows, which can display
parts of different buffers, or different parts of one buffer.  If you are
running Lucid GNU Emacs under X, that means you can have the X window
that contains the Emacs screen have multiple subwindows. 

@menu
* Basic Window::     Introduction to Emacs windows.
* Split Window::     New windows are made by splitting existing windows.
* Other Window::     Moving to another window or doing something to it.
* Pop Up Window::    Finding a file or buffer in another window.
* Change Window::    Deleting windows and changing their sizes.
@end menu

@node Basic Window, Split Window, Windows, Windows
@section Concepts of Emacs Windows

  When Emacs displays multiple windows, each window has one Emacs
buffer designated for display.  The same buffer may appear in more
than one window; if it does, any changes in its text are displayed in all
the windows that display it.  Windows showing the same buffer can
show different parts of it, because each window has its own value of point.

@cindex selected window
  At any time, one  windows is the @dfn{selected window}; the buffer
 displayed by that window is the current buffer.  The cursor
shows the location of point in that window.  Each other window has a
location of point as well, but since the terminal has only one cursor it
cannot show the location of point in the other windows. 

  Commands to move point affect the value of point for the selected Emacs
window only.  They do not change the value of point in any other Emacs
window, including those showing the same buffer.  The same is true for commands
such as @kbd{C-x b} to change the selected buffer in the selected window;
they do not affect other windows at all.  However, there are other commands
such as @kbd{C-x 4 b} that select a different window and switch buffers in
it.  Also, all commands that display information in a window, including
(for example) @kbd{C-h f} (@code{describe-function}) and @kbd{C-x C-b}
(@code{list-buffers}), work by switching buffers in a non-selected window
without affecting the selected window.

  Each window has its own mode line, which displays the buffer name,
modification status and major and minor modes of the buffer that is
displayed in the window.  @xref{Mode Line}, for details on the mode
line.

@node Split Window, Other Window, Basic Window, Windows
@section Splitting Windows

@table @kbd
@item C-x 2
Split the selected window into two windows, one above the other
(@code{split-window-vertically}).
@item C-x 5
Split the selected window into two windows positioned side by side
(@code{split-window-horizontally}).

In Lucid GNU Emacs, horizontal window splitting is not implemented.  In
this version, @kbd{C-x 5} creates a new screen.
@item C-x 6
Save the current window configuration in register @var{reg} (a letter).
@item C-x 7
Restore (make current) the window configuration in register
@var{reg} (a letter).  Use with a register previously set with @kbd{C-x 6}.
@end table

@kindex C-x 2
@findex split-window-vertically
  The command @kbd{C-x 2} (@code{split-window-vertically}) breaks the
selected window into two windows, one above the other.  Both windows
start out displaying the same buffer, with the same value of point.  By
default each of the two windows gets half the height of the window that
was split.  A numeric argument specifies how many lines to give to the
top window.

@kindex C-x 5
@findex split-window-horizontally
  @kbd{C-x 5} (@code{split-window-horizontally}) breaks the selected
window into two side-by-side windows.  A numeric argument specifies how
many columns to give the one on the left.  A line of vertical bars
separates the two windows.  Windows that are not the full width of the
screen have truncated mode lines which do not always appear in inverse
video, because Emacs display routines cannot display a region of inverse
video that is only part of a line on the screen.

@vindex truncate-partial-width-windows
  When a window is less than the full width, many text lines are too
long to fit.  Continuing all those lines might be confusing.  Set the
variable @code{truncate-partial-width-windows} to non-@code{nil} to
force truncation in all windows less than the full width of the screen,
independent of the buffer and its value for @code{truncate-lines}.
@xref{Continuation Lines}.@refill

  Horizontal scrolling is often used in side-by-side windows.
@xref{Display}.

@findex window-config-to-register
@findex register-to-window-config
You can resize a window and store that configuration in a register by
supplying a @var{register} argument to @code{register-to-window-config}
(@kbd{C-x 6}). To return to the window configuration established with
(@code{window-config-to-register}, use @code{register-to-window-config}
(@kbd{C-x 7}).

@node Other Window, Pop Up Window, Split Window, Windows
@section Using Other Windows

@table @kbd
@item C-x o
Select another window (@code{other-window}).  That is @kbd{o}, not zero.
@item C-M-v
Scroll the next window (@code{scroll-other-window}).
@page
@item M-x compare-windows
Find the next place where the text in the selected window does not match
the text in the next window.
@item M-x other-window-any-screen @var{n}
Select the @var{n}th different window on any screen.
@end table

@kindex C-x o
@findex other-window
  To select a different window, use @kbd{C-x o} (@code{other-window}).
That is an @kbd{o}, for `other', not a zero.  When there are more than
two windows, the command moves through all the windows in a cyclic
order, generally top to bottom and left to right.  From the rightmost
and bottommost window, it goes back to the one at the upper left corner.
A numeric argument, @var{n}, moves several steps in the cyclic order of
windows. A negative numeric argument moves around the cycle in the
opposite order.  If the optional second argument @var{all_screens} is
non-@code{nil}, the function cycles through all screens.  When the
minibuffer is active, the minibuffer is the last window in the cycle;
you can switch from the minibuffer window to one of the other windows,
and later switch back and finish supplying the minibuffer argument that
is requested.  @xref{Minibuffer Edit}.

@findex other-window-any-screen
 The command @kbd{M-x other-window-any-screen} also selects the window
@var{n} steps away in the cyclic order.  However, unlike @code{other-window},
this command selects a window on the next or previous screen instead of
wrapping around to the top or bottom of the current screen, when there
are no more windows.

@kindex C-M-v
@findex scroll-other-window
  The usual scrolling commands (@pxref{Display}) apply to the selected
window only.  @kbd{C-M-v} (@code{scroll-other-window}) scrolls the
window that @kbd{C-x o} would select.  Like @kbd{C-v}, it takes positive
and negative arguments. 

@findex compare-windows
  The command @kbd{M-x compare-windows} compares the text in the current
window with the text in the next window.  Comparison starts at point in each
window.  Point moves forward in each window, a character at a time,
until the next set of characters in the two windows are different.  Then the
command is finished.

A prefix argument @var{ignore-whitespace} means ignore changes in
whitespace.  The variable @code{compare-windows-whitespace} controls how
whitespace is skipped.

If @code{compare-ignore-case} is non-@code{nil}, changes in case are
also ignored.

@node Pop Up Window, Change Window, Other Window, Windows
@section Displaying in Another Window

@kindex C-x 4
  @kbd{C-x 4} is a prefix key for commands that select another window
(splitting the window if there is only one) and select a buffer in that
window.  Different @kbd{C-x 4} commands have different ways of finding the
buffer to select.

@findex switch-to-buffer-other-window
@findex find-file-other-window
@findex find-tag-other-window
@findex dired-other-window
@findex mail-other-window
@table @kbd
@item C-x 4 b @var{bufname} @key{RET}
Select buffer @var{bufname} in another window.  This runs 
@code{switch-to-buffer-other-window}.
@item C-x 4 f @var{filename} @key{RET}
Visit file @var{filename} and select its buffer in another window.  This
runs @code{find-file-other-window}.  @xref{Visiting}.
@item C-x 4 d @var{directory} @key{RET}
Select a Dired buffer for directory @var{directory} in another window.
This runs @code{dired-other-window}.  @xref{Dired}.
@item C-x 4 m
Start composing a mail message in another window.  This runs
@code{mail-other-window}, and its same-window version is @kbd{C-x m}
(@pxref{Sending Mail}).
@item C-x 4 .
Find a tag in the current tag table in another window.  This runs
@code{find-tag-other-window}, the multiple-window variant of @kbd{M-.}
(@pxref{Tags}).
@end table

@vindex display-buffer-function
If the variable @code{display-buffer-function} is non-@code{nil}, it is
the function to call to handle display-buffer. It receives two
arguments, the buffer and a flag, that if non-@code{nil}, means that the
currently selected window is not acceptable. Commands, such as
@code{switch-to-buffer-other-window} and @code{find-file-other-window}
work using this function.

@node Change Window,, Pop Up Window, Windows
@section Deleting and Rearranging Windows

@table @kbd
@item C-x 0
Get rid of the selected window (@code{delete-window}).  That is a zero. 
If there is more than one Emacs screen, deleting the sole remaining
window on that screen deletes the screen as well. If the current screen
is the only screen, it is not deleted. 
@item C-x 1
Get rid of all windows except the selected one
(@code{delete-other-windows}).
@item C-x ^
Make the selected window taller, at the expense of the other(s)
@*(@code{enlarge-window}).
@item C-x @}
Make the selected window wider (@code{enlarge-window-horizontally}).
@end table

@kindex C-x 0
@findex delete-window
  To delete a window, type @kbd{C-x 0} (@code{delete-window}).  (That is a
zero.)  The space occupied by the deleted window is distributed among the
other active windows (but not the minibuffer window, even if that is active
at the time).  Once a window is deleted, its attributes are forgotten;
there is no automatic way to make another window of the same shape or
showing the same buffer.  The buffer continues to exist, and you can
select it in any window with @kbd{C-x b}.

@kindex C-x 1
@findex delete-other-windows
  @kbd{C-x 1} (@code{delete-other-windows}) is more powerful than @kbd{C-x 0};
it deletes all the windows except the selected one (and the minibuffer).
The selected window expands to use the whole screen except for the echo
area.

@kindex C-x ^
@findex enlarge-window
@kindex C-x @}
@findex enlarge-window-horizontally
@vindex window-min-height
@vindex window-min-width
  To readjust the division of space among existing windows, use @kbd{C-x
^} (@code{enlarge-window}).  It makes the currently selected window
longer by one line or as many lines as a numeric argument specifies.
With a negative argument, it makes the selected window smaller.
@kbd{C-x @}} (@code{enlarge-window-horizontally}) makes the selected
window wider by the specified number of columns.  The extra screen space
given to a window comes from one of its neighbors, if that is possible;
otherwise, all the competing windows are shrunk in the same proportion.
If this makes some windows too small, those windows are deleted and their
space is divided up.   Minimum window size is specified by the variables
@code{window-min-height} and @code{window-min-width}.

@node Major Modes, Indentation, Windows, Top
@chapter Major Modes
@cindex major modes
@kindex TAB
@kindex DEL
@kindex LFD

  Emacs has many different @dfn{major modes}, each of which customizes
Emacs for editing text of a particular sort.  The major modes are mutually
exclusive;  at any time, each buffer has one major mode.  The mode line
normally contains the name of the current major mode in parentheses.
@xref{Mode Line}.

  The least specialized major mode is called @dfn{Fundamental mode}.  This
mode has no mode-specific redefinitions or variable settings.  Each
Emacs command behaves in its most general manner, and each option is in its
default state.  For editing any specific type of text, such as Lisp code or
English text, you should switch to the appropriate major mode, such as Lisp
mode or Text mode.

  Selecting a major mode changes the meanings of a few keys to become
more specifically adapted to the language being edited.  @key{TAB},
@key{DEL}, and @key{LFD} are changed frequently.  In addition, commands
which handle comments use the mode to determine how to delimit comments.
Many major modes redefine the syntactical properties of characters
appearing in the buffer.  @xref{Syntax}.

  The major modes fall into three major groups.  Lisp mode (which has
several variants), C mode, and Muddle mode are for specific programming
languages.  Text mode, Nroff mode, @TeX{} mode and Outline mode are for
editing English text.  The remaining major modes are not intended for use
on users' files; they are used in buffers created by Emacs for specific
purposes and include Dired mode for buffers made by Dired (@pxref{Dired}),
Mail mode for buffers made by @kbd{C-x m} (@pxref{Sending Mail}), and Shell
mode for buffers used for communicating with an inferior shell process
(@pxref{Interactive Shell}).

  Most programming language major modes specify that only blank lines
separate paragraphs.  This is so that the paragraph commands remain useful.
@xref{Paragraphs}.  They also cause Auto Fill mode to use the definition of
@key{TAB} to indent the new lines it creates.  This is because most lines
in a program are usually indented.  @xref{Indentation}.

@menu
* Choosing Modes::     How major modes are specified or chosen.
@end menu

@node Choosing Modes,,Major Modes,Major Modes
@section Choosing Major Modes

  You can select a major mode explicitly for the current buffer, but
most of the time Emacs determines which mode to use based on the file
name or some text in the file.

  Use a @kbd{M-x} command to explicitly select a new major mode.  Add
@code{-mode} to the name of a major mode to get the name of a command to
select that mode.  For example, to enter Lisp mode, execute @kbd{M-x
lisp-mode}.

@vindex auto-mode-alist
  When you visit a file, Emacs usually chooses the right major mode
based on the file's name.  For example, files whose names end in
@code{.c} are edited in C mode.  The variable @code{auto-mode-alist}
controls the correspondence between file names and major mode.  Its value
is a list in which each element has the form

@example
(@var{regexp} . @var{mode-function})
@end example

@noindent
For example, one element normally found in the list has the form
@code{(@t{"\\.c$"} . c-mode)}. It is responsible for selecting C mode
for files whose names end in @file{.c}.  (Note that @samp{\\} is needed in
Lisp syntax to include a @samp{\} in the string, which is needed to
suppress the special meaning of @samp{.} in regexps.)  The only practical
way to change this variable is with Lisp code.

  You can specify which major mode should be used for editing a certain
file by a special sort of text in the first non-blank line of the file.
The mode name should appear in this line both preceded and followed by
@samp{-*-}.  Other text may appear on the line as well.  For example,

@example
;-*-Lisp-*-
@end example

@noindent
tells Emacs to use Lisp mode.  Note how the semicolon is used to make Lisp
treat this line as a comment.  Such an explicit specification overrides any
default mode based on the file name.

  Another format of mode specification is

@example
-*-Mode: @var{modename};-*-
@end example

@noindent
which allows other things besides the major mode name to be specified.
However, Emacs does not look for anything except the mode name.

The major mode can also be specified in a local variables list.
@xref{File Variables}.

@vindex default-major-mode
  When you visit a file that does not specify a major mode to use, or
when you create a new buffer with @kbd{C-x b}, Emacs uses the major mode
specified by the variable @code{default-major-mode}.  Normally this
value is the symbol @code{fundamental-mode}, which specifies Fundamental
mode.  If @code{default-major-mode} is @code{nil}, the major mode is
taken from the previously selected buffer.

@node Indentation, Text, Major Modes, Top
@chapter Indentation
@cindex indentation

@c WideCommands
@table @kbd
@item @key{TAB}
Indent current line ``appropriately'' in a mode-dependent fashion.
@item @key{LFD}
Perform @key{RET} followed by @key{TAB} (@code{newline-and-indent}).
@item M-^
Merge two lines (@code{delete-indentation}).  This would cancel out
the effect of @key{LFD}.
@item C-M-o
Split line at point; text on the line after point becomes a new line
indented to the same column that it now starts in (@code{split-line}).
@item M-m
Move (forward or back) to the first non-blank character on the current
line (@code{back-to-indentation}).
@item C-M-\
Indent several lines to same column (@code{indent-region}).
@item C-x @key{TAB}
Shift block of lines rigidly right or left (@code{indent-rigidly}).
@item M-i
Indent from point to the next prespecified tab stop column
(@code{tab-to-tab-stop}).
@item M-x indent-relative
Indent from point to under an indentation point in the previous line.
@end table

@kindex TAB
@cindex indentation
  Most programming languages have some indentation convention.  For Lisp
code, lines are indented according to their nesting in parentheses.  The
same general idea is used for C code, though details differ.

   Use the @key{TAB} command to indent a line whatever the language,
Each major mode defines this command to perform indentation appropriate
for the particular language.  In Lisp mode, @key{TAB} aligns a line
according to its depth in parentheses.  No matter where in the line you
are when you type @key{TAB}, it aligns the line as a whole.  In C mode,
@key{TAB} implements a subtle and sophisticated indentation style that
knows about many aspects of C syntax.

@kindex TAB
  In Text mode, @key{TAB} runs the command @code{tab-to-tab-stop}, which
indents to the next tab stop column.  You can set the tab stops with
@kbd{M-x edit-tab-stops}.

@menu
* Indentation Commands:: Various commands and techniques for indentation.
* Tab Stops::            You can set arbitrary "tab stops" and then
                         indent to the next tab stop when you want to.
* Just Spaces::          You can request indentation using just spaces.
@end menu

@node Indentation Commands, Tab Stops, Indentation, Indentation
@section Indentation Commands and Techniques
@c ??? Explain what Emacs has instead of space-indent-flag.

  If you just want to insert a tab character in the buffer, you can type
@kbd{C-q @key{TAB}}.

@kindex M-m
@findex back-to-indentation
  To move over the indentation on a line, type @kbd{Meta-m}
(@code{back-to-indentation}).  This command, given anywhere on a line,
positions point at the first non-blank character on the line.

  To insert an indented line before the current line, type @kbd{C-a C-o
@key{TAB}}.  To make an indented line after the current line, use
@kbd{C-e @key{LFD}}.

@kindex C-M-o
@findex split-line
  @kbd{C-M-o} (@code{split-line}) moves the text from point to the end of
the line vertically down, so that the current line becomes two lines.
@kbd{C-M-o} first moves point forward over any spaces and tabs.  Then it
inserts after point a newline and enough indentation to reach the same
column point is on.  Point remains before the inserted newline; in this
regard, @kbd{C-M-o} resembles @kbd{C-o}.

@kindex M-\
@kindex M-^
@findex delete-horizontal-space
@findex delete-indentation
  To join two lines cleanly, use the @kbd{Meta-^}
(@code{delete-indentation}) command to delete the indentation at the
front of the current line, and the line boundary as well.  Empty spaces
are replaced by a single space, or by no space if at the beginning of a
line, before a @samp{)}, or after a @samp{(}.  To delete just the
indentation of a line, go to the beginning of the line and use
@kbd{Meta-\} (@code{delete-horizontal-space}), which deletes all spaces
and tabs around the cursor.

@kindex C-M-\
@kindex C-x TAB
@findex indent-region
@findex indent-rigidly
  There are also commands for changing the indentation of several lines at
once.  @kbd{Control-Meta-\} (@code{indent-region}) gives each line which
begins in the region the ``usual'' indentation by invoking @key{TAB} at the
beginning of the line.  A numeric argument specifies the column to indent
to.  Each line is shifted left or right so that its first non-blank
character appears in that column.  @kbd{C-x @key{TAB}}
(@code{indent-rigidly}) moves all the lines in the region right by its
argument (left, for negative arguments).  The whole group of lines moves
rigidly sideways, which is how the command gets its name.@refill

@findex indent-relative
  @kbd{M-x indent-relative} indents at point based on the previous line
(actually, the last non-empty line.)  It inserts whitespace at point, moving
point, until it is underneath an indentation point in the previous line.
An indentation point is the end of a sequence of whitespace or the end of
the line.  If point is farther right than any indentation point in the
previous line, the whitespace before point is deleted and the first
indentation point then applicable is used.  If no indentation point is
applicable even then, @code{tab-to-tab-stop} is run (see next section).

  @code{indent-relative} is the definition of @key{TAB} in Indented Text
mode.  @xref{Text}.

@node Tab Stops, Just Spaces, Indentation Commands, Indentation
@section Tab Stops

@kindex M-i
@findex tab-to-tab-stop
  For typing in tables, you can use Text mode's definition of @key{TAB},
@code{tab-to-tab-stop}.  This command inserts indentation before point,
enough to reach the next tab stop column.  Even if you are not in Text mode,
this function is associated with @kbd{M-i} anyway.

@findex edit-tab-stops
@findex edit-tab-stops-note-changes
@kindex C-c C-c (Edit Tab Stops)
@vindex tab-stop-list
  You can arbitrarily set the tab stops used by @kbd{M-i}.  They are
stored as a list of column-numbers in increasing order in the variable
@code{tab-stop-list}.

  The convenient way to set the tab stops is using @kbd{M-x edit-tab-stops},
which creates and selects a buffer containing a description of the tab stop
settings.  You can edit this buffer to specify different tab stops, and
then type @kbd{C-c C-c} to make those new tab stops take effect.  In the
tab stop buffer, @kbd{C-c C-c} runs the function
@code{edit-tab-stops-note-changes} rather than the default
@code{save-buffer}.  @code{edit-tab-stops} records which buffer was current
when you invoked it, and stores the tab stops in that buffer.  Normally
all buffers share the same tab stops and changing them in one buffer
affects all.  If you make @code{tab-stop-list} local in one
buffer @code{edit-tab-stops} in that buffer edits only the local
settings.

  Here is the text representing ordinary tab stops every eight columns.

@example
        :       :       :       :       :       :
0         1         2         3         4
0123456789012345678901234567890123456789012345678
To install changes, type C-c C-c
@end example

  The first line contains a colon at each tab stop.  The remaining lines
help you see where the colons are and tell you what to do.

  Note that the tab stops that control @code{tab-to-tab-stop} have nothing
to do with displaying tab characters in the buffer.  @xref{Display Vars},
for more information on that.

@node Just Spaces,, Tab Stops, Indentation
@section Tabs vs. Spaces

@vindex indent-tabs-mode
  Emacs normally uses both tabs and spaces to indent lines.  If you prefer,
all indentation can be made from spaces only.  To request this, set
@code{indent-tabs-mode} to @code{nil}.  This is a per-buffer variable;
altering the variable affects only the current buffer, but there is a
default value which you can change as well.  @xref{Locals}.

@findex tabify
@findex untabify
  There are also commands to convert tabs to spaces or vice versa, always
preserving the columns of all non-blank text.  @kbd{M-x tabify} scans the
region for sequences of spaces, and converts sequences of at least three
spaces to tabs if that is possible without changing indentation.  @kbd{M-x
untabify} changes all tabs in the region to corresponding numbers of spaces.

@node Text, Programs, Indentation, Top
@chapter Commands for Human Languages
@cindex text

  The term @dfn{text} has two widespread meanings in our area of the
computer field.  One is data that is a sequence of characters.  In this
sense of the word any file that you edit with Emacs is text.  The other
meaning is more restrictive: a sequence of characters in a human
language for humans to read (possibly after processing by a text
formatter), as opposed to a program or commands for a program.

  Human languages have syntactic and stylistic conventions that editor
commands should support or use to advantage: conventions involving
words, sentences, paragraphs, and capital letters.  This chapter describes
Emacs commands for all these things.  There are also commands for
@dfn{filling}, or rearranging paragraphs into lines of approximately equal
length.  The commands for moving over and killing words, sentences
and paragraphs, while intended primarily for editing text, are also often
useful for editing programs.

  Emacs has several major modes for editing human language text.
If a file contains plain text, use Text mode, which customizes
Emacs in small ways for the syntactic conventions of text.  For text which
contains embedded commands for text formatters, Emacs has other major modes,
each for a particular text formatter.  Thus, for input to @TeX{}, you can
use @TeX{} mode; for input to nroff, Nroff mode.

@menu
* Text Mode::   The major modes for editing text files.
* Nroff Mode::  The major mode for editing input to the formatter nroff.
* TeX Mode::    The major modes for editing input to the formatter TeX.
* Outline Mode::The major mode for editing outlines.
* Words::       Moving over and killing words.
* Sentences::   Moving over and killing sentences.
* Paragraphs::	Moving over paragraphs.
* Pages::	Moving over pages.
* Filling::     Filling or justifying text
* Case::        Changing the case of text
@end menu

@node Text Mode, Words, Text, Text
@section Text Mode

@findex tab-to-tab-stop
@findex edit-tab-stops
@cindex Text mode
@kindex TAB
@findex text-mode
  You should use Text mode---rather than Fundamental or Lisp mode---to
edit files of text in a human language.  Invoke @kbd{M-x text-mode} to
enter Text mode.  In Text mode, @key{TAB} runs the function
@code{tab-to-tab-stop}, which allows you to use arbitrary tab stops set
with @kbd{M-x edit-tab-stops} (@pxref{Tab Stops}).  Features concerned
with comments in programs are turned off unless they are explicitly invoked.
The syntax table is changed so that periods are not considered part of a
word, while apostrophes, backspaces and underlines are.

@findex indented-text-mode
  A similar variant mode is Indented Text mode, intended for editing
text in which most lines are indented.  This mode defines @key{TAB} to
run @code{indent-relative} (@pxref{Indentation}), and makes Auto Fill
indent the lines it creates.  As a result, a line made by Auto Filling,
or by @key{LFD}, is normally indented just like the previous line.  Use
@kbd{M-x indented-text-mode} to select this mode.

@vindex text-mode-hook
  Entering Text mode or Indented Text mode calls the value of the
variable @code{text-mode-hook} with no arguments, if that value exists
and is not @code{nil}.  This value is also called when modes related to
Text mode are entered; this includes Nroff mode, @TeX{} mode, Outline
mode and Mail mode.  Your hook can look at the value of
@code{major-mode} to see which of these modes is actually being entered.

  Two modes similar to Text mode are of use for editing text that is to
be passed through a text formatter before achieving the form in which
humans are to read it.

@menu
* Nroff Mode::  The major mode for editing input to the formatter nroff.
* TeX Mode::    The major modes for editing input to the formatter TeX.


  Another similar mode is used for editing outlines.  It allows you
to view the text at various levels of detail.  You can view either
the outline headings alone or both headings and text; you can also
hide some of the headings at lower levels from view to make the high
level structure more visible.


* Outline Mode::The major mode for editing outlines.
@end menu

@node Nroff Mode, TeX Mode, Text Mode, Text Mode
@subsection Nroff Mode

@cindex nroff
@findex nroff-mode
  Nroff mode is a mode like Text mode but modified to handle nroff
commands present in the text.  Invoke @kbd{M-x nroff-mode} to enter this
mode.  Nroff mode differs from Text mode in only a few ways.  All nroff
command lines are considered paragraph separators, so that filling never
garbles the nroff commands.  Pages are separated by @samp{.bp} commands.
Comments start with backslash-doublequote.  There are also three special
commands that are not available in Text mode:

@findex forward-text-line
@findex backward-text-line
@findex count-text-lines
@kindex M-n
@kindex M-p
@kindex M-?
@table @kbd
@item M-n
Move to the beginning of the next line that isn't an nroff command
(@code{forward-text-line}).  An argument is a repeat count.
@item M-p
Like @kbd{M-n} but move up (@code{backward-text-line}).
@item M-?
Prints in the echo area the number of text lines (lines that are not
nroff commands) in the region (@code{count-text-lines}).
@end table

@findex electric-nroff-mode
  The other feature of Nroff mode is that you can turn on Electric Nroff
newline mode.  This is a minor mode that you can turn on or off with
@kbd{M-x electric-nroff-mode} (@pxref{Minor Modes}).  When the mode is
on and you use @key{RET} to end a line that containing an nroff command
that opens a kind of grouping, Emacs automatically inserts the matching
nroff command to close that grouping on the following line.  For
example, if you are at the beginning of a line and type @kbd{.@: ( b
@key{RET}}, the matching command @samp{.)b} will be inserted on a new
line following point.

@vindex nroff-mode-hook
  Entering Nroff mode calls the value of the variable
@code{text-mode-hook} with no arguments, if that value exists and is not
@code{nil}; then does the same with the variable
@code{nroff-mode-hook}.

@node TeX Mode, Outline Mode, Nroff Mode, Text Mode
@subsection @TeX{} Mode
@cindex TeX
@cindex LaTeX
@findex TeX-mode
@findex tex-mode
@findex plain-tex-mode
@findex LaTeX-mode
@findex plain-TeX-mode
@findex latex-mode

  @TeX{} is a powerful text formatter written by Donald Knuth; like GNU
Emacs, it is free.  La@TeX{} is a simplified input format for @TeX{},
implemented by @TeX{} macros.  It is part of @TeX{}.@refill

  Emacs has a special @TeX{} mode for editing @TeX{} input files.
It provides facilities for checking the balance of delimiters and for
invoking @TeX{} on all or part of the file.

  @TeX{} mode has two variants, Plain @TeX{} mode and La@TeX{} mode,
which are two distinct major modes that differ only slightly.  These
modes are designed for editing the two different input formats.  The
command @kbd{M-x tex-mode} looks at the contents of a buffer to
determine whether it appears to be La@TeX{} input or not; it then
selects the appropriate mode.  If it can't tell which is right (e.g.,
the buffer is empty), the variable @code{TeX-default-mode} controls
which mode is used.

  The commands @kbd{M-x plain-tex-mode} and @kbd{M-x latex-mode}
explicitly select one of the variants of @TeX{} mode.  Use these
commands when @kbd{M-x tex-mode} does not guess right.@refill

@menu
* Editing: TeX Editing.   Special commands for editing in TeX mode.
* Printing: TeX Print.    Commands for printing part of a file with TeX.
@end menu

  @TeX{} for Unix systems can be obtained from the University of Washington
for a distribution fee.

  To order a full distribution, send $140.00 for a 1/2 inch
9-track tape, $165.00 for two 4-track 1/4 inch cartridge tapes
(foreign sites $150.00, for 1/2 inch, $175.00 for 1/4 inch, to cover
the extra postage) payable to the University of Washington to:

@display
The Director
Northwest Computer Support Group,  DW-10
University of Washington
Seattle, Washington 98195
@end display

@noindent
Purchase orders are acceptable, but there is an extra charge of
$10.00, to pay for processing charges. (Total of $150 for domestic
sites, $175 for foreign sites).

  The normal distribution is a tar tape, blocked 20, 1600 bpi, on an
industry standard 2400 foot half-inch reel.  The physical format for
the 1/4 inch streamer cartridges uses QIC-11, 8000 bpi, 4-track
serpentine recording for the SUN.  Also, SystemV tapes can be written
in cpio format, blocked 5120 bytes, ASCII headers.

@node TeX Editing,TeX Print,TeX Mode,TeX Mode
@subsubsection @TeX{} Editing Commands

  Here are the special commands provided in @TeX{} mode for editing the
text of the file.

@table @kbd
@item "
Insert, according to context, either @samp{@`@`} or @samp{"} or
@samp{@'@'} (@code{TeX-insert-quote}).
@item @key{LFD}
Insert a paragraph break (two newlines) and check the previous
paragraph for unbalanced braces or dollar signs
(@code{TeX-terminate-@*paragraph}).
@item M-x validate-TeX-buffer
Check each paragraph in the buffer for unbalanced braces or dollar signs.
@item M-@{
Insert @samp{@{@}} and position point between them (@code{TeX-insert-braces}).
@item M-@}
Move forward past the next unmatched close brace (@code{up-list}).
@item C-c C-f
Close a block for La@TeX{} (@code{TeX-close-LaTeX-block}).
@end table

@findex TeX-insert-quote
@kindex " (TeX mode)
  In @TeX{}, the character @samp{"} is not normally used; you use @samp{``}
to start a quotation and @samp{''} to end one.  @TeX{} mode defines the key
@kbd{"} to insert @samp{``} after whitespace or an open brace, @samp{"}
after a backslash, or @samp{''} otherwise.  This is done by the command
@code{TeX-insert-quote}.  If you need the character @samp{"} itself in
unusual contexts, use @kbd{C-q} to insert it.  Also, @kbd{"} with a
numeric argument always inserts that number of @samp{"} characters.

  In @TeX{} mode, @samp{$} has a special syntax code which attempts to
understand the way @TeX{} math mode delimiters match.  When you insert a
@samp{$} that is meant to exit math mode, the position of the matching
@samp{$} that entered math mode is displayed for a second.  This is the
same feature that displays the open brace that matches a close brace that
is inserted.  However, there is no way to tell whether a @samp{$} enters
math mode or leaves it; so when you insert a @samp{$} that enters math
mode, the previous @samp{$} position is shown as if it were a match, even
though they are actually unrelated.

@findex TeX-insert-braces
@kindex M-@{ (TeX mode)
@findex up-list
@kindex M-@} (TeX mode)
  If you prefer to keep braces balanced at all times, you can use @kbd{M-@{}
(@code{TeX-insert-braces}) to insert a pair of braces.  It leaves point
between the two braces so you can insert the text that belongs inside.
Afterward, use the command @kbd{M-@}} (@code{up-list}) to move forward
past the close brace.

@findex validate-TeX-buffer
@findex TeX-terminate-paragraph
@kindex LFD (TeX mode)
  There are two commands for checking the matching of braces.  @key{LFD}
(@code{TeX-terminate-paragraph}) checks the paragraph before point, and
inserts two newlines to start a new paragraph.  It prints a message in the
echo area if any mismatch is found.  @kbd{M-x validate-TeX-buffer} checks
the entire buffer, paragraph by paragraph.  When it finds a paragraph that
contains a mismatch, it displays point at the beginning of the paragraph
for a few seconds and pushes a mark at that spot.  Scanning continues
until the whole buffer has been checked or until you type another key.
The positions of the last several paragraphs with mismatches can be
found in the mark ring (@pxref{Mark Ring}).

  Note that square brackets and parentheses, not just braces, are
matched in @TeX{} mode.  This is wrong if you want to  check @TeX{} syntax.
However, parentheses and square brackets are likely to be used in text as
matching delimiters and it is useful for the various motion commands and
automatic match display to work with them.

@findex TeX-close-LaTeX-block
@kindex C-c C-f (LaTeX mode)
  In La@TeX{} input, @samp{\begin} and @samp{\end} commands must balance.
After you insert a @samp{\begin}, use @kbd{C-c C-f}
(@code{TeX-close-LaTeX-block}) to insert automatically a matching
@samp{\end} (on a new line following the @samp{\begin}).  A blank line is
inserted between the two, and point is left there.@refill

@node TeX Print,,TeX Editing,TeX Mode
@subsubsection @TeX{} Printing Commands

  You can invoke @TeX{} as an inferior of Emacs on either the entire
contents of the buffer or just a region at a time.  Running @TeX{} in
this way on just one chapter is a good way to see what your changes
look like without taking the time to format the entire file.

@table @kbd
@item C-c C-r
Invoke @TeX{} on the current region, plus the buffer's header
(@code{TeX-region}).
@item C-c C-b
Invoke @TeX{} on the entire current buffer (@code{TeX-buffer}).
@item C-c C-l
Recenter the window showing output from the inferior @TeX{} so that
the last line can be seen (@code{TeX-recenter-output-buffer}).
@item C-c C-k
Kill the inferior @TeX{} (@code{TeX-kill-job}).
@item C-c C-p
Print the output from the last @kbd{C-c C-r} or @kbd{C-c C-b} command
(@code{TeX-print}).
@item C-c C-q
Show the printer queue (@code{TeX-show-print-queue}).
@end table

@findex TeX-buffer
@kindex C-c C-b (TeX mode)
@findex TeX-print
@kindex C-c C-p (TeX mode)
@findex TeX-show-print-queue
@kindex C-c C-q (TeX mode)
  You can pass the current buffer through an inferior @TeX{} using
@kbd{C-c C-b} (@code{TeX-buffer}).  The formatted output appears in a file
in @file{/tmp}; to print it, type @kbd{C-c C-p} (@code{TeX-print}).
Afterward use @kbd{C-c C-q} (@code{TeX-show-print-queue}) to view the
progress of your output towards being printed.

@findex TeX-kill-job
@kindex C-c C-k (TeX mode)
@findex TeX-recenter-output-buffer
@kindex C-c C-l (TeX mode)
  The console output from @TeX{}, including any error messages, appears in a
buffer called @samp{*TeX-shell*}.  If @TeX{} gets an error, you can switch
to this buffer and feed it input (this works as in Shell mode;
@pxref{Interactive Shell}).  Without switching to this buffer, you can scroll
it so that its last line is visible by typing @kbd{C-c C-l}.

  Type @kbd{C-c C-k} (@code{TeX-kill-job}) to kill the @TeX{} process if
you see that its output is no longer useful.  Using @kbd{C-c C-b} or
@kbd{C-c C-r} also kills any @TeX{} process still running.@refill

@findex TeX-region
@kindex C-c C-r (TeX mode)
  You can pass an arbitrary region through an inferior @TeX{} by typing
@kbd{C-c C-r} (@code{TeX-region}).  This is tricky, however, because
most files of @TeX{} input contain commands at the beginning to set
parameters and define macros.  Without them, no later part of the file
will format correctly.  To solve this problem, @kbd{C-c C-r} allows you
to designate a part of the file as containing essential commands; it is
included before the specified region as part of the input to @TeX{}.
The designated part of the file is called the @dfn{header}.

@cindex header (TeX mode)
  To indicate the bounds of the header in Plain @TeX{} mode, insert two
special strings in the file: @samp{%**start of header} before the
header, and @samp{%**end of header} after it.  Each string must appear
entirely on one line, but there may be other text on the line before or
after.  The lines containing the two strings are included in the header.
If @samp{%**start of header} does not appear within the first 100 lines of
the buffer, @kbd{C-c C-r} assumes there is no header.

  In La@TeX{} mode, the header begins with @samp{\documentstyle} and ends
with @*@samp{\begin@{document@}}.  These are commands that La@TeX{} requires
you to use, so you don't need to do anything special to identify the
header.

@vindex TeX-mode-hook
@vindex LaTeX-mode-hook
@vindex plain-TeX-mode-hook
  When you enter either kind of @TeX{} mode, Emacs calls with no
arguments the value of the variable @code{text-mode-hook}, if that value
exists and is not @code{nil}.  Emacs then calls the variable
@code{TeX-mode-hook} and either @code{plain-TeX-mode-hook} or
@code{LaTeX-mode-hook} under the same conditions.

@node Outline Mode,, TeX Mode, Text Mode
@subsection Outline Mode
@cindex outlines
@cindex selective display
@cindex invisible lines

  Outline mode is a major mode similar to Text mode but intended for editing
outlines.  It allows you to make parts of the text temporarily invisible
so that you can see just the overall structure of the outline.  Type
@kbd{M-x outline-mode} to turn on Outline mode in the current buffer.

@vindex outline-mode-hook
  When you enter Outline mode, Emacs calls with no arguments the value
of the variable @code{text-mode-hook}, if that value exists and is not
@code{nil}; then it does the same with the variable
@code{outline-mode-hook}.

  When a line is invisible in outline mode, it does not appear on the
screen.  The screen appears exactly as if the invisible line
were deleted, except that an ellipsis (three periods in a row) appears
at the end of the previous visible line (only one ellipsis no matter
how many invisible lines follow).

  All editing commands treat the text of the invisible line as part of the
previous visible line.  For example, @kbd{C-n} moves onto the next visible
line.  Killing an entire visible line, including its terminating newline,
really kills all the following invisible lines as well; yanking
everything back yanks the invisible lines and they remain invisible.

@menu
* Format: Outline Format.	  What the text of an outline looks like.
* Motion: Outline Motion.	  Special commands for moving through outlines.
* Visibility: Outline Visibility. Commands to control what is visible.
@end menu

@node Outline Format,Outline Motion,Outline Mode, Outline Mode
@subsubsection Format of Outlines

@cindex heading lines (Outline mode)
@cindex body lines (Outline mode)
  Outline mode assumes that the lines in the buffer are of two types:
@dfn{heading lines} and @dfn{body lines}.  A heading line represents a
topic in the outline.  Heading lines start with one or more stars; the
number of stars determines the depth of the heading in the outline
structure.  Thus, a heading line with one star is a major topic; all the
heading lines with two stars between it and the next one-star heading
are its subtopics; and so on.  Any line that is not a heading line is a
body line.  Body lines belong to the preceding heading line.  Here is an
example:

@example
* Food

This is the body,
which says something about the topic of food.

** Delicious Food

This is the body of the second-level header.

** Distasteful Food

This could have
a body too, with
several lines.

*** Dormitory Food

* Shelter

A second first-level topic with its header line.
@end example

  A heading line together with all following body lines is called
collectively an @dfn{entry}.  A heading line together with all following
deeper heading lines and their body lines is called a @dfn{subtree}.

@vindex outline-regexp
 You can customize the criterion for distinguishing heading lines by
setting the variable @code{outline-regexp}.  Any line whose beginning
has a match for this regexp is considered a heading line.  Matches that
start within a line (not at the beginning) do not count.  The length of
the matching text determines the level of the heading; longer matches
make a more deeply nested level.  Thus, for example, if a text formatter
has commands @samp{@@chapter}, @samp{@@section} and @samp{@@subsection}
to divide the document into chapters and sections, you can make those
lines count as heading lines by setting @code{outline-regexp} to
@samp{"@@chap\\|@@\\(sub\\)*section"}.  Note the trick: the two words
@samp{chapter} and @samp{section} are the same length, but by defining
the regexp to match only @samp{chap} we ensure that the length of the
text matched on a chapter heading is shorter, so that Outline mode will
know that sections are contained in chapters.  This works as long as no
other command starts with @samp{@@chap}.

@page
  Outline mode makes a line invisible by changing the newline before it
into an ASCII Control-M (code 015).  Most editing commands that work on
lines treat an invisible line as part of the previous line because,
strictly speaking, it @i{is} part of that line, since there is no longer a
newline in between.  When you save the file in Outline mode, Control-M
characters are saved as newlines, so the invisible lines become ordinary
lines in the file.  Saving does not change the visibility status of a
line inside Emacs.

@node Outline Motion,Outline Visibility,Outline Format,Outline Mode
@subsubsection Outline Motion Commands

   Some special commands in Outline mode move backward and forward to
heading lines.

@table @kbd
@item C-c C-n
Move point to the next visible heading line
(@code{outline-next-visible-heading}).
@item C-c C-p
Move point to the previous visible heading line @*
(@code{outline-previous-visible-heading}).
@item C-c C-f
Move point to the next visible heading line at the same level
as the one point is on (@code{outline-forward-same-level}).
@item C-c C-b
Move point to the previous visible heading line at the same level
(@code{outline-backward-same-level}).
@item C-c C-u
Move point up to a lower-level (more inclusive) visible heading line
(@code{outline-up-heading}).
@end table

@findex outline-next-visible-heading
@findex outline-previous-visible-heading
@kindex C-c C-n (Outline mode)
@kindex C-c C-p (Outline mode)
  @kbd{C-c C-n} (@code{next-visible-heading}) moves down to the next
heading line.  @kbd{C-c C-p} (@code{previous-visible-heading}) moves
similarly backward.  Both accept numeric arguments as repeat counts.  The
names emphasize that invisible headings are skipped, but this is not really
a special feature.  All editing commands that look for lines ignore the
invisible lines automatically.@refill

@findex outline-up-heading
@findex outline-forward-same-level
@findex outline-backward-same-level
@kindex C-c C-f (Outline mode)
@kindex C-c C-b (Outline mode)
@kindex C-c C-u (Outline mode)
  More advanced motion commands understand the levels of headings.
The commands @kbd{C-c C-f} (@code{outline-forward-same-level}) and
@kbd{C-c C-b} (@code{outline-backward-same-level}) move from one
heading line to another visible heading at the same depth in
the outline.  @kbd{C-c C-u} (@code{outline-up-heading}) moves
backward to another heading that is less deeply nested.

@node Outline Visibility,,Outline Motion,Outline Mode
@subsubsection Outline Visibility Commands

  The other special commands of outline mode are used to make lines visible
or invisible.  Their names all start with @code{hide} or @code{show}.
Most of them exist as pairs of opposites.  They are not undoable; instead,
you can undo right past them.  Making lines visible or invisible is simply
not recorded by the undo mechanism.

@table @kbd
@item M-x hide-body
Make all body lines in the buffer invisible.
@item M-x show-all
Make all lines in the buffer visible.
@item C-c C-h
Make everything under this heading invisible, not including this
heading itself (@code{hide-subtree}).
@item C-c C-s
Make everything under this heading visible, including body,
subheadings, and their bodies (@code{show-subtree}).
@item M-x hide-leaves
Make the body of this heading line, and of all its subheadings,
invisible.
@item M-x show-branches
Make all subheadings of this heading line, at all levels, visible.
@item C-c C-i
Make immediate subheadings (one level down) of this heading line
visible (@code{show-children}).
@item M-x hide-entry
Make this heading line's body invisible.
@item M-x show-entry
Make this heading line's body visible.
@end table

@findex hide-entry
@findex show-entry
  Two commands that are exact opposites are @kbd{M-x hide-entry} and
@kbd{M-x show-entry}.  They are used with point on a heading line, and
apply only to the body lines of that heading.  The subtopics and their
bodies are not affected.

@findex hide-subtree
@findex show-subtree
@kindex C-c C-s (Outline mode)
@kindex C-c C-h (Outline mode)
@cindex subtree (Outline mode)
  Two more powerful opposites are @kbd{C-c C-h} (@code{hide-subtree}) and
@kbd{C-c C-s} (@code{show-subtree}).  Both should be used when point is
on a heading line, and both apply to all the lines of that heading's
@dfn{subtree}: its body, all its subheadings, both direct and indirect, and
all of their bodies.  In other words, the subtree contains everything
following this heading line, up to and not including the next heading of
the same or higher rank.@refill

@findex hide-leaves
@findex show-branches
  Intermediate between a visible subtree and an invisible one is having
all the subheadings visible but none of the body.  There are two commands
for doing this, one that hides the bodies and one that
makes the subheadings visible.  They are @kbd{M-x hide-leaves} and
@kbd{M-x show-branches}.

@kindex C-c C-i (Outline mode)
@findex show-children
  A little weaker than @code{show-branches} is @kbd{C-c C-i}
(@code{show-children}).  It makes just the direct subheadings
visible---those one level down.  Deeper subheadings remain
invisible.@refill

@findex hide-body
@findex show-all
  Two commands have a blanket effect on the whole file.  @kbd{M-x
hide-body} makes all body lines invisible, so that you see just the
outline structure.  @kbd{M-x show-all} makes all lines visible.  You can
think of these commands as a pair of opposites even though @kbd{M-x
show-all} applies to more than just body lines.

@vindex selective-display-ellipses
You can turn off the use of ellipses at the ends of visible lines by
setting @code{selective-display-ellipses} to @code{nil}.  The result is
no visible indication of the presence of invisible lines.

@node Words, Sentences, Text Mode, Text
@section Words
@cindex words
@cindex Meta

  Emacs has commands for moving over or operating on words.  By convention,
the keys for them are all @kbd{Meta-} characters.

@c widecommands
@table @kbd
@item M-f
Move forward over a word (@code{forward-word}).
@item M-b
Move backward over a word (@code{backward-word}).
@item M-d
Kill up to the end of a word (@code{kill-word}).
@item M-@key{DEL}
Kill back to the beginning of a word (@code{backward-kill-word}).
@item M-@@
Mark the end of the next word (@code{mark-word}).
@item M-t
Transpose two words;  drag a word forward
or backward across other words (@code{transpose-words}).
@end table

  Notice how these keys form a series that parallels the
character-based @kbd{C-f}, @kbd{C-b}, @kbd{C-d}, @kbd{C-t} and
@key{DEL}.  @kbd{M-@@} is related to @kbd{C-@@}, which is an alias for
@kbd{C-@key{SPC}}.@refill

@kindex M-f
@kindex M-b
@findex forward-word
@findex backward-word
  The commands @kbd{Meta-f} (@code{forward-word}) and @kbd{Meta-b}
(@code{backward-word}) move forward and backward over words.  They are
analogous to @kbd{Control-f} and @kbd{Control-b}, which move over single
characters.  Like their @kbd{Control-} analogues, @kbd{Meta-f} and
@kbd{Meta-b} move several words if given an argument.  @kbd{Meta-f} with a
negative argument moves backward, and @kbd{Meta-b} with a negative argument
moves forward.  Forward motion stop after the last letter of the
word, while backward motion stops before the first letter.@refill

@kindex M-d
@findex kill-word
  @kbd{Meta-d} (@code{kill-word}) kills the word after point.  To be
precise, it kills everything from point to the place @kbd{Meta-f} would
move to.  Thus, if point is in the middle of a word, @kbd{Meta-d} kills
just the part after point.  If some punctuation comes between point and the
next word, it is killed along with the word.  (To kill only the
next word but not the punctuation before it, simply type @kbd{Meta-f} to get
the end, and kill the word backwards with @kbd{Meta-@key{DEL}}.)
@kbd{Meta-d} takes arguments just like @kbd{Meta-f}.

@findex backward-kill-word
@kindex M-DEL
  @kbd{Meta-@key{DEL}} (@code{backward-kill-word}) kills the word before
point.  It kills everything from point back to where @kbd{Meta-b} would
move to.  If point is after the space in @w{@samp{FOO, BAR}}, then
@w{@samp{FOO, }} is killed.   To kill just @samp{FOO}, type
@kbd{Meta-b Meta-d} instead of @kbd{Meta-@key{DEL}}.)

@cindex transposition
@kindex M-t
@findex transpose-words
  @kbd{Meta-t} (@code{transpose-words}) exchanges the word before or
containing point with the following word.  The delimiter characters
between the words do not move.  For example, transposing @w{@samp{FOO,
BAR}} results in @w{@samp{BAR, FOO}} rather than @samp{@w{BAR FOO,}}.
@xref{Transpose}, for more on transposition and on arguments to
transposition commands.

@kindex M-@@
@findex mark-word
  To operate on the next @var{n} words with an operation which applies
between point and mark, you can either set the mark at point and then move
over the words, or you can use the command @kbd{Meta-@@} (@code{mark-word})
which does not move point, but sets the mark where @kbd{Meta-f} would move
to.  It can be given arguments just like @kbd{Meta-f}.

@cindex syntax table
  The word commands' understanding of syntax is completely controlled by
the syntax table.  Any character can, for example, be declared to be a word
delimiter.  @xref{Syntax}.

@node Sentences, Paragraphs, Words, Text
@section Sentences
@cindex sentences

  The Emacs commands for manipulating sentences and paragraphs are mostly
on @kbd{Meta-} keys, and therefore like the word-handling commands.

@table @kbd
@item M-a
Move back to the beginning of the sentence (@code{backward-sentence}).
@item M-e
Move forward to the end of the sentence (@code{forward-sentence}).
@item M-k
Kill forward to the end of the sentence (@code{kill-sentence}).
@item C-x @key{DEL}
Kill back to the beginning of the sentence @*(@code{backward-kill-sentence}).
@end table

@kindex M-a
@kindex M-e
@findex backward-sentence
@findex forward-sentence
  The commands @kbd{Meta-a} and @kbd{Meta-e} (@code{backward-sentence}
and @code{forward-sentence}) move to the beginning and end of the
current sentence, respectively.  They resemble @kbd{Control-a} and
@kbd{Control-e}, which move to the beginning and end of a line.  Unlike
their counterparts, @kbd{Meta-a} and @kbd{Meta-e} move over successive
sentences if repeated or given numeric arguments.  Emacs assumes
the typist's convention is followed, and thus considers a sentence to
end wherever there is a @samp{.}, @samp{?} or @samp{!} followed by the
end of a line or two spaces, with any number of @samp{)}, @samp{]},
@samp{'}, or @samp{"} characters allowed in between.  A sentence also
begins or ends wherever a paragraph begins or ends.@refill

  Neither @kbd{M-a} nor @kbd{M-e} moves past the newline or spaces beyond
the sentence edge at which it is stopping.

@kindex M-k
@kindex C-x DEL
@findex kill-sentence
@findex backward-kill-sentence
 @kbd{M-a} and @kbd{M-e} have a corresponding kill command, just like
@kbd{C-a} and @kbd{C-e} have @kbd{C-k}.  The command is  @kbd{M-k}
(@code{kill-sentence}) which kills from point to the end of the
sentence.  With minus one as an argument it kills back to the beginning
of the sentence.  Larger arguments serve as repeat counts.@refill

  There is a special command, @kbd{C-x @key{DEL}}
(@code{backward-kill-sentence}) for killing back to the beginning of a
sentence, which is useful when you change your mind in the middle of
composing text.@refill

@vindex sentence-end
  The variable @code{sentence-end} controls recognition of the end of a
sentence.  It is a regexp that matches the last few characters of a
sentence, together with the whitespace following the sentence.  Its
normal value is

@example
"[.?!][]\"')]*\\($\\|\t\\|  \\)[ \t\n]*"
@end example

@noindent
This example is explained in the section on regexps.  @xref{Regexps}.

@node Paragraphs, Pages, Sentences, Text
@section Paragraphs
@cindex paragraphs
@kindex M-[
@kindex M-]
@findex backward-paragraph
@findex forward-paragraph

  The Emacs commands for manipulating paragraphs are also @kbd{Meta-}
keys.

@table @kbd
@item M-[
Move back to previous paragraph beginning @*(@code{backward-paragraph}).
@item M-]
Move forward to next paragraph end (@code{forward-paragraph}).
@item M-h
Put point and mark around this or next paragraph (@code{mark-paragraph}).
@end table

  @kbd{Meta-[} moves to the beginning of the current or previous paragraph,
while @kbd{Meta-]} moves to the end of the current or next paragraph.
Blank lines and text formatter command lines separate paragraphs and are
not part of any paragraph.  An indented line starts a new paragraph.

  In major modes for programs (as opposed to Text mode), paragraphs begin
and end only at blank lines.  As a result, the paragraph commands continue to
be useful even though there are no paragraphs per se.

  When there is a fill prefix, paragraphs are delimited by all lines
which don't start with the fill prefix.  @xref{Filling}.

@kindex M-h
@findex mark-paragraph
   To operate on a paragraph, you can use the command
@kbd{Meta-h} (@code{mark-paragraph}) to set the region around it.  This
command puts point at the beginning and mark at the end of the paragraph
point was in.  If point is between paragraphs (in a run of blank lines, or
at a boundary), the paragraph following point is surrounded by point and
mark.  If there are blank lines preceding the first line of the paragraph,
one of the blank lines is included in the region.  Thus, for example,
@kbd{M-h C-w} kills the paragraph around or after point.

@vindex paragraph-start
@vindex paragraph-separate
  The precise definition of a paragraph boundary is controlled by the
variables @code{paragraph-separate} and @code{paragraph-start}.  The value
of @code{paragraph-start} is a regexp that matches any line that
either starts or separates paragraphs.  The value of
@code{paragraph-separate} is another regexp that  matches only lines
that separate paragraphs without being part of any paragraph.  Lines that
start a new paragraph and are contained in it must match both regexps.  For
example, normally @code{paragraph-start} is @code{"^[ @t{\}t@t{\}n@t{\}f]"}
and @code{paragraph-separate} is @code{"^[ @t{\}t@t{\}f]*$"}.@refill

  Normally it is desirable for page boundaries to separate paragraphs.
The default values of these variables recognize the usual separator for
pages.

@node Pages, Filling, Paragraphs, Text
@section Pages

@cindex pages
@cindex formfeed
  Files are often thought of as divided into @dfn{pages} by the
@dfn{formfeed} character (ASCII Control-L, octal code 014).  For
example, if a file is printed on a line printer, each ``page'' of the
file starts on a new page of paper.  Emacs treats a page-separator
character just like any other character.  It can be inserted with
@kbd{C-q C-l}, or deleted with @key{DEL}.  You are free to
paginate your file or not.  However, since pages are often meaningful
divisions of the file, commands are provided to move over them and
operate on them.

@c WideCommands
@table @kbd
@item C-x [
Move point to previous page boundary (@code{backward-page}).
@item C-x ]
Move point to next page boundary (@code{forward-page}).
@item C-x C-p
Put point and mark around this page (or another page) (@code{mark-page}).
@item C-x l
Count the lines in this page (@code{count-lines-page}).
@end table

@kindex C-x [
@kindex C-x ]
@findex forward-page
@findex backward-page
  The @kbd{C-x [} (@code{backward-page}) command moves point to
immediately after the previous page delimiter.  If point is already
right after a page delimiter, the command skips that one and stops at
the previous one.  A numeric argument serves as a repeat count.  The
@kbd{C-x ]} (@code{forward-page}) command moves forward past the next
page delimiter.

@kindex C-x C-p
@findex mark-page
  The @kbd{C-x C-p} command (@code{mark-page}) puts point at the beginning
of the current page and the mark at the end.  The page delimiter at the end
is included (the mark follows it).  The page delimiter at the front is
excluded (point follows it).  You can follow this command  by @kbd{C-w} to
kill a page you want to move elsewhere.  If you insert the page after a page
delimiter, at a place where @kbd{C-x ]} or @kbd{C-x [} would take you,
the page will be properly delimited before and after once again.

  A numeric argument to @kbd{C-x C-p} is used to specify which page to go
to, relative to the current one.  Zero means the current page.  One means
the next page, and @minus{}1 means the previous one.

@kindex C-x l
@findex count-lines-page
  The @kbd{C-x l} command (@code{count-lines-page}) can help you decide
where to break a page in two.  It prints the total number of lines in
the current page in the echo area, then divides the lines into those
preceding the current line and those following it, for example

@example
Page has 96 (72+25) lines
@end example

@noindent
  Notice that the sum is off by one; this is correct if point is not at the
beginning of a line.

@vindex page-delimiter
  The variable @code{page-delimiter} should have as its value a regexp that
matches the beginning of a line that separates pages.  This defines
where pages begin.  The normal value of this variable is @code{"^@t{\}f"},
which matches a formfeed character at the beginning of a line.

@node Filling, Case, Pages, Text
@section Filling Text
@cindex filling

  If you use Auto Fill mode, Emacs @dfn{fills} text (breaks it up into
lines that fit in a specified width) as you insert it.  When you alter
existing text it is often no longer be properly filled afterwards and
you can use explicit commands for filling.

@menu
* Auto Fill::	  Auto Fill mode breaks long lines automatically.
* Fill Commands:: Commands to refill paragraphs and center lines.
* Fill Prefix::   Filling when every line is indented or in a comment, etc.
@end menu

@node Auto Fill, Fill Commands, Filling, Filling
@subsection Auto Fill Mode

@cindex Auto Fill mode

  @dfn{Auto Fill} mode is a minor mode in which lines are broken
automatically when they become too wide.  Breaking happens only when
you type a @key{SPC} or @key{RET}.

@table @kbd
@item M-x auto-fill-mode
Enable or disable Auto Fill mode.
@item @key{SPC}
@itemx @key{RET}
In Auto Fill mode, break lines when appropriate.
@end table

@findex auto-fill-mode
  @kbd{M-x auto-fill-mode} turns Auto Fill mode on if it was off, or off
if it was on.  With a positive numeric argument the command always turns
Auto Fill mode on, and with a negative argument it always turns it off.
The presence of the word @samp{Fill} in the mode line, inside the
parentheses, indicates that Auto Fill mode is in effect.  Auto Fill mode
is a minor mode; you can turn it on or off for each buffer individually.
@xref{Minor Modes}.

  In Auto Fill mode, lines are broken automatically at spaces when they get
longer than desired.  Line breaking and rearrangement takes place
only when you type @key{SPC} or @key{RET}.  To insert a space
or newline without permitting line-breaking, type @kbd{C-q @key{SPC}} or
@kbd{C-q @key{LFD}} (recall that a newline is really a linefeed).
@kbd{C-o} inserts a newline without line breaking.

  Auto Fill mode works well with Lisp mode: when it makes a new line in
Lisp mode it indents that line with @key{TAB}.  If a line ending in a
Lisp comment gets too long, the text of the comment is split into two
comment lines.  Optionally new comment delimiters are inserted at the
end of the first line and the beginning of the second, so that each line
is a separate comment.  The variable @code{comment-multi-line} controls
the choice (@pxref{Comments}).

  Auto Fill mode does not refill entire paragraphs.  It can break lines but
cannot merge lines.  Editing in the middle of a paragraph can result in
a paragraph that is not correctly filled.  The easiest way to make the
paragraph properly filled again is using an explicit fill commands.

  Many users like Auto Fill mode and want to use it in all text files.
The section on init files explains how you can arrange this
permanently for yourself.  @xref{Init File}.

@node Fill Commands, Fill Prefix, Auto Fill, Filling
@subsection Explicit Fill Commands

@table @kbd
@item M-q
Fill current paragraph (@code{fill-paragraph}).
@item M-g
Fill each paragraph in the region (@code{fill-region}).
@item C-x f
Set the fill column (@code{set-fill-column}).
@item M-x fill-region-as-paragraph.
Fill the region, considering it as one paragraph.
@item M-s
Center a line.
@end table

@kindex M-q
@findex fill-paragraph
  To refill a paragraph, use the command @kbd{Meta-q}
(@code{fill-paragraph}).  It causes the paragraph containing point, or
the one after point if point is between paragraphs, to be refilled.  All
line breaks are removed, and new ones are inserted where necessary.
@kbd{M-q} can be undone with @kbd{C-_}.  @xref{Undo}.@refill

@kindex M-g
@findex fill-region
  To refill many paragraphs, use @kbd{M-g} (@code{fill-region}), which
divides the region into paragraphs and fills each of them.

@findex fill-region-as-paragraph
  @kbd{Meta-q} and @kbd{Meta-g} use the same criteria as @kbd{Meta-h} for
finding paragraph boundaries (@pxref{Paragraphs}).  For more control, you
can use @kbd{M-x fill-region-as-paragraph}, which refills everything
between point and mark.  This command recognizes only blank lines as
paragraph separators.@refill

@page
@cindex justification
  A numeric argument to @kbd{M-g} or @kbd{M-q} causes it to
@dfn{justify} the text as well as filling it.  Extra spaces are inserted
to make the right margin line up exactly at the fill column.  To remove
the extra spaces, use @kbd{M-q} or @kbd{M-g} with no argument.@refill

@vindex auto-fill-inhibit-regexp
The variable @code{auto-fill-inhibit-regexp}takes as a value a regexp to
match lines that should not be auto-filled.

@kindex M-s
@cindex centering
@findex center-line
  The command @kbd{Meta-s} (@code{center-line}) centers the current line
within the current fill column.  With an argument, it centers several lines
individually and moves past them.

@vindex fill-column
  The maximum line width for filling is in the variable
@code{fill-column}.  Altering the value of @code{fill-column} makes it
local to the current buffer; until then, the default value---initially
70---is in effect. @xref{Locals}.

@kindex C-x f
@findex set-fill-column
  The easiest way to set @code{fill-column} is to use the command @kbd{C-x
f} (@code{set-fill-column}).  With no argument, it sets @code{fill-column}
to the current horizontal position of point.  With a numeric argument, it
uses that number as the new fill column.

@node Fill Prefix,, Fill Commands, Filling
@subsection The Fill Prefix

@cindex fill prefix
  To fill a paragraph in which each line starts with a special marker
(which might be a few spaces, giving an indented paragraph), use the
@dfn{fill prefix} feature.  The fill prefix is a string which is not
included in filling.  Emacs expects every line to start with a fill
prefix.

@table @kbd
@item C-x .
Set the fill prefix (@code{set-fill-prefix}).
@item M-q
Fill a paragraph using current fill prefix (@code{fill-paragraph}).
@item M-x fill-individual-paragraphs
Fill the region, considering each change of indentation as starting a
new paragraph.
@end table

@kindex C-x .
@findex set-fill-prefix
  To specify a fill prefix, move to a line that starts with the desired
prefix, put point at the end of the prefix, and give the command
@w{@kbd{C-x .}}@: (@code{set-fill-prefix}).  That's a period after the
@kbd{C-x}.  To turn off the fill prefix, specify an empty prefix: type
@w{@kbd{C-x .}}@: with point at the beginning of a line.@refill

  When a fill prefix is in effect, the fill commands remove the fill
prefix from each line before filling and insert it on each line after
filling.  Auto Fill mode also inserts the fill prefix inserted on new
lines it creates.  Lines that do not start with the fill prefix are
considered to start paragraphs, both in @kbd{M-q} and the paragraph
commands; this is just right if you are using paragraphs with hanging
indentation (every line indented except the first one).  Lines which are
blank or indented once the prefix is removed also separate or start
paragraphs; this is what you want if you are writing multi-paragraph
comments with a comment delimiter on each line.

@vindex fill-prefix
  The fill prefix is stored in the variable @code{fill-prefix}.  Its value
is a string, or @code{nil} when there is no fill prefix.  This is a
per-buffer variable; altering the variable affects only the current buffer,
but there is a default value which you can change as well.  @xref{Locals}.

@findex fill-individual-paragraphs
  Another way to use fill prefixes is through @kbd{M-x
fill-individual-paragraphs}.  This function divides the region into groups
of consecutive lines with the same amount and kind of indentation and fills
each group as a paragraph, using its indentation as a fill prefix.

@node Case,, Filling, Text
@section Case Conversion Commands
@cindex case conversion

  Emacs has commands for converting either a single word or any arbitrary
range of text to upper case or to lower case.

@c WideCommands
@table @kbd
@item M-l
Convert following word to lower case (@code{downcase-word}).
@item M-u
Convert following word to upper case (@code{upcase-word}).
@item M-c
Capitalize the following word (@code{capitalize-word}).
@item C-x C-l
Convert region to lower case (@code{downcase-region}).
@item C-x C-u
Convert region to upper case (@code{upcase-region}).
@end table

@kindex M-l
@kindex M-u
@kindex M-c
@cindex words
@findex downcase-word
@findex upcase-word
@findex capitalize-word
  The word conversion commands are used most frequently.  @kbd{Meta-l}
(@code{downcase-word}) converts the word after point to lower case,
moving past it.  Thus, repeating @kbd{Meta-l} converts successive words.
@kbd{Meta-u} (@code{upcase-word}) converts to all capitals instead,
while @kbd{Meta-c} (@code{capitalize-word}) puts the first letter of the
word into upper case and the rest into lower case.  The word conversion
commands convert several words at once if given an argument.  They are
especially convenient for converting a large amount of text from all
upper case to mixed case: you can move through the text using
@kbd{M-l}, @kbd{M-u} or @kbd{M-c} on each word as appropriate,
occasionally using @kbd{M-f} instead to skip a word.

  When given a negative argument, the word case conversion commands apply
to the appropriate number of words before point, but do not move point.
This is convenient when you have just typed a word in the wrong case: you
can give the case conversion command and continue typing.

  If a word case conversion command is given in the middle of a word, it
applies only to the part of the word which follows point.  This is just
like what @kbd{Meta-d} (@code{kill-word}) does.  With a negative argument,
case conversion applies only to the part of the word before point.

@kindex C-x C-l
@kindex C-x C-u
@cindex region
@findex downcase-region
@findex upcase-region
  The other case conversion commands are @kbd{C-x C-u}
(@code{upcase-region}) and @kbd{C-x C-l} (@code{downcase-region}), which
convert everything between point and mark to the specified case.  Point and
mark do not move.@refill

@node Programs, Running, Text, Top
@chapter Editing Programs
@cindex Lisp
@cindex C

  Emacs has many commands designed to understand the syntax of programming
languages such as Lisp and C.  These commands can

@itemize @bullet
@item
Move over or kill balanced expressions or @dfn{sexps} (@pxref{Lists}).
@item
Move over or mark top-level balanced expressions (@dfn{defuns}, in Lisp;
functions, in C).
@item
Show how parentheses balance (@pxref{Matching}).
@item
Insert, kill or align comments (@pxref{Comments}).
@item
Follow the usual indentation conventions of the language
(@pxref{Grinding}).
@end itemize

  The commands available for words, sentences and paragraphs are useful in
editing code even though their canonical application is for editing human
language text.  Most symbols contain words (@pxref{Words}); sentences can
be found in strings and comments (@pxref{Sentences}).  Paragraphs per se
are not present in code, but the paragraph commands are useful anyway,
because Lisp mode and C mode define paragraphs to begin and end at blank
lines (@pxref{Paragraphs}).  Judicious use of blank lines to make the
program clearer also provides interesting chunks of text for the
paragraph commands to work on.

  The selective display feature is useful for looking at the overall
structure of a function (@pxref{Selective Display}).  This feature causes
only the lines that are indented less than a specified amount to appear
on the screen.

@menu
* Program Modes::       Major modes for editing programs.
* Lists::               Expressions with balanced parentheses.
                         There are editing commands to operate on them.
* Defuns::              Each program is made up of separate functions.
                         There are editing commands to operate on them.
* Grinding::            Adjusting indentation to show the nesting.
* Matching::            Insertion of a close-delimiter flashes matching open.
* Comments::            Inserting, illing and aligning comments.
* Balanced Editing::    Inserting two matching parentheses at once, etc.
* Lisp Completion::     Completion on symbol names in Lisp code.
* Documentation::       Getting documentation of functions you plan to call.
* Change Log::          Maintaining a change history for your program.
* Tags::                Go direct to any function in your program in one
                         command.  Tags remembers which file it is in.
* Fortran::		Fortran mode and its special features.
@end menu

@node Program Modes, Lists, Programs, Programs
@section Major Modes for Programming Languages

@cindex Lisp mode
@cindex C mode
@cindex Scheme mode
  Emacs has several major modes for the programming languages Lisp, Scheme (a
variant of Lisp), C, Fortran and Muddle.  Ideally, a major mode should be
implemented for each programming language you might want to edit with
Emacs; but often the mode for one language can serve for other
syntactically similar languages.  The language modes that exist are those
that someone decided to take the trouble to write.

  There are several variants of Lisp mode, which differ in the way they
interface to Lisp execution.  @xref{Lisp Modes}.

  Each of the programming language modes defines the @key{TAB} key to run
an indentation function that knows the indentation conventions of that
language and updates the current line's indentation accordingly.  For
example, in C mode @key{TAB} is bound to @code{c-indent-line}.  @key{LFD}
is normally defined to do @key{RET} followed by @key{TAB}; thus, it too
indents in a mode-specific fashion.

@kindex DEL
@findex backward-delete-char-untabify
  In most programming languages, indentation is likely to vary from line to
line.  So the major modes for those languages rebind @key{DEL} to treat a
tab as if it were the equivalent number of spaces (using the command
@code{backward-delete-char-untabify}).  This makes it possible to rub out
indentation one column at a time without worrying whether it is made up of
spaces or tabs.  In these modes, use @kbd{C-b C-d} to delete a tab
character before point. 

  Programming language modes define paragraphs to be separated only by
blank lines, so that the paragraph commands remain useful.  Auto Fill mode,
if enabled in a programming language major mode, indents the new lines
which it creates.

@cindex mode hook
@vindex c-mode-hook
@vindex lisp-mode-hook
@vindex emacs-lisp-mode-hook
@vindex lisp-interaction-mode-hook
@vindex scheme-mode-hook
@vindex muddle-mode-hook
  Turning on a major mode calls a user-supplied function called the
@dfn{mode hook}, which is the value of a Lisp variable.  For example,
turning on C mode calls the value of the variable @code{c-mode-hook} if
that value exists and is non-@code{nil}.  Mode hook variables for other
programming language modes include @code{lisp-mode-hook},
@code{emacs-lisp-mode-hook}, @code{lisp-interaction-mode-hook},
@code{scheme-mode-hook} and @code{muddle-mode-hook}.  The mode hook
function receives no arguments.@refill

@node Lists, Defuns, Program Modes, Programs
@section Lists and Sexps

@cindex Control-Meta
  By convention, Emacs keys for dealing with balanced expressions are
usually @kbd{Control-Meta-} characters.  They tend to be analogous in
function to their @kbd{Control-} and @kbd{Meta-} equivalents.  These commands
are usually thought of as pertaining to expressions in programming
languages, but can be useful with any language in which some sort of
parentheses exist (including English).

@cindex list
@cindex sexp
@cindex expression
  The commands fall into two classes.  Some commands deal only with
@dfn{lists} (parenthetical groupings).  They see nothing except
parentheses, brackets, braces (depending on what must balance in the
language you are working with), and escape characters that might be used
to quote those.

  The other commands deal with expressions or @dfn{sexps}.  The word `sexp'
is derived from @dfn{s-expression}, the term for an expression in
Lisp.  In Emacs, the notion of `sexp' is not limited to Lisp.  It
refers to an expression in the language  your program is written in.
Each programming language has its own major mode, which customizes the
syntax tables so that expressions in that language count as sexps.

  Sexps typically include symbols, numbers, and string constants, as well
as anything contained in parentheses, brackets or braces.

  In languages that use prefix and infix operators, such as C, it is not
possible for all expressions to be sexps.  For example, C mode does not
recognize @samp{foo + bar} as an sexp, even though it @i{is} a C expression;
it recognizes @samp{foo} as one sexp and @samp{bar} as another, with the
@samp{+} as punctuation between them.  This is a fundamental ambiguity:
both @samp{foo + bar} and @samp{foo} are legitimate choices for the sexp to
move over if point is at the @samp{f}.  Note that @samp{(foo + bar)} is a
sexp in C mode.

  Some languages have obscure forms of syntax for expressions that nobody
has bothered to make Emacs understand properly.

@c doublewidecommands
@table @kbd
@item C-M-f
Move forward over an sexp (@code{forward-sexp}).
@item C-M-b
Move backward over an sexp (@code{backward-sexp}).
@item C-M-k
Kill sexp forward (@code{kill-sexp}).
@item C-M-u
Move up and backward in list structure (@code{backward-up-list}).
@item C-M-d
Move down and forward in list structure (@code{down-list}).
@item C-M-n
Move forward over a list (@code{forward-list}).
@item C-M-p
Move backward over a list (@code{backward-list}).
@item C-M-t
Transpose expressions (@code{transpose-sexps}).
@item C-M-@@
Put mark after following expression (@code{mark-sexp}).
@end table

@kindex C-M-f
@kindex C-M-b
@findex forward-sexp
@findex backward-sexp
  To move forward over an sexp, use @kbd{C-M-f} (@code{forward-sexp}).  If
the first significant character after point is an opening delimiter
(@samp{(} in Lisp; @samp{(}, @samp{[} or @samp{@{} in C), @kbd{C-M-f}
moves past the matching closing delimiter.  If the character begins a
symbol, string, or number, @kbd{C-M-f} moves over that.  If the character
after point is a closing delimiter, @kbd{C-M-f} just moves past it.  (This
last is not really moving across an sexp; it is an exception which is
included in the definition of @kbd{C-M-f} because it is as useful a
behavior as anyone can think of for that situation.)@refill

  The command @kbd{C-M-b} (@code{backward-sexp}) moves backward over a
sexp.  The detailed rules are like those above for @kbd{C-M-f}, but with
directions reversed.  If there are any prefix characters (singlequote,
backquote and comma, in Lisp) preceding the sexp, @kbd{C-M-b} moves back
over them as well.

  @kbd{C-M-f} or @kbd{C-M-b} with an argument repeats that operation the
specified number of times; with a negative argument, it moves in the
opposite direction.

In languages such as C where the comment-terminator can be recognized,
the sexp commands move across comments as if they were whitespace.  In
Lisp, and other languages where comments run until the end of a line, it
is very difficult to ignore comments when parsing backwards; therefore,
in such languages the sexp commands treat the text of comments as if it
were code.

@kindex C-M-k
@findex kill-sexp
  Killing an sexp at a time can be done with @kbd{C-M-k} (@code{kill-sexp}).
@kbd{C-M-k} kills the characters that @kbd{C-M-f} would move over.

@kindex C-M-n
@kindex C-M-p
@findex forward-list
@findex backward-list
  The @dfn{list commands}, @kbd{C-M-n} (@code{forward-list}) and
@kbd{C-M-p} (@code{backward-list}), move over lists like the sexp
commands but skip over any number of other kinds of sexps (symbols,
strings, etc).  In some situations, these commands are useful because
they usually ignore comments, since the comments usually do not contain
any lists.@refill

@kindex C-M-u
@kindex C-M-d
@findex backward-up-list
@findex down-list
  @kbd{C-M-n} and @kbd{C-M-p} stay at the same level in parentheses, when
that is possible.  To move @i{up} one (or @var{n}) levels, use @kbd{C-M-u}
(@code{backward-up-list}).
@kbd{C-M-u} moves backward up past one unmatched opening delimiter.  A
positive argument serves as a repeat count; a negative argument reverses
direction of motion and also requests repetition, so it moves forward and
up one or more levels.@refill

  To move @i{down} in list structure, use @kbd{C-M-d}
(@code{down-list}).  In Lisp mode, where @samp{(} is the only opening
delimiter, this is nearly the same as searching for a @samp{(}.  An
argument specifies the number of levels of parentheses to go down.

@cindex transposition
@kindex C-M-t
@findex transpose-sexps
@kbd{C-M-t} (@code{transpose-sexps}) drags the previous sexp across
the next one.  An argument serves as a repeat count, and a negative
argument drags backwards (thus canceling out the effect of @kbd{C-M-t} with
a positive argument).  An argument of zero, rather than doing nothing,
transposes the sexps ending after point and the mark.

@kindex C-M-@@
@findex mark-sexp
  To make the region be the next sexp in the buffer, use @kbd{C-M-@@}
(@code{mark-sexp}) which sets the mark at the same place that
@kbd{C-M-f} would move to.  @kbd{C-M-@@} takes arguments like
@kbd{C-M-f}.  In particular, a negative argument is useful for putting
the mark at the beginning of the previous sexp.

  The list and sexp commands' understanding of syntax is completely
controlled by the syntax table.  Any character can, for example, be
declared to be an opening delimiter and act like an open parenthesis.
@xref{Syntax}.

@node Defuns, Grinding, Lists, Programs
@section Defuns
@cindex defuns

  In Emacs, a parenthetical grouping at the top level in the buffer is
called a @dfn{defun}.  The name derives from the fact that most
top-level lists in a Lisp file are instances of the special form
@code{defun}, but in Emacs terminology, any top-level parenthetical
grouping counts as a defun regardless of its contents, and regardless of
the programming language in use.  For example, in C, the body of a
function definition is a defun.

@c doublewidecommands
@table @kbd
@item C-M-a
Move to beginning of current or preceding defun
(@code{beginning-of-defun}).
@item C-M-e
Move to end of current or following defun (@code{end-of-defun}).
@item C-M-h
Put region around whole current or following defun (@code{mark-defun}).
@end table

@kindex C-M-a
@kindex C-M-e
@kindex C-M-h
@findex beginning-of-defun
@findex end-of-defun
@findex mark-defun
  The commands to move to the beginning and end of the current defun are
@kbd{C-M-a} (@code{beginning-of-defun}) and @kbd{C-M-e} (@code{end-of-defun}).

   To operate on the current defun, use @kbd{C-M-h} (@code{mark-defun})
which puts point at the beginning and the mark at the end of the current
or next defun.  This is the easiest way to prepare for moving the defun
to a different place.  In C mode, @kbd{C-M-h} runs the function
@code{mark-c-function}, which is almost the same as @code{mark-defun},
but which backs up over the argument declarations, function name, and
returned data type so that the entire C function is inside th e region.

@findex elisp-compile-defun
To compile and evaluate the current defun, use @kbd{M-x
elisp-compile-defun}. This function prints the results in the
minibuffer. If you include an argument, it inserts the value in the
current buffer after the defun.

  Emacs assumes that any open-parenthesis found in the leftmost column is
the start of a defun.  Therefore, @i{never put an open-parenthesis at the
left margin in a Lisp file unless it is the start of a top level list.
Never put an open-brace or other opening delimiter at the beginning of a
line of C code unless it starts the body of a function.}  The most likely
problem case is when you want an opening delimiter at the start of a line
inside a string.  To avoid trouble, put an escape character (@samp{\}, in C
and Emacs Lisp, @samp{/} in some other Lisp dialects) before the opening
delimiter.  It will not affect the contents of the string.

  The original Emacs found defuns by moving upward a
level of parentheses until there were no more levels to go up.  This always
required scanning back to the beginning of the buffer, even for
a small function.  To speed up the operation, Emacs was changed to assume
that any @samp{(} (or other character assigned the syntactic class of
opening-delimiter) at the left margin is the start of a defun.  This
heuristic is nearly always right and avoids the costly scan; however,
it mandates the convention described above.

@node Grinding, Matching, Defuns, Programs
@section Indentation for Programs
@cindex indentation
@cindex grinding

  The best way to keep a program properly indented (``ground'') is to
use Emacs to re-indent it as you change the program.  Emacs has commands
to indent properly either a single line, a specified number of lines, or
all of the lines inside a single parenthetical grouping.

@menu
* Basic Indent::
* Multi-line Indent::   Commands to reindent many lines at once.
* Lisp Indent::		Specifying how each Lisp function should be indented.
* C Indent::		Choosing an indentation style for C code.
@end menu

@node Basic Indent, Multi-line Indent, Grinding, Grinding
@subsection Basic Program Indentation Commands

@c WideCommands
@table @kbd
@item @key{TAB}
Adjust indentation of current line.
@item @key{LFD}
Equivalent to @key{RET} followed by @key{TAB} (@code{newline-and-indent}).
@end table

@kindex TAB
@findex c-indent-line
@findex lisp-indent-line
  The basic indentation command is @key{TAB}, which gives the current
line the correct indentation as determined from the previous lines.  The
function that @key{TAB} runs depends on the major mode; it is
@code{lisp-indent-line} in Lisp mode, @code{c-indent-line} in C mode,
etc.  These functions understand different syntaxes for different
languages, but they all do about the same thing.  @key{TAB} in any
programming language major mode inserts or deletes whitespace at the
beginning of the current line, independent of where point is in the
line.  If point is inside the whitespace at the beginning of the line,
@key{TAB} leaves it at the end of that whitespace; otherwise, @key{TAB}
leaves point fixed with respect to the characters around it.

  Use @kbd{C-q @key{TAB}} to insert a tab at point.

@kindex LFD
@findex newline-and-indent
  When entering a large amount of new code, use @key{LFD}
(@code{newline-and-indent}), which is equivalent to a @key{RET} followed
by a @key{TAB}.  @key{LFD} creates a blank line, then gives it the
appropriate indentation.

  @key{TAB} indents the second and following lines of the body of a
parenthetical grouping each under the preceding one; therefore, if you
alter one line's indentation to be nonstandard, the lines below tend
to follow it.  This is the right behavior in cases where the standard
result of @key{TAB} does not look good.

  Remember that Emacs assumes that an open-parenthesis, open-brace or
other opening delimiter at the left margin (including the indentation
routines) is the start of a function.  You should therefore never have
an opening delimiter in column zero that is not the beginning of a
function, not even inside a string.  This restriction is vital for
making the indentation commands fast. @xref{Defuns}, for more
information on this behavior.

@node Multi-line Indent, Lisp Indent, Basic Indent, Grinding
@subsection Indenting Several Lines

  Several commands are available to re-indent several lines of code
which have been altered or moved to a different level in a list
structure,


@table @kbd
@item C-M-q
Re-indent all the lines within one list (@code{indent-sexp}).
@item C-u @key{TAB}
Shift an entire list rigidly sideways so that its first line
is properly indented.
@item C-M-\
Re-indent all lines in the region (@code{indent-region}).
@end table

@kindex C-M-q
@findex indent-sexp
@findex indent-c-exp
 To re-indent the contents of a single list, position point before the
beginning of it and type @kbd{C-M-q}. This key is bound to
@code{indent-sexp} in Lisp mode, @code{indent-c-exp} in C mode, and
bound to other suitable functions in other modes.  The indentation of
the line the sexp starts on is not changed; therefore, only the relative
indentation within the list, and not its position, is changed.  To
correct the position as well, type a @key{TAB} before @kbd{C-M-q}.

@kindex C-u TAB
  If the relative indentation within a list is correct but the
indentation of its beginning is not, go to the line on which the list
begins and type @kbd{C-u @key{TAB}}.  When you give @key{TAB} a numeric
argument, it moves all the lines in the group, starting on the current
line, sideways the same amount that the current line moves.  The command
does not move lines that start inside strings, or C
preprocessor lines when in C mode.

@kindex C-M-\
@findex indent-region
  Another way to specify a range to be re-indented is with point and
mark.  The command @kbd{C-M-\} (@code{indent-region}) applies @key{TAB}
to every line whose first character is between point and mark.

@node Lisp Indent, C Indent, Multi-line Indent, Grinding
@subsection Customizing Lisp Indentation
@cindex customization

  The indentation pattern for a Lisp expression can depend on the function
called by the expression.  For each Lisp function, you can choose among
several predefined patterns of indentation, or define an arbitrary one with
a Lisp program.

  The standard pattern of indentation is as follows: the second line of the
expression is indented under the first argument, if that is on the same
line as the beginning of the expression; otherwise, the second line is
indented underneath the function name.  Each following line is indented
under the previous line whose nesting depth is the same.

@vindex lisp-indent-offset
  If the variable @code{lisp-indent-offset} is non-@code{nil}, it overrides
the usual indentation pattern for the second line of an expression, so that
such lines are always indented @code{lisp-indent-offset} more columns than
the containing list.

@vindex lisp-body-indention
  Certain functions override the standard pattern.  Functions
whose names start with @code{def} always indent the second line by
@code{lisp-body-indention} extra columns beyond the open-parenthesis
starting the expression.

  Individual functions can override the standard pattern in various
ways, according to the @code{lisp-indent-function} property of the
function name.  (Note: @code{lisp-indent-function} was formerly called
@code{lisp-indent-hook}).  There are four possibilities for this
property:

@table @asis
@item @code{nil}
This is the same as no property; the standard indentation pattern is used.
@item @code{defun}
The pattern used for function names that start with @code{def} is used for
this function also.
@item a number, @var{number}
The first @var{number} arguments of the function are
@dfn{distinguished} arguments; the rest are considered the @dfn{body}
of the expression.  A line in the expression is indented according to
whether the first argument on it is distinguished or not.  If the
argument is part of the body, the line is indented @code{lisp-body-indent}
more columns than the open-parenthesis starting the containing
expression.  If the argument is distinguished and is either the first
or second argument, it is indented @i{twice} that many extra columns.
If the argument is distinguished and not the first or second argument,
the standard pattern is followed for that line.
@item a symbol, @var{symbol}
@var{symbol} should be a function name; that function is called to
calculate the indentation of a line within this expression.  The
function receives two arguments:
@table @asis
@item @var{state}
The value returned by @code{parse-partial-sexp} (a Lisp primitive for
indentation and nesting computation) when it parses up to the
beginning of this line.
@item @var{pos}
The position at which the line being indented begins.
@end table
@noindent
It should return either a number, which is the number of columns of
indentation for that line, or a list whose first element is such a
number.  The difference between returning a number and returning a list
is that a number says that all following lines at the same nesting level
should be indented just like this one; a list says that following lines
might call for different indentations.  This makes a difference when the
indentation is computed by @kbd{C-M-q}; if the value is a number,
@kbd{C-M-q} need not recalculate indentation for the following lines
until the end of the list.
@end table

@node C Indent,, Lisp Indent, Grinding
@subsection Customizing C Indentation

  Two variables control which commands perform C indentation and when.

@vindex c-auto-newline
  If @code{c-auto-newline} is non-@code{nil}, newlines are inserted both
before and after braces that you insert, and after colons and semicolons.
Correct C indentation is done on all the lines that are made this way.

@vindex c-tab-always-indent
  If @code{c-tab-always-indent} is non-@code{nil}, the @key{TAB} command
in C mode does indentation only if point is at the left margin or within
the line's indentation.  If there is non-whitespace to the left of point,
@key{TAB} just inserts a tab character in the buffer.  Normally,
this variable is @code{nil}, and @key{TAB} always reindents the current line.

  C does not have anything analogous to particular function names for which
special forms of indentation are desirable.  However, it has a different
need for customization facilities: many different styles of C indentation
are in common use.

  There are six variables you can set to control the style that Emacs C
mode will use.

@table @code
@item c-indent-level
Indentation of C statements within surrounding block.  The surrounding
block's indentation is the indentation of the line on which the
open-brace appears.
@item c-continued-statement-offset
Extra indentation given to a substatement, such as the then-clause of
an if or body of a while.
@item c-brace-offset
Extra indentation for lines that start with an open brace.
@item c-brace-imaginary-offset
An open brace following other text is treated as if it were this far
to the right of the start of its line.
@item c-argdecl-indent
Indentation level of declarations of C function arguments.
@item c-label-offset
Extra indentation for a line that is a label, case, or default.
@end table

@vindex c-indent-level
  The variable @code{c-indent-level} controls the indentation for C
statements with respect to the surrounding block.  In the example

@example
    @{
      foo ();
@end example

@noindent
the difference in indentation between the lines is @code{c-indent-level}.
Its standard value is 2.

If the open-brace beginning the compound statement is not at the beginning
of its line, the @code{c-indent-level} is added to the indentation of the
line, not the column of the open-brace.  For example,

@example
if (losing) @{
  do_this ();
@end example

@page
@noindent
One popular indentation style is that which results from setting
@code{c-indent-level} to 8 and putting open-braces at the end of a line
in this way.  Another popular style prefers to put the open-brace on a
separate line.

@vindex c-brace-imaginary-offset
  In fact, the value of the variable @code{c-brace-imaginary-offset} is
also added to the indentation of such a statement.  Normally this variable
is zero.  Think of this variable as the imaginary position of the open
brace, relative to the first non-blank character on the line.  By setting
the variable to 4 and @code{c-indent-level} to 0, you can get this style:

@example
if (x == y) @{
    do_it ();
    @}
@end example

  When @code{c-indent-level} is zero, the statements inside most braces
line up exactly under the open brace.  An exception are braces in column
zero, like those surrounding a function's body.  The statements inside
those braces are not placed at column zero.  Instead,
@code{c-brace-offset} and @code{c-continued-statement-offset} (see
below) are added to produce a typical offset between brace levels, and
the statements are indented that far.

@vindex c-continued-statement-offset
  @code{c-continued-statement-offset} controls the extra indentation for
a line that starts within a statement (but not within parentheses or
brackets).  These lines are usually statements inside other statements,
like the then-clauses of @code{if} statements and the bodies of
@code{while} statements.  The @code{c-continued-statement-offset}
parameter determines the difference in indentation between the two lines in

@example
if (x == y)
  do_it ();
@end example

@noindent
The default value for @code{c-continued-statement-offset} is 2.  Some
popular indentation styles correspond to a value of zero for
@code{c-continued-statement-offset}.

@vindex c-brace-offset
  @code{c-brace-offset} is the extra indentation given to a line that
starts with an open-brace.  Its standard value is zero;
compare

@example
if (x == y)
  @{
@end example

@noindent
with

@example
if (x == y)
  do_it ();
@end example

@noindent
if you set @code{c-brace-offset} to 4, the first example becomes

@example
if (x == y)
      @{
@end example

@vindex c-argdecl-indent
  @code{c-argdecl-indent} controls the indentation of declarations of the
arguments of a C function.  It is absolute: argument declarations receive
exactly @code{c-argdecl-indent} spaces.  The standard value is 5 and
results in code like this:

@example
char *
index (string, char)
     char *string;
     int char;
@end example

@vindex c-label-offset
  @code{c-label-offset} is the extra indentation given to a line that
contains a label, a case statement, or a @code{default:} statement.  Its
standard value is @minus{}2 and results in code like this

@example
switch (c)
  @{
  case 'x':
@end example

@noindent
If @code{c-label-offset} were zero, the same code would be indented as

@example
switch (c)
  @{
    case 'x':
@end example

@noindent
This example assumes that the other variables above also have their
default values. 

Using the indentation style produced by the default settings of the
variables just discussed, and putting open braces on separate lines
produces clear and readable files.  For an example, look at any of the C
source files of GNU Emacs.

@node Matching, Comments, Grinding, Programs
@section Automatic Display of Matching Parentheses
@cindex matching parentheses
@cindex parentheses

  The Emacs parenthesis-matching feature shows you automatically how
parentheses match in the text.  Whenever a self-inserting character that
is a closing delimiter is typed, the cursor moves momentarily to the
location of the matching opening delimiter, provided that is visible on
the screen.  If it is not on the screen, some text starting with that
opening delimiter is displayed in the echo area.  Either way, you see
the grouping you are closing off. 

@page
  In Lisp, automatic matching applies only to parentheses.  In C, it
also applies to braces and brackets.  Emacs knows which characters to regard
as matching delimiters based on the syntax table set by the major
mode.  @xref{Syntax}.

  If the opening delimiter and closing delimiter are mismatched---such as
in @samp{[x)}---the echo area displays a warning message.  The
correct matches are specified in the syntax table.

@vindex blink-matching-paren
@vindex blink-matching-paren-distance
  Two variables control parenthesis matching displays.
@code{blink-matching-paren} turns the feature on or off: @code{nil}
turns it off, but the default is @code{t} to turn match display on.
@code{blink-matching-paren-distance} specifies how many characters back
Emacs searches to find a matching opening delimiter.  If the match is
not found in the specified region, scanning stops, and nothing is
displayed.  This prevents wasting lots of time scanning when there is no
match.  The default is 4000.

@node Comments, Balanced Editing, Matching, Programs
@section Manipulating Comments
@cindex comments
@kindex M-;
@cindex indentation
@findex indent-for-comment

  The comment commands insert, kill and align comments.

@c WideCommands
@table @kbd
@item M-;
Insert or align comment (@code{indent-for-comment}).
@item C-x ;
Set comment column (@code{set-comment-column}).
@item C-u - C-x ;
Kill comment on current line (@code{kill-comment}).
@item M-@key{LFD}
Like @key{RET} followed by inserting and aligning a comment
(@code{indent-new-comment-line}).
@end table

  The command that creates a comment is @kbd{Meta-;}
(@code{indent-for-comment}).  If there is no comment already on the
line, a new comment is created and aligned at a specific column called
the @dfn{comment column}.  Emacs creates the comment by inserting the
string at the value of @code{comment-start}; see below.  Point is left
after that string.  If the text of the line extends past the comment
column, indentation is done to a suitable boundary (usually, at least
one space is inserted).  If the major mode has specified a string to
terminate comments, that string is inserted after point, to keep the
syntax valid.

  You can also use @kbd{Meta-;} to align an existing comment.  If a line
already contains the string that starts comments, @kbd{M-;} just moves
point after it and re-indents it to the conventional place.  Exception:
comments starting in column 0 are not moved.

  Some major modes have special rules for indenting certain kinds of
comments in certain contexts.  For example, in Lisp code, comments which
start with two semicolons are indented as if they were lines of code,
instead of at the comment column.  Comments which start with three
semicolons are supposed to start at the left margin.  Emacs understands
these conventions by indenting a double-semicolon comment using @key{TAB},
and by not changing the indentation of a triple-semicolon comment at all.

@example
;; This function is just an example
;;; Here either two or three semicolons are appropriate.
(defun foo (x)
;;; And now, the first part of the function:
  ;; The following line adds one.
  (1+ x))           ; This line adds one.
@end example

  In C code, a comment preceded on its line by nothing but whitespace
is indented like a line of code.

  Even when an existing comment is properly aligned, @kbd{M-;} is still
useful for moving directly to the start of the comment.

@kindex C-u - C-x ;
@findex kill-comment
  @kbd{C-u - C-x ;} (@code{kill-comment}) kills the comment on the
current line, if there is one.  The indentation before the start of the
comment is killed as well.  If there does not appear to be a comment in
the line, nothing happens.  To reinsert the comment on another line,
move to the end of that line, type first @kbd{C-y}, and then @kbd{M-;}
to realign the comment.  Note that @kbd{C-u - C-x ;} is not a distinct
key; it is @kbd{C-x ;} (@code{set-comment-column}) with a negative
argument.  That command is programmed to call @code{kill-comment} when
called with a negative argument.  However, @code{kill-comment} is a
valid command which you could bind directly to a key if you wanted to.

@subsection Multiple Lines of Comments

@kindex M-LFD
@cindex blank lines
@cindex Auto Fill mode
@findex indent-new-comment-line
  If you are typing a comment and want to continue it on another line,
use the command @kbd{Meta-@key{LFD}} (@code{indent-new-comment-line}),
which terminates the comment you are typing, creates a new blank line
afterward, and begins a new comment indented under the old one.  If
Auto Fill mode is on and you go past the fill column while typing, the 
comment is continued in just this fashion.  If point is
not at the end of the line when you type @kbd{M-@key{LFD}}, the text on
the rest of the line becomes part of the new comment line.

@page
@subsection Options Controlling Comments

@vindex comment-column
@kindex C-x ;
@findex set-comment-column
  The comment column is stored in the variable @code{comment-column}.  You
can explicitly set it to a number.  Alternatively, the command @kbd{C-x ;}
(@code{set-comment-column}) sets the comment column to the column point is
at.  @kbd{C-u C-x ;} sets the comment column to match the last comment
before point in the buffer, and then calls @kbd{Meta-;} to align the
current line's comment under the previous one.  Note that @kbd{C-u - C-x ;}
runs the function @code{kill-comment} as described above.

  @code{comment-column} is a per-buffer variable; altering the variable
affects only the current buffer.  You can also change the default value.
@xref{Locals}.  Many major modes initialize this variable
for the current buffer.

@vindex comment-start-skip
  The comment commands recognize comments based on the regular expression
that is the value of the variable @code{comment-start-skip}.  This regexp
should not match the null string.  It may match more than the comment
starting delimiter in the strictest sense of the word; for example, in C
mode the value of the variable is @code{@t{"/\\*+ *"}}, which matches extra
stars and spaces after the @samp{/*} itself.  (Note that @samp{\\} is
needed in Lisp syntax to include a @samp{\} in the string, which is needed
to deny the first star its special meaning in regexp syntax.  @xref{Regexps}.)

@vindex comment-start
@vindex comment-end
  When a comment command makes a new comment, it inserts the value of
@code{comment-start} to begin it.  The value of @code{comment-end} is
inserted after point, and will follow the text you will insert
into the comment.  In C mode, @code{comment-start} has the value
@w{@code{"/* "}} and @code{comment-end} has the value @w{@code{" */"}}.

@vindex comment-multi-line
  @code{comment-multi-line} controls how @kbd{M-@key{LFD}}
(@code{indent-new-comment-line}) behaves when used inside a comment.  If
@code{comment-multi-line} is @code{nil}, as it normally is, then
@kbd{M-@key{LFD}} terminates the comment on the starting line and starts
a new comment on the new following line.  If @code{comment-multi-line}
is not @code{nil}, then @kbd{M-@key{LFD}} sets up the new following line
as part of the same comment that was found on the starting line.  This
is done by not inserting a terminator on the old line, and not inserting
a starter on the new line.  In languages where multi-line comments are legal,
the value you choose for this variable is a matter of taste.

@vindex comment-indent-hook
  The variable @code{comment-indent-hook} should contain a function that
is called to compute the indentation for a newly inserted comment or for
aligning an existing comment.  Major modes set this variable differently.
The function is called with no arguments, but with point at the
beginning of the comment, or at the end of a line if a new comment is to
be inserted.  The function should return the column in which the comment
ought to start.  For example, in Lisp mode, the indent hook function
bases its decision on the number of semicolons that begin an existing
comment, and on the code in the preceding lines.

@node Balanced Editing, Lisp Completion, Comments, Programs
@section Editing Without Unbalanced Parentheses

@table @kbd
@item M-(
Put parentheses around next sexp(s) (@code{insert-parentheses}).
@item M-)
Move past next close parenthesis and re-indent
(@code{move-over-close-and-reindent}).
@end table

@kindex M-(
@kindex M-)
@findex insert-parentheses
@findex move-over-close-and-reindent
  The commands @kbd{M-(} (@code{insert-parentheses}) and @kbd{M-)}
(@code{move-over-close-@*and-reindent}) are designed to facilitate a style of
editing which keeps parentheses balanced at all times.  @kbd{M-(} inserts a
pair of parentheses, either together as in @samp{()}, or, if given an
argument, around the next several sexps, and leaves point after the open
parenthesis.  Instead of typing @kbd{( F O O )}, you can type @kbd{M-( F O
O}, which has the same effect except for leaving the cursor before the
close parenthesis.  You can then type @kbd{M-)}, which moves past the
close parenthesis, deletes any indentation preceding it (in this example
there is none), and indents with @key{LFD} after it.

@node Lisp Completion, Documentation, Balanced Editing, Programs
@section Completion for Lisp Symbols
@cindex completion (symbol names)

   Completion usually happens in the minibuffer.  An exception is
completion for Lisp symbol names, which is available in all buffers.

@kindex M-TAB
@findex lisp-complete-symbol
  The command @kbd{M-@key{TAB}} (@code{lisp-complete-symbol}) takes the
partial Lisp symbol before point to be an abbreviation, and compares it
against all non-trivial Lisp symbols currently known to Emacs.  Any
additional characters that they all have in common are inserted at point.
Non-trivial symbols are those that have function definitions, values or
properties.

  If there is an open-parenthesis immediately before the beginning of
the partial symbol, only symbols with function definitions are considered
as completions.

  If the partial name in the buffer has more than one possible completion
and they have no additional characters in common, a list of all possible
completions is displayed in another window.

@node Documentation, Change Log, Lisp Completion, Programs
@section Documentation Commands

@kindex C-h f
@findex describe-function
@kindex C-h v
@findex describe-variable
  As you edit Lisp code to be run in Emacs, you can use the commands
@kbd{C-h f} (@code{describe-function}) and @kbd{C-h v}
(@code{describe-variable}) to print documentation of functions and
variables you want to call.  These commands use the minibuffer to
read the name of a function or variable to document, and display the
documentation in a window.

  For extra convenience, these commands provide default arguments based on
the code in the neighborhood of point.  @kbd{C-h f} sets the default to the
function called in the innermost list containing point.  @kbd{C-h v} uses
the symbol name around or adjacent to point as its default.

@findex manual-entry
  The @kbd{M-x manual-entry} command gives you access to documentation
on Unix commands, system calls, and libraries.  The command reads a
topic as an argument, and displays the Unix manual page for that topic. 
@code{manual-entry} always searches all 8 sections of the
manual, and concatenates all the entries it finds.  For example,
the topic @samp{termcap} finds the description of the termcap library
from section 3, followed by the description of the termcap data base
from section 5.

@node Change Log, Tags, Documentation, Programs
@section Change Logs

@cindex change log
@findex add-change-log-entry
  The Emacs command @kbd{M-x add-change-log-entry} helps you keep a record
of when and why you have changed a program.  It assumes that you have a
file in which you write a chronological sequence of entries describing
individual changes.  The default is to store the change entries in a file
called @file{ChangeLog} in the same directory as the file you are editing.
The same @file{ChangeLog} file therefore records changes for all the files
in a directory.

  A change log entry starts with a header line that contains your name
and the current date.  Except for these header lines, every line in the
change log starts with a tab.  One entry can describe several changes;
each change starts with a line starting with a tab and a star.  @kbd{M-x
add-change-log-entry} visits the change log file and creates a new entry
unless the most recent entry is for today's date and your name.  In
either case, it adds a new line to start the description of another
change just after the header line of the entry.  When @kbd{M-x
add-change-log-entry} is finished, all is prepared for you to edit in
the description of what you changed and how.  You must then save the
change log file yourself.

  The change log file is always visited in Indented Text mode, which means
that @key{LFD} and auto-filling indent each new line like the previous
line.  This is convenient for entering the contents of an entry, which must
be indented.  @xref{Text Mode}.

  Here is an example of the formatting conventions used in the change log
for Emacs:

@page
@smallexample
Wed Jun 26 19:29:32 1985  Richard M. Stallman  (rms at mit-prep)

        * xdisp.c (try_window_id):
        If C-k is done at end of next-to-last line,
        this fn updates window_end_vpos and cannot leave
        window_end_pos nonnegative (it is zero, in fact).
        If display is preempted before lines are output,
        this is inconsistent.  Fix by setting
        blank_end_of_window to nonzero.

Tue Jun 25 05:25:33 1985  Richard M. Stallman  (rms at mit-prep)

        * cmds.c (Fnewline):
        Call the auto fill hook if appropriate.

        * xdisp.c (try_window_id):
        If point is found by compute_motion after xp, record that
        permanently.  If display_text_line sets point position wrong
        (case where line is killed, point is at eob and that line is
        not displayed), set it again in final compute_motion.
@end smallexample

@node Tags, Fortran, Change Log, Programs
@section Tag Tables
@cindex tag table

  A @dfn{tag table} is a description of how a multi-file program is
broken up into files.  It lists the names of the component files and the
names and positions of the functions in each file.  Grouping the related
files makes it possible to search or replace through all the files with
one command.  Recording the function names and positions makes it
possible to use the @kbd{Meta-.} command which finds the definition of a
function without asking for information on the file it is in.

  Tag tables are stored in files called @dfn{tag table files}.  The
conventional name for a tag table file is @file{TAGS}.

  Each entry in the tag table records the name of one tag, the name of the
file that the tag is defined in (implicitly), and the position in that file
of the tag's definition.

   The programming language of a file determines what names are recorded
in the tag table depends on.  Normally, Emacs includes all functions and
subroutines, and may also include global variables, data types, and
anything else convenient.  Each recorded name is called a @dfn{tag}.

@menu
* Tag Syntax::
* Create Tag Table::
* Select Tag Table::
* Find Tag::
* Tags Search::
* Tags Stepping::
* List Tags::
@end menu

@node Tag Syntax, Create Tag Table, Tags, Tags
@subsection Source File Tag Syntax

  In Lisp code, any function defined with @code{defun}, any variable
defined with @code{defvar} or @code{defconst}, and the first argument of
any expression that starts with @samp{(def} in column zero, is a tag.

  In C code, any C function is a tag, and so is any typedef if @code{-t} is
specified when the tag table is constructed.

  In Fortran code, functions and subroutines are tags.

  In La@TeX{} text, the argument of any of the commands @code{\chapter},
@code{\section}, @code{\subsection}, @code{\subsubsection}, @code{\eqno},
@code{\label}, @code{\ref}, @code{\cite}, @code{\bibitem} and
@*@code{\typeout} is a tag.@refill

@node Create Tag Table, Select Tag Table, Tag Syntax, Tags
@subsection Creating Tag Tables
@cindex etags program

  The @code{etags} program is used to create a tag table file.  It knows
the syntax of C, Fortran, La@TeX{}, Scheme and Emacs Lisp/Common Lisp.  To
use @code{etags}, use it as a shell command

@example
etags @var{inputfiles}@dots{}
@end example
@noindent

The program reads the specified files and writes a tag table
named @file{TAGS} in the current working directory.  @code{etags}
recognizes the language used in an input file based on the name and
contents of the file; there are no switches for specifying the language.
The @code{-t} switch tells @code{etags} to record typedefs in C code as
tags.

  If the tag table data become outdated due to changes in the files
described in the table, you can update the tag table by running the
program from the shell again.  It is not necessary to do this often.

  If the tag table fails to record a tag, or records it for the wrong file,
Emacs cannot find its definition.  However, if the position
recorded in the tag table becomes a little bit wrong (due to some editing
in the file that the tag definition is in), the only consequence is to slow
down finding the tag slightly.  Even if the stored position is very wrong,
Emacs will still find the tag, but it must search the entire file for it.

   You should update a tag table when you define new tags you want
to have listed, or when you move tag definitions from one file to another,
or when changes become substantial.  You don't have to update
the tag table after each edit, or even every day.

@node Select Tag Table, Find Tag, Create Tag Table, Tags
@subsection Selecting a Tag Table

@vindex tag-table-alist
   At any time Emacs has one @dfn{selected} tag table, and all the commands
for working with tag tables use the selected one.  To select a tag table,
use the variable @code{tag-table-alist}.

The value of @code{tag-table-alist} is a list that determines which
@code{TAGS} files should be active for a given buffer.  This is not
really an association list, in that all elements are checked.  The car
of each element of this list is a pattern against which the buffers file
name is compared; if it matches, then the cdr of the list should be the
name of the tags table to use.  If more than one element of this list
matches the buffers file name, all of the associated tags tables are
used.  Earlier ones are searched first.

If the car of elements of this list are strings, they are treated
as regular-expressions against which the file is compared (like the
@code{auto-mode-alist}).  If they are not strings, they are evaluated.
If they evaluate to non-@code{nil}, the current buffer is considered to
match.

If the cdr of the elements of this list are strings, they are
assumed to name a tags file.  If they name a directory, the string
@file{tags} is appended to them to get the file name.  If they are not 
strings, they are evaluated, and must return an appropriate string.

For example:

@example
  (setq tag-table-alist
	(("/usr/src/public/perl/" . "/usr/src/public/perl/perl-3.0/")
	 ("\\.el$" . "/usr/local/emacs/src/")
	 ("/jbw/gnu/" . "/usr15/degree/stud/jbw/gnu/")
	 ("" . "/usr/local/emacs/src/")
	 ))
@end example

The example defines the tag table alist in the following way:
 
@itemize @bullet
@item
Anything in the directory @file{/usr/src/public/perl/} 
should use the @file{TAGS} file @file{/usr/src/public/perl/perl-3.0/TAGS}. 
@item
Files ending in @file{.el} should use the @file{TAGS} file
@file{/usr/local/emacs/src/TAGS}. 
@item
Anything in or below the directory @file{/jbw/gnu/} should use the 
@file{TAGS} file @file{/usr15/degree/stud/jbw/gnu/TAGS}.  
@end itemize

If you had a file called @file{/usr/jbw/foo.el}, it would use both
@file{TAGS} files, @* @file{/usr/local/emacs/src/TAGS} and
@file{/usr15/degree/stud/jbw/gnu/TAGS} (in that order), because it
matches both patterns.

@page
If the buffer-local variable @code{buffer-tag-table} is set, it names a
tags table that is searched before all others when @code{find-tag} is
executed from this buffer.

If there is a file called @file{TAGS} in the same directory as the file
in question, then that tags file will always be used as well (after the
@code{buffer-tag-table} but before the tables specified by this list).

If the variable @code{tags-file-name} is set, the @file{TAGS} file it names
will apply to all buffers (for backwards compatibility.)  It is searched
first.

@vindex tags-always-build-completion-table
If the value of the variable @code{tags-always-build-completion-table}
is @code{t}, the tags file will always be added to the completion table
without asking first, regardless of the size of the tags file.

@vindex tags-file-name
@findex visit-tags-table
The function @kbd{M-x visit-tags-table}, which is largely obsoleted by
the variable @code{tag-table-alist}, tells tags commands to use the tags
table file @var{file} first.  The @var{file} should be the name of a
file created with the @code{etags} program.  A directory name is also
acceptable; it means the file @file{TAGS} in that directory.  The
function only stores the file name you provide in the variable
@code{tags-file-name}.  Emacs does not actually read in the tag table
contents until you try to use them.  You can set the variable explicitly
instead of using @code{visit-tags-table}.  The value of the variable
@code{tags-file-name} is the name of the tags table used by all buffers.
This is for backward compatibility, and is largely supplanted by the
variable @code{tag-table-alist}.
 
@node Find Tag, Tags Search, Select Tag Table, Tags
@subsection Finding a Tag

  The most important thing that a tag table enables you to do is to find
the definition of a specific tag.

@table @kbd
@item M-.@: @var{tag &optional other-window}
Find first definition of @var{tag} (@code{find-tag}).
@item C-u M-.
Find next alternate definition of last tag specified.
@item C-x 4 . @var{tag}
Find first definition of @var{tag}, but display it in another window
(@code{find-tag-other-window}).
@end table

@kindex M-.
@findex find-tag
  @kbd{M-.}@: (@code{find-tag}) is the command to find the definition of
a specified tag.  It searches through the tag table for that tag, as a
string, then uses the tag table information to determine the file in
which the definition is used and the approximate character position of
the definition in the file.  Then @code{find-tag} visits the file,
moves point to the approximate character position, and starts searching
ever-increasing distances away for the text that should appear at
the beginning of the definition.

  If an empty argument is given (just type @key{RET}), the sexp in the
buffer before or around point is used as the name of the tag to find.
@xref{Lists}, for information on sexps.

  The argument to @code{find-tag} need not be the whole tag name; it can
be a substring of a tag name.  However, there can be many tag names
containing the substring you specify.  Since @code{find-tag} works by
searching the text of the tag table, it finds the first tag in the table
that the specified substring appears in.  To find other tags that match
the substring, give @code{find-tag} a numeric argument, as in @kbd{C-u
M-.}.  This does not read a tag name, but continues searching the tag
table's text for another tag containing the same substring last used.
If your keyboard has a real @key{META} key, @kbd{M-0 M-.}@: is an easier
alternative to @kbd{C-u M-.}.

If the optional second argument @var{other-window} is non-@code{nil}, it uses
another window to display the tag.
Multiple active tags tables and completion are supported.

Variables of note:

@vindex tag-table-alist
@vindex tags-file-name
@vindex tags-build-completion-table
@vindex buffer-tag-table
@vindex make-tags-files-invisible
@vindex tag-mark-stack-max

@table @kbd
@item tag-table-alist
Controls which tables apply to which buffers.
@item tags-file-name		
Stores a default tags table.
@item tags-build-completion-table   
Controls completion behavior.
@item buffer-tag-table		
Specifies a buffer-local table.
@item make-tags-files-invisible	
Sets whether tags tables should be very hidden.
@item tag-mark-stack-max		
Specifies how many tags-based hops to remember.
@end table

@kindex C-x 4 .
@findex find-tag-other-window
  Like most commands that can switch buffers, @code{find-tag} has another
similar command that displays the new buffer in another window.  @kbd{C-x 4
.}@: invokes the function @code{find-tag-other-window}.  (This key sequence
ends with a period.)

  Emacs comes with a tag table file @file{TAGS}, in the directory
containing Lisp libraries, which includes all the Lisp libraries and all
the C sources of Emacs.  By specifying this file with @code{visit-tags-table}
and then using @kbd{M-.}@: you can quickly look at the source of any Emacs
function.

@page
@node Tags Search, Tags Stepping, Find Tag, Tags
@subsection Searching and Replacing with Tag Tables

  The commands in this section visit and search all the files listed in the
selected tag table, one by one.  For these commands, the tag table serves
only to specify a sequence of files to search.  A related command is
@kbd{M-x grep} (@pxref{Compilation}).

@table @kbd
@item M-x tags-search
Search for the specified regexp through the files in the selected tag
table.
@item M-x tags-query-replace
Perform a @code{query-replace} on each file in the selected tag table.
@item M-,
Restart one of the commands above, from the current location of point
(@code{tags-loop-continue}).
@end table

@findex tags-search
  @kbd{M-x tags-search} reads a regexp using the minibuffer, then visits
the files of the selected tag table one by one, and searches through each
file for that regexp.  It displays the name of the file being searched so
you can follow its progress.  As soon as an occurrence is found,
@code{tags-search} returns.

@kindex M-,
@findex tags-loop-continue
 After you have found one match, you probably want to find all the rest.
To find one more match, type @kbd{M-,} (@code{tags-loop-continue}) to
resume the @code{tags-search}.  This searches the rest of the current
buffer, followed by the remaining files of the tag table.

@findex tags-query-replace
  @kbd{M-x tags-query-replace} performs a single @code{query-replace}
through all the files in the tag table.  It reads a string to search for
and a string to replace with, just like ordinary @kbd{M-x query-replace}.
It searches much like @kbd{M-x tags-search} but repeatedly, processing
matches according to your input.  @xref{Replace}, for more information on
@code{query-replace}.@refill

  It is possible to get through all the files in the tag table with a
single invocation of @kbd{M-x tags-query-replace}.  But since any
unrecognized character causes the command to exit, you may need to continue
where you left off.  You can use @kbd{M-,} to do this.  It resumes the last
tags search or replace command that you did.

  It may have struck you that @code{tags-search} is a lot like @code{grep}.
You can also run @code{grep} itself as an inferior of Emacs and have Emacs
show you the matching lines one by one.  This works mostly the same as
running a compilation and having Emacs show you where the errors were.
@xref{Compilation}.

@page
@node Tags Stepping, List Tags, Tags Search, Tags
@subsection Stepping Through a Tag Table
@findex next-file

  If you wish to process all the files in a selected tag table, but
@kbd{M-x tags-search} and @kbd{M-x tags-query-replace} are not giving
you the desired result, you can use @kbd{M-x next-file}.

@table @kbd
@item C-u M-x next-file
With a numeric argument, regardless of its value, visit the first
file in the tag table, and prepare to advance sequentially by files.
@item M-x next-file
Visit the next file in the selected tag table.
@end table

@node List Tags,, Tags Stepping, Tags
@subsection Tag Table Inquiries

@table @kbd
@item M-x list-tags
Display a list of the tags defined in a specific program file.
@item M-x tags-apropos
Display a list of all tags matching a specified regexp.
@end table

@findex list-tags
  @kbd{M-x list-tags} reads the name of one of the files described by the
selected tag table, and displays a list of all the tags defined in that
file.  The ``file name'' argument is really just a string to compare
against the names recorded in the tag table; it is read as a string rather
than a file name.  Therefore, completion and defaulting are not
available, and you must enter the string the same way it appears in the tag
table.  Do not include a directory as part of the file name unless the file
name recorded in the tag table contains that directory.

@findex tags-apropos
  @kbd{M-x tags-apropos} is like @code{apropos} for tags.  It reads a regexp,
then finds all the tags in the selected tag table whose entries match that
regexp, and displays the tag names found.

@node Fortran,, Tags, Programs
@section Fortran Mode
@cindex Fortran mode

  Fortran mode provides special motion commands for Fortran statements and
subprograms, and indentation commands that understand Fortran conventions
of nesting, line numbers and continuation statements.

  Special commands for comments are provided because Fortran comments are
unlike those of other languages.

  Built-in abbrevs optionally save typing when you insert Fortran keywords.

@findex fortran-mode
  Use @kbd{M-x fortran-mode} to switch to this major mode.  Doing so calls
the value of @code{fortran-mode-hook} as a function of no arguments if
that variable has a non- @code{nil} value.

@menu
* Motion: Fortran Motion.     Moving point by statements or subprograms.
* Indent: Fortran Indent.     Indentation commands for Fortran.
* Comments: Fortran Comments. Inserting and aligning comments.
* Columns: Fortran Columns.   Measuring columns for valid Fortran.
* Abbrev: Fortran Abbrev.     Built-in abbrevs for Fortran keywords.
@end menu

  Fortran mode was contributed by Michael Prange.

@node Fortran Motion, Fortran Indent, Fortran, Fortran
@subsection Motion Commands

  Fortran mode provides special commands to move by subprograms (functions
and subroutines) and by statements.  There is also a command to put the
region around one subprogram, convenient for killing it or moving it.

@kindex C-M-a (Fortran mode)
@kindex C-M-e (Fortran mode)
@kindex C-M-h (Fortran mode)
@kindex C-c C-p (Fortran mode)
@kindex C-c C-n (Fortran mode)
@findex beginning-of-fortran-subprogram
@findex end-of-fortran-subprogram
@findex mark-fortran-subprogram
@findex fortran-previous-statement
@findex fortran-next-statement

@table @kbd
@item C-M-a
Move to beginning of subprogram@*
(@code{beginning-of-fortran-subprogram}).
@item C-M-e
Move to end of subprogram (@code{end-of-fortran-subprogram}).
@item C-M-h
Put point at beginning of subprogram and mark at end
(@code{mark-fortran-subprogram}).
@item C-c C-n
Move to beginning of current or next statement
(@code{fortran-next-@*statement}).
@item C-c C-p
Move to beginning of current or previous statement
(@code{fortran-@*previous-statement}).
@end table

@node Fortran Indent, Fortran Comments, Fortran Motion, Fortran
@subsection Fortran Indentation

  Special commands and features are available for indenting Fortran
code.  They make sure various syntactic entities (line numbers, comment line
indicators and continuation line flags) appear in the columns that are
required for standard Fortran.

@menu
* Commands: ForIndent Commands. Commands for indenting Fortran.
* Numbers:  ForIndent Num.      How line numbers auto-indent.
* Conv:     ForIndent Conv.     Conventions you must obey to avoid trouble.
* Vars:     ForIndent Vars.     Variables controlling Fortran indent style.
@end menu

@node ForIndent Commands, ForIndent Num, Fortran Indent, Fortran Indent
@subsubsection Fortran Indentation Commands

@table @kbd
@item @key{TAB}
Indent the current line (@code{fortran-indent-line}).
@item M-@key{LFD}
Break the current line and set up a continuation line.
@item C-M-q
Indent all the lines of the subprogram point is in
(@code{fortran-indent-subprogram}).
@end table

@findex fortran-indent-line
  @key{TAB} is redefined by Fortran mode to reindent the current line for
Fortran (@code{fortran-indent-line}).  Line numbers and continuation
markers are indented to their required columns, and the body of the
statement is independently indented based on its nesting in the program.

@kindex C-M-q (Fortran mode)
@findex fortran-indent-subprogram
  The key @kbd{C-M-q} is redefined as @code{fortran-indent-subprogram}, a
command that reindents all the lines of the Fortran subprogram (function or
subroutine) containing point.

@kindex M-LFD (Fortran mode)
@findex fortran-split-line
  The key @kbd{M-@key{LFD}} is redefined as @code{fortran-split-line}, a
command to split a line in the appropriate fashion for Fortran.  In a
non-comment line, the second half becomes a continuation line and is
indented accordingly.  In a comment line, both halves become separate
comment lines.

@node ForIndent Num, ForIndent Conv, ForIndent Commands, Fortran Indent
@subsubsection Line Numbers and Continuation

  If a number is the first non-whitespace in the line, it is assumed to be
a line number and is moved to columns 0 through 4.  (Columns are always
counted from 0 in GNU Emacs.)  If the text on the line starts with the
conventional Fortran continuation marker @samp{$}, it is moved to column 5.
If the text begins with any non whitespace character in column 5, it is
assumed to be an unconventional continuation marker and remains in column
5.

@vindex fortran-line-number-indent
  Line numbers of four digits or less are normally indented one space.
This amount is controlled by the variable @code{fortran-line-number-indent}
which is the maximum indentation a line number can have.  Line numbers
are indented to right-justify them to end in column 4 unless that would
require more than the maximum indentation.  The default value of the
variable is 1.

@vindex fortran-electric-line-number
  Simply inserting a line number is enough to indent it according to these
rules.  As each digit is inserted, the indentation is recomputed.  To turn
off this feature, set the variable @code{fortran-electric-line-number} to
@code{nil}.  Then inserting line numbers is like inserting anything else.

@node ForIndent Conv, ForIndent Vars, ForIndent Num, Fortran Indent
@subsubsection Syntactic Conventions

  Fortran mode assumes that you follow certain conventions that simplify
the task of understanding a Fortran program well enough to indent it
properly:

@vindex fortran-continuation-char
@itemize @bullet
@item
Two nested @samp{do} loops never share a @samp{continue} statement.

@item
The same character appears in column 5 of all continuation lines.  It
is the value of the variable @code{fortran-continuation-char}.
By default, this character is @samp{$}.
@end itemize

@noindent
If you fail to follow these conventions, the indentation commands may
indent some lines unaesthetically.  However, a correct Fortran program will
retain its meaning when reindented even if the conventions are not
followed.

@node ForIndent Vars,, ForIndent Conv, Fortran Indent
@subsubsection Variables for Fortran Indentation

@vindex fortran-do-indent
@vindex fortran-if-indent
@vindex fortran-continuation-indent
@vindex fortran-check-all-num-for-matching-do
@vindex fortran-minimum-statement-indent
  Several additional variables control how Fortran indentation works.

@table @code
@item fortran-do-indent
Extra indentation within each level of @samp{do} statement (default 3).

@item fortran-if-indent
Extra indentation within each level of @samp{if} statement (default 3).

@item fortran-continuation-indent
Extra indentation for bodies of continuation lines (default 5).

@item fortran-check-all-num-for-matching-do
If this is @code{nil}, indentation assumes that each @samp{do}
statement ends on a @samp{continue} statement.  Therefore, when
computing indentation for a statement other than @samp{continue}, it
can save time by not checking for a @samp{do} statement ending there.
If this is non-@code{nil}, indenting any numbered statement must check
for a @samp{do} that ends there.  The default is @code{nil}.

@item fortran-minimum-statement-indent
Minimum indentation for Fortran statements.  For standard Fortran,
this is 6.  Statement bodies are always indented at least this much.
@end table

@node Fortran Comments, Fortran Columns, Fortran Indent, Fortran
@subsection Comments

  The usual Emacs comment commands assume that a comment can follow a line
of code.  In Fortran, the standard comment syntax requires an entire line
to be just a comment.  Therefore, Fortran mode replaces the standard Emacs
comment commands and defines some new variables.

  Fortran mode can also handle a non-standard comment syntax where comments
start with @samp{!} and can follow other text.  Because only some Fortran
compilers accept this syntax, Fortran mode will not insert such comments
unless you have specified to do so in advance by setting the variable
@code{comment-start} to @samp{"!"} (@pxref{Variables}).

@table @kbd
@item M-;
Align comment or insert new comment (@code{fortran-comment-indent}).

@item C-x ;
Applies to nonstandard @samp{!} comments only.

@item C-c ;
Turn all lines of the region into comments, or (with arg)
turn them back into real code (@code{fortran-comment-region}).
@end table

  @kbd{M-;} in Fortran mode is redefined as the command
@code{fortran-comment-indent}.  Like the usual @kbd{M-;} command,
recognizes an existing comment and aligns its text appropriately.
If there is no existing comment, a comment is inserted and aligned.

Inserting and aligning comments is not the same in Fortran mode as in
other modes.  When a new comment must be inserted, a full-line comment is
inserted if the current line is blank.  On a non-blank line, a
non-standard @samp{!} comment is inserted if you previously specified
you wanted to use them.  Otherwise a full-line comment is inserted on a
new line before the current line.

  Non-standard @samp{!} comments are aligned like comments in other
languages, but full-line comments are aligned differently.  In a
standard full-line comment, the comment delimiter itself must always
appear in column zero.  What can be aligned is the text within the
comment.  You can choose from three styles of alignment by setting the
variable @code{fortran-comment-indent-style} to one of these values:

@vindex fortran-comment-indent-style
@vindex fortran-comment-line-column
@table @code
@item fixed
The text is aligned at a fixed column, which is the value of
@code{fortran-comment-line-column}.  This is the default.
@item relative
The text is aligned as if it were a line of code, but with an
additional @code{fortran-comment-line-column} columns of indentation.
@item nil
Text in full-line columns is not moved automatically.
@end table

@vindex fortran-comment-indent-char
  You can also specify the character to be used to indent within
full-line comments by setting the variable @code{fortran-comment-indent-char}
to the character you want to use.

@vindex comment-line-start
@vindex comment-line-start-skip
  Fortran mode introduces two variables @code{comment-line-start} and
@code{comment-line-start-skip} which play for full-line comments the same
roles played by @code{comment-start} and @code{comment-start-skip} for
ordinary text-following comments.  Normally these are set properly by
Fortran mode so you do not need to change them.

  The normal Emacs comment command @kbd{C-x ;} has not been redefined.
It can therefore be used if you use @samp{!} comments, but is useless in
Fortran mode otherwise. 

@kindex C-c ; (Fortran mode)
@findex fortran-comment-region
@vindex fortran-comment-region
  The command @kbd{C-c ;} (@code{fortran-comment-region}) turns all the
lines of the region into comments by inserting the string @samp{C$$$} at
the front of each one.  With a numeric arg, the region is turned back into
live code by deleting @samp{C$$$} from the front of each line.  You can
control the string used for the comments by setting the variable
@code{fortran-comment-region}.  Note that here we have an example of a
command and a variable with the same name; the two uses of the name never
conflict because in Lisp and in Emacs it is always clear from the context
which one is referred to.

@node Fortran Columns, Fortran Abbrev, Fortran Comments, Fortran
@subsection Columns

@table @kbd
@item C-c C-r
Displays a ``column ruler'' momentarily above the current line
(@code{fortran-column-ruler}).
@item C-c C-w
Splits the current window horizontally so that it is 72 columns wide.
This may help you avoid going over that limit (@code{fortran-window-create}).
@end table

@kindex C-c C-r (Fortran mode)
@findex fortran-column-ruler
@vindex fortran-column-ruler
  The command @kbd{C-c C-r} (@code{fortran-column-ruler}) shows a column
ruler above the current line.  The comment ruler consists of two lines
of text that show you the locations of columns with special significance
in Fortran programs.  Square brackets show the limits of the columns for
line numbers, and curly brackets show the limits of the columns for the
statement body.  Column numbers appear above them.

  Note that the column numbers count from zero, as always in GNU Emacs.  As
a result, the numbers may not be those you are familiar with; but the
actual positions in the line are standard Fortran.

  The text used to display the column ruler is the value of the variable
@code{fortran-comment-ruler}.  By changing this variable, you can change
the display.

@kindex C-c C-w (Fortran mode)
@findex fortran-window-create
  For even more help, use @kbd{C-c C-w} (@code{fortran-window-create}), a
command which splits the current window horizontally, resulting in a window 72
columns wide.  When you edit in this window, you can immediately see
when a line gets too wide to be correct Fortran.

@node Fortran Abbrev,, Fortran Columns, Fortran
@subsection Fortran Keyword Abbrevs

  Fortran mode provides many built-in abbrevs for common keywords and
declarations.  These are the same sort of abbrev that you can define
yourself.  To use them, you must turn on Abbrev mode.  @pxref{Abbrevs}.

  The built-in abbrevs are unusual in one way: they all start with a
semicolon.  You cannot normally use semicolon in an abbrev, but Fortran
mode makes this possible by changing the syntax of semicolon to ``word
constituent''.

  For example, one built-in Fortran abbrev is @samp{;c} for
@samp{continue}.  If you insert @samp{;c} and then insert a punctuation
character such as a space or a newline, the @samp{;c} changes
automatically to @samp{continue}, provided Abbrev mode is enabled.@refill

  Type @samp{;?} or @samp{;C-h} to display a list of all built-in
Fortran abbrevs and what they stand for.

@node Running, Abbrevs, Programs, Top
@chapter Compiling and Testing Programs

  The previous chapter discusses the Emacs commands that are useful for
making changes in programs.  This chapter deals with commands that assist
in the larger process of developing and maintaining programs.

@menu
* Compilation::        Compiling programs in languages other than Lisp
                        (C, Pascal, etc.)
* Modes: Lisp Modes.   Various modes for editing Lisp programs, with
                       different facilities for running the Lisp programs.
* Libraries: Lisp Libraries.      Creating Lisp programs to run in Emacs.
* Interaction: Lisp Interaction.  Executing Lisp in an Emacs buffer.
* Eval: Lisp Eval.     Executing a single Lisp expression in Emacs.
* Debug: Lisp Debug.   Debugging Lisp programs running in Emacs.
* External Lisp::      Communicating through Emacs with a separate Lisp.
@end menu

@node Compilation, Lisp Modes, Running, Running
@section Running `make', or Compilers Generally
@cindex inferior process
@cindex make
@cindex compilation errors
@cindex error log

  Emacs can run compilers for non-interactive languages like C and
Fortran as inferior processes, feeding the error log into an Emacs buffer.
It can also parse the error messages and visit the files in which errors
are found, moving point to the line where the error occurred.

@table @kbd
@item M-x compile
Run a compiler asynchronously under Emacs, with error messages to
@samp{*compilation*} buffer.
@item M-x grep
Run @code{grep} asynchronously under Emacs, with matching lines
listed in the buffer named @samp{*compilation*}.
@item M-x kill-compilation
Kill the process made by the M-x compile command.
@itemx M-x kill-grep
Kill the running compilation or @code{grep} subprocess.
@item C-x `
Visit the next compiler error message or @code{grep} match.
@end table

@findex compile
  To run @code{make} or another compiler, type @kbd{M-x compile}.  This
command reads a shell command line using the minibuffer, then executes
the specified command line in an inferior shell with output going to the
buffer named @samp{*compilation*}.  By default, the current buffer's
default directory is used as the working directory for the execution of
the command; therefore, the makefile comes from this directory.

@vindex compile-command
  When the shell command line is read, the minibuffer appears containing a
default command line (the command you used the last time you typed
@kbd{M-x compile}).  If you type just @key{RET}, the same command line is used
again.  The first @kbd{M-x compile} provides @code{make -k} as the default.
The default is taken from the variable @code{compile-command}; if the
appropriate compilation command for a file is something other than
@code{make -k}, it can be useful to have the file specify a local value for
@code{compile-command} (@pxref{File Variables}).

@cindex compiling files
  When you start a compilation, the buffer @samp{*compilation*} is
displayed in another window but not selected.  Its mode line displays
the word @samp{run} or @samp{exit} in the parentheses tells you whether
compilation is finished.  You do not have to keep this buffer visible;
compilation continues in any case.

@findex kill-compilation
  To kill the compilation process, type @kbd{M-x-compilation}.  The mode
line of the @samp{*compilation*} buffer changes to say @samp{signal}
instead of @samp{run}.  Starting a new compilation also kills any
running compilation, as only one can occur at any time.  Starting a new
compilation prompts for confirmation before actually killing a
compilation that is running.@refill

@kindex C-x `
@findex next-error
  To parse the compiler error messages, type @kbd{C-x `}
(@code{next-error}).  The character following @kbd{C-x} is the grave
accent, not the single quote.  The command displays the buffer
@samp{*compilation*} in one window and the buffer in which the next
error occurred in another window.  Point in that buffer is moved to the
line where the error was found.  The corresponding error message is
scrolled to the top of the window in which @samp{*compilation*} is
displayed.

  The first time you use @kbd{C-x `} after the start of a compilation, it
parses all the error messages, visits all the files that have error
messages, and creates markers pointing at the lines the error messages
refer to.  It then moves to the first error message location.  Subsequent
uses of @kbd{C-x `} advance down the data set up by the first use.  When
the preparsed error messages are exhausted, the next @kbd{C-x `} checks for
any more error messages that have come in; this is useful if you start
editing compiler errors while compilation is still going on.  If no
additional error messages have come in, @kbd{C-x `} reports an error.

  @kbd{C-u C-x `} discards the preparsed error message data and parses the
@samp{*compilation*} buffer again, then displays the first error.
This way, you can process the same set of errors again.

  Instead of running a compiler, you can run @code{grep} and see the
lines on which matches were found.  To do this, type @kbd{M-x grep} with
an argument line that contains the same arguments you would give to
@code{grep} a @code{grep}-style regexp (usually in single quotes to
quote the shell's special characters) followed by filenames, which may
use wildcard characters.  The output from @code{grep} goes in the
@samp{*compilation*} buffer.  You can use @kbd{C-x `} to find the lines that
match as if they were compilation errors.

  Note: a shell is used to run the compile command, but the shell is not
run in interactive mode.  This means in particular that the shell starts
up with no prompt.  If you find your usual shell prompt making an
unsightly appearance in the @samp{*compilation*} buffer, it means you
have made a mistake in your shell's initialization file (@file{.cshrc}
or @file{.shrc} or @dots{}) by setting the prompt unconditionally.  The
shell initialization file should set the prompt only if there already is
a prompt.  Here's how to do it in @code{csh}:

@example
if ($?prompt) set prompt = ...
@end example

@node Lisp Modes, Lisp Libraries, Compilation, Running
@section Major Modes for Lisp

  Emacs has four different major modes for Lisp.  They are the same in
terms of editing commands, but differ in the commands for executing Lisp
expressions.

@table @asis
@item Emacs-Lisp mode
The mode for editing source files of programs to run in Emacs Lisp.
This mode defines @kbd{C-M-x} to evaluate the current defun.
@xref{Lisp Libraries}.
@item Lisp Interaction mode
The mode for an interactive session with Emacs Lisp.  It defines
@key{LFD} to evaluate the sexp before point and insert its value in the
buffer.  @xref{Lisp Interaction}.
@item Lisp mode
The mode for editing source files of programs that run in other dialects
of Lisp than Emacs Lisp.  This mode defines @kbd{C-M-x} to send the
current defun to an inferior Lisp process.  @xref{External Lisp}.
@item Inferior Lisp mode
The mode for an interactive session with an inferior Lisp process.
This mode combines the special features of Lisp mode and Shell mode
(@pxref{Shell Mode}).
@item Scheme mode
Like Lisp mode but for Scheme programs.
@item Inferior Scheme mode
The mode for an interactive session with an inferior Scheme process.
@end table

@node Lisp Libraries, Lisp Eval, Lisp Modes, Running
@section Libraries of Lisp Code for Emacs
@cindex libraries
@cindex loading Lisp code

  Lisp code for Emacs editing commands is stored in files whose names
conventionally end in @file{.el}.  This ending tells Emacs to edit them in
Emacs-Lisp mode (@pxref{Lisp Modes}).

@menu
* Loading::		Loading libraries of Lisp code into Emacs for use.
* Compiling Libraries:: Compiling a library makes it load and run faster.
* Mocklisp::		Converting Mocklisp to Lisp so GNU Emacs can run it.
@end menu

@node Loading, Compiling Libraries, Lisp Libraries, Lisp Libraries
@subsection Loading Libraries

@table @kbd
@item M-x load-file @var{file}
Load the file @var{file} of Lisp code.
@item M-x load-library @var{library}
Load the library named @var{library}.
@item M-x locate-library @var{library} &optional @var{nosuffix}
Show the full path name of Emacs library @file{library}.
@end table

@findex load-file
  To execute a file of Emacs Lisp, use @kbd{M-x load-file}.  This
command reads the file name you provide in the minibuffer, then executes
the contents of that file as Lisp code.  It is not necessary to visit
the file first; in fact, this command reads the file as found on
disk, not the text in an Emacs buffer.

@findex load
@findex load-library
  Once a file of Lisp code is installed in the Emacs Lisp library
directories, users can load it using @kbd{M-x load-library}.  Programs can
load it by calling @code{load-library}, or with @code{load}, a more primitive
function that is similar but accepts some additional arguments.

  @kbd{M-x load-library} differs from @kbd{M-x load-file} in that it
searches a sequence of directories and tries three file names in each
directory.  The three names are: first, the specified name with @file{.elc}
appended; second, the name with @file{.el} appended; third, the specified
name alone.  A @file{.elc} file would be the result of compiling the Lisp
file into byte code;  if possible, it is loaded in preference to the Lisp
file itself because the compiled file loads and runs faster.

@cindex loading libraries
  Because the argument to @code{load-library} is usually not in itself
a valid file name, file name completion is not available.  In fact, when
using this command, you usually do not know exactly what file name
will be used.

@vindex load-path
  The sequence of directories searched by @kbd{M-x load-library} is
specified by the variable @code{load-path}, a list of strings that are
directory names.  The elements of this list may not begin with "@samp{~}",
so you must call @code{expand-file-name} on them before adding them to
the list.  The default value of the list contains the directory where
the Lisp code for Emacs itself is stored.  If you have libraries of your
own, put them in a single directory and add that directory to
@code{load-path}.  @code{nil} in this list stands for the current
default directory, but it is probably not a good idea to put @code{nil}
in the list.  If you start wishing that @code{nil} were in the list, you
should probably use @kbd{M-x load-file} for this case.

The variable is initialized by the @b{EMACSLOADPATH} environment
variable. If no value is specified, the variable takes the default value
specified in the file @file{paths.h} when Emacs was built. If a path
isn't specified in @file{paths.h}, a default value is obtained from the
file system, near the directory in which the Emacs executable resides.

@findex locate-library
 @kbd{M-x locate-library} searches the directories in @code{load-path}
like @kbd{M-x load-library} to find the file that @kbd{M-x load-library}
would load.  If the optional second argument @var{nosuffix} is
non-@code{nil}, the suffixes @file{.elc} or @file{.el} are not added to
the specified name @var{library} (a la calling load instead of
load-library).

@cindex autoload
   You often do not have to give any command to load a library, because the
commands defined in the library are set up to @dfn{autoload} that library.
Running any of those commands causes @code{load} to be called to load the
library; this replaces the autoload definitions with the real ones from the
library.

  If autoloading a file does not finish, either because of an error or
because of a @kbd{C-g} quit, all function definitions made by the file
are undone automatically.  So are any calls to @code{provide}.  As a
consequence, the entire file is loaded a second time if you use one of
the autoloadable commands again.  This prevents problems when the
command is no longer autoloading but works incorrectly because the file
was only partially loaded.  Function definitions are undone only for
autoloading; explicit calls to @code{load} do not undo anything if
loading is not completed.

@vindex after-load-alist
The variable @code{after-load-alist} takes an alist of expressions to be
evalled when particular files are loaded.  Each element looks like
@code{(@var{filename} forms...)}.  When load is run and the filename
argument is @var{filename}, the forms in the corresponding element are
executed at the end of loading.

@var{filename} must match exactly.  Normally @var{filename} is the
name of a library, with no directory specified, since that is how load
is normally called.  An error in @code{forms} does not undo the load, but
does prevent execution of the rest of the @code{forms}.

@node Compiling Libraries, Mocklisp, Loading, Lisp Libraries
@subsection Compiling Libraries

@cindex byte code
  Emacs Lisp code can be compiled into byte-code which loads faster,
takes up less space when loaded, and executes faster.

@table @kbd
@item M-x batch-byte-compile
Run byte-compile-file on the files remaining on the command line.
@item M-x byte-compile-buffer &optional @var{buffer}
Byte-compile and evaluate contents of @var{buffer} (default is current 
buffer).
@item M-x byte-compile-file
Compile a file of Lisp code named @var{filename} into a file of byte code.
@item M-x byte-compile-and-load-file @var{filename}
Compile a file of Lisp code named @var{filename} into a file of byte
code and load it.
@item M-x byte-recompile-directory @var{directory}
Recompile every @file{.el} file in @var{directory} that needs recompilation.
@item M-x disassemble
Print disassembled code for @var{object} on (optional) @var{stream}.
@item M-x make-obsolete @var{function new} 
Make the byte-compiler warn that @var{function} is obsolete and @var{new} 
should be used instead.
@end table

@findex byte-compile-file
@findex byte-compile-and-load-file
@findex byte-compile-buffer
 @kbd{byte-compile-file} creates a byte-code compiled file from an
Emacs-Lisp source file.  The default argument for this function is the
file visited in the current buffer.  The function reads the specified
file, compiles it into byte code, and writes an output file whose name
is made by appending @file{c} to the input file name.  Thus, the file
@file{rmail.el} would be compiled into @file{rmail.elc}. To compile a
file of Lisp code named @var{filename} into a file of byte code and
then load it, use @code{byte-compile-and-load-file}. To compile and
evaluate Lisp code in a given buffer, use @code{byte-compile-buffer}.

@findex byte-recompile-directory
  To recompile all changed Lisp files in a directory, use @kbd{M-x
byte-recompile-directory}.  Specify just the directory name as an argument.
Each @file{.el} file that has been byte-compiled before is byte-compiled
again if it has changed since the previous compilation.  A numeric argument
to this command tells it to offer to compile each @file{.el} file that has
not been compiled yet.  You must answer @kbd{y} or @kbd{n} to each
offer.

@findex batch-byte-compile
  You can use the function @code{batch-byte-compile} to invoke Emacs
non-interactively from the shell to do byte compilation.  When you use
this function, the files to be compiled are specified with command-line
arguments.  Use a shell command of the form

@example
emacs -batch -f batch-byte-compile @var{files}...
@end example

  Directory names may also be given as arguments; in that case,
@code{byte-recompile-directory} is invoked on each such directory.
@code{batch-byte-compile} uses all remaining command-line arguments as
file or directory names, then kills the Emacs process.

@findex disassemble
  @kbd{M-x disassemble} explains the result of byte compilation.  Its
argument is a function name.  It displays the byte-compiled code in a help
window in symbolic form, one instruction per line.  If the instruction
refers to a variable or constant, that is shown too.

@node Mocklisp,,Compiling Libraries,Lisp Libraries
@subsection Converting Mocklisp to Lisp

@cindex mocklisp
@findex convert-mocklisp-buffer
  GNU Emacs can run Mocklisp files by converting them to Emacs Lisp first.
To convert a Mocklisp file, visit it and then type @kbd{M-x
convert-mocklisp-buffer}.  Then save the resulting buffer of Lisp file in a
file whose name ends in @file{.el} and use the new file as a Lisp library.

  You cannot currently byte-compile converted Mocklisp code.
The reason is that converted Mocklisp code uses some special Lisp features
to deal with Mocklisp's incompatible ideas of how arguments are evaluated
and which values signify ``true'' or ``false''.

@node Lisp Eval, Lisp Debug, Lisp Libraries, Running
@section Evaluating Emacs-Lisp Expressions
@cindex Emacs-Lisp mode

@findex emacs-lisp-mode
  Lisp programs intended to be run in Emacs should be edited in
Emacs-Lisp mode; this will happen automatically for file names ending in
@file{.el}.  By contrast, Lisp mode itself should be used for editing
Lisp programs intended for other Lisp systems.  Emacs-Lisp mode can be
selected with the command @kbd{M-x emacs-lisp-mode}.

  For testing of Lisp programs to run in Emacs, it is useful to be able
to evaluate part of the program as it is found in the Emacs buffer.  For
example, if you change the text of a Lisp function definition, and then
evaluate the definition, Emacs installs the change for future calls to the
function.  Evaluation of Lisp expressions is also useful in any kind of
editing task for invoking non-interactive functions (functions that are
not commands).

@table @kbd
@item M-@key{ESC}
Read a Lisp expression in the minibuffer, evaluate it, and print the
value in the minibuffer (@code{eval-expression}).
@item C-x C-e
Evaluate the Lisp expression before point, and print the value in the
minibuffer (@code{eval-last-sexp}).
@item C-M-x
Evaluate the defun containing point or after point, and print the value in
the minibuffer (@code{eval-defun}).
@item M-x eval-region
Evaluate all the Lisp expressions in the region.
@item M-x eval-current-buffer
Evaluate all the Lisp expressions in the buffer.
@end table

@kindex M-ESC
@findex eval-expression
  @kbd{M-@key{ESC}} (@code{eval-expression}) is the most basic command
for evaluating a Lisp expression interactively.  It reads the expression
using the minibuffer, so you can execute any expression on a buffer
regardless of what the buffer contains.  When evaluation is complete,
the current buffer is once again the buffer that was current when
@kbd{M-@key{ESC}} was typed.

  @kbd{M-@key{ESC}} can easily confuse users, especially on keyboards
with autorepeat where it can result from holding down the @key{ESC} key
for too long.  Therefore, @code{eval-expression} is normally a disabled
command.  Attempting to use this command asks for confirmation and gives
you the option of enabling it; once you enable the command, you are no
longer required to confirm.  @xref{Disabling}.@refill

@kindex C-M-x
@findex eval-defun
  In Emacs-Lisp mode, the key @kbd{C-M-x} is bound to the function
@code{eval-defun}, which parses the defun containing point or following point
as a Lisp expression and evaluates it.  The value is printed in the echo
area.  This command is convenient for installing in the Lisp environment
changes that you have just made in the text of a function definition.

@kindex C-x C-e
@findex eval-last-sexp
  The command @kbd{C-x C-e} (@code{eval-last-sexp}) performs a similar job
but is available in all major modes, not just Emacs-Lisp mode.  It finds
the sexp before point, reads it as a Lisp expression, evaluates it, and
prints the value in the echo area.  It is sometimes useful to type in an
expression and then, with point still after it, type @kbd{C-x C-e}.

  If @kbd{C-M-x} or @kbd{C-x C-e} are given a numeric argument, they
print the value by inserting it into the current buffer at point, rather
than in the echo area.  The argument value does not matter.

@findex eval-region
@findex eval-current-buffer
  The most general command for evaluating Lisp expressions from a buffer
is @code{eval-region}.  @kbd{M-x eval-region} parses the text of the
region as one or more Lisp expressions, evaluating them one by one.
@kbd{M-x eval-current-buffer} is similar but evaluates the entire
buffer.  This is a reasonable way to install the contents of a file of
Lisp code that you are just ready to test.  After finding and fixing a
bug, use @kbd{C-M-x} on each function that you change, to keep the Lisp
world in step with the source file.

@node Lisp Debug, Lisp Interaction, Lisp Eval, Running
@section The Emacs-Lisp Debugger
@cindex debugger

@vindex debug-on-error
@vindex debug-on-quit
  GNU Emacs contains a debugger for Lisp programs executing inside it.
This debugger is normally not used; many commands frequently get Lisp
errors when invoked in inappropriate contexts (such as @kbd{C-f} at the
end of the buffer) and it would be unpleasant to enter a special
debugging mode in this case.  When you want to make Lisp errors invoke
the debugger, you must set the variable @code{debug-on-error} to
non-@code{nil}.  Quitting with @kbd{C-g} is not considered an error, and
@code{debug-on-error} has no effect on the handling of @kbd{C-g}.
However, if you set @code{debug-on-quit} non-@code{nil}, @kbd{C-g} will
invoke the debugger.  This can be useful for debugging an infinite loop;
type @kbd{C-g} once the loop has had time to reach its steady state.
@code{debug-on-quit} has no effect on errors.@refill

@findex debug-on-entry
@findex cancel-debug-on-entry
@findex debug
  You can make Emacs enter the debugger when a specified function
is called, or at a particular place in Lisp code.  Use @kbd{M-x
debug-on-entry} with argument @var{fun-name} to have Emacs enter the
debugger as soon as @var{fun-name} is called.Use
@kbd{M-x cancel-debug-on-entry} to make the function stop entering the
debugger when called.  (Redefining the function also does this.)  To enter
the debugger from some other place in Lisp code, you must insert the
expression @code{(debug)} there and install the changed code with
@kbd{C-M-x}.  @xref{Lisp Eval}.@refill

  When the debugger is entered, it displays the previously selected buffer
in one window and a buffer named @samp{*Backtrace*} in another window.  The
backtrace buffer contains one line for each level of Lisp function
execution currently going on.  At the beginning of the buffer is a message
describing the reason that the debugger was invoked, for example, an
error message if it was invoked due to an error.

  The backtrace buffer is read-only, and is in Backtrace mode, a special
major mode in which letters are defined as debugger commands.  The
usual Emacs editing commands are available; you can switch windows to
examine the buffer that was being edited at the time of the error, and
you can switch buffers, visit files, and perform any other editing
operations.  However, the debugger is a recursive editing level
(@pxref{Recursive Edit}); it is a good idea to return to the backtrace
buffer and explictly exit the debugger when you don't want to use it any
more.  Exiting the debugger kills the backtrace buffer.

@cindex current stack frame
  The contents of the backtrace buffer show you the functions that are
executing and the arguments that were given to them.  It also allows you
to specify a stack frame by moving point to the line describing that
frame.  The frame whose line point is on is considered the @dfn{current
frame}.  Some of the debugger commands operate on the current frame.
Debugger commands are mainly used for stepping through code one
expression at a time.  Here is a list of them:

@table @kbd
@item c
Exit the debugger and continue execution.  In most cases, execution of
the program continues as if the debugger had never been entered (aside
from the effect of any variables or data structures you may have changed
while inside the debugger).  This includes entry to the debugger due to
function entry or exit, explicit invocation, and quitting or certain
errors.  Most errors cannot be continued; trying to continue an error usually
causes the same error to occur again.
@item d
Continue execution, but enter the debugger the next time a Lisp
function is called.  This allows you to step through the
subexpressions of an expression, and see what the subexpressions do and
what values they compute.

When you enter the debugger this way, Emacs flags the stack frame for the
function call from which you entered.  The same function is then called
when you exit the frame.  To cancel this flag, use @kbd{u}.
@item b
Set up to enter the debugger when the current frame is exited.  Frames
that invoke the debugger on exit are flagged with stars.
@page
@item u
Don't enter the debugger when the current frame is exited.  This
cancels a @kbd{b} command on a frame.
@item e
Read a Lisp expression in the minibuffer, evaluate it, and print the
value in the echo area.  This is equivalent to the command @kbd{M-@key{ESC}},
except that @kbd{e} is not normally disabled like @kbd{M-@key{ESC}}.
@item q
Terminate the program being debugged; return to top-level Emacs
command execution.

If the debugger was entered due to a @kbd{C-g} but you really want
to quit, not to debug, use the @kbd{q} command.
@item r
Return a value from the debugger.  The value is computed by reading an
expression with the minibuffer and evaluating it.

The value returned by the debugger makes a difference when the debugger
was invoked due to exit from a Lisp call frame (as requested with @kbd{b});
then the value specified in the @kbd{r} command is used as the value of
that frame.

The debugger's return value also matters with many errors.  For example,
@code{wrong-type-argument} errors will use the debugger's return value
instead of the invalid argument; @code{no-catch} errors will use the
debugger value as a throw tag instead of the tag that was not found.
If an error was signaled by calling the Lisp function @code{signal},
the debugger's return value is returned as the value of @code{signal}.
@end table

@node Lisp Interaction, External Lisp, Lisp Debug, Running
@section Lisp Interaction Buffers

  The buffer @samp{*scratch*}, which is selected when Emacs starts up, is
provided for evaluating Lisp expressions interactively inside Emacs.  Both
the expressions you evaluate and their output goes in the buffer.

  The @samp{*scratch*} buffer's major mode is Lisp Interaction mode, which
is the same as Emacs-Lisp mode except for one command, @key{LFD}.  In
Emacs-Lisp mode, @key{LFD} is an indentation command.  In Lisp
Interaction mode, @key{LFD} is bound to @code{eval-print-last-sexp}.  This
function reads the Lisp expression before point, evaluates it, and inserts
the value in printed representation before point.

 The way to use the @samp{*scratch*} buffer is to insert Lisp
expressions at the end, ending each one with @key{LFD} so that it will
be evaluated.  The result is a complete typescript of the expressions
you have evaluated and their values.

@findex lisp-interaction-mode
  The rationale for this feature is that Emacs must have a buffer when it
starts up, but that buffer is not useful for editing files since a new
buffer is made for every file that you visit.  The Lisp interpreter
typescript is the most useful thing I can think of for the initial buffer
to do.  @kbd{M-x lisp-interaction-mode} will put any buffer in Lisp
Interaction mode.

@node External Lisp,, Lisp Interaction, Running
@section Running an External Lisp

  Emacs has facilities for running programs in other Lisp systems.  You can
run a Lisp process as an inferior of Emacs, and pass expressions to it to
be evaluated.  You can also pass changed function definitions directly from
the Emacs buffers in which you edit the Lisp programs to the inferior Lisp
process.

@findex run-lisp
  To run an inferior Lisp process, type @kbd{M-x run-lisp}.  This runs the
program named @code{lisp}, the same program you would run by typing
@code{lisp} as a shell command, with both input and output going through an
Emacs buffer named @samp{*lisp*}.  In other words, any ``terminal output''
from Lisp will go into the buffer, advancing point, and any ``terminal
input'' for Lisp comes from text in the buffer.  To give input to Lisp, go
to the end of the buffer and type the input, terminated by @key{RET}.  The
@samp{*lisp*} buffer is in Inferior Lisp mode, which has all the
special characteristics of Lisp mode and Shell mode (@pxref{Shell Mode}).

@findex lisp-mode
  Use Lisp mode to run the source files of programs in external Lisps.
You can select this mode with @kbd{M-x lisp-mode}.  It is used automatically
for files whose names end in @file{.l} or @file{.lisp}, as most Lisp
systems usually expect.

@kindex C-M-x
@findex lisp-send-defun
  When you edit a function in a Lisp program you are running, the easiest
way to send the changed definition to the inferior Lisp process is the key
@kbd{C-M-x}.  In Lisp mode, this key runs the function @code{lisp-send-defun},
which finds the defun around or following point and sends it as input to
the Lisp process.  (Emacs can send input to any inferior process regardless
of what buffer is current.)

  Contrast the meanings of @kbd{C-M-x} in Lisp mode (for editing programs
to be run in another Lisp system) and Emacs-Lisp mode (for editing Lisp
programs to be run in Emacs): in both modes it has the effect of installing
the function definition that point is in, but the way of doing so is
different according to where the relevant Lisp environment is found.
@xref{Lisp Modes}.

@node Abbrevs, Picture, Running, Top
@chapter Abbrevs
@cindex abbrevs
@cindex expansion (of abbrevs)

  An @dfn{abbrev} is a word which @dfn{expands}, if you insert it, into some
different text.  Abbrevs are defined by the user to expand in specific
ways.  For example, you might define @samp{foo} as an abbrev expanding to
@samp{find outer otter}.  With this abbrev defined, you would be able to
get @samp{find outer otter } into the buffer by typing @kbd{f o o @key{SPC}}.

@findex abbrev-mode
@vindex abbrev-mode
  Abbrevs expand only when Abbrev mode (a minor mode) is enabled.
Disabling Abbrev mode does not cause abbrev definitions to be discarded,
but they do not expand until Abbrev mode is enabled again.  The command
@kbd{M-x abbrev-mode} toggles Abbrev mode; with a numeric argument, it
turns Abbrev mode on if the argument is positive, off otherwise.
@xref{Minor Modes}.  @code{abbrev-mode} is also a variable; Abbrev mode is
on when the variable is non-@code{nil}.  The variable @code{abbrev-mode}
automatically becomes local to the current buffer when it is set.

  Abbrev definitions can be @dfn{mode-specific}---active only in one major
mode.  Abbrevs can also have @dfn{global} definitions that are active in
all major modes.  The same abbrev can have a global definition and various
mode-specific definitions for different major modes.  A mode specific
definition for the current major mode overrides a global definition.

 You can define Abbrevs interactively during an editing session.  You
can also save lists of abbrev definitions in files and reload them in later
sessions.  Some users keep extensive lists of abbrevs that they load in
every session.

  A second kind of abbreviation facility is called the @dfn{dynamic
expansion}.  Dynamic abbrev expansion happens only when you give an
explicit command and the result of the expansion depends only on the
current contents of the buffer.  @xref{Dynamic Abbrevs}.

@menu
* Defining Abbrevs::  Defining an abbrev, so it will expand when typed.
* Expanding Abbrevs:: Controlling expansion: prefixes, canceling expansion.
* Editing Abbrevs::   Viewing or editing the entire list of defined abbrevs.
* Saving Abbrevs::    Saving the entire list of abbrevs for another session.
* Dynamic Abbrevs::   Abbreviations for words already in the buffer.
@end menu

@node Defining Abbrevs, Expanding Abbrevs, Abbrevs, Abbrevs
@section Defining Abbrevs

@table @kbd
@item C-x +
Define an abbrev to expand into some text before point
(@code{add-global-abbrev}).
@item C-x C-a
Similar, but define an abbrev available only in the current major mode
(@code{add-mode-abbrev}).
@item C-x -
Define a word in the buffer as an abbrev (@code{inverse-add-global-abbrev}).
@item C-x C-h
Define a word in the buffer as a mode-specific abbrev
(@code{inverse-add-mode-abbrev}).
@item M-x kill-all-abbrevs
After this command, no abbrev definitions remain in effect.
@end table

@kindex C-x +
@findex add-global-abbrev
  The usual way to define an abbrev is to enter the text you want the
abbrev to expand to, position point after it, and type @kbd{C-x +}
(@code{add-global-abbrev}).  This reads the abbrev itself using the
minibuffer, and then defines it as an abbrev for one or more words
before point.  Use a numeric argument to say how many words before point
should be taken as the expansion.  For example, to define the abbrev
@samp{foo} as in the example above, insert the text @samp{find outer
otter}, then type @*@kbd{C-u 3 C-x + f o o @key{RET}}.

  An argument of zero to @kbd{C-x +} means to use the contents of the
region as the expansion of the abbrev being defined.

@kindex C-x C-a
@findex add-mode-abbrev
  The command @kbd{C-x C-a} (@code{add-mode-abbrev}) is similar, but
defines a mode-specific abbrev.  Mode specific abbrevs are active only in a
particular major mode.  @kbd{C-x C-a} defines an abbrev for the major mode
in effect at the time @kbd{C-x C-a} is typed.  The arguments work the
same way they do for @kbd{C-x +}.

@kindex C-x -
@findex inverse-add-global-abbrev
@kindex C-x C-h
@findex inverse-add-mode-abbrev
  If the text of an abbrev you want is already in the buffer instead of
the expansion, use command @kbd{C-x -} (@code{inverse-add-global-abbrev})
instead of @kbd{C-x +}, or use @kbd{C-x C-h}
(@code{inverse-add-mode-abbrev}) instead of @kbd{C-x C-a}.  These commands
are called ``inverse'' because they invert the meaning of the argument
found in the buffer and the argument read using the minibuffer.@refill

  To change the definition of an abbrev, just add the new definition.  You
will be asked to confirm if the abbrev has a prior definition.  To remove
an abbrev definition, give a negative argument to @kbd{C-x +} or @kbd{C-x
C-a}.  You must choose the command to specify whether to kill a global
definition or a mode-specific definition for the current mode, since those
two definitions are independent for one abbrev.

@findex kill-all-abbrevs
  @kbd{M-x kill-all-abbrevs} removes all existing abbrev definitions.

@node Expanding Abbrevs, Editing Abbrevs, Defining Abbrevs, Abbrevs
@section Controlling Abbrev Expansion

  An abbrev expands whenever it is in a buffer just before point and you
type and a self-inserting punctuation character (@key{SPC}, comma,
etc.@:).  Most often an abbrev is used by inserting the abbrev followed
by punctuation.

@vindex abbrev-all-caps
  Abbrev expansion preserves case; thus, @samp{foo} expands into @samp{find
outer otter}; @samp{Foo} into @samp{Find outer otter}, and @samp{FOO} into
@samp{FIND OUTER OTTER} or @samp{Find Outer Otter} according to the
variable @code{abbrev-all-caps} (a non-@code{nil} value chooses the first
of the two expansions).@refill

   Two commands are available to control abbrev expansion:

@table @kbd
@item M-'
Separate a prefix from a following abbrev to be expanded
(@code{abbrev-prefix-mark}).
@item C-x '
@findex expand-abbrev
Expand the abbrev before point (@code{expand-abbrev}).
This is effective even when Abbrev mode is not enabled.
@item M-x unexpand-abbrev
Undo last abbrev expansion.
@item M-x expand-region-abbrevs
Expand some or all abbrevs found in the region.
@end table

@kindex M-'
@findex abbrev-prefix-mark
  You may wish to expand an abbrev with a prefix attached.  For example,
if @samp{cnst} expands into @samp{construction}, you may want to use it
to enter @samp{reconstruction}.  It does not work to type @kbd{recnst},
because that is not necessarily a defined abbrev.  Instead, you can use
the command @kbd{M-'} (@code{abbrev-prefix-mark}) between the prefix
@samp{re} and the abbrev @samp{cnst}.  First, insert @samp{re}.  Then
type @kbd{M-'}; this inserts a minus sign in the buffer to indicate that
it has done its work.  Then insert the abbrev @samp{cnst}.  The buffer
now contains @samp{re-cnst}.  Now insert a punctuation character to
expand the abbrev @samp{cnst} into @samp{construction}.  The minus sign
is deleted at this point, by @kbd{M-'}.  The resulting text is the
desired @samp{reconstruction}.@refill

  If you actually want the text of the abbrev in the buffer, rather than
its expansion, insert the following punctuation with @kbd{C-q}.  Thus,
@kbd{foo C-q -} leaves @samp{foo-} in the buffer.

@findex unexpand-abbrev
  If you expand an abbrev by mistake, you can undo the expansion (replace
the expansion by the original abbrev text) with @kbd{M-x unexpand-abbrev}.
You can also use @kbd{C-_} (@code{undo}) to undo the expansion; but that
will first undo the insertion of the punctuation character.

@findex expand-region-abbrevs
  @kbd{M-x expand-region-abbrevs} searches through the region for defined
abbrevs, and  offers to replace each one it finds with its expansion.
This command is useful if you have typed text using abbrevs but forgot
to turn on Abbrev mode first.  It may also be useful together with a
special set of abbrev definitions for making several global replacements at
once.  The command is effective even if Abbrev mode is not enabled.

@node Editing Abbrevs, Saving Abbrevs, Expanding Abbrevs, Abbrevs
@section Examining and Editing Abbrevs

@table @kbd
@item M-x list-abbrevs
Print a list of all abbrev definitions.
@item M-x edit-abbrevs
Edit a list of abbrevs; you can add, alter or remove definitions.
@end table

@findex list-abbrevs
  The output from @kbd{M-x list-abbrevs} looks like this:

@example
(lisp-mode-abbrev-table)
"dk"	       0    "define-key"
(global-abbrev-table)
"dfn"	       0    "definition"
@end example

@noindent
(Some blank lines of no semantic significance, and some other abbrev
tables, have been omitted.)

  A line containing a name in parentheses is the header for abbrevs in a
particular abbrev table; @code{global-abbrev-table} contains all the global
abbrevs, and the other abbrev tables that are named after major modes
contain the mode-specific abbrevs.

  Within each abbrev table, each non-blank line defines one abbrev.  The
word at the beginning is the abbrev.  The number that appears is the number
of times the abbrev has been expanded.  Emacs keeps track of this to help
you see which abbrevs you actually use, in case you want to eliminate
those that you don't use often.  The string at the end of the line is the
expansion.

@findex edit-abbrevs
@kindex C-c C-c (Edit Abbrevs)
@findex edit-abbrevs-redefine
  @kbd{M-x edit-abbrevs} allows you to add, change or kill abbrev
definitions by editing a list of them in an Emacs buffer.  The list has
the format described above.  The buffer of abbrevs is called
@samp{*Abbrevs*}, and is in Edit-Abbrevs mode.  This mode redefines the
key @kbd{C-c C-c} to install the abbrev definitions as specified in the
buffer.  The  @code{edit-abbrevs-redefine} command does this.
Any abbrevs not described in the buffer are eliminated when this is
done.

  @code{edit-abbrevs} is actually the same as @code{list-abbrevs} except
that it selects the buffer @samp{*Abbrevs*} whereas @code{list-abbrevs}
merely displays it in another window.

@node Saving Abbrevs, Dynamic Abbrevs, Editing Abbrevs, Abbrevs
@section Saving Abbrevs

  These commands allow you to keep abbrev definitions between editing
sessions.

@table @kbd
@item M-x write-abbrev-file
Write a file describing all defined abbrevs.
@item M-x read-abbrev-file
Read such an abbrev file and define abbrevs as specified there.
@item M-x quietly-read-abbrev-file
Similar, but do not display a message about what is going on.
@item M-x define-abbrevs
Define abbrevs from buffer.
@item M-x insert-abbrevs
Insert all abbrevs and their expansions into the buffer.
@end table

@findex write-abbrev-file
  Use @kbd{M-x write-abbrev-file} to save abbrev definitions for use in
a later session.  The command reads a file name using the minibuffer and
writes a description of all current abbrev definitions into the
specified file.  The text stored in the file looks like the output of
@kbd{M-x list-abbrevs}.


@findex read-abbrev-file
@findex quietly-read-abbrev-file
@vindex abbrev-file-name
  @kbd{M-x read-abbrev-file} prompts for a file name using the
minibuffer and reads the specified file, defining abbrevs according to
its contents.  @kbd{M-x quietly-read-abbrev-file} is the same but does
not display a message in the echo area; it is actually useful primarily
in the @file{.emacs} file.  If you give an empty argument to either of
these functions, the file name Emacs uses is the value of the variable
@code{abbrev-file-name}, which is by default @code{"~/.abbrev_defs"}.

@vindex save-abbrevs
  Emacs offers to save abbrevs automatically if you have changed any of
them, whenever it offers to save all files (for @kbd{C-x s} or @kbd{C-x
C-c}).  Set the variable @code{save-abbrevs} to @code{nil} to inhibit
this feature.

@findex insert-abbrevs
@findex define-abbrevs
  The commands @kbd{M-x insert-abbrevs} and @kbd{M-x define-abbrevs} are
similar to the previous commands but work on text in an Emacs buffer.
@kbd{M-x insert-abbrevs} inserts text into the current buffer before point,
describing all current abbrev definitions; @kbd{M-x define-abbrevs} parses
the entire current buffer and defines abbrevs accordingly.@refill

@node Dynamic Abbrevs,, Saving Abbrevs, Abbrevs
@section Dynamic Abbrev Expansion

  The abbrev facility described above operates automatically as you insert
text, but all abbrevs must be defined explicitly.  By contrast,
@dfn{dynamic abbrevs} allow the meanings of abbrevs to be determined
automatically from the contents of the buffer, but dynamic abbrev expansion
happens only when you request it explicitly.

@kindex M-/
@findex dabbrev-expand
@table @kbd
@item M-/
Expand the word in the buffer before point as a @dfn{dynamic abbrev},
by searching in the buffer for words starting with that abbreviation
(@code{dabbrev-expand}).
@end table

  For example, if the buffer contains @samp{does this follow } and you type
@kbd{f o M-/}, the effect is to insert @samp{follow} because that is the
last word in the buffer that starts with @samp{fo}.  A numeric argument to
@kbd{M-/} says to take the second, third, etc.@: distinct expansion found
looking backward from point.  Repeating @kbd{M-/} searches for an
alternative expansion by looking farther back.  After the entire buffer
before point has been considered, the buffer after point is searched.

  Dynamic abbrev expansion is completely independent of Abbrev mode; the
expansion of a word with @kbd{M-/} is completely independent of whether it
has a definition as an ordinary abbrev.

@node Picture, Sending Mail, Abbrevs, Top
@chapter Editing Pictures
@cindex pictures
@findex edit-picture

  If you want to create a picture made out of text characters (for example,
a picture of the division of a register into fields, as a comment in a
program), use the command @code{edit-picture} to enter Picture mode.

  In Picture mode, editing is based on the @dfn{quarter-plane} model of
text.  In this model, the text characters lie studded on an area that
stretches infinitely far to the right and downward.  The concept of the end
of a line does not exist in this model; the most you can say is where the
last non-blank character on the line is found.

  Of course, Emacs really always considers text as a sequence of
characters, and lines really do have ends.  But in Picture mode most
frequently-used keys are rebound to commands that simulate the
quarter-plane model of text.  They do this by inserting spaces or by
converting tabs to spaces.

  Most of the basic editing commands of Emacs are redefined by Picture mode
to do essentially the same thing but in a quarter-plane way.  In addition,
Picture mode defines various keys starting with the @kbd{C-c} prefix to
run special picture editing commands.

  One of these keys, @kbd{C-c C-c}, is pretty important.  Often a picture
is part of a larger file that is usually edited in some other major mode.
@kbd{M-x edit-picture} records the name of the previous major mode. 
You can then use the @kbd{C-c C-c} command (@code{picture-mode-exit}) to
restore that mode.  @kbd{C-c C-c} also deletes spaces from the ends of
lines, unless you give it a numeric argument.

  The commands used in Picture mode all work in other modes (provided the
@file{picture} library is loaded), but are only  bound to keys in
Picture mode.  Note that the descriptions below talk of moving ``one
column'' and so on, but all the picture mode commands handle numeric
arguments as their normal equivalents do.

@vindex picture-mode-hook
  Turning on Picture mode calls the value of the variable
@code{picture-mode-hook} as a function, with no arguments, if that value
exists and is non-@code{nil}.

@menu
* Basic Picture::         Basic concepts and simple commands of Picture Mode.
* Insert in Picture::     Controlling direction of cursor motion
                           after "self-inserting" characters.
* Tabs in Picture::       Various features for tab stops and indentation.
* Rectangles in Picture:: Clearing and superimposing rectangles.
@end menu

@node Basic Picture, Insert in Picture, Picture, Picture
@section Basic Editing in Picture Mode

@findex picture-forward-column
@findex picture-backward-column
@findex picture-move-down
@findex picture-move-up
  Most keys do the same thing in Picture mode that they usually do, but do
it in a quarter-plane style.  For example, @kbd{C-f} is rebound to run
@code{picture-forward-column}, which moves point one column to
the right, by inserting a space if necessary, so that the actual end of the
line makes no difference.  @kbd{C-b} is rebound to run
@code{picture-backward-column}, which always moves point left one column,
converting a tab to multiple spaces if necessary.  @kbd{C-n} and @kbd{C-p}
are rebound to run @code{picture-move-down} and @code{picture-move-up},
which can either insert spaces or convert tabs as necessary to make sure
that point stays in exactly the same column.  @kbd{C-e} runs
@code{picture-end-of-line}, which moves to after the last non-blank
character on the line.  There was no need to change @kbd{C-a}, as the choice
of screen model does not affect beginnings of lines.@refill

@findex picture-newline
  Insertion of text is adapted to the quarter-plane screen model through
the use of Overwrite mode (@pxref{Minor Modes}).  Self-inserting characters
replace existing text, column by column, rather than pushing existing text
to the right.  @key{RET} runs @code{picture-newline}, which just moves to
the beginning of the following line so that new text will replace that
line.

@findex picture-backward-clear-column
@findex picture-clear-column
@findex picture-clear-line
  Text is erased instead of deleted and killed.  @key{DEL}
(@code{picture-backward-clear-column}) replaces the preceding character
with a space rather than removing it.  @kbd{C-d}
(@code{picture-clear-column}) does the same in a forward direction.
@kbd{C-k} (@code{picture-clear-line}) really kills the contents of lines,
but never removes the newlines from a buffer.@refill

@findex picture-open-line
  To do actual insertion, you must use special commands.  @kbd{C-o}
(@code{picture-open-line}) creates a blank line, but does so after
the current line; it never splits a line.  @kbd{C-M-o}, @code{split-line},
makes sense in Picture mode, so it remains unchanged.  @key{LFD}
(@code{picture-duplicate-line}) inserts another line
with the same contents below the current line.@refill

@kindex C-c C-d (Picture mode)
@findex delete-char
 
  To actually delete parts of the picture, use @kbd{C-w}, or with
@kbd{C-c C-d} (which is defined as @code{delete-char}, as @kbd{C-d} is
in other modes), or with one of the picture rectangle commands
(@pxref{Rectangles in Picture}).

@node Insert in Picture, Tabs in Picture, Basic Picture, Picture
@section Controlling Motion after Insert

@findex picture-movement-up
@findex picture-movement-down
@findex picture-movement-left
@findex picture-movement-right
@findex picture-movement-nw
@findex picture-movement-ne
@findex picture-movement-sw
@findex picture-movement-se
@kindex C-c < (Picture mode)
@kindex C-c > (Picture mode)
@kindex C-c ^ (Picture mode)
@kindex C-c . (Picture mode)
@kindex C-c ` (Picture mode)
@kindex C-c ' (Picture mode)
@kindex C-c / (Picture mode)
@kindex C-c \ (Picture mode)
  Since ``self-inserting'' characters just overwrite and move point in
Picture mode, there is no essential restriction on how point should be
moved.  Normally point moves right, but you can specify any of the eight
orthogonal or diagonal directions for motion after a ``self-inserting''
character.  This is useful for drawing lines in the buffer.

@table @kbd
@item C-c <
Move left after insertion (@code{picture-movement-left}).
@item C-c >
Move right after insertion (@code{picture-movement-right}).
@item C-c ^
Move up after insertion (@code{picture-movement-up}).
@item C-c .
Move down after insertion (@code{picture-movement-down}).
@item C-c `
Move up and left (``northwest'') after insertion @*(@code{picture-movement-nw}).
@item C-c '
Move up and right (``northeast'') after insertion @*
(@code{picture-movement-ne}).
@item C-c /
Move down and left (``southwest'') after insertion
@*(@code{picture-movement-sw}).
@item C-c \
Move down and right (``southeast'') after insertion
@*(@code{picture-movement-se}).
@end table

@kindex C-c C-f (Picture mode)
@kindex C-c C-b (Picture mode)
@findex picture-motion
@findex picture-motion-reverse
  Two motion commands move based on the current Picture insertion
direction.  The command @kbd{C-c C-f} (@code{picture-motion}) moves in the
same direction as motion after ``insertion'' currently does, while @kbd{C-c
C-b} (@code{picture-motion-reverse}) moves in the opposite direction.

@node Tabs in Picture, Rectangles in Picture, Insert in Picture, Picture
@section Picture Mode Tabs
 
@kindex M-TAB
@findex picture-tab-search
@vindex picture-tab-chars
  Two kinds of tab-like action are provided in Picture mode.
Context-based tabbing is done with @kbd{M-@key{TAB}}
(@code{picture-tab-search}).  With no argument, it moves to a point
underneath the next ``interesting'' character that follows whitespace in
the previous non-blank line.  ``Next'' here means ``appearing at a
horizontal position greater than the one point starts out at''.  With an
argument, as in @kbd{C-u M-@key{TAB}}, the command moves to the next such
interesting character in the current line.  @kbd{M-@key{TAB}} does not
change the text; it only moves point.  ``Interesting'' characters are
defined by the variable @code{picture-tab-chars}, which contains a string
of characters considered interesting.  Its default value is
@code{"!-~"}.@refill

@findex picture-tab
  @key{TAB} itself runs @code{picture-tab}, which operates based on the
current tab stop settings; it is the Picture mode equivalent of
@code{tab-to-tab-stop}.  Without arguments it just moves point, but with
a numeric argument it clears the text that it moves over.

@kindex C-c TAB (Picture mode)
@findex picture-set-tab-stops
  The context-based and tab-stop-based forms of tabbing are brought
together by the command @kbd{C-c @key{TAB}}, @code{picture-set-tab-stops}.
This command sets the tab stops to the positions which @kbd{M-@key{TAB}}
would consider significant in the current line.  If you use this command,
together with @key{TAB}, you can get the effect of context-based tabbing.  But
@kbd{M-@key{TAB}} is more convenient in the cases where it is sufficient.

@page
@node Rectangles in Picture,, Tabs in Picture, Picture
@section Picture Mode Rectangle Commands
@cindex rectangle

  Picture mode defines commands for working on rectangular pieces of the
text in ways that fit with the quarter-plane model.  The standard rectangle
commands may also be useful (@pxref{Rectangles}).

@table @kbd
@item C-c C-k
Clear out the region-rectangle (@code{picture-clear-rectangle}).  With
argument, kill it.
@item C-c C-w @var{r}
Similar but save rectangle contents in register @var{r} first
(@code{picture-clear-rectangle-to-register}).
@item C-c C-y
Copy last killed rectangle into the buffer by overwriting, with upper
left corner at point (@code{picture-yank-rectangle}).  With argument,
insert instead.
@item C-c C-x @var{r}
Similar, but use the rectangle in register @var{r}@*
(@code{picture-yank-rectangle-from-register}).
@end table

@kindex C-c C-k (Picture mode)
@kindex C-c C-w (Picture mode)
@findex picture-clear-rectangle
@findex picture-clear-rectangle-to-register
  The picture rectangle commands @kbd{C-c C-k}
(@code{picture-clear-rectangle}) and @kbd{C-c C-w}
(@code{picture-clear-rectangle-to-register}) differ from the standard
rectangle commands in that they normally clear the rectangle instead of
deleting it; this is analogous with the way @kbd{C-d} is changed in Picture
mode.@refill

  However, deletion of rectangles can be useful in Picture mode, so these
commands delete the rectangle if given a numeric argument.

@kindex C-c C-y (Picture mode)
@kindex C-c C-x (Picture mode)
@findex picture-yank-rectangle
@findex picture-yank-rectangle-from-register
  The Picture mode commands for yanking rectangles differ from the standard
ones in overwriting instead of inserting.  This is the same way that
Picture mode insertion of other text is different from other modes.
@kbd{C-c C-y} (@code{picture-yank-rectangle}) inserts (by overwriting) the
rectangle that was most recently killed, while @kbd{C-c C-x}
(@code{picture-yank-rectangle-from-register}) does for the
rectangle found in a specified register.

@node Sending Mail, Rmail, Picture, Top
@chapter Sending Mail
@cindex mail
@cindex message

  To send a message in Emacs, start by typing the command (@kbd{C-x m})
to select and initialize the @samp{*mail*} buffer.  You can then edit the text
and headers of the message in the mail buffer, and type the command
(@kbd{C-c C-c}) to send the message.

@table @kbd
@item C-x m
Begin composing a message to send (@code{mail}).
@item C-x 4 m
Likewise, but display the message in another window
(@code{mail-other-window}).
@item C-c C-c
In Mail mode, send the message and switch to another buffer
(@code{mail-send-and-exit}).
@end table

@kindex C-x m
@findex mail
@kindex C-x 4 m
@findex mail-other-window
  The command @kbd{C-x m} (@code{mail}) selects a buffer named
@samp{*mail*} and initializes it with the skeleton of an outgoing message.
@kbd{C-x 4 m} (@code{mail-other-window}) selects the @samp{*mail*} buffer
in a different window, leaving the previous current buffer visible.@refill

  Because the buffer for mail composition is an ordinary Emacs buffer, you can
switch to other buffers while in the middle of composing mail, and switch
back later (or never).  If you use the @kbd{C-x m} command again when you
have been composing another message but have not sent it, a new mail
buffer will be created; in this way, you can compose multiple messages
at once.  You can switch back to and complete an unsent message by using
the normal buffer selection mechanisms.  

@kbd{C-u C-x m} is another way to switch back to a message in progress:
it will search for an existing, unsent mail message buffer and select it.

@menu
* Format: Mail Format.    Format of the mail being composed.
* Headers: Mail Headers.  Details of allowed mail header fields.
* Mode: Mail Mode.        Special commands for editing mail being composed.
@end menu

@node Mail Format, Mail Headers, Sending Mail, Sending Mail
@section The Format of the Mail Buffer

  In addition to the @dfn{text} or contents, a message has @dfn{header
fields} which say who sent it, when, to whom, why, and so on.  Some header
fields such as the date and sender are created automatically after the
message is sent.  Others, such as the recipient names, must be specified by
you in order to send the message properly.

  Mail mode provides a few commands to help you edit some header fields,
and some are preinitialized in the buffer automatically at times.  You can
insert or edit any header fields using ordinary editing commands.

  The line in the buffer that says

@example
--text follows this line--
@end example

@vindex mail-header-separator
@noindent
is a special delimiter that separates the headers you have specified from
the text.  Whatever follows this line is the text of the message; the
headers precede it.  The delimiter line itself does not appear in the
message actually sent.  The text used for the delimiter line is controlled
by the variable @code{mail-header-separator}.

Here is an example of what the headers and text in the @samp{*mail*} buffer
might look like.

@example
To: rms@@mc
CC: mly@@mc, rg@@oz
Subject: The Emacs Manual
--Text follows this line--
Please ignore this message.
@end example

@node Mail Headers, Mail Mode, Mail Format, Sending Mail
@section Mail Header Fields
@cindex headers (of mail message)

  There are several header fields you can use in the @samp{*mail*} buffer.
Each header field starts with a field name at the beginning of a line,
terminated by a colon.  It does not matter whether you use upper or lower
case in the field name.  After the colon and optional whitespace comes the
contents of the field.

@table @samp
@item To
This field contains the mailing addresses of the message.

@item Subject
The contents of the @samp{Subject} field should be a piece of text that
says what the message is about.  Subject fields are useful because most
mail-reading programs can provide a summary of messages, listing the
subject of each message but not its text.

@item CC
This field contains additional mailing addresses to send the message
to, but whose readers should not regard the message as addressed to
them.

@item BCC
This field contains additional mailing addresses to send the message
to, but which should not appear in the header of the message actually
sent.

@item FCC
This field contains the name of one file (in Unix mail file format) to
which a copy of the message should be appended when the message is
sent.

@item From
Use the @samp{From} field to say who you are, when the account you are
using to send the mail is not your own.  The contents of the
@samp{From} field should be a valid mailing address, since replies
will normally go there.

@page
@item Reply-To
Use the @samp{Reply-to} field to direct replies to a different
address, not your own. @samp{From} and
@samp{Reply-to} have the same effect on where replies go, but they convey a
different meaning to the person who reads the message.

@item In-Reply-To
This field contains a piece of text describing a message you are
replying to.  Some mail systems can use the information to correlate
related pieces of mail.  Normally this field is filled in by Rmail
when you are replying to a message in Rmail, and you never need to
think about it (@pxref{Rmail}).
@end table

@noindent
The @samp{To}, @samp{CC}, @samp{BCC} and @samp{FCC} fields can appear
any number of times, to specify many places to send the message.

@noindent
The @samp{To}, @samp{CC}, and @samp{BCC} fields can have continuation
lines.  All the lines starting with whitespace, following the line on
which the field starts, are considered part of the field.  For
example,@refill

@example
To: foo@@here, this@@there,
  me@@gnu.cambridge.mass.usa.earth.spiral3281
@end example

@noindent
@vindex mail-abbrev-mailrc-file
If you have a @file{~/.mailrc} file, Emacs scans it for mail aliases the
first time you try to send mail in an Emacs session.  Emacs expands
aliases found in the @samp{To}, @samp{CC}, and @samp{BCC} fields where
appropriate. You can set the variable @code{mail-abbrev-mailrc-file} to
the name of the file with mail aliases.  If @code{nil}, @file{~/.mailrc}
is used.

@cindex .mailrc file
Your @file{.mailrc} file ensures that word-abbrevs are defined for each
of your mail aliases when point is in a @samp{To}, @samp{CC},
@samp{BCC}, or @samp{From} field.  The aliases are defined in your
@file{.mailrc} file or in a file specified by the @code{MAILRC}
environment variable if it exists.  Your mail aliases expand any time
you type a word-delimiter at the end of an abbreviation.

In this version of Emacs, what you see is what you get: in contrast to
some other versions, no abbreviations are expanded after you have sent the
mail.  This means you don't suffer the annoyance of having the system do
things behind your back --- if the system rewrites an address you typed,
you know it immediately, instead of after the mail has been sent and
it's too late to do anything about it.  For example, you will never
again be in trouble because you forgot to delete an old alias from your
@file{.mailrc} and a new local user is given a userid which conflicts
with one of your aliases.

@vindex mail-abbrev-mode-regexp 
Your mail alias abbrevs are in effect only when point is in an
appropriate header field. The mail aliases will not expand in the body
of the message, or in other header fields.  The default mode-specific
abbrev table @code{mail-mode-abbrev-table} is used instead if defined.
That means if you have been using mail-mode specific abbrevs, this code
will not adversely affect you.  You can control which header fields the
abbrevs are used in by changing the variable @code{mail-abbrev-mode-regexp}.

If auto-fill mode is on, abbrevs wrap at commas instead of at word
boundaries and header continuation lines will be properly indented.

@findex mail-interactive-insert-alias
You can also insert a mail alias with @code{mail-interactive-insert-alias}.
This function, which is bound to @kbd{C-c C-a}, prompts you for an alias
(with completion) and inserts its expansion at point.

In this version of Emacs, it is possible to have lines like the
following in your @file{.mailrc} file:

@example
     alias someone "John Doe <doe@@quux.com>"
@end example

That is, if you want an address to have embedded spaces, simply surround
it with double-quotes.  The quotes are necessary because the format of
the @file{.mailrc} file uses spaces as address delimiters.  

Aliases in the mailrc file may be nested. For example, assume you define
aliases like:
@example
     alias group1 fred ethel
     alias group2 larry curly moe
     alias everybody group1 group2
@end example

When you now type @samp{everybody} on the @samp{To} line, it will expand to
@example
     fred, ethyl, larry, curly, moe
@end example

Aliases may contain forward references; the alias of @samp{everybody} in the
example above can preceed the aliases of @samp{group1} and @samp{group2}.

In this version of Emacs, you can use the @code{source} .mailrc command
for reading aliases from some other file as well.

Aliases may contain hyphens, as in @code{"alias foo-bar foo@@bar"} even
though word-abbrevs normally cannot contain hyphens.

To read in the contents of another .mailrc-type file from Emacs, use the
command @code{M-x merge-mail-aliases}.  The @code{rebuild-mail-aliases}
command is similar, but deletes existing aliases first.

@page
If you would like your aliases to be expanded when you type @kbd{M->} or
@kbd{C-n} to move from the mail-header into the message body, instead of
having to type @key{SPC} at the end of the abbrev before moving away,
use the following code:

@example
 (setq mail-setup-hook
      '(lambda ()
         (define-key mail-mode-map "\C-n"
           'abbrev-hacking-next-line)
         (define-key mail-mode-map "\M->"
           'abbrev-hacking-end-of-buffer)))
@end example

@vindex mail-alias-seperator-string
If you want multiple addresses seperated by a string other than ", " then
you set the variable @code{mail-alias-seperator-string} to it.  This has to
be a comma bracketed by whitespace if you want any kind of reasonable
behavior.

@vindex mail-archive-file-name
  If the variable @code{mail-archive-file-name} is non-@code{nil}, it
should be a string naming a file.  Each time you start to edit a message
to send, an @samp{FCC} field is entered for that file.  Unless you
remove the @samp{FCC} field, every message is written into that
file when it is sent.

@node Mail Mode,, Mail Headers, Sending Mail
@section Mail Mode

  The major mode used in the @samp{*mail*} buffer is Mail mode.  Mail
mode is similar to Text mode, but several commands are provided on
the @kbd{C-c} prefix.  These commands all deal specifically with
editing or sending the message.

@table @kbd
@item C-c C-s
Send the message, and leave the @samp{*mail*} buffer selected
(@code{mail-send}).
@item C-c C-c
Send the message, and select some other buffer (@code{mail-send-and-exit}).
@item C-c C-f C-t
Move to the @samp{To} header field, creating one if there is none
(@code{mail-to}).
@item C-c C-f C-s
Move to the @samp{Subject} header field, creating one if there is
none (@code{mail-subject}).
@item C-c C-f C-c
Move to the @samp{CC} header field, creating one if there is none
(@code{mail-cc}).
@item C-c C-w
Insert the file @file{~/.signature} at the end of the message text
(@code{mail-signature}).
@item C-c C-y
Yank the selected message from Rmail (@code{mail-yank-original}).
This command does nothing unless your command to start sending a
message was issued with Rmail.
@item C-c C-q
Fill all paragraphs of yanked old messages, each individually
(@code{mail-fill-yanked-message}).
@item @key{button3}
Pops up a menu of useful mail-mode commands.
@end table

@kindex C-c C-s (Mail mode)
@kindex C-c C-c (Mail mode)
@findex mail-send
@findex mail-send-and-exit
  There are two ways to send a message.  @kbd{C-c C-c}
(@code{mail-send-and-exit}) is the usual way to send the message.  It
sends the message and then deletes the window (if there is another
window) or switches to another buffer.  It puts the @samp{*mail*} buffer
at the lowest priority for automatic reselection, since you are finished
with using it.  @kbd{C-c C-s} (@code{mail-send}) sends the
message and marks the @samp{*mail*} buffer unmodified, but leaves that
buffer selected so that you can modify the message (perhaps with new
recipients) and send it again.

@kindex C-c C-f C-t (Mail mode)
@findex mail-to
@kindex C-c C-f C-s (Mail mode)
@findex mail-subject
@kindex C-c C-f C-c (Mail mode)
@findex mail-cc
  Mail mode provides some other special commands that are useful for
editing the headers and text of the message before you send it.  There are
three commands defined to move point to particular header fields, all based
on the prefix @kbd{C-c C-f} (@samp{C-f} is for ``field'').  They are
@kbd{C-c C-f C-t} (@code{mail-to}) to move to the @samp{To} field, @kbd{C-c
C-f C-s} (@code{mail-subject}) for the @samp{Subject} field, and @kbd{C-c
C-f C-c} (@code{mail-cc}) for the @samp{CC} field.  These fields have
special motion commands because they are edited most frequently. 


@kindex C-c C-w (Mail mode)
@findex mail-signature
  @kbd{C-c C-w} (@code{mail-signature}) adds a standard piece of text at
the end of the message to say more about who you are.  The text comes
from the file @file{.signature} in your home directory.

@kindex C-c C-y (Mail mode)
@findex mail-yank-original
  When you use an Rmail command to send mail from the Rmail mail reader,
you can use @kbd{C-c C-y} inside the @samp{*mail*} buffer to insert the
text of the message you are replying to.  Normally Rmail indents each line
of that message four spaces and eliminates most header fields.  A
numeric argument specifies the number of spaces to indent.  An argument
of just @kbd{C-u} says not to indent at all and not to eliminate
anything.  @kbd{C-c C-y} always uses the current message from the
@samp{RMAIL} buffer, so you can insert several old messages by selecting
one in @samp{RMAIL}, switching to @samp{*mail*} and yanking it, then
switching back to @samp{RMAIL} to select another.@refill

@kindex C-c C-q (Mail mode)
@findex mail-fill-yanked-message
  After using @kbd{C-c C-y}, you can use the command @kbd{C-c C-q}
(@code{mail-fill-yanked-message}) to fill the paragraphs of the yanked
old message or messages.  One use of @kbd{C-c C-q} fills all such
paragraphs, each one separately.

  Clicking the right mouse button in a mail buffer pops up a menu of
the above commands, for easy access.

@vindex mail-mode-hook
  Turning on Mail mode (which @kbd{C-x m} does automatically) calls the
value of @code{text-mode-hook}, if it is not void or @code{nil}, and
then calls the value of @code{mail-mode-hook} if that is not void or
@code{nil}.

@node Rmail, Recursive Edit, Sending Mail, Top
@chapter Reading Mail with Rmail
@cindex Rmail
@cindex message
@findex rmail

  Rmail is an Emacs subsystem for reading and disposing of mail that you
receive.  Rmail stores mail messages in files called Rmail files.  You read
the messages in an Rmail file in a special major mode, Rmail mode,
which redefines most letters to run commands for managing mail.  To enter
Rmail, type @kbd{M-x rmail}.  This reads your primary mail file, merges
new mail in from your inboxes, displays the first new message, and lets
you begin reading.

@cindex primary mail file
  Using Rmail in the simplest fashion, you have one Rmail file
@file{~/RMAIL} in which all of your mail is saved.  It is called your
@dfn{primary mail file}.  You can also copy messages into other Rmail
files and then edit those files with Rmail.

  Rmail displays only one message at a time.  It is called the
@dfn{current message}.  Rmail mode's special commands can move to
another message, delete the message, copy the message into another file,
or send a reply.

@cindex message number
  Within the Rmail file, messages are arranged sequentially in order
of receipt.  They are also assigned consecutive integers as their
@dfn{message numbers}.  The number of the current message is displayed
in Rmail's mode line, followed by the total number of messages in the
file.  You can move to a message by specifying its message number
using the @kbd{j} key (@pxref{Rmail Motion}).

@kindex s (Rmail)
@findex rmail-save
  Following the usual conventions of Emacs, changes in an Rmail file become
permanent only when the file is saved.  You can do this with @kbd{s}
(@code{rmail-save}), which also expunges deleted messages from the file
first (@pxref{Rmail Deletion}).  To save the file without expunging, use
@kbd{C-x C-s}.  Rmail saves the Rmail file automatically when moving new
mail from an inbox file (@pxref{Rmail Inbox}).

@kindex q (Rmail)
@findex rmail-quit
  You can exit Rmail with @kbd{q} (@code{rmail-quit}); this expunges and
saves the Rmail file and then switches to another buffer.  But there is
no need to `exit' formally.  If you switch from Rmail to editing in
other buffers, and never happen to switch back, you have exited.  Just
make sure to save the Rmail file eventually (like any other file you
have changed).  @kbd{C-x s} is a good enough way to do this
(@pxref{Saving}).

@menu
* Scroll: Rmail Scrolling.   Scrolling through a message.
* Motion: Rmail Motion.      Moving to another message.
* Deletion: Rmail Deletion.  Deleting and expunging messages.
* Inbox: Rmail Inbox.        How mail gets into the Rmail file.
* Files: Rmail Files.        Using multiple Rmail files.
* Output: Rmail Output.	     Copying message out to files.
* Labels: Rmail Labels.      Classifying messages by labeling them.
* Summary: Rmail Summary.    Summaries show brief info on many messages.
* Reply: Rmail Reply.        Sending replies to messages you are viewing.
* Editing: Rmail Editing.    Editing message text and headers in Rmail.
* Digest: Rmail Digest.      Extracting the messages from a digest message.
@end menu

@node Rmail Scrolling, Rmail Motion, Rmail, Rmail
@section Scrolling Within a Message

  When Rmail displays a message that does not fit on the screen, you
have to scroll through it.  You could use @kbd{C-v}, @kbd{M-v}
and @kbd{M-<}, but scrolling is so frequent in Rmail that it deserves to be
easier to type.

@table @kbd
@item @key{SPC}
Scroll forward (@code{scroll-up}).
@item @key{DEL}
Scroll backward (@code{scroll-down}).
@item .
Scroll to start of message (@code{rmail-beginning-of-message}).
@end table

@kindex SPC (Rmail)
@kindex DEL (Rmail)
  Since the most common thing to do while reading a message is to scroll
through it by screenfuls, Rmail makes @key{SPC} and @key{DEL} synonyms of
@kbd{C-v} (@code{scroll-up}) and @kbd{M-v} (@code{scroll-down})

@kindex . (Rmail)
@findex rmail-beginning-of-message
  The command @kbd{.} (@code{rmail-beginning-of-message}) scrolls back to the
beginning of a selected message.  This is not quite the same as @kbd{M-<}:
first, it does not set the mark; secondly, it resets the buffer
boundaries to the current message if you have changed them.

@node Rmail Motion, Rmail Deletion, Rmail Scrolling, Rmail
@section Moving Among Messages

  The most basic thing to do with a message is to read it.  The way to
do this in Rmail is to make the message current.  You can make any
message current given its message number using the @kbd{j} command, but
people most often move sequentially through the file, since this is the
order of receipt of messages.  When you enter Rmail, you are positioned
at the first new message (new messages are those received after you last
used Rmail), or at the last message if there are no new messages this
time.  Move forward to see other new messages if there are any; move
backward to re-examine old messages.

@table @kbd
@item n
Move to the next non-deleted message, skipping any intervening deleted @*
messages (@code{rmail-next-undeleted-message}).
@item p
Move to the previous non-deleted message @*
(@code{rmail-previous-undeleted-message}).
@item M-n
Move to the next message, including deleted messages
(@code{rmail-next-message}).
@item M-p
Move to the previous message, including deleted messages
(@code{rmail-previous-message}).
@item j
Move to the first message.  With argument @var{n}, move to
message number @var{n} (@code{rmail-show-message}).
@item >
Move to the last message (@code{rmail-last-message}).

@item M-s @var{regexp} @key{RET}
Move to the next message containing a match for @var{regexp}
(@code{rmail-search}).  If @var{regexp} is empty, the last regexp used is
used again.

@item - M-s @var{regexp} @key{RET}
Move to the previous message containing a match for @var{regexp}.
If @var{regexp} is empty, the last regexp used is used again.
@end table

@kindex n (Rmail)
@kindex p (Rmail)
@kindex M-n (Rmail)
@kindex M-p (Rmail)
@findex rmail-next-undeleted-message
@findex rmail-previous-undeleted-message
@findex rmail-next-message
@findex rmail-previous-message
  To move among messages in Rmail, you can use @kbd{n} and @kbd{p}.
These keys move through the messages sequentially, but skip over deleted
messages, which is usually what you want to do.  Their command
definitions are named @code{rmail-next-undeleted-message} and
@code{rmail-previous-undeleted-message}.  If you do not want to skip
deleted messages---for example, if you want to move to a message to
undelete it---use the variants @kbd{M-n} (@code{rmail-next-message} and
@kbd{M-p} @code{rmail-previous-message}).  A numeric argument to any of
these commands serves as a repeat count.@refill

  In Rmail, you can specify a numeric argument by just typing the digits.
It is not necessary to type @kbd{C-u} first.

@kindex M-s (Rmail)
@findex rmail-search
  The @kbd{M-s} (@code{rmail-search}) command is Rmail's version of
search.  The usual incremental search command @kbd{C-s} works in Rmail,
but searches only within the current message.  The purpose of @kbd{M-s}
is to search for another message.  It reads a regular expression
(@pxref{Regexps}) non-incrementally, then searches starting at the
beginning of the following message for a match.  The message containing
the match is selected.

  To search backward in the file for another message, give @kbd{M-s} a
negative argument.  In Rmail you can do this with @kbd{- M-s}.

  It is also possible to search for a message based on labels.
@xref{Rmail Labels}.

@kindex j (Rmail)
@kindex > (Rmail)
@findex rmail-show-message
@findex rmail-last-message
  To move to a message specified by absolute message number, use @kbd{j}
(@code{rmail-show-message}) with the message number as argument.  With
no argument, @kbd{j} selects the first message.  @kbd{>}
(@code{rmail-last-message}) selects the last message.

@node Rmail Deletion, Rmail Inbox, Rmail Motion, Rmail
@section Deleting Messages

@cindex deletion (Rmail)
  When you no longer need to keep a message, you can @dfn{delete} it.  This
flags it as ignorable, and some Rmail commands will pretend it is no longer
present; but it still has its place in the Rmail file, and still has its
message number.

@cindex expunging (Rmail)
  @dfn{Expunging} the Rmail file actually removes the deleted messages.
The remaining messages are renumbered consecutively.  Expunging is the only
action that changes the message number of any message, except for
undigestifying (@pxref{Rmail Digest}).

@table @kbd
@item d
Delete the current message, and move to the next non-deleted message
(@code{rmail-delete-forward}).
@item C-d
Delete the current message, and move to the previous non-deleted
message (@code{rmail-delete-backward}).
@item u
Undelete the current message, or move back to a deleted message and
undelete it (@code{rmail-undelete-previous-message}).
@itemx x
@item e
Expunge the Rmail file (@code{rmail-expunge}).  These two
commands are synonyms.
@end table

@kindex d (Rmail)
@kindex C-d (Rmail)
@findex rmail-delete-forward
@findex rmail-delete-backward
  There are two Rmail commands for deleting messages.  Both delete the
current message and select another message.  @kbd{d}
(@code{rmail-delete-forward}) moves to the following message, skipping
messages already deleted, while @kbd{C-d} (@code{rmail-delete-backward})
moves to the previous non-deleted message.  If there is no non-deleted
message to move to in the specified direction, the message that was just
deleted remains current.

@cindex undeletion (Rmail)
@kindex e (Rmail)
@findex rmail-expunge
  To make all deleted messages disappear from the Rmail file, type
@kbd{e} (@code{rmail-expunge}).  Until you do this, you can still
@dfn{undelete} the deleted messages.

@kindex u (Rmail)
@findex rmail-undelete-previous-message
  To undelete, type
@kbd{u} (@code{rmail-undelete-previous-message}), which cancels the
effect of a @kbd{d} command (usually).  It undeletes the current message
if the current message is deleted.  Otherwise it moves backward to previous
messages until a deleted message is found, and undeletes that message.

  You can usually undo a @kbd{d} with a @kbd{u} because the @kbd{u}
moves back to and undeletes the message that the @kbd{d} deleted.  This
does not work when the @kbd{d} skips a few already-deleted messages that
follow the message being deleted; in that case the @kbd{u} command
undeletes the last of the messages that were skipped.  There is no clean
way to avoid this problem.  However, by repeating the @kbd{u} command,
you can eventually get back to the message you intended to
undelete.  You can also reach that message with @kbd{M-p} commands and
then type @kbd{u}.@refill

  A deleted message has the @samp{deleted} attribute, and as a result
@samp{deleted} appears in the mode line when the current message is
deleted.  In fact, deleting or undeleting a message is nothing more than
adding or removing this attribute.  @xref{Rmail Labels}.

@node Rmail Inbox, Rmail Files, Rmail Deletion, Rmail
@section Rmail Files and Inboxes
@cindex inbox file

  Unix places your incoming mail in a file called your @dfn{inbox}.
When you start up Rmail, it copies the new messages from your inbox into
your primary mail file, an Rmail file which also contains other messages
saved from previous Rmail sessions.  In this file, you actually
read the mail with Rmail.  The operation is called @dfn{getting new mail}.
You can repeat it at any time using the @kbd{g} key in Rmail.  The inbox
file name is @file{/usr/spool/mail/@var{username}} in Berkeley Unix,
@file{/usr/mail/@var{username}} in system V.

  There are two reason for having separate Rmail files and inboxes.

@enumerate
@item
The format in which Unix delivers the mail in the inbox is not
adequate for Rmail mail storage.  It has no way to record attributes
(such as @samp{deleted}) or user-specified labels; it has no way to record
old headers and reformatted headers; it has no way to record cached
summary line information.

@item
It is very cumbersome to access an inbox file without danger of losing
mail, because it is necessary to interlock with mail delivery.
Moreover, different Unix systems use different interlocking
techniques.  The strategy of moving mail out of the inbox once and for
all into a separate Rmail file avoids the need for interlocking in all
the rest of Rmail, since only Rmail operates on the Rmail file.
@end enumerate

  When getting new mail, Rmail first copies the new mail from the inbox
file to the Rmail file and saves the Rmail file.  It then deletes the
inbox file.  This way, a system crash may cause duplication of mail between
the inbox and the Rmail file, but cannot lose mail.

  Copying mail from an inbox in the system's mailer directory actually puts
it in an intermediate file @file{~/.newmail}.  This is because the
interlocking is done by a C program that copies to another file.
@file{~/.newmail} is deleted after mail merging is successful.  If there is
a crash at the wrong time, this file will continue to exist and will be
used as an inbox the next time you get new mail.

@node Rmail Files, Rmail Output, Rmail Inbox, Rmail
@section Multiple Mail Files

  Rmail operates by default on your @dfn{primary mail file}, which is named
@file{~/RMAIL} and receives your incoming mail from your system inbox file.
But you can also have other mail files and edit them with Rmail.  These
files can receive mail through their own inboxes, or you can move messages
into them by explicit command in Rmail (@pxref{Rmail Output}).

@table @kbd
@item i @var{file} @key{RET}
Read @var{file} into Emacs and run Rmail on it (@code{rmail-input}).

@item M-x set-rmail-inbox-list @key{RET} @var{files} @key{RET}
Specify inbox file names for current Rmail file to get mail from.

@item g
Merge new mail from current Rmail file's inboxes
(@code{rmail-get-new-mail}).

@item C-u g @var{file}
Merge new mail from inbox file @var{file}.
@end table

@kindex i (Rmail)
@findex rmail-input
  To run Rmail on a file other than your primary mail file, you may use
the @kbd{i} (@code{rmail-input}) command in Rmail.  This visits the
file, puts it in Rmail mode, and then gets new mail from the file's
inboxes if any.  You can also use @kbd{M-x rmail-input} even when not in
Rmail.

  The file you read with @kbd{i} does not have to be in Rmail file format.
It could also be Unix mail format, or mmdf format; or it could be a mixture
of all three, as long as each message has one of the three formats.
Rmail recognizes all three and converts all the messages to proper Rmail
format before showing you the file.

@findex set-rmail-inbox-list
  Each Rmail file can contain a list of inbox file names; you can specify
this list with @kbd{M-x set-rmail-inbox-list @key{RET} @var{files}
@key{RET}}.  The argument can contain any number of file names, separated
by commas.  It can also be empty, which specifies that this file should
have no inboxes.  Once a list of inboxes is specified, the Rmail file
remembers it permanently until it is explicitly changed.@refill

@kindex g (Rmail)
@findex rmail-get-new-mail
  If an Rmail file has inboxes, new mail is merged in from the inboxes
when you bring the Rmail file into Rmail, and when you use the @kbd{g}
(@code{rmail-get-new-mail}) command.  If the Rmail file
specifies no inboxes, then no new mail is merged in at these times.  A
special exception is made for your primary mail file: Rmail uses the
standard system inbox for it if it does not specify an inbox.

  To merge mail from a file that is not the usual inbox, give the
@kbd{g} key a numeric argument, as in @kbd{C-u g}.  Rmail prompts you
for a file name and merges mail from the file you specify.  The inbox
file is not deleted or changed in any way when you use @kbd{g} with an
argument.  This is, therefore, a general way of merging one file
of messages into another.

@node Rmail Output, Rmail Labels, Rmail Files, Rmail
@section Copying Messages Out to Files

@table @kbd
@item o @var{file} @key{RET}
Append a copy of the current message to the file @var{file},
writing it in Rmail file format (@code{rmail-output-to-rmail-file}).

@item C-o @var{file} @key{RET}
Append a copy of the current message to the file @var{file},
writing it in Unix mail file format (@code{rmail-output}).
@end table

@kindex o (Rmail)
@findex rmail-output-to-rmail-file
@kindex C-o (Rmail)
@findex rmail-output
  If an Rmail file has no inboxes, use explicit @kbd{o} commands to
write Rmail files.

  @kbd{o} (@code{rmail-output-to-rmail-file}) appends the current
message in Rmail format to the end of a specified file.  This is the
best command to use to move messages between Rmail files.  If you are
currently visiting the other Rmail file, copying is done into the other
file's Emacs buffer instead.  You should eventually save the buffer on
disk.

  The @kbd{C-o} (@code{rmail-output}) command in Rmail appends a copy of
the current message to a specified file, in Unix mail file format.  This
is useful for moving messages into files to be read by other mail
processors that do not understand Rmail format.

  Copying a message with @kbd{o} or @kbd{C-o} gives the original copy of the
message the @samp{filed} attribute, @samp{filed} appears in the mode
line when such a message is current.

  Normally you should use only @kbd{o} to output messages to other Rmail
files, never @kbd{C-o}.  But it is also safe if you always use @kbd{C-o},
never @kbd{o}.  When a file is visited in Rmail, the last message is
checked, and if it is in Unix format, the entire file is scanned and all
Unix-format messages are converted to Rmail format.  (The reason for
checking the last message is that scanning the file is slow and most Rmail
files have only Rmail format messages.)  If you use @kbd{C-o} consistently,
the last message is guaranteed to be in Unix format, so Rmail will convert all
messages properly.

When you and other users want to append mail to the same file, you
probably always want to use @kbd{C-o} instead of @kbd{o}.  Other mail
processors may not know Rmail format but will know Unix format.

  In any case, always use @kbd{o} to add to an Rmail file that is being
visited in Rmail.  Adding messages with @kbd{C-o} to the actual disk file
will trigger a ``simultaneous editing'' warning when you ask to save the
Emacs buffer, and the messages will be lost if you do save.

@node Rmail Labels, Rmail Summary, Rmail Output, Rmail
@section Labels
@cindex label (Rmail)
@cindex attribute (Rmail)

  Each message can have various @dfn{labels} assigned to it as a means of
classification.  A label has a name; different names mean different labels.
Any given label is either present or absent on a particular message.  A few
label names have standard meanings and are given to messages automatically
by Rmail when appropriate; these special labels are called @dfn{attributes}.
All other labels are assigned by the user.

@table @kbd
@item a @var{label} @key{RET}
Assign the label @var{label} to the current message (@code{rmail-add-label}).
@item k @var{label} @key{RET}
Remove the label @var{label} from the current message (@code{rmail-kill-label}).
@item C-M-n @var{labels} @key{RET}
Move to the next message that has one of the labels @var{labels}
(@code{rmail-next-labeled-message}).
@item C-M-p @var{labels} @key{RET}
Move to the previous message that has one of the labels @var{labels}
(@code{rmail-previous-labeled-message}).
@item C-M-l @var{labels} @key{RET}
Make a summary of all messages containing any of the labels @var{labels}
(@code{rmail-summary-by-labels}).
@end table

@noindent
Specifying an empty string for one these commands means to use the last
label specified for any of these commands.

@kindex a (Rmail)
@kindex k (rmail)
@findex rmail-add-label
@findex rmail-kill-label
  The @kbd{a} (@code{rmail-add-label}) and @kbd{k}
(@code{rmail-kill-label}) commands allow you to assign or remove any
label on the current message.  If the @var{label} argument is empty, it
means to assign or remove the label most recently assigned or
removed.

  Once you have given messages labels to classify them as you wish, there
are two ways to use the labels: in moving and in summaries.

@kindex C-M-n (Rmail)
@kindex C-M-p (Rmail)
@findex rmail-next-labeled-message
@findex rmail-previous-labeled-message
  The command @kbd{C-M-n @var{labels} @key{RET}}
(@code{rmail-next-labeled-message}) moves to the next message that has one
of the labels @var{labels}.  @var{labels} is one or more label names,
separated by commas.  @kbd{C-M-p} (@code{rmail-previous-labeled-message})
is similar, but moves backwards to previous messages.  A preceding numeric
argument to either one serves as a repeat count.@refill

@kindex C-M-l (Rmail)
@findex rmail-summary-by-labels
  The command @kbd{C-M-l @var{labels} @key{RET}}
(@code{rmail-summary-by-labels}) displays a summary containing only the
messages that have at least one of a specified set of messages.  The
argument @var{labels} is one or more label names, separated by commas.
@xref{Rmail Summary}, for information on summaries.@refill

  If the @var{labels} argument to @kbd{C-M-n}, @kbd{C-M-p} or
@kbd{C-M-l} is empty, it means to use the last set of labels specified
for any of these commands.

  Some labels such as @samp{deleted} and @samp{filed} have built-in
meanings and are assigned to or removed from messages automatically at
appropriate times; these labels are called @dfn{attributes}.  Here is a
list of Rmail attributes:

@table @samp
@item unseen
Means the message has never been current.  Assigned to messages when
they come from an inbox file, and removed when a message is made
current.
@item deleted
Means the message is deleted.  Assigned by deletion commands and
removed by undeletion commands (@pxref{Rmail Deletion}).
@item filed
Means the message has been copied to some other file.  Assigned by the
file output commands (@pxref{Rmail Files}).
@item answered
Means you have mailed an answer to the message.  Assigned by the @kbd{r}
command (@code{rmail-reply}).  @xref{Rmail Reply}.
@item forwarded
Means you have forwarded the message to other users.  Assigned by the
@kbd{f} command (@code{rmail-forward}).  @xref{Rmail Reply}.
@item edited
Means you have edited the text of the message within Rmail.
@xref{Rmail Editing}.
@end table

  All other labels are assigned or removed only by the user, and it is up
to the user to decide what they mean.

@node Rmail Summary, Rmail Reply, Rmail Labels, Rmail
@section Summaries
@cindex summary (Rmail)

  A @dfn{summary} is a buffer Rmail creates and displays to give you an
overview of the mail in an Rmail file.  It contains one line per message;
each line shows the message number, the sender, the labels, and the
subject.  When you select the summary buffer, you can use a number of
commands to select messages by moving in the summary buffer, or to
delete or undelete messages.

  A summary buffer applies to a single Rmail file only; if you are
editing multiple Rmail files, they have separate summary buffers.  The
summary buffer name is generated by appending @samp{-summary} to the
Rmail buffer's name.  Only one summary buffer is displayed at a
time unless you make several windows and select the summary buffers by
hand.

@menu
* Rmail Make Summary::  Making various sorts of summaries.
* Rmail Summary Edit::  Manipulating messages from the summary.
@end menu

@node Rmail Make Summary, Rmail Summary Edit, Rmail Summary, Rmail Summary
@subsection Making Summaries

  Here are the commands to create a summary for the current Rmail file.
Summaries do not update automatically; to make an updated summary, you
must use one of the commands again.

@table @kbd
@item h
@itemx C-M-h
Summarize all messages (@code{rmail-summary}).
@item l @var{labels} @key{RET}
@itemx C-M-l @var{labels} @key{RET}
Summarize message that have one or more of the specified labels
(@code{rmail-summary-by-labels}).
@item C-M-r @var{rcpts} @key{RET}
Summarize messages that have one or more of the specified recipients
(@code{rmail-summary-by-recipients})
@end table

@kindex h (Rmail)
@findex rmail-summary
  The @kbd{h} or @kbd{C-M-h} (@code{rmail-summary}) command fills the
summary buffer for the current Rmail file with a summary of all the
messages in the file.  It then displays and selects the summary buffer
in another window.

@kindex l (Rmail)
@kindex C-M-l (Rmail)
@findex rmail-summary-by-labels
  @kbd{C-M-l @var{labels} @key{RET}} (@code{rmail-summary-by-labels}) makes
a partial summary mentioning only the messages that have one or more of the
labels @var{labels}.  @var{labels} should contain label names separated by
commas.@refill

@kindex C-M-r (Rmail)
@findex rmail-summary-by-recipients
  @kbd{C-M-r @var{rcpts} @key{RET}} (@code{rmail-summary-by-recipients})
makes a partial summary mentioning only the messages that have one or more
of the recipients @var{rcpts}.  @var{rcpts} should contain mailing
addresses separated by commas.@refill

  Note that there is only one summary buffer for any Rmail file; making one
kind of summary discards any previously made summary.

@node Rmail Summary Edit,, Rmail Make Summary, Rmail Summary
@subsection Editing in Summaries

  Summary buffers are given the major mode Rmail Summary mode, which
provides the following special commands:

@table @kbd
@item j
Select the message described by the line that point is on
(@code{rmail-summary-goto-msg}).
@item C-n
Move to next line and select its message in Rmail
(@code{rmail-summary-next-all}).
@item C-p
Move to previous line and select its message
(@code{rmail-summary-@*previous-all}).
@item n
Move to next line, skipping lines saying `deleted', and select its
message (@code{rmail-summary-next-msg}).
@item p
Move to previous line, skipping lines saying `deleted', and select
its message (@code{rmail-summary-previous-msg}).
@item d
Delete the current line's message, then do like @kbd{n}
(@code{rmail-summary-delete-forward}).
@item u
Undelete and select this message or the previous deleted message in
the summary (@code{rmail-summary-undelete}).
@item @key{SPC}
Scroll the other window (presumably Rmail) forward
(@code{rmail-summary-scroll-msg-up}).
@item @key{DEL}
Scroll the other window backward (@code{rmail-summary-scroll-msg-down}).
@item x
Kill the summary window (@code{rmail-summary-exit}).
@item q
Exit Rmail (@code{rmail-summary-quit}).
@end table

@kindex C-n (Rmail summary)
@kindex C-p (Rmail summary)
@findex rmail-summary-next-all
@findex rmail-summary-previous-all
  The keys @kbd{C-n} and @kbd{C-p} are modified in Rmail Summary mode.
In addition to moving point in the summary buffer they also cause
the line's message to become current in the associated Rmail buffer.
That buffer is also made visible in another window if it is not
currently visible.

@kindex n (Rmail summary)
@kindex p (Rmail summary)
@findex rmail-summary-next-msg
@findex rmail-summary-previous-msg
  @kbd{n} and @kbd{p} are similar to @kbd{C-n} and @kbd{C-p}, but skip
lines that say `message deleted'.  They are like the @kbd{n} and @kbd{p}
keys of Rmail itself.  Note, however, that in a partial summary these
commands move only among the message listed in the summary.@refill

@kindex j (Rmail summary)
@findex rmail-summary-goto-msg
  The other Emacs cursor motion commands are not changed in Rmail
Summary mode, so it is easy to get the point on a line whose message is
not selected in Rmail.  This can also happen if you switch to the Rmail
window and switch messages there.  To get the Rmail buffer back in sync
with the summary, use the @kbd{j} (@code{rmail-summary-goto-msg})
command, which selects the message of the current summary line in Rmail.

@kindex d (Rmail summary)
@kindex u (Rmail summary)
@findex rmail-summary-delete-forward
@findex rmail-summary-undelete
  Deletion and undeletion can also be done from the summary buffer.
They always work based on where point is located in the summary buffer,
ignoring which message is selected in Rmail.  @kbd{d}
(@code{rmail-summary-delete-forward}) deletes the current line's
message, then moves to the next line whose message is not deleted and
selects that message.  The inverse is @kbd{u}
(@code{rmail-summary-undelete}), which moves back (if necessary) to a
line whose message is deleted, undeletes that message, and selects it in
Rmail.

@kindex SPC (Rmail summary)
@kindex DEL (Rmail summary)
@findex rmail-summary-scroll-msg-down
@findex rmail-summary-scroll-msg-up
  When moving through messages with the summary buffer, it is convenient
to be able to scroll the message while remaining in the summary window.
The commands @key{SPC} (@code{rmail-summary-scroll-msg-up}) and
@key{DEL} (@code{rmail-summary-scroll-msg-down}) do this.  They scroll
the message just as they do when the Rmail buffer is selected.@refill

@kindex x (Rmail summary)
@findex rmail-summary-exit
  When you are finished using the summary, type @kbd{x}
(@code{rmail-summary-exit}) to kill the summary buffer's window.

@kindex q (Rmail summary)
@findex rmail-summary-quit
  You can also exit Rmail while in the summary.  @kbd{q}
(@code{rmail-summary-quit}) kills the summary window, then saves the
Rmail file and switches to another buffer.

@node Rmail Reply, Rmail Editing, Rmail Summary, Rmail
@section Sending Replies

  Rmail has several commands that use Mail mode to send mail.
@xref{Sending Mail}, for information on using Mail mode.  Only the
special commands of Rmail for entering Mail mode are documented here.
Note that the usual keys for sending mail, @kbd{C-x m} and @kbd{C-x 4
m}, are available in Rmail mode and work just as they usually do.@refill

@table @kbd
@item m
Send a message (@code{rmail-mail}).
@item c
Continue editing already started outgoing message (@code{rmail-continue}).
@item r
Send a reply to the current Rmail message (@code{rmail-reply}).
@item f
Forward current message to other users (@code{rmail-forward}).
@end table

@kindex r (Rmail)
@findex rmail-reply
@vindex rmail-dont-reply-to
@cindex reply to a message
 To reply to a the message you are reading in Rmail, type @kbd{r}
(@code{rmail-reply}).  This displays the @samp{*mail*} buffer in another
window, much like @kbd{C-x 4 m}, but pre-initializes the @samp{Subject},
@samp{To}, @samp{CC} and @samp{In-reply-to} header fields based on the
message you reply to.  The @samp{To} field is given the sender of
that message, and the @samp{CC} gets all the recipients of that message.
Recipients that match elements of the list
@code{rmail-dont-reply-to} are omitted; by default, this list contains
your own mailing address.@refill

  Once you have initialized the @samp{*mail*} buffer this way, sending the
mail goes as usual (@pxref{Sending Mail}).  You can edit the presupplied
header fields if they are not what you want.

@kindex C-c C-y (Mail mode)
@findex mail-yank-original
  One additional Mail mode command is available when you invoke mail
from Rmail: @kbd{C-c C-y} (@code{mail-yank-original}) inserts into the
outgoing message a copy of the current Rmail message.  Normally this is
the message you are replying to, but you can also switch to the Rmail
buffer, select a different message, switch back, and yank the new current
message.  Normally the yanked message is indented four spaces and has
most header fields deleted from it; an argument to @kbd{C-c C-y}
specifies the amount to indent, and @kbd{C-u C-c C-y} does not indent at
all and does not delete any header fields.@refill

@kindex f (Rmail)
@findex rmail-forward
@cindex forward a message
  Another frequent reason to send mail in Rmail is to forward the current
message to other users.  @kbd{f} (@code{rmail-forward}) makes this easy by
preinitializing the @samp{*mail*} buffer with the current message as the
text, and a subject designating a forwarded message.  All you have to do is
fill in the recipients and send.@refill

@kindex m (Rmail)
@findex rmail-mail
  You can use the @kbd{m} (@code{rmail-mail}) command to start editing an
outgoing message that is not a reply.  It leaves the header fields empty.
Its only difference from @kbd{C-x 4 m} is that it makes the Rmail buffer
accessible for @kbd{C-c y}, just as @kbd{r} does.  Thus, @kbd{m} can be
used to reply to or forward a message; it can do anything @kbd{r} or @kbd{f}
can do.@refill

@kindex c (Rmail)
@findex rmail-continue
  The @kbd{c} (@code{rmail-continue}) command resumes editing the
@samp{*mail*} buffer, to finish editing an outgoing message you were
already composing, or to alter a message you have sent.@refill

@node Rmail Editing, Rmail Digest, Rmail Reply, Rmail
@section Editing Within a Message

  Rmail mode provides a few special commands for moving within and
editing the current message.  In addition, the usual Emacs commands are
available (except for a few, such as @kbd{C-M-n} and @kbd{C-M-h}, that
are redefined by Rmail for other purposes).  However, the Rmail buffer
is normally read-only, and to alter it you must use the Rmail command
@kbd{w} described below.

@table @kbd
@item t
Toggle display of original headers (@code{rmail-toggle-headers}).
@item w
Edit current message (@code{rmail-edit-current-message}).
@end table

@kindex t (Rmail)
@findex rmail-toggle-header
@vindex rmail-ignored-headers
  Rmail reformats the header of each message before displaying it.
Normally this involves deleting most header fields, on the grounds that
they are not interesting.  The variable @code{rmail-ignored-headers}
should contain a regexp that matches the header fields to discard in
this way.  The original headers are saved permanently; to see what they
look like, use the @kbd{t} (@code{rmail-toggle-headers}) command.  This
discards the reformatted headers of the current message and displays it
with the original headers.  Repeating @kbd{t} reformats the message
again.  Selecting the message again also reformats.

@kindex w (Rmail)
@findex rmail-edit-current-message
  The Rmail buffer is normally read only, and most of the characters you
would type to modify it (including most letters) are redefined as Rmail
commands.  This is usually not a problem since people rarely want to
change the text of a message.  When you do want to do this, type @kbd{w}
(@code{rmail-edit-current-message}), which changes from Rmail mode to
Rmail Edit mode, another major mode which is nearly the same as Text
mode.  The mode line indicates this change.

  In Rmail Edit mode, letters insert themselves as usual and the Rmail
commands are not available.  When you are finished editing the message
and are ready to go back to Rmail, type @kbd{C-c C-c}, which switches
back to Rmail mode.  To return to Rmail mode but cancel all the editing
you have done, type @kbd{C-c C-]}.

@vindex rmail-edit-mode-hook
  Entering Rmail Edit mode calls the value of the variable
@code{text-mode-hook} with no arguments, if that value exists and is not
@code{nil}.  It then does the same with the variable
@code{rmail-edit-mode-hook} and finally adds the attribute @samp{edited}
to the message.

@node Rmail Digest,, Rmail Editing, Rmail
@section Digest Messages
@cindex digest message
@cindex undigestify

  A @dfn{digest message} is a message which exists to contain and carry
several other messages.  Digests are used on moderated mailing lists.  All
messages that arrive for the list during a period of time, such as one
day, are put inside a single digest which is then sent to the subscribers.
Transmitting the single digest uses much less computer time than
transmitting the individual messages even though the total size is the
same, because the per-message overhead in network mail transmission is
considerable.

@findex undigestify-rmail-message
  When you receive a digest message, the most convenient way to read it is
to @dfn{undigestify} it: to turn it back into many individual messages.
You can then read and delete the individual messages as it suits you.

  To undigestify a message, select it and then type @kbd{M-x
undigestify-rmail-message}.  This copies each submessage as a separate
Rmail message and inserts them all following the digest.  The digest
message itself is flagged as deleted.

@iftex
@chapter Miscellaneous Commands

  This chapter contains several brief topics that do not fit anywhere else.

@end iftex
@node Recursive Edit, Narrowing, Rmail, Top
@section Recursive Editing Levels
@cindex recursive editing level
@cindex editing level, recursive

  A @dfn{recursive edit} is a situation in which you are using Emacs
commands to perform arbitrary editing while in the middle of another
Emacs command.  For example, when you type @kbd{C-r} inside a
@code{query-replace}, you enter a recursive edit in which you can change
the current buffer.  When you exit from the recursive edit, you go back to
the @code{query-replace}.

@kindex C-M-c
@findex exit-recursive-edit
@cindex exiting
  @dfn{Exiting} a recursive edit means returning to the unfinished
command, which continues execution.  For example, exiting the recursive
edit requested by @kbd{C-r} in @code{query-replace} causes query replacing
to resume.  Exiting is done with @kbd{C-M-c} (@code{exit-recursive-edit}).

@kindex C-]
@findex abort-recursive-edit
  You can also @dfn{abort} a recursive edit.  This is like exiting, but
also quits the unfinished command immediately.  Use the command @kbd{C-]}
(@code{abort-recursive-edit}) for this.  @xref{Quitting}.

  The mode line shows you when you are in a recursive edit by displaying
square brackets around the parentheses that always surround the major
and minor mode names.  Every window's mode line shows the square
brackets, since Emacs as a whole, rather than any particular buffer, is
in a recursive edit.

@findex top-level
  It is possible to be in recursive edits within recursive edits.  For
example, after typing @kbd{C-r} in a @code{query-replace}, you might
type a command that entered the debugger.  In such circumstances, two or
more sets of square brackets appear in the mode line(s).  Exiting the
inner recursive edit (for example, with the debugger @kbd{c} command)
resumes the query-replace command where it called the debugger.  After
the end of the query-replace command, you would be able to exit the
first recursive edit.  Aborting exits only one level of recursive edit;
it returns to the command level of the previous recursive edit.  You can
then abort that one as well.

  The command @kbd{M-x top-level} aborts all levels of
recursive edits, returning immediately to the top level command reader.

  The text you edit inside the recursive edit need not be the same text
that you were editing at top level.  It depends on what the recursive
edit is for.  If the command that invokes the recursive edit selects a
different buffer first, that is the buffer you will edit recursively.
You can switch buffers within the recursive edit in the normal manner
(as long as the buffer-switching keys have not been rebound).  While you
could theoretically do the rest of your editing inside the recursive
edit, including visiting files, this could have surprising effects (such
as stack overflow) from time to time.  It is recommended you always exit
or abort a recursive edit when you no longer need it.

  In general, GNU Emacs tries to avoid using recursive edits.  It is
usually preferable to allow users to switch among the possible editing
modes in any order they like.  With recursive edits, the only way to get
to another state is to go ``back'' to the state that the recursive edit
was invoked from.

@node Narrowing, Sorting, Recursive Edit, Top
@section Narrowing
@cindex widening
@cindex restriction
@cindex narrowing

  @dfn{Narrowing} means focusing in on some portion of the buffer, making
the rest temporarily invisible and inaccessible.  Cancelling the narrowing,
and making the entire buffer once again visible, is called @dfn{widening}.
The amount of narrowing in effect in a buffer at any time is called the
buffer's @dfn{restriction}.

@c WideCommands
@table @kbd
@item C-x n
Narrow down to between point and mark (@code{narrow-to-region}).
@item C-x w
Widen to make the entire buffer visible again (@code{widen}).
@end table

  Narrowing sometimes makes it easier to concentrate on a single
subroutine or paragraph by eliminating clutter.  It can also be used to
restrict the range of operation of a replace command or repeating
keyboard macro.  The word @samp{Narrow} appears in the mode line
whenever narrowing is in effect.  When you have narrowed to a part of the
buffer, that part appears to be all there is.  You can't see the rest,
can't move into it (motion commands won't go outside the visible part),
and can't change it in any way.  However, the invisible text is not
gone; if you save the file, it will be saved.

@kindex C-x n
@findex narrow-to-region
  The primary narrowing command is @kbd{C-x n} (@code{narrow-to-region}).
It sets the current buffer's restrictions so that the text in the current
region remains visible but all text before the region or after the region
is invisible.  Point and mark do not change.

  Because narrowing can easily confuse users who do not understand it,
@code{narrow-to-region} is normally a disabled command.  Attempting to use
this command asks for confirmation and gives you the option of enabling it;
once you enable the command, confirmation will no longer be required.  @xref{Disabling}.

@kindex C-x w
@findex widen
   To undo narrowing, use @kbd{C-x w} (@code{widen}).  This makes all
text in the buffer accessible again.

   Use the @kbd{C-x =} command to get information on what part of the
buffer you narrowed down.  @xref{Position Info}.

@node Sorting, Shell, Narrowing, Top
@section Sorting Text
@cindex sorting

  Emacs provides several commands for sorting text in a buffer.  All
operate on the contents of the region (the text between point and the
mark).  They divide the text of the region into many @dfn{sort records},
identify a @dfn{sort key} for each record, and then reorder the records
using the order determined by the sort keys.  The records are ordered so
that their keys are in alphabetical order, or, for numeric sorting, in
numeric order.  In alphabetic sorting, all upper case letters `A' through
`Z' come before lower case `a', in accord with the ASCII character
sequence.

  The sort commands differ in how they divide the text into sort
records and in which part of each record they use as the sort key.  Most of
the commands make each line a separate sort record, but some commands use
paragraphs or pages as sort records.  Most of the sort commands use each
entire sort record as its own sort key, but some use only a portion of the
record as the sort key.

@findex sort-lines
@findex sort-paragraphs
@findex sort-pages
@findex sort-fields
@findex sort-numeric-fields
@table @kbd
@item M-x sort-lines
Divide the region into lines, and sort by comparing the entire
text of a line.  A prefix argument means sort in descending order.

@item M-x sort-paragraphs
Divide the region into paragraphs and sort by comparing the entire
text of a paragraph (except for leading blank lines).  A prefix
argument means sort in descending order.

@item M-x sort-pages
Divide the region into pages, and sort by comparing the entire
text of a page (except for leading blank lines).  A prefix
argument means sort in descending order.

@item M-x sort-fields
Divide the region into lines, and sort by comparing the contents of
one field in each line.  Fields are defined as separated by
whitespace, so the first run of consecutive non-whitespace characters
in a line constitutes field 1, the second such run constitutes field
2, etc.

You specify which field to sort by with a numeric argument: 1 to sort
by field 1, etc.  A negative argument means sort in descending
order.  Thus, minus 2 means sort by field 2 in reverse-alphabetical
order.

@item M-x sort-numeric-fields
Like @kbd{M-x sort-fields} except the specified field is converted
to a number for each line, and the numbers are compared.  @samp{10}
comes before @samp{2} when considered as text, but after it when
considered as a number.

@page
@item M-x sort-columns
Like @kbd{M-x sort-fields} except that the text within each line
used for comparison comes from a fixed range of columns.  An explanation
is given below.
@end table

For example, if the buffer contains

@smallexample
On systems where clash detection (locking of files being edited) is
implemented, Emacs also checks the first time you modify a buffer
whether the file has changed on disk since it was last visited or
saved.  If it has, you are asked to confirm that you want to change
the buffer.
@end smallexample

@noindent
then if you apply @kbd{M-x sort-lines} to the entire buffer you get

@smallexample
On systems where clash detection (locking of files being edited) is
implemented, Emacs also checks the first time you modify a buffer
saved.  If it has, you are asked to confirm that you want to change
the buffer.
whether the file has changed on disk since it was last visited or
@end smallexample

@noindent
where the upper case `O' comes before all lower case letters.  If you apply
instead @kbd{C-u 2 M-x sort-fields} you get

@smallexample
implemented, Emacs also checks the first time you modify a buffer
saved.  If it has, you are asked to confirm that you want to change
the buffer.
On systems where clash detection (locking of files being edited) is
whether the file has changed on disk since it was last visited or
@end smallexample

@noindent
where the sort keys were @samp{Emacs}, @samp{If}, @samp{buffer},
@samp{systems} and @samp{the}.@refill

@findex sort-columns
  @kbd{M-x sort-columns} requires more explanation.  You specify the
columns by putting point at one of the columns and the mark at the other
column.  Because this means you cannot put point or the mark at the
beginning of the first line to sort, this command uses an unusual
definition of `region': all of the line point is in is considered part of
the region, and so is all of the line the mark is in.

  For example, to sort a table by information found in columns 10 to 15,
you could put the mark on column 10 in the first line of the table, and
point on column 15 in the last line of the table, and then use this command.
Or you could put the mark on column 15 in the first line and point on
column 10 in the last line.

@page
  This can be thought of as sorting the rectangle specified by point and
the mark, except that the text on each line to the left or right of the
rectangle moves along with the text inside the rectangle.
@xref{Rectangles}.

@node Shell, Hardcopy, Sorting, Top
@section Running Shell Commands from Emacs
@cindex subshell
@cindex shell commands

  Emacs has commands for passing single command lines to inferior shell
processes; it can also run a shell interactively with input and output to
an Emacs buffer @samp{*shell*}.

@table @kbd
@item M-!
Run a specified shell command line and display the output
(@code{shell-command}).
@item M-|
Run a specified shell command line with region contents as input;
optionally replace the region with the output
(@code{shell-command-on-region}).
@item M-x shell
Run a subshell with input and output through an Emacs buffer.
You can then give commands interactively.
@end table

@menu
* Single Shell::         How to run one shell command and return.
* Interactive Shell::    Permanent shell taking input via Emacs.
* Shell Mode::           Special Emacs commands used with permanent shell.
@end menu

@node Single Shell, Interactive Shell, Shell, Shell
@subsection Single Shell Commands

@kindex M-!
@findex shell-command
  @kbd{M-!} (@code{shell-command}) reads a line of text using the
minibuffer and creates an inferior shell to execute the line as a command.
Standard input from the command comes from the null device.  If the shell
command produces any output, the output goes to an Emacs buffer named
@samp{*Shell Command Output*}, which is displayed in another window but not
selected.  A numeric argument, as in @kbd{M-1 M-!}, directs this command to
insert any output into the current buffer.  In that case, point is left
before the output and the mark is set after the output.

@kindex M-|
@findex shell-command-on-region
  @kbd{M-|} (@code{shell-command-on-region}) is like @kbd{M-!} but passes
the contents of the region as input to the shell command, instead of no
input.  If a numeric argument is used to direct  output to the current
buffer, then the old region is deleted first and the output replaces it as
the contents of the region.@refill

@vindex shell-file-name
@cindex environment
  Both @kbd{M-!} and @kbd{M-|} use @code{shell-file-name} to specify the
shell to use.  This variable is initialized based on your @code{SHELL}
environment variable when you start Emacs.  If the file name does not
specify a directory, the directories in the list @code{exec-path} are
searched; this list is initialized based on the @code{PATH} environment
variable when you start Emacs.  You can override either or both of these
default initializations in your @file{.emacs} file .@refill

  When you use @kbd{M-!} and @kbd{M-|}, Emacs has to wait until the
shell command completes.  You can quit with @kbd{C-g}; that terminates
the shell command.

@node Interactive Shell, Shell Mode, Single Shell, Shell
@subsection Interactive Inferior Shell

@findex shell
  To run a subshell interactively with its typescript in an Emacs
buffer, use @kbd{M-x shell}.  This creates (or reuses) a buffer named
@samp{*shell*} and runs a subshell with input coming from and output going
to that buffer.  That is to say, any ``terminal output'' from the subshell
will go into the buffer, advancing point, and any ``terminal input'' for
the subshell comes from text in the buffer.  To give input to the subshell,
go to the end of the buffer and type the input, terminated by @key{RET}.

  Emacs does not wait for the subshell to do anything.  You can switch
windows or buffers and edit them while the shell is waiting, or while it is
running a command.  Output from the subshell waits until Emacs has time to
process it; this happens whenever Emacs is waiting for keyboard input or
for time to elapse.

 To get multiple subshells, change the name of buffer
@samp{*shell*} to something different by using @kbd{M-x rename-buffer}.  The
next use of @kbd{M-x shell} creates a new buffer @samp{*shell*} with
its own subshell.  By renaming this buffer as well you can create a third
one, and so on.  All the subshells run independently and in parallel.

@vindex explicit-shell-file-name
  The file name used to load the subshell is the value of the variable
@code{explicit-shell-file-name}, if that is non-@code{nil}.  Otherwise, the
environment variable @code{ESHELL} is used, or the environment variable
@code{SHELL} if there is no @code{ESHELL}.  If the file name specified
is relative, the directories in the list @code{exec-path} are searched
(@pxref{Single Shell,Single Shell Commands}).@refill

  As soon as the subshell is started, it is sent as input the contents of
the file @file{~/.emacs_@var{shellname}}, if that file exists, where
@var{shellname} is the name of the file that the shell was loaded from.
For example, if you use @code{csh}, the file sent to it is
@file{~/.emacs_csh}.@refill

@vindex shell-pushd-regexp
@vindex shell-popd-regexp
@vindex shell-cd-regexp
  @code{cd}, @code{pushd} and @code{popd} commands given to the inferior
shell are watched by Emacs so it can keep the @samp{*shell*} buffer's
default directory the same as the shell's working directory.  These
commands are recognized syntactically by examining lines of input that are
sent.  If you use aliases for these commands, you can tell Emacs to
recognize them also.  For example, if the value of the variable
@code{shell-pushd-regexp} matches the beginning of a shell command line,
that line is regarded as a @code{pushd} command.  Change this variable when
you add aliases for @samp{pushd}.  Likewise, @code{shell-popd-regexp} and
@code{shell-cd-regexp} are used to recognize commands with the meaning of
@samp{popd} and @samp{cd}.@refill

@kbd{M-x shell-resync-dirs} queries the shell and resynchronizes Emacs' idea
of what the current directory stack is.
@kbd{M-x shell-dirtrack-toggle} turns directory tracking on and off.

@vindex input-ring-size
Emacs keeps a history of the most recent commands you have typed in the
@samp{*shell*} buffer.  If you are at the beginning of a shell command
line and type @key{M-p}, the previous shell input is inserted into the
buffer before point.  Immediately typing @key{M-p} again deletes that
input and inserts the one before it.  By repeating @key{M-p} you can
move backward through your commands until you find one you want to
repeat.  You may then edit the command before typing @key{RET} if you
wish. @key{M-n} moves forward through the command history, in case you
moved backward past the one you wanted while using @key{M-p}.  If you
type the first few characters of a previous command and then type
@key{M-p}, the most recent shell input starting with those characters is
inserted.  This can be very convenient when you are repeating a sequence
of shell commands.  The variable @code{input-ring-size} controls how
many commands are saved in your input history.  The default is 30.


@node Shell Mode,, Interactive Shell, Shell
@subsection Shell Mode

@cindex Shell mode
  The shell buffer uses Shell mode, which defines several special keys
attached to the @kbd{C-c} prefix.  They are chosen to resemble the usual
editing and job control characters present in shells that are not under
Emacs, except that you must type @kbd{C-c} first.  Here is a list
of the special key bindings of Shell mode:

@kindex RET (Shell mode)
@kindex C-c C-d (Shell mode)
@kindex C-d (Shell mode)
@kindex C-c C-u (Shell mode)
@kindex C-c C-w (Shell mode)
@kindex C-c C-c (Shell mode)
@kindex C-c C-z (Shell mode)
@kindex C-c C-\ (Shell mode)
@kindex C-c C-o (Shell mode)
@kindex C-c C-r (Shell mode)
@kindex C-c C-y (Shell mode)
@kindex M-p (Shell mode)
@kindex M-n (Shell mode)
@kindex TAB (Shell mode)
@findex send-shell-input
@findex shell-send-eof
@findex comint-delchar-or-maybe-eof
@findex interrupt-shell-subjob
@findex stop-shell-subjob
@findex quit-shell-subjob
@findex kill-output-from-shell
@findex show-output-from-shell
@findex copy-last-shell-input
@findex comint-previous-input
@findex comint-next-input
@findex comint-dynamic-complete
@vindex shell-prompt-pattern
@table @kbd
@item @key{RET}
At end of buffer send line as input; otherwise, copy current line to end of
buffer and send it (@code{send-shell-input}).  When a line is copied, any
text at the beginning of the line that matches the variable
@code{shell-prompt-pattern} is left out; this variable's value should be a
regexp string that matches the prompts that you use in your subshell.
@item C-c C-d
Send end-of-file as input, probably causing the shell or its current
subjob to finish (@code{shell-send-eof}).
@item C-d
If point is not at the end of the buffer, delete the next character just
like most other modes.  If point is at the end of the buffer, send
end-of-file as input (instead of generating an error as in other modes).
@item C-c C-u
Kill all text that has yet to be sent as input (@code{kill-shell-input}).
@item C-c C-w
Kill a word before point (@code{backward-kill-word}).
@item C-c C-c
Interrupt the shell or its current subjob if any
(@code{interrupt-shell-subjob}).
@item C-c C-z
Stop the shell or its current subjob if any (@code{stop-shell-subjob}).
@item C-c C-\
Send quit signal to the shell or its current subjob if any
(@code{quit-shell-subjob}).
@item C-c C-o
Delete last batch of output from shell (@code{kill-output-from-shell}).
@item C-c C-r
Scroll top of last batch of output to top of window
(@code{show-output-from-shell}).
@item C-c C-y
Copy the previous bunch of shell input, and insert it into the
buffer before point (@code{copy-last-shell-input}).  No final newline
is inserted, and the input copied is not resubmitted until you type
@key{RET}.
@item M-p
Move backward through the input history.  Search for a matching command
if you have typed the beginning of a command.
@item M-n
Move forward through the input history.  Useful when you are
using @key{M-p} quickly and go past the desired command.
@item @key{TAB}
Complete the file name preceding point.
@end table

@node Hardcopy, Dissociated Press, Shell, Top
@section Hardcopy Output
@cindex hardcopy

  The Emacs commands for making hardcopy derive their names from the
Unix commands @samp{print} and @samp{lpr}.

@table @kbd
@item M-x print-buffer
Print hardcopy of current buffer using Unix command @samp{print}
@*(@samp{lpr -p}).  This command adds page headings containing the file name
and page number.
@item M-x lpr-buffer
Print hardcopy of current buffer using Unix command @samp{lpr}.
This command does not add page headings.
@item M-x print-region
Like @code{print-buffer} but prints only the current region.
@item M-x lpr-region
Like @code{lpr-buffer} but prints only the current region.
@end table

@findex print-buffer
@findex print-region
@findex lpr-buffer
@findex lpr-region
@vindex lpr-switches
  All the hardcopy commands pass extra switches to the @code{lpr} program
based on the value of the variable @code{lpr-switches}.  Its value should
be a list of strings, each string a switch starting with @samp{-}.  For
example, the value could be @code{("-Pfoo")} to print on printer
@samp{foo}.

@node Dissociated Press, CONX, Hardcopy, Top
@section Dissociated Press

@findex dissociated-press
  @kbd{M-x dissociated-press} is a command for scrambling a file of text
either word by word or character by character.  Starting from a buffer of
straight English, it produces extremely amusing output.  The input comes
from the current Emacs buffer.  Dissociated Press writes its output in a
buffer named @samp{*Dissociation*}, and redisplays that buffer after every
couple of lines (approximately) to facilitate reading it.

  @code{dissociated-press} asks every so often whether to continue
operating.  Answer @kbd{n} to stop it.  You can also stop at any time by
typing @kbd{C-g}.  The dissociation output remains in the @samp{*Dissociation*}
buffer for you to copy elsewhere if you wish.

@cindex presidentagon
  Dissociated Press operates by jumping at random from one point in the
buffer to another.  In order to produce plausible output rather than
gibberish, it insists on a certain amount of overlap between the end of one
run of consecutive words or characters and the start of the next.  That is,
if it has just printed out `president' and then decides to jump to a
different point in the file, it might spot the `ent' in `pentagon' and
continue from there, producing `presidentagon'.  Long sample texts produce
the best results.

@cindex againformation
  A positive argument to @kbd{M-x dissociated-press} tells it to operate
character by character, and specifies the number of overlap characters.  A
negative argument tells it to operate word by word and specifies the number
of overlap words.  In this mode, whole words are treated as the elements to
be permuted, rather than characters.  No argument is equivalent to an
argument of two.  For your againformation, the output goes only into the
buffer @samp{*Dissociation*}.  The buffer you start with is not changed.

@cindex Markov chain
@cindex ignoriginal
@cindex techniquitous
  Dissociated Press produces nearly the same results as a Markov chain
based on a frequency table constructed from the sample text.  It is,
however, an independent, ignoriginal invention.  Dissociated Press
techniquitously copies several consecutive characters from the sample
between random choices, whereas a Markov chain would choose randomly for
each word or character.  This makes for more plausible sounding results,
and runs faster.

@cindex outragedy
@cindex buggestion
@cindex properbose
  It is a mustatement that too much use of Dissociated Press can be a
developediment to your real work.  Sometimes to the point of outragedy.
And keep dissociwords out of your documentation, if you want it to be well
userenced and properbose.  Have fun.  Your buggestions are welcome.

@node CONX, Amusements, Dissociated Press, Top
@section CONX
@cindex random sentences

Besides producing a file of scrambled text with Dissociated Press, you 
can generate random sentences by using CONX.

@table @kbd
@item M-x conx
Generate random sentences in the *conx* buffer.
@item M-x conx-buffer
Absorb the text in the current buffer into the @code{conx} database.
@item M-x conx-init
Forget the current word-frequency tree.
@item M-x conx-load
Load a @code{conx} database that has been previously saved with 
@code{M-x conx-save}.
@item M-x conx-region
Absorb the text in the current buffer into the conx database.
@item M-x conx-save
Save the current conx database to a file for future retrieval.
@end table

@findex conx
@findex conx-buffer
@findex conx-load
@findex conx-region
@findex conx-init
@findex conx-save

Copy text from a buffer using @kbd{M-x conx-buffer} or @kbd{M-x conx-region}
and then type @kbd{M-x conx}.  Output is continuously generated until you
type @key{^G}. You can save the @code{conx} database to a file with
@kbd{M-x conx-save}, which you can retrieve with @code{M-x conx-load}. 
To clear the database, use @code{M-x conx-init}.

@page
@node Amusements, Emulation, CONX, Top
@section Other Amusements
@cindex boredom
@findex hanoi
@findex yow

  If you are a little bit bored, you can try @kbd{M-x hanoi}.  If you are
considerably bored, give it a numeric argument.  If you are very very
bored, try an argument of 9.  Sit back and watch.

  When you are frustrated, try the famous Eliza program.  Just do
@kbd{M-x doctor}.  End each input by typing @kbd{RET} twice.

  When you are feeling strange, type @kbd{M-x yow}.

@node Emulation, Customization, Amusements, Top
@comment  node-name,  next,  previous,  up
@section Emulation
@cindex other editors
@cindex EDT
@cindex vi

  GNU Emacs can be programmed to emulate (more or less) most other
editors.  Standard facilities can emulate these:

@table @asis
@item EDT (DEC VMS editor)
@findex edt-emulation-on
@findex edt-emulation-off
Turn on EDT emulation with @kbd{M-x edt-emulation-on}.  @kbd{M-x
@*edt-emulation-off} restores normal Emacs command bindings.

Most of the EDT emulation commands are keypad keys, and most standard
Emacs key bindings are still available.  The EDT emulation rebindings
are done in the global keymap, so there is no problem switching
buffers or major modes while in EDT emulation.

@item Gosling Emacs
@findex set-gosmacs-bindings
@findex set-gnu-bindings
Turn on emulation of Gosling Emacs (aka Unipress Emacs) with @kbd{M-x
set-gosmacs-bindings}.  This redefines many keys, mostly on the
@kbd{C-x} and @kbd{ESC} prefixes, to work as they do in Gosmacs.
@kbd{M-x set-gnu-bindings} returns to normal GNU Emacs by rebinding
the same keys to the definitions they had at the time @kbd{M-x
set-gosmacs-bindings} was done.

It is also possible to run Mocklisp code written for Gosling Emacs.
@xref{Mocklisp}.

@item evi (vi emulation in Lucid GNU Emacs)
@cindex evi
In Lucid GNU Emacs, evi is the emulation of vi within Emacs.  
By default, evi-mode is as close as possible to regular vi.
To start evi mode from Emacs, type:
@code{Meta-x evi}.

If you want be in evi mode whenever you bring up Emacs, include this
line in your @code{.emacs} file:
@example
(setq term-setup-hook 'evi)
@end example   
@xref{evi Mode} for more information on evi Mode. 

@item vi (Berkeley Unix editor)
@findex vi-mode
Turn on vi emulation with @kbd{M-x vi-mode}.  This is a major mode
that replaces the previously established major mode.  All of the
vi commands that, in real vi, enter ``input'' mode are programmed
in the Emacs emulator to return to the previous major mode.  Thus,
ordinary Emacs serves as vi's ``input'' mode.

Because vi emulation works through major modes, it does not work
to switch buffers during emulation.  Return to normal Emacs first.

If you plan to use vi emulation much, you probably want to bind a key
to the @code{vi-mode} command.

@item vi (alternate emulator)
@findex vip-mode
Another vi emulator said to resemble real vi more thoroughly is
invoked by @kbd{M-x vip-mode}.  ``Input'' mode in this emulator is
changed from ordinary Emacs so you can use @key{ESC} to go back to
emulated vi command mode.  To get from emulated vi command mode back
to ordinary Emacs, type @kbd{C-z}.

This emulation does not work through major modes, and it is possible
to switch buffers in various ways within the emulator.  It is not
so necessary to assign a key to the command @code{vip-mode} as
it is with @code{vi-mode} because terminating insert mode does
not use it.

For full information, see the long comment at the beginning of the
source file, which is @file{lisp/vip.el} in the Emacs distribution.
@end table

Warning: loading more than one vi emulator at once may cause name
conficts; no one has checked.

@menu
* evi Mode:: Brief discussion of evi, the vi Emulation mode within Lucid
             GNU Emacs
@end menu

@node evi Mode, , Emulation, Emulation
@comment  node-name,  next,  previous,  up
@subsection Using evi Mode
@cindex evi
In Lucid GNU Emacs, evi provides vi emulation within Emacs.  
By default, evi-mode is as close as possible to regular vi.
To start evi mode from Emacs, type:
@code{Meta-x evi}
If you want be in evi mode whenever you bring up Emacs, include this
line in your @file{.emacs} file:
@example
(setq term-setup-hook 'evi)
@end example   

You can find a customization file for evi-mode in @file{~/.evirc}.
This file has to contain Lisp code, just like the @file{.emacs} file,
and is loaded whenever you invoke evi mode.  The file allows you to
rebind keys in evi mode, just as you can in other Emacs modes. 

Note that evi also loads a file of vi commands from @file{.exrc}, just
like vi. 

By default, all Emacs commands are disabled in evi mode.  This leaves you
with only vi commands.  You may customize evi mode to make certain
keybindings accessible.  For example, to enable all emacs command
sequences that begin with @kbd{Control-x} or with @kbd{Meta}, include
the following lines in your @file{.evirc} file:
@example
(evi-define-key evi-all-keymaps "\C-x" ctl-x-map)
(setq evi-meta-prefix-char ?\C-a)
(evi-define-key evi-all-keymaps "\C-a" esc-map)
@end example

When you are in evi mode, typing @kbd{Control-z} stops vi emulation,
leaving you in Emacs.  To get back into evi mode, use @code{Meta-x evi}
again.  To exit Emacs, use @kbd{Control-x Control-c}. 

The file management commands used by vi have been adapted to Emacs.
They have slightly different meanings than the vi commands itself: 

@table @code
@item :e
Edit a file in the current window.  With no argument, brings in a new
copy of the file, if it has been subsequently modified on disk.
@code{:e} overrides any complaints about the current buffer being modified
and discards all modifications.  With a filename argument, it edits that
file in the current window, using the copy already in the editor if it
was previously read in.  There is no difference between @code{:e!
filename} and @code{:e filename}.  As a shorthand for editing the most
recently accessed buffer not in the window, use @code{:e#}. @refill
@item :E
Same as @code{:e}, but edits the file in another window, creating that
window if necessary.  If used with no filename, this command splits the
current buffer into two windows. @refill
@item :n
Switch to the next file in the buffer list that is not currently
displayed.  Rotates the current file to the end of the buffer list, so
the command effectively cycles through all buffers. @refill
@item :N
Same as @code{:n}, but switches to another window or creates another window
and puts the next file into it. @refill
@end table

All @code{ex} commands that accept filenames as arguments perform file
completion using @kbd{Space} or @kbd{Tab}.  Completion begins after the
space that separates the command from the filename. 

Many of the @code{ex} commands are not implemented.  The following
commands are implemented: 
@example
cd, chdir, copy, delete, edit, file, global, map, move, next print,
put, quit, read, set, source, substitute, tag, write, wq, yank, !, <, >
@end example

The following @code{ex} options are implemented:
@example 
autoindent, ignorecase, magic, notimeout, shiftwidth, showmatch,
tabstop, wrapscan
@end example

@node Customization, Quitting, Emulation, Top
@chapter Customization
@cindex customization

  This chapter talks about various topics relevant to adapting the
behavior of Emacs in minor ways.

  All kinds of customization affect only the particular Emacs job that you
do them in.  They are completely lost when you kill the Emacs job, and have
no effect on other Emacs jobs you may run at the same time or later.  The
only way an Emacs job can affect anything outside of it is by writing a
file; in particular, the only way to make a customization `permanent' is to
put something in your @file{.emacs} file or other appropriate file to do the
customization in each session.  @xref{Init File}.

@menu
* Minor Modes::     Each minor mode is one feature you can turn on
                     independently of any others.
* Variables::       Many Emacs commands examine Emacs variables
                     to decide what to do; by setting variables,
                     you can control their functioning.
* Keyboard Macros:: A keyboard macro records a sequence of keystrokes
                     to be replayed with a single command.
* Key Bindings::    The keymaps say what command each key runs.
                     By changing them, you can "redefine keys".
* Syntax::          The syntax table controls how words and expressions
                     are parsed.
* Init File::       How to write common customizations in the @file{.emacs} 
                     file.
* Audible Bell::    Changing how Emacs sounds the bell. 
* Faces::
                    Changing the fonts and colors of a region of text. 
@end menu

@node Minor Modes, Variables, , Customization
@section Minor Modes
@cindex minor modes

@cindex mode line
  Minor modes are options which you can use or not.  For example, Auto
Fill mode is a minor mode in which @key{SPC} breaks lines between words
as you type.  All the minor modes are independent of each other and of
the selected major mode.  Most minor modes inform you in the mode line
when they are on; for example, @samp{Fill} in the mode line means that
Auto Fill mode is on.

  Append @code{-mode} to the name of a minor mode to get the name of a
command function that turns the mode on or off.  Thus, the command to
enable or disable Auto Fill mode is called @kbd{M-x auto-fill-mode}.  These
commands are usually invoked with @kbd{M-x}, but you can bind keys to them
if you wish.  With no argument, the function turns the mode on if it was
off and off if it was on.  This is known as @dfn{toggling}.  A positive
argument always turns the mode on, and an explicit zero argument or a
negative argument always turns it off.

@cindex Auto Fill mode
@findex auto-fill-mode
  Auto Fill mode allows you to enter filled text without breaking lines
explicitly.  Emacs inserts newlines as necessary to prevent lines from
becoming too long.  @xref{Filling}.

@cindex Overwrite mode
@findex overwrite-mode
  Overwrite mode causes ordinary printing characters to replace existing
text instead of moving it to the right.  For example, if point is in
front of the @samp{B} in @samp{FOOBAR}, and you type a @kbd{G} in Overwrite
mode, it changes to @samp{FOOGAR}, instead of @samp{FOOGBAR}.@refill

@cindex Abbrev mode
@findex abbrev-mode
  Abbrev mode allows you to define abbreviations that automatically expand
as you type them.  For example, @samp{amd} might expand to @samp{abbrev
mode}.  @xref{Abbrevs}, for full information.

@node Variables, Keyboard Macros, Minor Modes, Customization
@section Variables
@cindex variable
@cindex option

  A @dfn{variable} is a Lisp symbol which has a value.  The symbol's name
is also called the name of the variable.  Variable names can contain any
characters, but conventionally they are chosen to be words separated by
hyphens.  A variable can have a documentation string which describes what
kind of value it should have and how the value will be used.

  Lisp allows any variable to have any kind of value, but most variables
that Emacs uses require a value of a certain type.  Often the value has
to be a string, or has to be a number.  Sometimes we say that a
certain feature is turned on if a variable is ``non-@code{nil},'' meaning
that if the variable's value is @code{nil}, the feature is off, but the
feature is on for @i{any} other value.  The conventional value to
turn on the feature---since you have to pick one particular value when you
set the variable---is @code{t}.

  Emacs uses many Lisp variables for internal recordkeeping, as any Lisp
program must, but the most interesting variables for you are the ones that
exist for the sake of customization.  Emacs does not (usually) change the
values of these variables; instead, you set the values, and thereby alter
and control the behavior of certain Emacs commands.  These variables are
called @dfn{options}.  Most options are documented in this manual, and
appear in the Variable Index (@pxref{Variable Index}).

  One example of a variable which is an option is @code{fill-column}, which
specifies the position of the right margin (as a number of characters from
the left margin) to be used by the fill commands (@pxref{Filling}).

@menu
* Examining::           Examining or setting one variable's value.
* Edit Options::        Examining or editing list of all variables' values.
* Locals::              Per-buffer values of variables.
* File Variables::      How files can specify variable values.
@end menu

@node Examining, Edit Options, Variables, Variables
@subsection Examining and Setting Variables
@cindex setting variables

@table @kbd
@item C-h v
@itemx M-x describe-variable
Print the value and documentation of a variable.
@item M-x set-variable
Change the value of a variable.
@end table

@kindex C-h v
@findex describe-variable
  To examine the value of a single variable, use @kbd{C-h v}
(@code{describe-variable}), which reads a variable name using the
minibuffer, with completion.  It prints both the value and the
documentation of the variable.

@example
C-h v fill-column @key{RET}
@end example

@noindent
prints something like

@smallexample
fill-column's value is 75

Documentation:
*Column beyond which automatic line-wrapping should happen.
Automatically becomes local when set in any fashion.
@end smallexample

@cindex option
@noindent
The star at the beginning of the documentation indicates that this variable
is an option.  @kbd{C-h v} is not restricted to options; it allows any
variable name.

@findex set-variable
  If you know which option you want to set, you can use @kbd{M-x
set-variable} to set it.  This prompts for the variable name in the
minibuffer (with completion), and then prompts for a Lisp expression for the
new value using the minibuffer a second time.  For example,

@example
M-x set-variable @key{RET} fill-column @key{RET} 75 @key{RET}
@end example

@noindent
sets @code{fill-column} to 75, like executing the Lisp expression

@example
(setq fill-column 75)
@end example

  Setting variables in this way, like all means of customizing Emacs
except where explicitly stated, affects only the current Emacs session.

@node Edit Options, Locals, Examining, Variables
@subsection Editing Variable Values

@table @kbd
@item M-x list-options
Display a buffer listing names, values and documentation of all options.
@item M-x edit-options
Change option values by editing a list of options.
@end table

@findex list-options
  @kbd{M-x list-options} displays a list of all Emacs option variables, in
an Emacs buffer named @samp{*List Options*}.  Each option is shown with its
documentation and its current value.  Here is what a portion of it might
look like:

@smallexample
;; exec-path:
("." "/usr/local/bin" "/usr/ucb" "/bin" "/usr/bin" "/u2/emacs/etc")
*List of directories to search programs to run in subprocesses.
Each element is a string (directory name)
or nil (try the default directory).
;;
;; fill-column:
75
*Column beyond which automatic line-wrapping should happen.
Automatically becomes local when set in any fashion.
;;
@end smallexample

@findex edit-options
  @kbd{M-x edit-options} goes one step further and immediately selects the
@samp{*List Options*} buffer; this buffer uses the major mode Options mode,
which provides commands that allow you to point at an option and change its
value:

@table @kbd
@item s
Set the variable point is in or near to a new value read using the
minibuffer.
@item x
Toggle the variable point is in or near: if the value was @code{nil},
it becomes @code{t}; otherwise it becomes @code{nil}.
@item 1
Set the variable point is in or near to @code{t}.
@item 0
Set the variable point is in or near to @code{nil}.
@item n
@itemx p
Move to the next or previous variable.
@end table

@node Locals, File Variables, Edit Options, Variables
@subsection Local Variables

@table @kbd
@item M-x make-local-variable
Make a variable have a local value in the current buffer.
@item M-x kill-local-variable
Make a variable use its global value in the current buffer.
@item M-x make-variable-buffer-local
Mark a variable so that setting it will make it local to the
buffer that is current at that time.
@end table

@cindex local variables
   You can make any variable @dfn{local} to a specific Emacs buffer.
This means that the variable's value in that buffer is independent of
its value in other buffers.  A few variables are always local in every
buffer.  All other Emacs variables have a @dfn{global} value which is in
effect in all buffers that have not made the variable local.

  Major modes always make the variables they set local to the buffer.
This is why changing major modes in one buffer has no effect on other
buffers.

@findex make-local-variable
  @kbd{M-x make-local-variable} reads the name of a variable and makes it
local to the current buffer.  Further changes in this buffer will not
affect others, and changes in the global value will not affect this
buffer.

@findex make-variable-buffer-local
@cindex per-buffer variables
  @kbd{M-x make-variable-buffer-local} reads the name of a variable and
changes the future behavior of the variable so that it automatically
becomes local when it is set.  More precisely, once you have marked a
variable in this way, the usual ways of setting the
variable will automatically invoke @code{make-local-variable} first.  We
call such variables @dfn{per-buffer} variables.

  Some important variables have been marked per-buffer already.  They
include @code{abbrev-mode}, @code{auto-fill-function},
@code{case-fold-search}, @code{comment-column}, @code{ctl-arrow},
@code{fill-column}, @code{fill-prefix}, @code{indent-tabs-mode},
@code{left-margin}, @*@code{mode-line-format}, @code{overwrite-mode},
@code{selective-display-ellipses}, @*@code{selective-display},
@code{tab-width}, and @code{truncate-lines}.  Some other variables are
always local in every buffer, but they are used for internal
purposes.@refill

Note: the variable @code{auto-fill-function} was formerly named
@code{auto-fill-hook}.

@findex kill-local-variable
  If you want that a variable ceases to be local to the current buffer,
call @kbd{M-x kill-local-variable} and provide the name of a variable to
the prompt.  The global value of the variable
is again in effect in this buffer.  Setting the major mode kills all
the local variables of the buffer.

@findex setq-default
  To set the global value of a variable, regardless of whether the
variable has a local value in the current buffer, you can use the
Lisp function @code{setq-default}.  It works like @code{setq}.
If there is a local value in the current buffer, the local value is
not affected by @code{setq-default}; thus, the new global value may
not be visible until you switch to another buffer.  For example,

@example
(setq-default fill-column 75)
@end example

@noindent
@code{setq-default} is the only way to set the global value of a variable
that has been marked with @code{make-variable-buffer-local}.

@findex default-value
  Programs can look at a variable's default value with @code{default-value}.
This function takes a symbol as an argument and returns its default value.
The argument is evaluated; usually you must quote it explicitly.  For
example,

@example
(default-value 'fill-column)
@end example

@node File Variables,, Locals, Variables
@subsection Local Variables in Files
@cindex local variables in files

  A file can contain a @dfn{local variables list}, which specifies the
values to use for certain Emacs variables when that file is edited.
Visiting the file checks for a local variables list and makes each variable
in the list local to the buffer in which the file is visited, with the
value specified in the file.

  A local variables list goes near the end of the file, in the last page.
(It is often best to put it on a page by itself.)  The local variables list
starts with a line containing the string @samp{Local Variables:}, and ends
with a line containing the string @samp{End:}.  In between come the
variable names and values, one set per line, as @samp{@var{variable}:@:
@var{value}}.  The @var{value}s are not evaluated; they are used literally.

  The line which starts the local variables list does not have to say
just @samp{Local Variables:}.  If there is other text before @samp{Local
Variables:}, that text is called the @dfn{prefix}, and if there is other
text after, that is called the @dfn{suffix}.  If a prefix or suffix are
present, each entry in the local variables list should have the prefix
before it and the suffix after it.  This includes the @samp{End:} line.
The prefix and suffix are included to disguise the local variables list
as a comment so the compiler or text formatter  will ignore it.
If you do not need to disguise the local variables list as a comment in
this way, there is no need to include a prefix or a suffix.@refill

  Two ``variable'' names are special in a local variables list: a value
for the variable @code{mode} sets the major mode, and a value for the
variable @code{eval} is simply evaluated as an expression and the value
is ignored.  These are not real variables; setting them in any other
context does not have the same effect.  If @code{mode} is used in a
local variables list, it should be the first entry in the list.

Here is an example of a local variables list:
@example
;;; Local Variables: ***
;;; mode:lisp ***
;;; comment-column:0 ***
;;; comment-start: ";;; "  ***
;;; comment-end:"***" ***
;;; End: ***
@end example

  Note that the prefix is @samp{;;; } and the suffix is @samp{ ***}.
Note also that comments in the file begin with and end with the same
strings.  Presumably the file contains code in a language which is
enough like Lisp for Lisp mode to be useful but in which comments
start and end differently.  The prefix and suffix are used in the local
variables list to make the list appear as several lines of comments when
the compiler or interpreter for that language reads the file. 

  The start of the local variables list must be no more than 3000
characters from the end of the file, and must be in the last page if the
file is divided into pages.  Otherwise, Emacs will not notice it is
there.  The purpose is twofold: A stray @samp{Local Variables:}@: not in
the last page does not confuse Emacs, and Emacs never needs to search a
long file that contains no page markers and has no local variables list.

  You may be tempted to turn on Auto Fill mode with a local variable
list.  That is inappropriate.  Whether you use Auto Fill mode or not is
a matter of personal taste, not a matter of the contents of particular
files.  If you want to use Auto Fill, set up major mode hooks with your
@file{.emacs} file to turn it on (when appropriate) for you alone
(@pxref{Init File}).  Don't try to use a local variable list that would
impose your taste on everyone working with the file.

Lucid GNU Emacs allows you to specify local variables in the first line
of a file, in addition to specifying them in the @code{Local variables}
section at the end of a file.

If the first line of a file contains two occurences of @code{`-*-'}, Emacs
uses the information between them to determine what the major mode and
variable settings should be.  For example, these are all legal:

@example
	;;; -*- mode: emacs-lisp -*-
	;;; -*- mode: postscript; version-control: never -*-
	;;; -*- tags-file-name: "/foo/bar/TAGS" -*-
@end example

For historical reasons, the syntax @code{`-*- modename -*-'} is allowed
as well, for example, you can use:

@example
	;;; -*- emacs-lisp -*-
@end example

@vindex enable-local-variables
The variable @code{enable-local-variables} controls the use of local
variables lists in files you visit.  The value can be @code{t},
@code{nil} or something else.  A value of @code{t} means local variables
lists are obeyed; @code{nil} means they are ignored; anything else means
query.

The command @code{M-x normal-mode} always obeys local variables lists
and ignores this variable.

@node Keyboard Macros, Key Bindings, Variables, Customization
@section Keyboard Macros

@cindex keyboard macros
  A @dfn{keyboard macro} is a command defined by the user to abbreviate a
sequence of keys.  For example, if you discover that you are about to type
@kbd{C-n C-d} forty times, you can speed your work by defining a keyboard
macro to invoke @kbd{C-n C-d} and calling it with a repeat count of forty.

@c widecommands
@table @kbd
@item C-x (
Start defining a keyboard macro (@code{start-kbd-macro}).
@item C-x )
End the definition of a keyboard macro (@code{end-kbd-macro}).
@item C-x e
Execute the most recent keyboard macro (@code{call-last-kbd-macro}).
@item C-u C-x (
Re-execute last keyboard macro, then add more keys to its definition.
@item C-x q
When this point is reached during macro execution, ask for confirmation
(@code{kbd-macro-query}).
@item M-x name-last-kbd-macro
Give a command name (for the duration of the session) to the most
recently defined keyboard macro.
@item M-x insert-kbd-macro
Insert in the buffer a keyboard macro's definition, as Lisp code.
@end table

@page
  Keyboard macros differ from other Emacs commands in that they are
written in the Emacs command language rather than in Lisp.  This makes it
easier for the novice to write them, and makes them more convenient as
temporary hacks.  However, the Emacs command language is not powerful
enough as a programming language to be useful for writing anything
general or complex.  For such things, Lisp must be used.

  You define a keyboard macro by executing the commands which are its
definition.  Put differently, as you are defining a keyboard macro, the
definition is being executed for the first time.  This way, you see
what the effects of your commands are, and don't have to figure
them out in your head.  When you are finished, the keyboard macro is
defined and also has been executed once.  You can then execute the same
set of commands again by invoking the macro.

@menu
* Basic Kbd Macro::     Defining and running keyboard macros.
* Save Kbd Macro::      Giving keyboard macros names; saving them in files.
* Kbd Macro Query::     Keyboard macros that do different things each use.
@end menu

@node Basic Kbd Macro, Save Kbd Macro, Keyboard Macros, Keyboard Macros
@subsection Basic Use

@kindex C-x (
@kindex C-x )
@kindex C-x e
@findex start-kbd-macro
@findex end-kbd-macro
@findex call-last-kbd-macro
  To start defining a keyboard macro, type @kbd{C-x (}
(@code{start-kbd-macro}).  From then on, anything you type continues to be
executed, but also becomes part of the definition of the macro.  @samp{Def}
appears in the mode line to remind you of what is going on.  When you are
finished, the @kbd{C-x )} command (@code{end-kbd-macro}) terminates the
definition, without becoming part of it. 

  For example

@example
C-x ( M-f foo C-x )
@end example

@noindent
defines a macro to move forward a word and then insert @samp{foo}.

You can give @kbd{C-x )} a repeat count as an argument, in which case it
repeats the macro that many times right after defining it, but defining
the macro counts as the first repetition (since it is executed as you
define it).  If you give @kbd{C-x )} an argument of 4, it executes the
macro immediately 3 additional times.  An argument of zero to @kbd{C-x
e} or @kbd{C-x )} means repeat the macro indefinitely (until it gets an
error or you type @kbd{C-g}).

  Once you have defined a macro, you can invoke it again with the
@kbd{C-x e} command (@code{call-last-kbd-macro}).  You can give the
command a repeat count numeric argument to execute the macro many times.

  To repeat an operation at regularly spaced places in the
text, define a macro and include as part of the macro the commands to move
to the next place you want to use it.  For example, if you want to change
each line, you should position point at the start of a line, and define a
macro to change that line and leave point at the start of the next line.
Repeating the macro will then operate on successive lines.

  After you have terminated the definition of a keyboard macro, you can add
to the end of its definition by typing @kbd{C-u C-x (}.  This is equivalent
to plain @kbd{C-x (} followed by retyping the whole definition so far.  As
a consequence it re-executes the macro as previously defined.

@node Save Kbd Macro, Kbd Macro Query, Basic Kbd Macro, Keyboard Macros
@subsection Naming and Saving Keyboard Macros

@findex name-last-kbd-macro
  To save a keyboard macro for longer than until you define the
next one, you must give it a name using @kbd{M-x name-last-kbd-macro}.
This reads a name as an argument using the minibuffer and defines that name
to execute the macro.  The macro name is a Lisp symbol, and defining it in
this way makes it a valid command name for calling with @kbd{M-x} or for
binding a key to with @code{global-set-key} (@pxref{Keymaps}).  If you
specify a name that has a prior definition other than another keyboard
macro, Emacs prints an error message and nothing is changed.

@findex insert-kbd-macro
  Once a macro has a command name, you can save its definition in a file.
You can then use it in another editing session.  First visit the file
you want to save the definition in.  Then use the command

@example
M-x insert-kbd-macro @key{RET} @var{macroname} @key{RET}
@end example

@noindent
This inserts some Lisp code that, when executed later, will define the same
macro with the same definition it has now.  You need not understand Lisp
code to do this, because @code{insert-kbd-macro} writes the Lisp code for you.
Then save the file.  You can load the file with @code{load-file}
(@pxref{Lisp Libraries}).  If the file you save in is your initialization file
@file{~/.emacs} (@pxref{Init File}) then the macro will be defined each
time you run Emacs.

  If you give @code{insert-kbd-macro} a prefix argument, it creates
additional Lisp code to record the keys (if any) that you have bound to the
keyboard macro, so that the macro is reassigned the same keys when you
load the file.

@node Kbd Macro Query,, Save Kbd Macro, Keyboard Macros
@subsection Executing Macros with Variations

@kindex C-x q
@findex kbd-macro-query
  You can use @kbd{C-x q} (@code{kbd-macro-query}), to get an effect similar
to that of @code{query-replace}.  The macro asks you  each time
whether to make a change.  When you are defining the macro, type @kbd{C-x
q} at the point where you want the query to occur.  During macro
definition, the @kbd{C-x q} does nothing, but when you invoke the macro,
@kbd{C-x q} reads a character from the terminal to decide whether to
continue.

  The special answers to a @kbd{C-x q} query are @key{SPC}, @key{DEL},
@kbd{C-d}, @kbd{C-l} and @kbd{C-r}.  Any other character terminates
execution of the keyboard macro and is then read as a command.
@key{SPC} means to continue.  @key{DEL} means to skip the remainder of
this repetition of the macro, starting again from the beginning in the
next repetition.  @kbd{C-d} means to skip the remainder of this
repetition and cancel further repetition.  @kbd{C-l} redraws the screen
and asks you again for a character to specify what to do.  @kbd{C-r} enters
a recursive editing level, in which you can perform editing which is not
part of the macro.  When you exit the recursive edit using @kbd{C-M-c},
you are asked again how to continue with the keyboard macro.  If you
type a @key{SPC} at this time, the rest of the macro definition is
executed.  It is up to you to leave point and the text in a state such
that the rest of the macro will do what you want.@refill

  @kbd{C-u C-x q}, which is @kbd{C-x q} with a numeric argument, performs a
different function.  It enters a recursive edit reading input from the
keyboard, both when you type it during the definition of the macro, and
when it is executed from the macro.  During definition, the editing you do
inside the recursive edit does not become part of the macro.  During macro
execution, the recursive edit gives you a chance to do some particularized
editing.  @xref{Recursive Edit}.


@iftex
@chapter Keystrokes, Key Sequences and Key Bindings

  This chapter discusses the character set Emacs uses for input commands
and inside files.  You have already learned that the more frequently
used Emacs commands are bound to keys.  For example, @kbd{Control-f} is
bound to @code{forward-char}.  The following discussion deals with:

@itemize @bullet
@item
How keystrokes can be represented
@item
How you can create key sequences from keystrokes
@item
How you can add to the available modifier keys by customizing your
keyboard: for example, you could have the
@key{Capslock} key be understood as the @key{Super} key by Emacs. A
@key{Super} key is used like @key{Control} or @key{Meta} in that you hold
it while typing another key. 
@end itemize

 You will also learn how to customize existing key bindings and
create new ones.
 
@end iftex

@node Keystrokes, Representing Keystrokes, Screen, Top
@section Keystrokes as Building Blocks of Key Sequences
@cindex character set
@cindex ASCII
@cindex keystroke

      Earlier versions of GNU Emacs used only the ASCII character set,
which defines 128 different character codes.  Some of these codes are
assigned graphic symbols like @samp{a} and @samp{=}; the rest are
control characters, such as @kbd{Control-a} (also called @kbd{C-a}).
@kbd{C-a} means you hold down the @key{CTRL} key and then press
@kbd{a}.@refill

   Keybindings in Lucid GNU Emacs are no longer restricted to the set of
keystrokes that can be represented in ASCII.  Emacs can now tell the
difference between, for example, @kbd{Control-h}, @kbd{Control-Shift-h},
and @kbd{Backspace}.
  
@cindex modifier key
@cindex keysym
@kindex meta key
@kindex control key
@kindex hyper key
@kindex super key
@kindex shift key
@kindex button1 
@kindex button2
@kindex button3
@kindex button1up
@kindex button2up
@kindex button3up

  A keystroke is like a piano chord: you get it by simultaneously
striking several keys.  To be more precise, a keystroke consists
of a possibly empty set of modifiers followed by a single
@dfn{keysym}.  The set of modifiers is small; it consists of
@kbd{Control}, @kbd{Meta}, @kbd{Super}, @kbd{Hyper}, and @kbd{Shift}.

  The rest of the keys on your keyboard, along with the mouse buttons,
make up the set of keysyms.  A keysym is usually what is printed on the
keys on your keyboard.  Here is a table of some of the symbolic names
for keysyms:
@table @kbd
@item a,b,c...
alphabetic keys
@item f1,f2...
function keys
@item button1
left mouse button
@item button2
middle mouse button
@item button3
right mouse button
@item button1up 
upstroke on the left mouse button
@item button2up
upstroke on the middle mouse button
@item button3up
upstroke on the right mouse button
@item return
Return key
@end table

@vindex keyboard-translate-table
Use the variable @code{keyboard-translate-table} only if you are on a
dumb tty, as it cannot handle input that cannot be represented as ASCII.
The value of this variable is a string used as a translate table for
keyboard input or @code{nil}.  Each character is looked up in this
string and the contents used instead.  If the string is of length
@code{n}, character codes @code{N} and up are untranslated.  If you are
running Emacs under X, you should do the translations with the
@code{xmodmap} program instead.


@menu
* Representing Keystrokes::  Using lists of modifiers and keysyms to
                             represent keystrokes.
* Key Sequences::            Combine key strokes into key sequences you can
                             bind to commands.
* String Key Sequences::     Available for upward compatibility.
* Meta Key::                 Using @key{ESC} to represent @key{Meta}
* Super and Hyper Keys::     Adding modifier keys on certain keyboards.
* Character Representation:: How characters appear in Emacs buffers.
* Commands::                 How commands are bound to key sequences.
@end menu

@node Representing Keystrokes, Key Sequences, Keystrokes, Top
@comment  node-name,  next,  previous,  up
@subsection Representing Keystrokes
@kindex hyper key
@kindex super key
@findex read-key-sequence

  Lucid GNU Emacs represents keystrokes as lists. Each list consists of
an arbitrary combination of modifiers followed by a single keysym at the
end of the list.  If the keysym corresponds to an ASCII character, you
can use its character code.  (A keystroke may also be represented by an
event object, as returned by the @code{read-key-sequence} function;
non-programmers need not worry about this.)

The following table gives some examples of how to list representations
for keystrokes.  Each list consists of sets of modifiers followed by
keysyms:

@table @kbd
@item (control a)
Pressing @key{CTRL} and @kbd{a} simultaneously.
@item (control ?a)
Another way of writing the keystroke @kbd{C-a}.
@item (control 65)
Yet another way of writing the keystroke @kbd{C-a}.
@item (break)
Pressing the @key{BREAK} key.
@item (control meta button2up)
Release the middle mouse button, while pressing @key{CTRL} and
@key{META}. 
@end table
@cindex shift modifer
  Note: As you define keystrokes, you can use the @kbd{shift} key only
as a modifier with characters that do not have a second keysym on the
same key, such as @kbd{backspace} and @kbd{tab}.  It is an error to
define a keystroke using the @key{shift} modifier with keysyms such as
@kbd{a} and @kbd{=}.  The correct forms are @kbd{A} and @kbd{+}.

@node Key Sequences, String Key Sequences, Representing Keystrokes, Keystrokes
@subsection Representing Key Sequences

  A @dfn{complete key sequence} is a sequence of keystrokes that Emacs
understands as a unit.  Key sequences are significant because you can
bind them to commands.  Note that not all sequences of keystrokes are
possible key sequences.  In particular, the initial keystrokes in a key
sequence must make up a @dfn{prefix key sequence}.

  Emacs represents a key sequence as a vector of keystrokes.  Thus, the
schematic representation of a complete key sequence is as follows:

@example
  [(modifier .. modifer keysym) ... (modifier .. modifier keysym)]
@end example

  Here are some examples of complete key sequences:

@table @kbd
@item [(control c) (control a)]	
Typing @kbd{C-c} followed by @kbd{C-a}
@item [(control c) (control 65)]	
Typing @kbd{C-c} followed by @kbd{C-a}. (Using the ASCII code
for the character `a')@refill
@item [(control c) (break)]
Typing @kbd{C-c} followed by the @kbd{break} character.@refill
@end table

@kindex C-c
@kindex C-x
@kindex C-h
@kindex ESC
@cindex prefix key sequence

  A @dfn{prefix key sequence} is the beginning of a series of longer
sequences that are valid key sequences; adding any single keystroke to
the end of a prefix results in a valid key sequence.  For example,
@kbd{control-x} is standardly defined as a prefix.  Thus, there is a
two-character key sequence starting with @kbd{C-x} for each valid
keystroke, giving numerous possibilities.  Here are some samples:

@itemize @bullet
@item
@kbd{[(control x) (c)]}
@item
@kbd{[(control x) (control c)]}
@end itemize

  Adding one character to a prefix key does not have to form a complete
key.  It could make another, longer prefix.  For example, @kbd{[(control
x) (\4)]} is itself a prefix that leads to any number of different
three-character keys, including @kbd{[(control x) (\4) (f)]},
@kbd{[(control x) (\4) (b)]} and so on.  It would be possible to define
one of those three-character sequences as a prefix, creating a series of
four-character keys, but we did not define any of them this way.@refill

  By contrast, the two-character sequence @kbd{[(control f) (control
k)]} is not a key, because the @kbd{(control f)} is a complete key
sequence in itself.  It's impossible to give @kbd{[(control f (control k)]} 
an independent meaning as a command as long as @kbd{(control f)} retains
its meaning, because what we have is really two commands.@refill

 The predefined prefix key sequences in Emacs are @kbd{(control c)},
@kbd{(control x)}, @kbd{(control h)}, @kbd{[(control x) (\4)]}, and
@kbd{escape}.  You can customize Emacs, and could make new prefix
keys, or eliminate the default key sequences.  @xref{Key Bindings}.@refill

  Whether a particular key sequence is valid can be changed by
customization.  For example, if you redefine @kbd{(control f)} as a
prefix, @kbd{[(control f) (control k)]} automatically becomes a valid key
sequence (complete, unless you define it as a prefix as well).  Conversely,
if you remove the prefix definition of @kbd{[(control x) (\4)]},
@kbd{[(control x) (\4) (f)]} (or @kbd{[(control x) (\4) @var{anything}]})
is no longer a valid key sequence.

Note that the above paragraphs uses \4 instead of simply 4, because \4
is the symbol whose name is "4", and plain 4 is the integer 4, which
would have been interpreted as the ASCII value.  Another way of
representing the symbol whose name is "4" is to write ?4, which would be
interpreted as the number 52, which is the ASCII code for the character
"4".  We could therefore actually have written 52 directly but that is
far less clear.

@node String Key Sequences, Meta Key, Key Sequences, Keystrokes
@comment  node-name,  next,  previous,  up
@subsection  String Key Sequences
For backward compatibility, you may also represent a key sequence using
strings.  For example, we have the following equivalent representations:

@table @kbd
@item "\C-c\C-c"
@code{[(control c) (control c)]}
@item "\e\C-c"
@code{[(meta control c)]}
@end table

@kindex LFD
@kindex TAB
@menu
* Meta Key:: Assignment of the @key{META} Key
* Super and Hyper Keys:: Assignment of the @key{SUPER} and @key{HYPER} Keys
@end menu

@node Meta Key, Super and Hyper Keys, String Key Sequences, Keystrokes
@comment  node-name,  next,  previous,  up
@subsection Assignment of the @key{META} Key
 
@kindex META
@kindex ESC
  Not all terminals have the complete set of modifiers.  
Terminals that have a @key{Meta} key allow you to type Meta characters
by just holding that key down.  To type @kbd{Meta-a}, hold down
@key{META} and press @kbd{a}.  On those terminals, the @key{META} key
works like the @key{SHIFT} key.  Such a key is not always labeled
@key{META}, however, as this function is often a special option for a
key with some other primary purpose.@refill

  If there is no @key{META} key, you can still type Meta characters
using two-character sequences starting with @key{ESC}.  To enter
@kbd{M-a}, you could type @kbd{@key{ESC} a}.  To enter @kbd{C-M-a}, you
would type @kbd{ESC C-a}.  @key{ESC} is allowed on terminals with
Meta keys, too, in case you have formed a habit of using it.@refill

If you are running under X and do not have a Meta key, it 
is possible to reconfigure some other key to be a Meta 
key.  @xref{Super and Hyper Keys}. @refill

@vindex meta-flag
  Emacs believes the terminal has a @key{META} key if the variable
@code{meta-flag} is non-@code{nil}.  Normally this is set automatically
according to the termcap entry for your terminal type.  However, sometimes
the termcap entry is wrong, and then it is useful to set this variable
yourself.  @xref{Variables}, for how to do this.

Note: If you are running under the X window system, the setting of
the @code{meta-flag} variable is irrelevant. 

@node Super and Hyper Keys, Character Representation, Meta Key, Keystrokes
@comment  node-name,  next,  previous,  up
@subsection Assignment of the @key{SUPER} and @key{HYPER} Keys
@kindex hyper key
@kindex super key

  Most keyboards do not, by default, have @key{SUPER} or @key{HYPER}
modifier keys.  Under X, you can simulate the @key{SUPER} or
@key{HYPER} key if you want to bind keys to sequences using @kbd{super}
and @kbd{hyper}.  You can use the @code{xmodmap} program to do this.

  For example, to turn your @key{CAPS-LOCK} key into a @key{SUPER} key,
do the following:

  Create a file called @code{~/.xmodmap}.  In this file, place the lines

@example
	remove Lock = Caps_Lock
	keysym Caps_Lock = Super_L
	add Mod2 = Super_L
@end example

The first line says that the key that is currently called @code{Caps_Lock}
should no longer behave as a ``lock'' key.  The second line says that
this should now be called @code{Super_L} instead.  The third line says that 
the key called @code{Super_L} should be a modifier key, which produces the
@code{Mod2} modifier.

To create a Meta or Hyper key instead of a Super key, replace the 
word ``Super'' above with Meta or Hyper.  

Just after you start up X, execute the command @code{xmodmap /.xmodmap}.
You can add this command to the appropriate initialization file to have
the command executed automatically.@refill

If you have problems, see the documentation for the @code{xmodmap}
program.  The X keyboard model is quite complicated, and explaining
it is beyond the scope of this manual.

@comment   ***  if we do want to explain this nonsense here, we could
@comment   ***  start with the doc I wrote for xkeycaps.  - jwz

@node Key Bindings, Syntax, Keyboard Macros, Customization
@section Customizing Key Bindings

  This section deals with the @dfn{keymaps} which define the bindings
between keys and functions, and shows how you can customize these bindings.
@cindex command
@cindex function
@cindex command name

  A command is a Lisp function whose definition provides for interactive
use.  Like every Lisp function, a command has a function name, a Lisp
symbol whose name usually consists of lower case letters and hyphens.

@menu
* Keymaps::    Definition of the keymap data structure.
               Names of Emacs's standard keymaps.
* Rebinding::  How to redefine one key's meaning conveniently.
* Disabling::  Disabling a command means confirmation is required
                before it can be executed.  This is done to protect
                beginners from surprises.
@end menu

@node Keymaps, Disabling,, Key Bindings
@subsection Keymaps
@cindex keymap

@cindex global keymap
@vindex global-map
  The bindings between characters and command functions are recorded in
data structures called @dfn{keymaps}.  Emacs has many of these.  One, the
@dfn{global} keymap, defines the meanings of the single-character keys that
are defined regardless of major mode.  It is the value of the variable
@code{global-map}.

@cindex local keymap
@vindex c-mode-map
@vindex lisp-mode-map
  Each major mode has another keymap, its @dfn{local keymap}, which
contains overriding definitions for the single-character keys that are
redefined in that mode.  Each buffer records which local keymap is
installed for it at any time, and the current buffer's local keymap is
the only one that directly affects command execution.  The local keymaps
for Lisp mode, C mode, and many other major modes always exist even when
not in use.  They are the values of the variables @code{lisp-mode-map},
@code{c-mode-map}, and so on.  For less frequently used major modes, the
local keymap is sometimes constructed only when the mode is used for the
first time in a session, to save space.

@cindex minibuffer
@vindex minibuffer-local-map
@vindex minibuffer-local-ns-map
@vindex minibuffer-local-completion-map
@vindex minibuffer-local-must-match-map
@vindex repeat-complex-command-map
@vindex isearch-mode-map
  There are local keymaps for the minibuffer too; they contain various
completion and exit commands.

@itemize @bullet
@item
@code{minibuffer-local-map} is used for ordinary input (no completion).
@item
@code{minibuffer-local-ns-map} is similar, except that @key{SPC} exits
just like @key{RET}.  This is used mainly for Mocklisp compatibility.
@item
@code{minibuffer-local-completion-map} is for permissive completion.
@item
@code{minibuffer-local-must-match-map} is for strict completion and
for cautious completion.
@item
@code{repeat-complex-command-map} is for use in @kbd{C-x @key{ESC}}.
@item
@code{isearch-mode-map} contains the bindings of the special keys which
are bound in the pseudo-mode entered with @kbd{C-s} and @kbd{C-r}.
@end itemize

@vindex ctl-x-map
@vindex help-map
@vindex esc-map
  Finally, each prefix key has a keymap which defines the key sequences
that start with it.  For example, @code{ctl-x-map} is the keymap used for
characters following a @kbd{C-x}.

@itemize @bullet
@item
@code{ctl-x-map} is the variable name for the map used for characters that
follow @kbd{C-x}.
@item
@code{help-map} is used for characters that follow @kbd{C-h}.
@item
@code{esc-map} is for characters that follow @key{ESC}. All Meta
characters are actually defined by this map.
@item
@code{ctl-x-4-map} is for characters that follow @kbd{C-x 4}.
@item
@code{mode-specific-map} is for characters that follow @kbd{C-c}.
@end itemize

  The definition of a prefix key is the keymap to use for looking up
the following character.  Sometimes, the definition is actually a Lisp
symbol whose function definition is the following character keymap.  The
effect is the same, but it provides a command name for the prefix key that
you can use as a description of what the prefix key is for.  Thus, the
binding of @kbd{C-x} is the symbol @code{Ctl-X-Prefix}, whose function
definition is the keymap for @kbd{C-x} commands, the value of
@code{ctl-x-map}.@refill

  Prefix key definitions can appear in either the global
map or a local map.  The definitions of @kbd{C-c}, @kbd{C-x}, @kbd{C-h}
and @key{ESC} as prefix keys appear in the global map, so these prefix
keys are always available.  Major modes can locally redefine a key as a
prefix by putting a prefix key definition for it in the local
map.@refill

  A mode can also put a prefix definition of a global prefix character such
as @kbd{C-x} into its local map.  This is how major modes override the
definitions of certain keys that start with @kbd{C-x}.  This case is
special, because the local definition does not entirely replace the global
one.  When both the global and local definitions of a key are other
keymaps, the next character is looked up in both keymaps, with the local
definition overriding the global one.  So, the character after the
@kbd{C-x} is looked up in both the major mode's own keymap for redefined
@kbd{C-x} commands and in @code{ctl-x-map}.  If the major mode's own keymap
for @kbd{C-x} commands contains @code{nil}, the definition from the global
keymap for @kbd{C-x} commands is used.@refill

@menu
* Rebinding::                 Changing Key Bindings Interactively   
* Programmatic Rebinding::    Changing Key Bindings Programmatically
* Key Bindings Using Strings::Using Strings for Changings Key Bindings 
@end menu

@node Rebinding, Programmatic Rebinding, Keymaps, Keymaps
@subsection Changing Key Bindings Interactively
@cindex key rebinding, this session
@cindex rebinding keys, this session

  You can redefine an Emacs key by changing its entry in a keymap.
You can change the global keymap, in which case the change is effective in
all major modes except those that have their own overriding local
definitions for the same key.  Or you can change the current buffer's
local map, which affects all buffers using the same major mode.
@findex global-set-key
@findex local-set-key
@findex local-unset-key

@table @kbd
@item M-x global-set-key @key{RET} @var{key} @var{cmd} @key{RET}
Defines @var{key} globally to run @var{cmd}.
@item M-x local-set-key @key{RET} @var{keys} @var{cmd} @key{RET} 
Defines @var{key} locally (in the major mode now in effect) to run
@var{cmd}.
@item M-x local-unset-key @key{RET} @var{keys} @key{RET}
Removes the local binding of @var{key}.
@end table

@var{cmd} is a symbol naming an interactively-callable function.

When called interactively, @var{key} is the next complete key sequence
that you type.  When called as a function, @var{key} is a string, a
vector of events or a vector of key-description lists as described in
the the @code{define-key} function description.  The binding goes in
the current buffer's local map, which is shared with other buffers in
the same major mode.

The following example,

@example
M-x global-set-key @key{RET} C-f next-line @key{RET}
@end example

@noindent
redefines @kbd{C-f} to move down a line.  The fact that @var{cmd} is
read second makes it serve as a kind of confirmation for @var{key}.

  These functions offer no way to specify a particular prefix keymap as
the one to redefine in, but that is not necessary, as you can include
prefixes in @var{key}.  @var{key} is read by reading characters one by
one until they amount to a complete key (that is, not a prefix key).
Thus, if you type @kbd{C-f} for @var{key}, Emacs enters
the minibuffer immediately to read @var{cmd}.  But if you type
@kbd{C-x}, another character is read; if that character is @kbd{4},
another character is read, and so on.  For example,@refill

@example
M-x global-set-key @key{RET} C-x 4 $ spell-other-window @key{RET}
@end example

@noindent
redefines @kbd{C-x 4 $} to run the (fictitious) command
@code{spell-other-window}.

@findex define-key
@findex substitute-key-definition
  The most general way to modify a keymap is the function
@code{define-key}, used in Lisp code (such as your @file{.emacs} file).
@code{define-key} takes three arguments: the keymap, the key to modify
in it, and the new definition.  @xref{Init File}, for an example.
@code{substitute-key-definition} is used similarly; it takes three
arguments, an old definition, a new definition and a keymap, and
redefines in that keymap all keys that were previously defined with the
old definition to have the new definition instead.

@comment  node-name,  next,  previous,  up
@node Programmatic Rebinding, Key Bindings Using Strings, Rebinding, Keymaps
@subsection Changing Key Bindings Programmatically

  You can use the functions @code{global-set-key} and @code{define-key}
to rebind keys under program control.

@findex define-key
@findex global-set-key

@table @kbd
@item  @code{(global-set-key @var{keys} @var{cmd})}
Defines @var{keys} globally to run @var{cmd}.
@item @code{(define-key @var{keymap} @var{keys} @var{def})}
Defines @var{keys} to run @var{cmd} in the keymap @var{keymap}.
@end table
 
@var{keymap} is a keymap object.

@var{keys} is the sequence of keystrokes to bind.

@var{def} is anything that can be a key's definition:

@itemize @bullet
@item
@code{nil} meaning key is undefined in this keymap.
@item
A command, that is, a Lisp function suitable for interactive calling.
@item
A string or key sequence vector, which is treated as a keyboard macro.
@item
A keymap to define a prefix key.
@item
A symbol so that when the key is looked up, the symbol stands for its
function definition, which should at that time be one of the above,
or another symbol whose function definition is used, and so on.
@item
A cons, @code{(string . defn)}, meaning that @var{defn} is the definition
(@var{defn} should be a valid definition in its own right).
@item
A cons, @code{(keymap . char)}, meaning use the definition of
@var{char} in map @var{keymap}.
@end itemize

For backward compatibility, Lucid GNU Emacs allows you to specify key
sequences as strings.  However, the preferred method is to use the
representations of key sequences as vectors of keystrokes.
@xref{Keystrokes}, for more information about the rules for constructing
key sequences.

Emacs allows you to abbreviate representations for key sequences in 
most places where there is no ambiguity.
Here are some rules for abbreviation:

@itemize @bullet
@item
The keysym by itself is equivalent to a list of just that keysym, i.e.
@code{f1} is equivalent to @code{(f1)}.
@item
A keystroke by itself is equivalent to a vector containing just that
keystroke, i.e.  @code{(control a)} is equivalent to @code{[(control a)]} 
@item
You can use ASCII codes for keysyms that have them. i.e.
@code{65} is equivalent to @code{A}. (This is not so much an
abbreviation as an alternate representation.)
@end itemize

Here are some examples of programmatically binding keys:

@example

;;;  Bind @code{my-command} to @key{f1}

(global-set-key 'f1 'my-command)		

;;;  Bind @code{my-command} to @kbd{Shift-f1}
(global-set-key '(shift f1) 'my-command)

;;; Bind @code{my-command} to @kbd{C-c Shift-f1}
(global-set-key '[(control c) (shift f1)] 'my-command)	
@page
;;; Bind @code{my-command} to the middle mouse button.
(global-set-key 'button2 'my-command)

;;; Bind @code{my-command} to @kbd{@key{META} @key{CTL} @key{Right Mouse Button}}
;;; in the keymap that is in force when you are running @code{dired}.
(define-key dired-mode-map '(meta control button3) 'my-command)

@end example

@comment ;; note that these next four lines are not synonymous:
@comment ;;
@comment (global-set-key '(meta control delete) 'my-command)
@comment (global-set-key '(meta control backspace) 'my-command)
@comment (global-set-key '(meta control h) 'my-command)
@comment (global-set-key '(meta control H) 'my-command)
@comment 
@comment ;; note that this binds two key sequences: ``control-j'' and ``linefeed''.
@comment ;;
@comment (global-set-key "\^J" 'my-command)

@node Key Bindings Using Strings,, Programmatic Rebinding, Keymaps
@comment  node-name,  next,  previous,  up

  For backward compatibility, you can still use strings to represent
key sequences.  Thus you can use comands like the following:

@example
;;; Bind @code{end-of-line} to @kbd{C-f}
(global-set-key "\C-f" 'end-of-line)
@end example

Note, however, that in some cases you may be binding more than one
key sequence by using a single command.  This situation can 
arise because in ASCII, @kbd{C-i} and @key{TAB} have
the same representation.  Therefore, when Emacs sees:

@example
(global-set-key "\C-i" 'end-of-line)
@end example

it is unclear whether the user intended to bind @kbd{C-i} or @key{TAB}.
The solution Lucid GNU Emacs adopts is to bind both of these key
sequences.

@cindex redefining keys
After binding a command to two key sequences with a form like

@example
	(define-key global-map "\^X\^I" 'command-1)
@end example

it is possible to redefine only one of those sequences like so:

@example
	(define-key global-map [(control x) (control i)] 'command-2)
	(define-key global-map [(control x) tab] 'command-3)
@end example

This applies only when running under a window system.  If you are
talking to Emacs through an ASCII-only channel, you do not get any of
these features.

Here is a table of pairs of key sequences that behave in a
similar fashion:

@example
        control h      backspace           
        control l      clear
        control i      tab 
        control m      return              
        control j      linefeed 
        control [      escape
        control @@      control space
@end example

@node Disabling,, Keymaps, Key Bindings
@subsection Disabling Commands
@cindex disabled command

  Disabling a command marks it as requiring confirmation before it
can be executed.  The purpose of disabling a command is to prevent
beginning users from executing it by accident and being confused.

  The direct mechanism for disabling a command is to have a non-@code{nil}
@code{disabled} property on the Lisp symbol for the command.  These
properties are normally set by the user's @file{.emacs} file with
Lisp expressions such as

@example
(put 'delete-region 'disabled t)
@end example

  If the value of the @code{disabled} property is a string, that string
is included in the message printed when the command is used:

@example
(put 'delete-region 'disabled
     "Text deleted this way cannot be yanked back!\n")
@end example

@findex disable-command
@findex enable-command
  You can disable a command either by editing the @file{.emacs} file
directly or with the command @kbd{M-x disable-command}, which edits the
@file{.emacs} file for you.  @xref{Init File}.

  When you attempt to invoke a disabled command interactively in Emacs,
a window is displayed containing the command's name, its
documentation, and some instructions on what to do next; then
Emacs asks for input saying whether to execute the command as requested,
enable it and execute, or cancel it.  If you decide to enable the
command, you are asked whether to do this permanently or just for the
current session.  Enabling permanently works by automatically editing
your @file{.emacs} file.  You can use @kbd{M-x enable-command} at any
time to enable any command permanently.

  Whether a command is disabled is independent of what key is used to
invoke it; it also applies if the command is invoked using @kbd{M-x}.
Disabling a command has no effect on calling it as a function from Lisp
programs.

@node Syntax, Init File, Key Bindings, Customization
@section The Syntax Table
@cindex syntax table

  All the Emacs commands which parse words or balance parentheses are
controlled by the @dfn{syntax table}.  The syntax table specifies which
characters are opening delimiters, which are parts of words, which are
string quotes, and so on.  Actually, each major mode has its own syntax
table (though sometimes related major modes use the same one) which it
installs in each buffer that uses that major mode.  The syntax table
installed in the current buffer is the one that all commands use, so we
call it ``the'' syntax table.  A syntax table is a Lisp object, a vector of
length 256 whose elements are numbers.

@menu
* Entry: Syntax Entry.    What the syntax table records for each character.
* Change: Syntax Change.  How to change the information.
@end menu

@node Syntax Entry, Syntax Change, Syntax, Syntax
@subsection Information about Each Character

  The syntax table entry for a character is a number that encodes six
pieces of information:

@itemize @bullet
@item
The syntactic class of the character, represented as a small integer.
@item
The matching delimiter, for delimiter characters only.
The matching delimiter of @samp{(} is @samp{)}, and vice versa.
@item
A flag saying whether the character is the first character of a
two-character comment starting sequence.
@item
A flag saying whether the character is the second character of a
two-character comment starting sequence.
@item
A flag saying whether the character is the first character of a
two-character comment ending sequence.
@item
A flag saying whether the character is the second character of a
two-character comment ending sequence.
@end itemize

  The syntactic classes are stored internally as small integers, but are
usually described to or by the user with characters.  For example, @samp{(}
is used to specify the syntactic class of opening delimiters.  Here is a
table of syntactic classes, with the characters that specify them.

@table @samp
@item @w{ }
The class of whitespace characters.
@item w
The class of word-constituent characters.
@item _
The class of characters that are part of symbol names but not words.
This class is represented by @samp{_} because the character @samp{_}
has this class in both C and Lisp.
@item .
The class of punctuation characters that do not fit into any other
special class.
@item (
The class of opening delimiters.
@item )
The class of closing delimiters.
@item '
The class of expression-adhering characters.  These characters are
part of a symbol if found within or adjacent to one, and are part
of a following expression if immediately preceding one, but are like
whitespace if surrounded by whitespace.
@item "
The class of string-quote characters.  They match each other in pairs,
and the characters within the pair all lose their syntactic
significance except for the @samp{\} and @samp{/} classes of escape
characters, which can be used to include a string-quote inside the
string.
@item $
The class of self-matching delimiters.  This is intended for @TeX{}'s
@samp{$}, which is used both to enter and leave math mode.  Thus,
a pair of matching @samp{$} characters surround each piece of math mode
@TeX{} input.  A pair of adjacent @samp{$} characters act like a single
one for purposes of matching

@item /
The class of escape characters that always just deny the following
character its special syntactic significance.  The character after one
of these escapes is always treated as alphabetic.
@item \
The class of C-style escape characters.  In practice, these are
treated just like @samp{/}-class characters, because the extra
possibilities for C escapes (such as being followed by digits) have no
effect on where the containing expression ends.
@item <
The class of comment-starting characters.  Only single-character
comment starters (such as @samp{;} in Lisp mode) are represented this
way.
@item >
The class of comment-ending characters.  Newline has this syntax in
Lisp mode.
@end table

@vindex parse-sexp-ignore-comments
  The characters flagged as part of two-character comment delimiters can
have other syntactic functions most of the time.  For example, @samp{/} and
@samp{*} in C code, when found separately, have nothing to do with
comments.  The comment-delimiter significance overrides when the pair of
characters occur together in the proper order.  Only the list and sexp
commands use the syntax table to find comments; the commands specifically
for comments have other variables that tell them where to find comments.
And the list and sexp commands notice comments only if
@code{parse-sexp-ignore-comments} is non-@code{nil}.  This variable is set
to @code{nil} in modes where comment-terminator sequences are liable to
appear where there is no comment; for example, in Lisp mode where the
comment terminator is a newline but not every newline ends a comment.

@node Syntax Change, , Syntax Entry, Syntax
@subsection Altering Syntax Information

  It is possible to alter a character's syntax table entry by storing a new
number in the appropriate element of the syntax table, but it would be hard
to determine what number to use.  Emacs therefore provides a command that
allows you to specify the syntactic properties of a character in a
convenient way.

@findex modify-syntax-entry
  @kbd{M-x modify-syntax-entry} is the command to change a character's
syntax.  It can be used interactively, and is also used by major
modes to initialize their own syntax tables.  Its first argument is the
character to change.  The second argument is a string that specifies the
new syntax.  When called from Lisp code, there is a third, optional
argument, which specifies the syntax table in which to make the change.  If
not supplied, or if this command is called interactively, the third
argument defaults to the current buffer's syntax table.

@enumerate
@item
The first character in the string specifies the syntactic class.  It
is one of the characters in the previous table (@pxref{Syntax Entry}).

@item
The second character is the matching delimiter.  For a character that
is not an opening or closing delimiter, this should be a space, and may
be omitted if no following characters are needed.

@item
The remaining characters are flags.  The flag characters allowed are

@table @samp
@item 1
Flag this character as the first of a two-character comment starting sequence.
@item 2
Flag this character as the second of a two-character comment starting sequence.
@item 3
Flag this character as the first of a two-character comment ending sequence.
@item 4
Flag this character as the second of a two-character comment ending sequence.
@end table
@end enumerate

@kindex C-h s
@findex describe-syntax
  Use @kbd{C-h s} (@code{describe-syntax}) to display a description of
the contents of the current syntax table.  The description of each
character includes both the string you have to pass to
@code{modify-syntax-entry} to set up that character's current syntax,
and some English to explain that string if necessary.

@node Init File, Audible Bell, Syntax, Customization
@section The Init File, .emacs
@cindex init file
@cindex Emacs initialization file
@cindex key rebinding, permanent
@cindex rebinding keys, permanently

  When you start Emacs, it normally loads the file @file{.emacs} in your
home directory.  This file, if it exists, should contain Lisp code.  It
is called your initialization file or @dfn{init file}.  Use the command
line switches @samp{-q} and @samp{-u} to tell Emacs whether to load an
init file (@pxref{Entering Emacs}).

@vindex init-file-user
When the @file{.emacs} file is read, the variable @code{init-file-user}
says which users init file it is.  The value may be the null string or a
string containing a user's name.  If the value is a null string, it means
that the init file was taken from the user that originally logged in.

In all cases, @code{(concat "~" init-file-user "/")} evaluates to the
directory name of the directory where the @file{.emacs} file was looked
for.

  At some sites, there is a @dfn{default init file}, which is the
library named @file{default.el}, found via the standard search path for
libraries.  The Emacs distribution contains no such library; your site
may create one for local customizations.  If this library exists, it is
loaded whenever you start Emacs.  But your init file, if any, is loaded
first; if it sets @code{inhibit-default-init} non-@code{nil}, then
@file{default} is not loaded.

  If you have a large amount of code in your @file{.emacs} file, you
should move it into another file named @file{@var{something}.el},
byte-compile it (@pxref{Lisp Libraries}), and load that file from your
@file{.emacs} file using @code{load}.

@menu
* Init Syntax::     Syntax of constants in Emacs Lisp.
* Init Examples::   How to do some things with an init file.
* Terminal Init::   Each terminal type can have an init file.
@end menu

@node Init Syntax, Init Examples, Init File, Init File
@subsection Init File Syntax

  The @file{.emacs} file contains one or more Lisp function call
expressions.  Each consists of a function name followed by
arguments, all surrounded by parentheses.  For example, @code{(setq
fill-column 60)} represents a call to the function @code{setq} which is
used to set the variable @code{fill-column} (@pxref{Filling}) to 60.

  The second argument to @code{setq} is an expression for the new value
of the variable.  This can be a constant, a variable, or a function call
expression.  In @file{.emacs}, constants are used most of the time.
They can be:

@table @asis
@item Numbers:
Integers are written in decimal, with an optional initial minus sign.

If a sequence of digits is followed by a period and another sequence
of digits, it is interpreted as a floating point number.

@item Strings:
Lisp string syntax is the same as C string syntax with a few extra
features.  Use a double-quote character to begin and end a string constant.

Newlines and special characters may be present literally in strings.  They
can also be represented as backslash sequences: @samp{\n} for newline,
@samp{\b} for backspace, @samp{\r} for return, @samp{\t} for tab,
@samp{\f} for formfeed (control-l), @samp{\e} for escape, @samp{\\} for a
backslash, @samp{\"} for a double-quote, or @samp{\@var{ooo}} for the
character whose octal code is @var{ooo}.  Backslash and double-quote are
the only characters for which backslash sequences are mandatory.

You can use @samp{\C-} as a prefix for a control character, as in
@samp{\C-s} for ASCII Control-S, and @samp{\M-} as a prefix for
a meta character, as in @samp{\M-a} for Meta-A or @samp{\M-\C-a} for
Control-Meta-A.@refill

@item Characters:
Lisp character constant syntax consists of a @samp{?} followed by
either a character or an escape sequence starting with @samp{\}.
Examples: @code{?x}, @code{?\n}, @code{?\"}, @code{?\)}.  Note that
strings and characters are not interchangeable in Lisp; some contexts
require one and some contexts require the other.

@item True:
@code{t} stands for `true'.

@item False:
@code{nil} stands for `false'.

@item Other Lisp objects:
Write a single-quote (') followed by the Lisp object you want.
@end table

@node Init Examples, Terminal Init, Init Syntax, Init File
@subsection Init File Examples

  Here are some examples of doing certain commonly desired things with
Lisp expressions:

@itemize @bullet
@item
Make @key{TAB} in C mode just insert a tab if point is in the middle of a
line.

@example
(setq c-tab-always-indent nil)
@end example

Here we have a variable whose value is normally @code{t} for `true'
and the alternative is @code{nil} for `false'.

@item
Make searches case sensitive by default (in all buffers that do not
override this).

@example
(setq-default case-fold-search nil)
@end example

This sets the default value, which is effective in all buffers that do
not have local values for the variable.  Setting @code{case-fold-search}
with @code{setq} affects only the current buffer's local value, which
is probably not what you want to do in an init file.

@item
Make Text mode the default mode for new buffers.

@example
(setq default-major-mode 'text-mode)
@end example

Note that @code{text-mode} is used because it is the command for entering
the mode we want.  A single-quote is written before it to make a symbol
constant; otherwise, @code{text-mode} would be treated as a variable name.

@item
Turn on Auto Fill mode automatically in Text mode and related modes.

@example
(setq text-mode-hook
  '(lambda () (auto-fill-mode 1)))
@end example

Here we have a variable whose value should be a Lisp function.  The
function we supply is a list starting with @code{lambda}, and a single
quote is written in front of it to make it (for the purpose of this
@code{setq}) a list constant rather than an expression.  Lisp functions
are not explained here; for mode hooks it is enough to know that
@code{(auto-fill-mode 1)} is an expression that will be executed when
Text mode is entered.  You could replace it with any other expression
that you like, or with several expressions in a row.

@example
(setq text-mode-hook 'turn-on-auto-fill)
@end example

This is another way to accomplish the same result.
@code{turn-on-auto-fill} is a symbol whose function definition is
@code{(lambda () (auto-fill-mode 1))}.

@item
Load the installed Lisp library named @file{foo} (actually a file
@file{foo.elc} or @file{foo.el} in a standard Emacs directory).

@example
(load "foo")
@end example

When the argument to @code{load} is a relative pathname, not starting
with @samp{/} or @samp{~}, @code{load} searches the directories in
@code{load-path} (@pxref{Loading}).

@item
Load the compiled Lisp file @file{foo.elc} from your home directory.

@example
(load "~/foo.elc")
@end example

Here an absolute file name is used, so no searching is done.

@item
Rebind the key @kbd{C-x l} to run the function @code{make-symbolic-link}.

@example
(global-set-key "\C-xl" 'make-symbolic-link)
@end example

or

@example
(define-key global-map "\C-xl" 'make-symbolic-link)
@end example

Note once again the single-quote used to refer to the symbol
@code{make-symbolic-link} instead of its value as a variable.

@item
Do the same thing for C mode only.

@example
(define-key c-mode-map "\C-xl" 'make-symbolic-link)
@end example

@item
Bind the function key @key{F1} to a command in C mode.
Note that the names of function keys must be lower case.

@example
(define-key c-mode-map 'f1 'make-symbolic-link)
@end example

@item
Bind the shifted version of @key{F1} to a command.

@example
(define-key c-mode-map '(shift f1) 'make-symbolic-link)
@end example

@item
Redefine all keys which now run @code{next-line} in Fundamental mode
to run @code{forward-line} instead.

@example
(substitute-key-definition 'next-line 'forward-line
                           global-map)
@end example

@item
Make @kbd{C-x C-v} undefined.

@example
(global-unset-key "\C-x\C-v")
@end example

One reason to undefine a key is so that you can make it a prefix.
Simply defining @kbd{C-x C-v @var{anything}} would make @kbd{C-x C-v}
a prefix, but @kbd{C-x C-v} must be freed of any non-prefix definition
first.

@item
Make @samp{$} have the syntax of punctuation in Text mode.
Note the use of a character constant for @samp{$}.

@example
(modify-syntax-entry ?\$ "." text-mode-syntax-table)
@end example

@item
Enable the use of the command @code{eval-expression} without confirmation.

@example
(put 'eval-expression 'disabled nil)
@end example
@end itemize

@node Terminal Init,, Init Examples, Init File
@subsection Terminal-specific Initialization

  Each terminal type can have a Lisp library to be loaded into Emacs when
it is run on that type of terminal.  For a terminal type named
@var{termtype}, the library is called @file{term/@var{termtype}} and it is
found by searching the directories @code{load-path} as usual and trying the
suffixes @samp{.elc} and @samp{.el}.  Normally it appears in the
subdirectory @file{term} of the directory where most Emacs libraries are
kept.@refill

  The usual purpose of the terminal-specific library is to define the
escape sequences used by the terminal's function keys using the library
@file{keypad.el}.  See the file
@file{term/vt100.el} for an example of how this is done.@refill

  When the terminal type contains a hyphen, only the part of the name
before the first hyphen is significant in choosing the library name.
Thus, terminal types @samp{aaa-48} and @samp{aaa-30-rv} both use
the library @file{term/aaa}.  The code in the library can use
@code{(getenv "TERM")} to find the full terminal type name.@refill

@vindex term-file-prefix
  The library's name is constructed by concatenating the value of the
variable @code{term-file-prefix} and the terminal type.  Your @file{.emacs}
file can prevent the loading of the terminal-specific library by setting
@code{term-file-prefix} to @code{nil}.

@vindex term-setup-hook
  The value of the variable @code{term-setup-hook}, if not @code{nil}, is
called as a function of no arguments at the end of Emacs initialization,
after both your @file{.emacs} file and any terminal-specific library have
been read.  You can set the value in the @file{.emacs} file to override
part of any of the terminal-specific libraries and to define
initializations for terminals that do not have a library.@refill

@comment  node-name,  next,  previous,  up
@node Audible Bell, Faces, Init File, Customization
@section Changing the Bell Sound
@cindex audible bell, changing
@cindex bell, changing
@vindex sound-alist
@findex load-default-sounds
@findex play-sound

You can now change how the audible bell sounds using the variable
@code{sound-alist}.

@code{sound-alist}'s value is an alist associating symbols with strings
of audio-data.  When @code{ding} is called with one of
the symbols, the associated sound data is played instead of the
standard beep.  This only works if you are logged in on the console of a
SPARCstation. To listen to a sound of the provided type, call the
function @code{play-sound} with the argument @var{sound}. You can also
set the volume of the sound with the optional arugment @var{volume}.
@cindex ding

Elements of the list should be of one of the following forms:

@example
   ( symbol . string-or-symbol )
   ( symbol integer string-or-symbol )
@end example

If @code{string-or-symbol} is a string, it should contain raw sound
data, the contents of a @file{.au} file.  If it is a symbol, the symbol
is considered an alias for some other element, and the sound-player
looks for that next.  If the integer is provided, it is the volume at
which the sound should be played, from 0 to 100.

If an element of this alist begins with the symbol @code{default}, that
sound is used when no other sound is appropriate.

@page
If the symbol @code{t} is in place of a sound-string, Emacs uses the
default X beep.  This allows you to define beep-types of 
different volumes even when not running on the console of a SPARCstation.

@findex load-sound-file
You can add things to this list by calling the function
@code{load-sound-file}, which reads in an audio-file and adds its data to
the sound-alist. You can specify the sound with the @var{sound-name}
argument and the file into which the sounds are loaded with the
@var{filename} argument. The optional @var{volume} argument sets the
volume.

@code{load-sound-file (filename sound-name &optional volume)}

To load and install some sound files as beep-types, use the function
@code{load-default-sounds} (note that this only works if you are on
display 0 of a SPARCstation).

The following beep-types are used by Emacs itself. Other Lisp
packages may use other beep types, but these are the ones that the C
kernel of Emacs uses.

@table @code
@item auto-save-error
An auto-save does not succeed

@item command-error
The Emacs command loop catches an error

@item undefined-key
You type a key that is undefined

@item undefined-click	
You use an undefined mouse-click combination

@item no-completion	
Completion was not possible

@item y-or-n-p		
You type something other than the required @code{y} or @code{n}

@item yes-or-no-p  	
When you type something other than @code{yes} or @code{no}
@end table

@comment  node-name,  next,  previous,  up
@node Faces, , Audible Bell, Customization
@section Faces

Lucid GNU Emacs has objects called extents and faces.  An @dfn{extent}
is a region of text and a @dfn{face} is a collection of textual
attributes, such as fonts and colors.  Every extent is displayed in some
face, therefore, changing the properties of a face immediately updates the
display of all associated extents.  Faces can be screen-local: you can
have a region of text that displays with completely different
attributes when its buffer is viewed from a different X window.

The display attributes of faces may be specified either in Lisp or through
the X resource manager.

@subsection Customizing Faces

You can change the face of an extent with the functions in
this section.  All the functions prompt for a @var{face} as an
argument; use completion for a list of possible values.

@table @kbd
@item M-x invert-face
Swap the foreground and background colors of the given @var{face}.
@item M-x make-face-bold
Make the font of the given @var{face} bold.  When called from a
program, returns @code{nil} if this is not possible.
@item M-x make-face-bold-italic
Make the font of the given @var{face} bold italic.  
When called from a program, returns @code{nil} if not possible.
@item M-x make-face-italic
Make the font of the given @var{face} italic.  
When called from a program, returns @code{nil} if not possible.
@item M-x make-face-unbold
Make the font of the given @var{face} non-bold.  
When called from a program, returns @code{nil} if not possible.
@item M-x make-face-unitalic
Make the font of the given @var{face} non-italic.
When called from a program, returns @code{nil} if not possible.
@item M-x set-face-background 
Change the background color of the given @var{face}.
@item M-x set-face-background-pixmap
Change the background pixmap of the given @var{face}.
@item M-x set-face-font 
Change the font of the given @var{face}.
@item M-x set-face-foreground
Change the foreground color of the given @var{face}.
@item M-x set-face-underline-p
Change whether the given @var{face} is underlined.
@end table

@findex make-face-bold
@findex make-face-bold-italic
@findex make-face-italic
@findex make-face-unbold
@findex make-face-unitalic
@findex invert-face
You can exchange the foreground and background color of the selected
@var{face} with the function @code{invert-face}. If the face does not
specify both foreground and background, then its foreground and
background are set to the background and foreground of the default face.
When calling this from a program, you can supply the optional argument 
@var{screen} to specify which screen is affected; otherwise, all screens
are affected.

@findex set-face-background
You can set the background color of the specified @var{face} with the
function @code{set-face-background}.  The argument @code{color} should
be a string, the name of a color.  When called from a program, if the
optional @var{screen} argument is provided, the face is changed only 
in that screen; otherwise, it is changed in all screens.

@findex set-face-background-pixmap
You can set the background pixmap of the specified @var{face} with the
function @code{set-face-background-pixmap}.  The pixmap argument
@var{name} should be a string, the name of a file of pixmap data.  The
directories listed in the @code{x-bitmap-file-path} variable are
searched.  The bitmap may also be a list of the form @code{(width height
data)} where width and height are the size in pixels, and data is a
string containing the raw bits of the bitmap.  If the optional
@var{screen} argument is provided, the face is changed only in that
screen; otherwise, it is changed in all screens.

The variable @code{x-bitmap-file-path} takes as a value a list of the
directories in which X bitmap files may be found.  If the value is
@code{nil}, the list is initialized from the @code{*bitmapFilePath}
resource.

@findex set-face-font
You can set the font of the specified @var{face} with the function
@code{set-face-font}.  The @var{font} argument should be a string, the
name of a font.  When called from a program, if the
optional @var{screen} argument is provided, the face is changed only 
in that screen; otherwise, it is changed in all screens.

@findex set-face-foreground 
You can set the foreground color of the specified @var{face} with the
function @code{set-face-foreground}.  The argument @var{color} should be
a string, the name of a color.  If the optional @var{screen} argument is
provided, the face is changed only in that screen; otherwise, it is
changed in all screens.

@findex set-face-underline-p
You can set underline the specified @var{face} with the function
@code{set-face-underline-p}. The argument @var{underline-p} can be used
to make underlining an attribute of the face or not. If the optional
@var{screen} argument is provided, the face is changed only in that
screen; otherwise, it is changed in all screens.


@iftex
@chapter Correcting Mistakes (Yours or Emacs's)

  If you type an Emacs command you did not intend, the results are often
mysterious.  This chapter discusses how you can undo your mistake or
recover from a mysterious situation.  Emacs bugs and system crashes are
also considered.
@end iftex


@node Quitting, Lossage, Customization, Top
@section Quitting and Aborting
@cindex quitting

@table @kbd
@item C-g
Quit.  Cancel running or partially typed command.
@item C-]
Abort innermost recursive editing level and cancel the command which
invoked it (@code{abort-recursive-edit}).
@item M-x top-level
Abort all recursive editing levels that are currently executing.
@item C-x u
Cancel an already-executed command, usually (@code{undo}).
@end table

  There are two ways of cancelling commands which are not finished
executing: @dfn{quitting} with @kbd{C-g}, and @dfn{aborting} with @kbd{C-]}
or @kbd{M-x top-level}.  Quitting is cancelling a partially typed command
or one which is already running.  Aborting is getting out of a recursive
editing level and cancelling the command that invoked the recursive edit.

@cindex quitting
@cindex C-g
  Quitting with @kbd{C-g} is used for getting rid of a partially typed
command, or a numeric argument that you don't want.  It also stops a
running command in the middle in a relatively safe way, so you can use
it if you accidentally start executing a command that takes a long
time.  In particular, it is safe to quit out of killing; either your
text will @var{all} still be there, or it will @var{all} be in the kill
ring (or maybe both).  Quitting an incremental search does special
things documented under searching; in general, it may take two
successive @kbd{C-g} characters to get out of a search.  @kbd{C-g} works
by setting the variable @code{quit-flag} to @code{t} the instant
@kbd{C-g} is typed; Emacs Lisp checks this variable frequently and quits
if it is non-@code{nil}.  @kbd{C-g} is only actually executed as a
command if it is typed while Emacs is waiting for input.

If you quit twice in a row before the first @kbd{C-g} is recognized, you
activate the ``emergency escape'' feature and return to the shell.
@xref{Emergency Escape}.

@cindex recursive editing level
@cindex editing level, recursive
@cindex aborting
@findex abort-recursive-edit
@kindex C-]
  You can use @kbd{C-]} (@code{abort-recursive-edit}) to get out
of a recursive editing level and cancel the command which invoked it.
Quitting with @kbd{C-g} does not do this, and could not do this, because it
is used to cancel a partially typed command @i{within} the recursive
editing level.  Both operations are useful.  For example, if you are in the
Emacs debugger (@pxref{Lisp Debug}) and have typed @kbd{C-u 8} to enter a
numeric argument, you can cancel that argument with @kbd{C-g} and remain in
the debugger.

@findex top-level
  The command @kbd{M-x top-level} is equivalent to ``enough'' @kbd{C-]}
commands to get you out of all the levels of recursive edits that you are
in.  @kbd{C-]} only gets you out one level at a time, but @kbd{M-x top-level}
goes out all levels at once.  Both @kbd{C-]} and @kbd{M-x top-level} are
like all other commands, and unlike @kbd{C-g}, in that they are effective
only when Emacs is ready for a command.  @kbd{C-]} is an ordinary key and
has its meaning only because of its binding in the keymap.
@xref{Recursive Edit}.

  @kbd{C-x u} (@code{undo}) is not strictly speaking a way of cancelling a
command, but you can think of it as cancelling a command already finished
executing.  @xref{Undo}.

@node Lossage, Bugs, Quitting, Top
@section Dealing with Emacs Trouble

  This section describes various conditions in which Emacs fails to work,
and how to recognize them and correct them.

@menu
* Stuck Recursive::    `[...]' in mode line around the parentheses
* Screen Garbled::     Garbage on the screen
* Text Garbled::       Garbage in the text
* Unasked-for Search:: Spontaneous entry to incremental search
* Emergency Escape::   Emergency escape---
                        What to do if Emacs stops responding
* Total Frustration::  When you are at your wits' end.
@end menu

@node Stuck Recursive, Screen Garbled, Lossage, Lossage
@subsection Recursive Editing Levels

  Recursive editing levels are important and useful features of Emacs, but
they can seem like malfunctions to the user who does not understand them.

  If the mode line has square brackets @samp{[@dots{}]} around the parentheses
that contain the names of the major and minor modes, you have entered a
recursive editing level.  If you did not do this on purpose, or if you
don't understand what that means, you should just get out of the recursive
editing level.  To do so, type @kbd{M-x top-level}.  This is called getting
back to top level.  @xref{Recursive Edit}.

@node Screen Garbled, Text Garbled, Stuck Recursive, Lossage
@subsection Garbage on the Screen

  If the data on the screen looks wrong, the first thing to do is see
whether the text is actually wrong.  Type @kbd{C-l}, to redisplay the
entire screen.  If the text appears correct after this, the problem was
entirely in the previous screen update.

  Display updating problems often result from an incorrect termcap entry
for the terminal you are using.  The file @file{etc/TERMS} in the Emacs
distribution gives the fixes for known problems of this sort.
@file{INSTALL} contains general advice for these problems in one of its
sections.  Very likely there is simply insufficient padding for certain
display operations.  To investigate the possibility that you have this
sort of problem, try Emacs on another terminal made by a different
manufacturer.  If problems happen frequently on one kind of terminal but
not another kind, the real problem is likely to be a bad termcap entry,
though it could also be due to a bug in Emacs that appears for terminals
that have or that lack specific features.

@node Text Garbled, Unasked-for Search, Screen Garbled, Lossage
@subsection Garbage in the Text

  If @kbd{C-l} shows that the text is wrong, try undoing the changes to it
using @kbd{C-x u} until it gets back to a state you consider correct.  Also
try @kbd{C-h l} to find out what command you typed to produce the observed
results.

  If a large portion of text appears to be missing at the beginning or
end of the buffer, check for the word @samp{Narrow} in the mode line.
If it appears, the text is still present, but marked off-limits.
To make it visible again, type @kbd{C-x w}.  @xref{Narrowing}.

@node Unasked-for Search, Emergency Escape, Text Garbled, Lossage
@subsection Spontaneous Entry to Incremental Search

  If Emacs spontaneously displays @samp{I-search:} at the bottom of the
screen, it means that the terminal is sending @kbd{C-s} and @kbd{C-q}
according to the poorly designed xon/xoff ``flow control'' protocol.  You
should try to prevent this by putting the terminal in a mode where it will
not use flow control or by giving it enough padding that it will never send a
@kbd{C-s}.  If that cannot be done, you must tell Emacs to expect flow
control to be used, until you can get a properly designed terminal.

  Information on how to do these things can be found in the file
@file{INSTALL} in the Emacs distribution.

@node Emergency Escape, Total Frustration, Unasked-for Search, Lossage
@subsection Emergency Escape

  Because at times there have been bugs causing Emacs to loop without
checking @code{quit-flag}, a special feature causes Emacs to be suspended
immediately if you type a second @kbd{C-g} while the flag is already set,
so you can always get out of GNU Emacs.  Normally Emacs recognizes and
clears @code{quit-flag} (and quits!) quickly enough to prevent this from
happening.

  When you resume Emacs after a suspension caused by multiple @kbd{C-g}, it
asks two questions before going back to what it had been doing:

@example
Auto-save? (y or n)
Abort (and dump core)? (y or n)
@end example

@noindent
Answer each one with @kbd{y} or @kbd{n} followed by @key{RET}.

  Saying @kbd{y} to @samp{Auto-save?} causes immediate auto-saving of all
modified buffers in which auto-saving is enabled.

  Saying @kbd{y} to @samp{Abort (and dump core)?} causes an illegal
instruction to be executed, dumping core.  This is to enable a wizard to
figure out why Emacs was failing to quit in the first place.  Execution
does not continue after a core dump.  If you answer @kbd{n}, execution
does continue.  With luck, Emacs will ultimately check
@code{quit-flag} and quit normally.  If not, and you type another
@kbd{C-g}, it is suspended again.

  If Emacs is not really hung, just slow, you may invoke the double
@kbd{C-g} feature without really meaning to.  Then just resume and answer
@kbd{n} to both questions, and you will arrive at your former state.
Presumably the quit you requested will happen soon.

  The double-@kbd{C-g} feature may be turned off when Emacs is running under
a window system, since the window system always enables you to kill Emacs
or to create another window and run another program.

@node Total Frustration,, Emergency Escape, Lossage
@subsection Help for Total Frustration
@cindex Eliza
@cindex doctor

  If using Emacs (or something else) becomes terribly frustrating and none
of the techniques described above solve the problem, Emacs can still help
you.

  First, if the Emacs you are using is not responding to commands, type
@kbd{C-g C-g} to get out of it and then start a new one.

@findex doctor
  Second, type @kbd{M-x doctor @key{RET}}.

  The doctor will make you feel better.  Each time you say something to
the doctor, you must end it by typing @key{RET} @key{RET}.  This lets the
doctor know you are finished.

@node Bugs, Manifesto, Lossage, Top
@section Reporting Bugs

@cindex bugs
  Sometimes you will encounter a bug in Emacs.  Although we cannot promise
we can or will fix the bug, and we might not even agree that it is a bug,
we want to hear about bugs you encounter in case we do want to fix them.

  To make it possible for us to fix a bug, you must report it.  In order
to do so effectively, you must know when and how to do it.

@subsection When Is There a Bug

  If Emacs executes an illegal instruction, or dies with an operating
system error message that indicates a problem in the program (as opposed to
something like ``disk full''), then it is certainly a bug.

  If Emacs updates the display in a way that does not correspond to what is
in the buffer, then it is certainly a bug.  If a command seems to do the
wrong thing but the problem corrects itself if you type @kbd{C-l}, it is a
case of incorrect display updating.

  Taking forever to complete a command can be a bug, but you must make
certain that it was really Emacs's fault.  Some commands simply take a long
time.  Type @kbd{C-g} and then @kbd{C-h l} to see whether the input Emacs
received was what you intended to type; if the input was such that you
@var{know} it should have been processed quickly, report a bug.  If you
don't know whether the command should take a long time, find out by looking
in the manual or by asking for assistance.

  If a command you are familiar with causes an Emacs error message in a
case where its usual definition ought to be reasonable, it is probably a
bug.

  If a command does the wrong thing, that is a bug.  But be sure you know
for certain what it ought to have done.  If you aren't familiar with the
command, or don't know for certain how the command is supposed to work,
then it might actually be working right.  Rather than jumping to
conclusions, show the problem to someone who knows for certain.

  Finally, a command's intended definition may not be best for editing
with.  This is a very important sort of problem, but it is also a matter of
judgment.  Also, it is easy to come to such a conclusion out of ignorance
of some of the existing features.  It is probably best not to complain
about such a problem until you have checked the documentation in the usual
ways, feel confident that you understand it, and know for certain that what
you want is not available.  If you are not sure what the command is
supposed to do after a careful reading of the manual, check the index and
glossary for any terms that may be unclear.  If you still do not
understand, this indicates a bug in the manual.  The manual's job is to
make everything clear.  It is just as important to report documentation
bugs as program bugs.

  If the on-line documentation string of a function or variable disagrees
with the manual, one of them must be wrong, so report the bug.

@page
@subsection How to Report a Bug

@findex emacs-version
  When you decide that there is a bug, it is important to report it and to
report it in a way which is useful.  What is most useful is an exact
description of what commands you type, starting with the shell command to
run Emacs, until the problem happens.  Always include the version number
of Emacs that you are using; type @kbd{M-x emacs-version} to print this.

  The most important principle in reporting a bug is to report @var{facts},
not hypotheses or categorizations.  It is always easier to report the facts,
but people seem to prefer to strain to posit explanations and report
them instead.  If the explanations are based on guesses about how Emacs is
implemented, they will be useless; we will have to try to figure out what
the facts must have been to lead to such speculations.  Sometimes this is
impossible.  But in any case, it is unnecessary work for us.

  For example, suppose that you type @kbd{C-x C-f /glorp/baz.ugh
@key{RET}}, visiting a file which (you know) happens to be rather large,
and Emacs prints out @samp{I feel pretty today}.  The best way to report
the bug is with a sentence like the preceding one, because it gives all the
facts and nothing but the facts.

  Do not assume that the problem is due to the size of the file and say,
``When I visit a large file, Emacs prints out @samp{I feel pretty today}.''
This is what we mean by ``guessing explanations''.  The problem is just as
likely to be due to the fact that there is a @samp{z} in the file name.  If
this is so, then when we got your report, we would try out the problem with
some ``large file'', probably with no @samp{z} in its name, and not find
anything wrong.  There is no way in the world that we could guess that we
should try visiting a file with a @samp{z} in its name.

  Alternatively, the problem might be due to the fact that the file starts
with exactly 25 spaces.  For this reason, you should make sure that you
inform us of the exact contents of any file that is needed to reproduce the
bug.  What if the problem only occurs when you have typed the @kbd{C-x C-a}
command previously?  This is why we ask you to give the exact sequence of
characters you typed since starting to use Emacs.

  You should not even say ``visit a file'' instead of @kbd{C-x C-f} unless
you @i{know} that it makes no difference which visiting command is used.
Similarly, rather than saying ``if I have three characters on the line,''
say ``after I type @kbd{@key{RET} A B C @key{RET} C-p},'' if that is
the way you entered the text.@refill

  If you are not in Fundamental mode when the problem occurs, you should
say what mode you are in.

  If the manifestation of the bug is an Emacs error message, it is
important to report not just the text of the error message but a backtrace
showing how the Lisp program in Emacs arrived at the error.  To make the
backtrace, you must execute the Lisp expression 
@code{(setq @w{debug-on-error t})} before the error happens (that is to
say, you must execute that expression and then make the bug happen).  This
causes the Lisp debugger to run (@pxref{Lisp Debug}).  The debugger's
backtrace can be copied as text into the bug report.  This use of the
debugger is possible only if you know how to make the bug happen again.  Do
note the error message the first time the bug happens, so if you can't make
it happen again, you can report at least that.

  Check whether any programs you have loaded into the Lisp world, including
your @file{.emacs} file, set any variables that may affect the functioning
of Emacs.  Also, see whether the problem happens in a freshly started Emacs
without loading your @file{.emacs} file (start Emacs with the @code{-q} switch
to prevent loading the init file.)  If the problem does @var{not} occur
then, it is essential that we know the contents of any programs that you
must load into the Lisp world in order to cause the problem to occur.

  If the problem does depend on an init file or other Lisp programs that
are not part of the standard Emacs system, then you should make sure it is
not a bug in those programs by complaining to their maintainers first.
After they verify that they are using Emacs in a way that is supposed to
work, they should report the bug.

  If you can tell us a way to cause the problem without visiting any files,
please do so.  This makes it much easier to debug.  If you do need files,
make sure you arrange for us to see their exact contents.  For example, it
can often matter whether there are spaces at the ends of lines, or a
newline after the last line in the buffer (nothing ought to care whether
the last line is terminated, but tell that to the bugs).

@findex open-dribble-file
@cindex dribble file
  The easy way to record the input to Emacs precisely is to to write a
dribble file; execute the Lisp expression

@example
(open-dribble-file "~/dribble")
@end example

@noindent
using @kbd{Meta-@key{ESC}} or from the @samp{*scratch*} buffer just after starting
Emacs.  From then on, all Emacs input will be written in the specified
dribble file until the Emacs process is killed.

@findex open-termscript
@cindex termscript file
  For possible display bugs, it is important to report the terminal type
(the value of environment variable @code{TERM}), the complete termcap entry
for the terminal from @file{/etc/termcap} (since that file is not identical
on all machines), and the output that Emacs actually sent to the terminal.
The way to collect this output is to execute the Lisp expression

@example
(open-termscript "~/termscript")
@end example

@noindent
using @kbd{Meta-@key{ESC}} or from the @samp{*scratch*} buffer just
after starting Emacs.  From then on, all output from Emacs to the terminal
will be written in the specified termscript file as well, until the Emacs
process is killed.  If the problem happens when Emacs starts up, put this
expression into your @file{.emacs} file so that the termscript file will
be open when Emacs displays the screen for the first time.  Be warned:
it is often difficult, and sometimes impossible, to fix a terminal-dependent
bug without access to a terminal of the type that stimulates the bug.@refill

  The address for reporting bugs is

@format
GNU Emacs Bugs
Lucid, Inc.
707 Laurel Street
Menlo Park, CA 94025
@end format

@noindent
or send email to @samp{sun!edsel!hotline} (Usenet) 

@noindent
or @samp{hotline@@lucid.com} (Internet).

You can also call the Lucid hotline at the following numbers:

@format
Phone:  (415) 327-1234
FAX:    (415) 321-2680
@end format

  Once again, we do not promise to fix the bug; but if the bug is serious,
or ugly, or easy to fix, chances are we will want to.

@iftex
@unnumbered Lucid GNU Emacs Features

This section describes the difference between Emacs Version 18 and 
Lucid GNU Emacs.

@unnumberedsec General Changes

@itemize @bullet
@item 
Lucid GNU Emacs has a new vi emulation mode called evi mode.  To
start evi mode in Emacs, type the command @kbd{M-x evi}.  If you want
Emacs to automatically put you in evi-mode all the time, include this
line in your @file{.emacs} file:
@example
(setq term-setup-hook 'evi)
@end example
@xref{evi Mode} for a brief discussion.
@item
Earlier versions of Emacs only allowed keybindings to ASCII character
sequences.  Lucid GNU Emacs has greatly expanded this by allowing you to
use a vector of key sequences which are, in turn, composed of a modifier
and a keysym. @xref{Keystrokes} for more information.

@item
The keymap data structure has been reimplemented to allow the use of a
character set larger than ASCII. Keymaps are no longer alists and/or
vectors; they are a new primary data type.  Consequently, code which
manipulated keymaps with list or array manipulation functions will no
longer work.  It must use the functions @code{define-key} or (the new
keymap functions) @code{map-keymap} and @code{set-keymap-parent}.
@xref{Key Bindings} for more information.

@item
You may now put variable settings in the first line using @code{-*-}.
@xref{File Variables} for more information.

@item
There is a new @file{tags} package and a new UNIX manual browsing
package. They are similar to earlier versions; for more information look
at the source code. 

@item
There is a new implementation of Dired, with many new features. The
online info for Dired @i{not} the Dired node of Emacs info, provides
more detail. 

@item
GNUS (a network news reader), VM (an alternative mail reader), ILISP (a
package for interacting with inferior Lisp processes), ANGE-FTP (a package
for making FTP-accessible files appear just like files on the local disk,
even to Dired) and Calendar (an Emacs-based calendar and appointment-
management tool) are a part of the Lucid GNU Emacs lisp library. See the
related documentation in the online info browser.

@item
Emacs now supports floating-point numbers.

@item
When you send mail, mail aliases are now expanded in the buffer. In
earlier versions, they were expanded after the mail-sending command was
executed.

@item
The initial value of @code{load-path} is computed when Emacs starts up,
instead of being hardcoded in when Emacs is compiled. As a result you
can now move the Emacs executable and Lisp library to a
different location in the file system without having to recompile.

@item
Any existing subdirectories of the Emacs Lisp directory are now added to the
@code{load-path} by default.

@item
On SPARCstations, you can change the audible bell using the
@code{sound-alist} variable. @xref{Audible Bell} for more information. 
@end itemize

Currently, Lucid GNU Emacs can only be used under the X window system.
It will soon be released for tty terminals as well.  If you run Lucid GNU
Emacs under X, there are a number of additional features:
@itemize @bullet
@item
You can use multiple X windows to display multiple Emacs screens. 
@item
You can use the X selection mechanism to copy material from other
applications and into other applications.  You can also use all Emacs
region commands on a region selected with the mouse. @xref{Mouse
Selection} for more information
@item
By default, the variable @code{zmacs-regions} is set to highlight the region
between point and the mark.  This unifies X selection and Emacs selection
behavior.  
@item
Lucid GNU Emacs has a menu bar for mouse-controlled operations in addition to
keystrokes.  @xref{Pull-down Menus}.
@item
Look in the file @file{/usr/lib/X11/app-defaults/Emacs} for a list of 
Emacs X resources.  You can set these resources in your X environment 
to set your preferences for color, fonts, location, and the size of Lucid 
Emacs screens.  Refer to your X documentation for more information 
about resources.
@item
A number of options for running under X can be set using a flag at
startup time.  @xref{Command Switches} for more information.
@item 
You can look up the release number of the X server in use with the
variable @code{x-release}.
@item 
You can look up the vendor supporting the X server in use with the
variable @code{x-vendor}.
@item 
You can allow synthetic (program-created) events to occur in Emacs by
setting the value of the variable @code{x-allow-sendevents} to
non-@code{nil}. A value of @code{nil} means that any synthetic event is
ignored. Caution: allowing Emacs to process @code{SendEvents} opens a big
security hole.
@end itemize
@vindex x-server
@vindex x-vendor
@vindex x-allow-sendevents

@unnumberedsec New Functions and Variables

You can conditionalize your @file{.emacs} file so that Lucid GNU Emacs 
commands are invoked only when you are in Lucid GNU Emacs as follows:

@cindex version number
@example
(cond ((string-match "Lucid" emacs-version)
        ...<any Lucid GNU Emacs commands>...))
@end example

These new functions are only present in Lucid GNU Emacs:

@itemize @bullet
@findex add-menu
@findex add-menu-item
@findex delete-menu-item
@findex disable-menu-item
@findex enable-menu-item
@findex relabel-menu-item
@item
@code{add-menu} lets you add a new menu to the menubar or a submenu to a
pull-down menu.  @code{add-menu-item}, @code{disable-menu-item},
@code{delete-menu-item}, @code{enable-menu-item}, and
@code{relabel-menu-item} allow you to customize the Lucid GNU Emacs
pull-down menus.

@findex byte-compile-and-load-file
@findex byte-compile-buffer
@item
@code{byte-compile-and-load-file} and @code{byte-compile-buffer}
byte-compiles the contents of a file or buffer.

@findex conx
The new @code{conx} function lets you generate random sentences for your
amusement.

@findex elisp-compile-defun
@item
@code{elisp-compile-defun} compiles and evaluates the current top-level
form.

@findex find-this-file
@findex find-this-file-other-window
@item
@code{find-this-file} and @code{find-this-file-other-window} can be used
interactively with a prefix argument to switch to the filename at point
in the buffer.  @code{find-this-file-other-window} displays the file in
another window.

@findex invert-face
@findex make-face-bold
@findex make-face-bold-italic
@findex make-face-italic
@findex make-face-unbold
@findex make-face-unitalic
@findex set-face-background
@findex set-face-background-pixmap
@findex set-face-font
@findex set-face-foreground
@findex set-face-underline-p
@item
Several new functions have been added that allow you to customize the
color and font attributes of a region of text: @code{invert-face},
@code{make-face-bold}, @code{make-face-bold-italic},
@code{make-face-italic}, @code{make-face-unbold},
@code{make-face-unitalic}, @code{set-face-background},
@code{set-face-background-pixmap}, @code{set-face-font},
@code{set-face-foreground}, and @code{set-face-underline-p}.

@findex load-default-sounds
@findex load-sound-file
@findex play-sound
@item
@code{load-default-sounds} and @code{load-sound-file} allow you to
customize the audible bell sound.  @code{load-default-sounds} loads and
install sound files.  @code{load-sound-file} reads in audio files and
adds them to the sound alist. @code{play-sound} plays the specified
sound type.

@findex local-unset-key
@item
@code{local-unset-key} removes the local binding of @var{key}.
@var{key} is a string, a vector of events, or a vector of
key-description lists as described in the definition of
@code{define-key}.

@findex locate-library
@item
@code{locate-library} finds the file that the function
@code{load-library} loads and displays the file's full pathname.

@findex make-directory
@findex remove-directory
@item
@code{make-directory} creates a directory, while @code{remove-directory}
removes a directory.

@findex make-obsolete
@item
@code{make-obsolete} makes the byte-compiler warn you if a specific
function is obsolete and what function to use instead.

@findex mark-bob
@findex mark-eob
@item
@code{mark-bob} and @code{mark-eob} pushes the mark to the beginning or
end of a buffer, respectively.

@findex mouse-del-char
@findex mouse-delete-window
@findex mouse-keep-one-window
@findex mouse-kill-line
@findex mouse-line-length
@findex mouse-scroll
@findex mouse-select
@findex mouse-select-and-split
@findex mouse-set-mark
@findex mouse-set-point
@findex mouse-track
@findex mouse-track-adjust
@findex mouse-track-and-copy-to-cutbuffer
@findex mouse-track-delete-and-insert
@findex mouse-track-insert
@findex mouse-window-to-region
Several functions have been added that allow you to perform various
editing, region, and window operations using the mouse:
@code{mouse-del-char}, @code{mouse-delete-window},
@code{mouse-keep-one-window}, @code{mouse-kill-line},
@code{mouse-line-length}, @code{mouse-scroll}, @code{mouse-select},
@code{mouse-select-and-split}, @code{mouse-set-mark},
@code{mouse-set-point}, @code{mouse-track}, @code{mouse-track-adjust},
@code{mouse-track-and-copy-to-cutbuffer},
@code{mouse-track-delete-and-insert}, @code{mouse-track-insert}, and
@code{mouse-window-to-region}.

@findex next-screen
@item
@code{next-screen} returns the next screen in the screen list after
specified screen.

@findex other-window-any-screen
@item
@code{other-window-any-screen} selects the nth specified window on any
screen.

@findex read-key-sequence
@item
@code{read-key-sequence} reads a sequence of keystrokes or mouse clicks
and returns a vector of the event objects read.

@findex register-to-window-config
@findex window-config-to-register
@item
@code{register-to-window-config} makes current the window
configuration in the given register.  @code{window-config-to-register}
saves the current window configuration in the given register.

@findex set-default-file-mode
@item
@code{set-default-file-mode} sets the UNIX @code{umask} value to the
specified value and returns the old value.

@findex start-timer
@item
@code{start-timer} creates a timer that you can set by providing values
to its arguments.

@findex switch-to-other-buffer
@item
@code{switch-to-other-buffer} switches you to the previous buffer in the
window.

@findex user-login-name
@item
@code{user-login-name} returns the login name of the user.

@findex x-copy-primary-selection
@findex x-delete-primary-selection
@findex x-kill-primary-selection
@findex x-own-selection
@findex x-own-secondary-selection
@findex x-set-point-and-insert-selection
@item 
Several functions have been added that allow you to make a selected
region of text the primary or secondary X selection, to insert the text
into a buffer at point, and to delete the selection with or without
copying it to the kill ring or X clipboard. The functions are:
@code{x-copy-primary-selection}, @code{x-delete-primary-selection},
@code{x-kill-primary-selection}, @code{x-own-selection},
@code{x-own-secondary-selection}, and
@code{x-set-point-and-insert-selection}.

@findex x-insert-selection
@findex x-yank-clipboard-selection
@item
@code{x-insert-selection} inserts selected text into a buffer at point.
@code{x-yank-clipboard-selection} inserts a clipboard selection into a
buffer at point.

@findex x-mouse-kill
@item
@code{x-mouse-kill} kills the text between point and mouse and copies it
to the X clipboard and the cut buffer.

@findex x-new-screen
@item
@code{x-new-screen} creates a new Emacs screen (X window).

@findex zmacs-activate-region
@findex zmacs-deactivate-region
@item
@code{zmacs-activate-region} puts the region of text between point and
the mark in the active state if @code{zmacs-region} is true.
@code{zmacs-deactivate-region} takes the region of text between point
andthe mark out of the active state.
@end itemize

These new variables are only present in Lucid GNU Emacs:

@itemize @bullet

@vindex activate-menubar-hook
@item
@code{activate-menubar-hook} takes as a value function or functions
called before a menu is pulled down and returns a list describing the
desired contents of the pull-down menu.

@vindex after-change-function
@item
@code{after-change-function} takes an alist of expressions to be evalled
when particular files are loaded.  It causes the corresponding elements
in the alist to be executed after files are loaded.

@vindex after-write-file-hooks
@item
@code{after-write-file-hooks} taskes a list of functions to be called
after writing out a buffer to a file.

@vindex auto-fill-inhibit-regexp
@item
@code{auto-fill-inhibit-regexp} matches lines that should not be
auto-filled.

@vindex auto-lower-screen
@vindex auto-raise-screen
@item
@code{auto-lower-screen} lowers screens to the bottom when they are no
longer selected.  @code{auto-raise-screen} raises the selected screen to
the top.

@vindex auto-save-timeout
@item
@code{auto-save-timeout} sets the amount of idle time.

@vindex buffer-file-truename
@vindex buffer-file-name
@vindex find-file-compare-truenames
@vindex find-file-use-truenames
@item 
The four variables  @code{buffer-file-name},
@code{buffer-file-truename}, @code{find-file-compare-truenames}, and
@code{find-file-use-truenames} allow Emacs to determine whether to consider
symbolic links when determining the path for a file. 

@vindex create-screen-hook
@item
@code{create-screen-hook} takes a function of one argument. The function
is called with each newly-created screen.

@vindex current-mouse-event
@item
@code{current-mouse-event} returns the last mouse event.

@vindex describe-function-show-arglist
@item
@code{describe-function-show-arglist} evaluates as a Lisp expression an
alist of major modes and their opinion on @code{default-directory}.

@vindex directory-abbrev-alist
@item
@code{directory-abbrev-alist} returns an alist of abbreviations for file
directories.

@vindex enable-local-variables
@item
@code{enable-local-variables} controls the use of local variables lists
in files that you visit.

@vindex float-output-format
@item
@code{float-output-format} returns the format descriptor string that
Lisp uses to print floats.

@vindex init-file-user
@item
@code{init-file-user} determines which init file @file{.emacs} uses when
it is read.

@vindex mail-abbrev-mailrc-file
@item
@code{mail-abbrev-mailrc-file} takes the name of a file with mail
aliases.

@vindex minibuffer-confirm-incomplete
@item
@code{minibuffer-confirm-incomplete} prompts for confirmation in 
contexts where completing-read allows answers that are not valid
completions.

@vindex screen-default-alist
@item
@code{screen-default-alist} takes an alist of default values for new
screens other than the first one.

@vindex screen-icon-title-format
@vindex screen-title-format
@item
@code{screen-title-format} and @code{screen-icon-title-format} determine
the title of the screen and the title of the icon that results if you
shrink the screen.

@vindex sound-alist
@item
@code{sound-alist} takes an alist of beep-types that are played when a
beep or ding is called.

@vindex tags-always-build-completion-table
@vindex tag-table-alist
@item
If @code{t}, @code{tags-always-build-completion-table} causes the tags
file to always be added to the completion table without asking first,
regardless of the size of the tags file.  @code{tag-table-alist} takes
a list that determines which @code{TAGS} file should be active for a
given buffer.

@vindex x-allow-sendevents
@item
If @code{t}, @code{x-allow-sendevents} allows synthetic events to occur.

@vindex x-mode-pointer-shape
@vindex x-nontext-pointer-shape
@vindex x-pointer-background-color
@vindex x-pointer-foreground-color
@vindex x-pointer-shape
@item
Several variables have been added that allow you to customize the color
and shape of the moue pointer: @code{x-pointer-background-color},
@code{x-pointer-foreground-color}, @code{x-mode-pointer-shape},
@code{x-pointer-shape}, and @* @code{x-nontext-pointer-shape}.

@vindex x-release
@vindex x-vendor
@item
@code{x-release} returns the release number of the X server in use.
@code{x-vendor} returns the vendor supporting the X server in use.

@vindex x-screen-count
@vindex x-screen-defaults
@vindex x-screen-height
@vindex x-screen-height-mm
@vindex x-screen-planes
@vindex x-screen-visual
@vindex x-screen-width
@vindex x-screen-width-mm
@item
@code{x-screen-count} returns the number of screens associated with the
current display. Several other new screen variables have been added that
allow you to customize an X screen:  @code{x-screen-defaults},
@code{x-screen-height}, @code{x-screen-height-mm},
@code{x-screen-planes}, @code{x-screen-visual}, @code{x-screen-width},
and @code{x-screen-width-mm}.

@vindex zmacs-regions
@item
@code{zmacs-regions} determines whether LISPM-style active regions
shoulde be used.
@end itemize

@unnumberedsec Changes in Emacs Functions and Bindings

In this version of Emacs, some commands are different from Emacs18.
This section describes changes to existing functions and function bindings,
and lists default key bindings. Some of the changes are due to the fact
that Lucid GNU Emacs is a multi-window X-based editor.

@itemize @bullet
@findex baud-rate
@item
The function @code{baud-rate} is now a variable instead of a function.

@findex buffer-disable-undo
@item
@code{buffer-flush-undo} has been renamed @code{buffer-
disable-undo}.

@findex compare-windows 
@item
@code{compare-windows} takes an argument @var{ignore-whitespace}.
The argument means ignore changes in whitespace.

@findex describe-key
@item
@code{describe-key} function's argument, @var{key}, is a string or
vector of events.  When called interactvely, @var{key} can also be a menu
selection.

@findex find-tag
@item
@code{find-tag} finds a tag whose name contains @var{tagname}.
The function selects the buffer that the tag is contained in
and puts point at its definition.

If the second argument @var{other-window} is non-@code{nil}, uses
another window to display the tag.

Multiple active tags tables and completion are supported.

Variables of note:
@vindex tag-table-alist
@vindex tags-file-name
@vindex tags-build-completion-table
@vindex buffer-tag-table
@vindex make-tags-files-invisible
@vindex tag-mark-stack-max

@itemize @bullet
@item tag-table-alist
@item tags-file-name		
@item tags-build-completion-table   
@item buffer-tag-table		
@item make-tags-files-invisible	
@item tag-mark-stack-max		
@end itemize

@findex interactive
@item
@code{interactive} has a new code, @code{e} for last mouse event. Also,
if the string begins with @samp{*}, an error is signaled if the buffer
is read-only.

@findex local-set-key 
@item
@code{local-set-key} gives @var{key} a local binding as
@code{command}.  The binding goes in the current buffer's local map,
which is shared with other buffers in the same major mode.

@findex make-shell
@findex make-comint
@item
@code{make-shell} has been changed.  It is defined as a 
synonym for @code{make-comint}.  See the Emacs extension Lisp file 
@file{comint.el} for documentation.  Note that @code{make-shell} in Emacs18 
takes @code{nil} as a command argument, but @code{make-shell} in
Lucid GNU Emacs does not.

@findex other-window
@item
@code{other-window} has an optional second argument
@var{all_screens}.  When @var{all_screens} is non-@code{nil}, the
function cycles through all screens.

@findex visit-tags-table
@item
@code{visit-tags-table} tells tags commands to use the tags
table file @var{file} first.  @var{file} should be the name of a file
created with the @code{etags} program.  This function is largely
obsoleted by the variable @code{tag-table-alist}.
@end itemize

Lucid GNU Emacs has the following default function keybindings:

@table @kbd
@item @key{HELP}	
Same as @kbd{C-h}.

@item @key{UNDO}	
Same as @kbd{M-x undo}.

@item @key{CUT}  	
Same as the cut menu item; that is, works on a 
selection, not a region.

@item @key{COPY}	
Same as the Copy menu item.

@item @key{PASTE}	
Same as the Paste menu item.

@item @key{PGUP}	
Same as @kbd{M-v}.

@item @key{PGDN}	
Same as @kbd{C-v}.

@item @key{HOME}	
Same as @kbd{M-<}.

@item @key{END} 	
Same as @kbd{M->}.

@item @key{LEFT-ARROW}	
Same as the function @code{backward-char}.

@item @key{RIGHT-ARROW}	
Same as the function @code{forward-char}.

@item @key{UP-ARROW}	
Same as the function @code{previous-line}.

@item @key{DOWN-ARROW}	
Same as the function @code{next-line}.

@end table

@unnumberedsec Changes in Emacs Variables

This section describes changes to existing Emacs variables.

@itemize @bullet

@vindex auto-fill-hook
@item
The variable @code{auto-fill-hook} has been renamed
@code{auto-fill-function}.

@vindex blink-paren-hook
@item
The variable @code{blink-paren-hook} has been renamed
@code{blink-paren-function}.

@vindex default-directory
@item
The value of the variable @code{default-directory} should end with a slash.

@vindex executing-kbd-macro
@item
The value of the variable @code{executing-kbd-macro} is the currently
executing keyboard macro (a vector of events); a value of @code{nil} is
returned if none are executing.

@vindex executing-macro
@item
The value of the variable @code{executing-macro} is the currently
executing keyboard macro (a vector of events); a value of @code{nil} is
returned if none are executing.

@vindex inhibit-local-variables
@item
The variable @code{inhibit-local-variables} has been replaced with
@code{enable-local-variables} with the definition reversed.

@vindex inverse-video
@item
The variable @code{inverse-video} is currently ignored; this will
become a parameter of the default face on tty screens.  For X
screens, simply set the foreground and background colors
appropriately.

@vindex keyboard-translate-table
@item
The value of the variable @code{keyboard-translate-table} is a string
used as translate table for keyboard input or @code{nil}.  Use this
variable only if you are on a dumb tty, since it can only handle input
that is represented as ASCII.  If you are running Emacs under X, you
should do the translations with the @code{xmodmap} program instead.

@vindex last-command-char
@vindex last-input-char
@item
The value of @code{last-command-cahr} is the last keyboard event that
was part of a command.  This is the value that the @code{self-insert}
command puts in the buffer.  The value of @code{last-input-char} is the
last keyboard event received.  Note that for both variables there is not
a 1:1 mapping between keyboard events and ASCII characters: the set of
keyboard events is much larger, it is not advisable to write code that
examines this variable to determine what key has been typed unless you
are certain that it will be one of a small set of characters.

@vindex last-kbd-macro
@item
The value of the variable @code{last-kbd-macro} is the last keyboard macro
defined as a vector of events; a value of @code{nil} is returned if none
are defined.

@vindex lisp-indent-hook
@item
The variable @code{lisp-indent-hook} has been renamed
@code{lisp-indent-function}.

@vindex load-path
@item
The value of the variable @code{load-path} is the list of directories to
search for files to load.  Note that the elements of this list should
not begin with "@samp{~}", you must call @code{expand-file-name} on
them before adding them to the list.

The value is specified in the EMACSLOADPATH environment variable;
otherwise, the default value is specified in the file @file{paths.h}
when Emacs was built.  If there were no paths specified in
@file{paths.h}, Emacs chooses a default value for this variable by
looking around in the file system near the directory in which the Emacs
executable resides.

@vindex mode-line-format
The @code{mode-line-format} and similar strings accept a new format
directive:

%S -- print name of selected screen (applicable only under X Windows).

@vindex mode-line-inverse-video
@item
The variable @code{mode-line-inverse-video} is currently ignored; this will
become a parameter of the @code{modeline} face on tty screens. 

@vindex tags-file-name
@item
The value of the variable @code{tags-file-name} is the name of the tags
table used by all buffers.  This variable is largely supplanted by the
variable @code{tag-table-alist}.

@vindex temp-buffer-show-hook
@item
The variable @code{temp-buffer-show-hook} has been renamed
@code{temp-buffer-show-function}.

@end itemize

@end iftex

@node Manifesto,, Bugs, Top
@unnumbered The GNU Manifesto

@unnumberedsec What's GNU?  Gnu's Not Unix!

GNU, which stands for Gnu's Not Unix, is the name for the complete
Unix-compatible software system which I am writing so that I can give it
away free to everyone who can use it.  Several other volunteers are helping
me.  Contributions of time, money, programs and equipment are greatly
needed.

So far we have an Emacs text editor with Lisp for writing editor commands,
a source level debugger, a yacc-compatible parser generator, a linker, and
around 35 utilities.  A shell (command interpreter) is nearly completed.  A
new portable optimizing C compiler has compiled itself and may be released
this year.  An initial kernel exists but many more features are needed to
emulate Unix.  When the kernel and compiler are finished, it will be
possible to distribute a GNU system suitable for program development.  We
will use @TeX{} as our text formatter, but an nroff is being worked on.  We
will use the free, portable X window system as well.  After this we will
add a portable Common Lisp, an Empire game, a spreadsheet, and hundreds of
other things, plus on-line documentation.  We hope to supply, eventually,
everything useful that normally comes with a Unix system, and more.

GNU will be able to run Unix programs, but will not be identical to Unix.
We will make all improvements that are convenient, based on our experience
with other operating systems.  In particular, we plan to have longer
filenames, file version numbers, a crashproof file system, filename
completion perhaps, terminal-independent display support, and perhaps
eventually a Lisp-based window system through which several Lisp programs
and ordinary Unix programs can share a screen.  Both C and Lisp will be
available as system programming languages.  We will try to support UUCP,
MIT Chaosnet, and Internet protocols for communication.

GNU is aimed initially at machines in the 68000/16000 class with virtual
memory, because they are the easiest machines to make it run on.  The extra
effort to make it run on smaller machines will be left to someone who wants
to use it on them.

To avoid horrible confusion, please pronounce the `G' in the word `GNU'
when it is the name of this project.

@page
@unnumberedsec Why I Must Write GNU

I consider that the golden rule requires that if I like a program I must
share it with other people who like it.  Software sellers want to divide
the users and conquer them, making each user agree not to share with
others.  I refuse to break solidarity with other users in this way.  I
cannot in good conscience sign a nondisclosure agreement or a software
license agreement.  For years I worked within the Artificial Intelligence
Lab to resist such tendencies and other inhospitalities, but eventually
they had gone too far: I could not remain in an institution where such
things are done for me against my will.

So that I can continue to use computers without dishonor, I have decided to
put together a sufficient body of free software so that I will be able to
get along without any software that is not free.  I have resigned from the
AI lab to deny MIT any legal excuse to prevent me from giving GNU away.

@unnumberedsec Why GNU Will Be Compatible with Unix

Unix is not my ideal system, but it is not too bad.  The essential features
of Unix seem to be good ones, and I think I can fill in what Unix lacks
without spoiling them.  And a system compatible with Unix would be
convenient for many other people to adopt.

@unnumberedsec How GNU Will Be Available

GNU is not in the public domain.  Everyone will be permitted to modify and
redistribute GNU, but no distributor will be allowed to restrict its
further redistribution.  That is to say, proprietary modifications will not
be allowed.  I want to make sure that all versions of GNU remain free.

@unnumberedsec Why Many Other Programmers Want to Help

I have found many other programmers who are excited about GNU and want to
help.

Many programmers are unhappy about the commercialization of system
software.  It may enable them to make more money, but it requires them to
feel in conflict with other programmers in general rather than feel as
comrades.  The fundamental act of friendship among programmers is the
sharing of programs; marketing arrangements now typically used essentially
forbid programmers to treat others as friends.  The purchaser of software
must choose between friendship and obeying the law.  Naturally, many decide
that friendship is more important.  But those who believe in law often do
not feel at ease with either choice.  They become cynical and think that
programming is just a way of making money.

By working on and using GNU rather than proprietary programs, we can be
hospitable to everyone and obey the law.  In addition, GNU serves as an
example to inspire and a banner to rally others to join us in sharing.
This can give us a feeling of harmony which is impossible if we use
software that is not free.  For about half the programmers I talk to, this
is an important happiness that money cannot replace.

@unnumberedsec How You Can Contribute

I am asking computer manufacturers for donations of machines and money.
I'm asking individuals for donations of programs and work.

One consequence you can expect if you donate machines is that GNU will run
on them at an early date.  The machines should be complete, ready to use
systems, approved for use in a residential area, and not in need of
sophisticated cooling or power.

I have found very many programmers eager to contribute part-time work for
GNU.  For most projects, such part-time distributed work would be very hard
to coordinate; the independently-written parts would not work together.
But for the particular task of replacing Unix, this problem is absent.  A
complete Unix system contains hundreds of utility programs, each of which
is documented separately.  Most interface specifications are fixed by Unix
compatibility.  If each contributor can write a compatible replacement for
a single Unix utility, and make it work properly in place of the original
on a Unix system, then these utilities will work right when put together.
Even allowing for Murphy to create a few unexpected problems, assembling
these components will be a feasible task.  (The kernel will require closer
communication and will be worked on by a small, tight group.)

If I get donations of money, I may be able to hire a few people full or
part time.  The salary won't be high by programmers' standards, but I'm
looking for people for whom building community spirit is as important as
making money.  I view this as a way of enabling dedicated people to devote
their full energies to working on GNU by sparing them the need to make a
living in another way.

@unnumberedsec Why All Computer Users Will Benefit

Once GNU is written, everyone will be able to obtain good system software
free, just like air.

This means much more than just saving everyone the price of a Unix license.
It means that much wasteful duplication of system programming effort will
be avoided.  This effort can go instead into advancing the state of the
art.

Complete system sources will be available to everyone.  As a result, a user
who needs changes in the system will always be free to make them himself,
or hire any available programmer or company to make them for him.  Users
will no longer be at the mercy of one programmer or company which owns the
sources and is in sole position to make changes.

Schools will be able to provide a much more educational environment by
encouraging all students to study and improve the system code.  Harvard's
computer lab used to have the policy that no program could be installed on
the system if its sources were not on public display, and upheld it by
actually refusing to install certain programs.  I was very much inspired by
this.

Finally, the overhead of considering who owns the system software and what
one is or is not entitled to do with it will be lifted.

Arrangements to make people pay for using a program, including licensing of
copies, always incur a tremendous cost to society through the cumbersome
mechanisms necessary to figure out how much (that is, which programs) a
person must pay for.  And only a police state can force everyone to obey
them.  Consider a space station where air must be manufactured at great
cost: charging each breather per liter of air may be fair, but wearing the
metered gas mask all day and all night is intolerable even if everyone can
afford to pay the air bill.  And the TV cameras everywhere to see if you
ever take the mask off are outrageous.  It's better to support the air
plant with a head tax and chuck the masks.

Copying all or parts of a program is as natural to a programmer as
breathing, and as productive.  It ought to be as free.

@unnumberedsec Some Easily Rebutted Objections to GNU's Goals

@quotation
``Nobody will use it if it is free, because that means they can't rely
on any support.''

``You have to charge for the program to pay for providing the
support.''
@end quotation

If people would rather pay for GNU plus service than get GNU free without
service, a company to provide just service to people who have obtained GNU
free ought to be profitable.

We must distinguish between support in the form of real programming work
and mere handholding.  The former is something one cannot rely on from a
software vendor.  If your problem is not shared by enough people, the
vendor will tell you to get lost.

If your business needs to be able to rely on support, the only way is to
have all the necessary sources and tools.  Then you can hire any available
person to fix your problem; you are not at the mercy of any individual.
With Unix, the price of sources puts this out of consideration for most
businesses.  With GNU this will be easy.  It is still possible for there to
be no available competent person, but this problem cannot be blamed on
distibution arrangements.  GNU does not eliminate all the world's problems,
only some of them.

Meanwhile, the users who know nothing about computers need handholding:
doing things for them which they could easily do themselves but don't know
how.

Such services could be provided by companies that sell just hand-holding
and repair service.  If it is true that users would rather spend money and
get a product with service, they will also be willing to buy the service
having got the product free.  The service companies will compete in quality
and price; users will not be tied to any particular one.  Meanwhile, those
of us who don't need the service should be able to use the program without
paying for the service.

@quotation
``You cannot reach many people without advertising,
and you must charge for the program to support that.''

``It's no use advertising a program people can get free.''
@end quotation

There are various forms of free or very cheap publicity that can be used to
inform numbers of computer users about something like GNU.  But it may be
true that one can reach more microcomputer users with advertising.  If this
is really so, a business which advertises the service of copying and
mailing GNU for a fee ought to be successful enough to pay for its
advertising and more.  This way, only the users who benefit from the
advertising pay for it.

On the other hand, if many people get GNU from their friends, and such
companies don't succeed, this will show that advertising was not really
necessary to spread GNU.  Why is it that free market advocates don't want
to let the free market decide this?
@page
@quotation
``My company needs a proprietary operating system
to get a competitive edge.''
@end quotation

GNU will remove operating system software from the realm of competition.
You will not be able to get an edge in this area, but neither will your
competitors be able to get an edge over you.  You and they will compete in
other areas, while benefitting mutually in this one.  If your business is
selling an operating system, you will not like GNU, but that's tough on
you.  If your business is something else, GNU can save you from being
pushed into the expensive business of selling operating systems.

I would like to see GNU development supported by gifts from many
manufacturers and users, reducing the cost to each.

@quotation
``Don't programmers deserve a reward for their creativity?''
@end quotation

If anything deserves a reward, it is social contribution.  Creativity can
be a social contribution, but only in so far as society is free to use the
results.  If programmers deserve to be rewarded for creating innovative
programs, by the same token they deserve to be punished if they restrict
the use of these programs.

@quotation
``Shouldn't a programmer be able to ask for a reward for his creativity?''
@end quotation

There is nothing wrong with wanting pay for work, or seeking to maximize
one's income, as long as one does not use means that are destructive.  But
the means customary in the field of software today are based on
destruction.

Extracting money from users of a program by restricting their use of it is
destructive because the restrictions reduce the amount and the ways that
the program can be used.  This reduces the amount of wealth that humanity
derives from the program.  When there is a deliberate choice to restrict,
the harmful consequences are deliberate destruction.

The reason a good citizen does not use such destructive means to become
wealthier is that, if everyone did so, we would all become poorer from the
mutual destructiveness.  This is Kantian ethics; or, the Golden Rule.
Since I do not like the consequences that result if everyone hoards
information, I am required to consider it wrong for one to do so.
Specifically, the desire to be rewarded for one's creativity does not
justify depriving the world in general of all or part of that creativity.

@quotation
``Won't programmers starve?''
@end quotation

I could answer that nobody is forced to be a programmer.  Most of us cannot
manage to get any money for standing on the street and making faces.  But
we are not, as a result, condemned to spend our lives standing on the
street making faces, and starving.  We do something else.

But that is the wrong answer because it accepts the questioner's implicit
assumption: that without ownership of software, programmers cannot possibly
be paid a cent.  Supposedly it is all or nothing.

The real reason programmers will not starve is that it will still be
possible for them to get paid for programming; just not paid as much as
now.

Restricting copying is not the only basis for business in software.  It is
the most common basis because it brings in the most money.  If it were
prohibited, or rejected by the customer, software business would move to
other bases of organization which are now used less often.  There are
always numerous ways to organize any kind of business.

Probably programming will not be as lucrative on the new basis as it is
now.  But that is not an argument against the change.  It is not considered
an injustice that sales clerks make the salaries that they now do.  If
programmers made the same, that would not be an injustice either.  (In
practice they would still make considerably more than that.)

@quotation
``Don't people have a right to control how their creativity is used?''
@end quotation

``Control over the use of one's ideas'' really constitutes control over
other people's lives; and it is usually used to make their lives more
difficult.

People who have studied the issue of intellectual property rights carefully
(such as lawyers) say that there is no intrinsic right to intellectual
property.  The kinds of supposed intellectual property rights that the
government recognizes were created by specific acts of legislation for
specific purposes.

For example, the patent system was established to encourage inventors to
disclose the details of their inventions.  Its purpose was to help society
rather than to help inventors.  At the time, the life span of 17 years for
a patent was short compared with the rate of advance of the state of the
art.  Since patents are an issue only among manufacturers, for whom the
cost and effort of a license agreement are small compared with setting up
production, the patents often do not do much harm.  They do not obstruct
most individuals who use patented products.

The idea of copyright did not exist in ancient times, when authors
frequently copied other authors at length in works of non-fiction.  This
practice was useful, and is the only way many authors' works have survived
even in part.  The copyright system was created expressly for the purpose
of encouraging authorship.  In the domain for which it was
invented---books, which could be copied economically only on a printing
press---it did little harm, and did not obstruct most of the individuals
who read the books.

All intellectual property rights are just licenses granted by society
because it was thought, rightly or wrongly, that society as a whole would
benefit by granting them.  But in any particular situation, we have to ask:
are we really better off granting such license?  What kind of act are we
licensing a person to do?

The case of programs today is very different from that of books a hundred
years ago.  The fact that the easiest way to copy a program is from one
neighbor to another, the fact that a program has both source code and
object code which are distinct, and the fact that a program is used rather
than read and enjoyed, combine to create a situation in which a person who
enforces a copyright is harming society as a whole both materially and
spiritually; in which a person should not do so regardless of whether the
law enables him to.

@quotation
``Competition makes things get done better.''
@end quotation

The paradigm of competition is a race: by rewarding the winner, we
encourage everyone to run faster.  When capitalism really works this way,
it does a good job; but its defenders are wrong in assuming it always works
this way.  If the runners forget why the reward is offered and become
intent on winning, no matter how, they may find other strategies---such as,
attacking other runners.  If the runners get into a fist fight, they will
all finish late.

Proprietary and secret software is the moral equivalent of runners in a
fist fight.  Sad to say, the only referee we've got does not seem to
object to fights; he just regulates them (``For every ten yards you run,
you can fire one shot'').  He really ought to break them up, and penalize
runners for even trying to fight.

@quotation
``Won't everyone stop programming without a monetary incentive?''
@end quotation

Actually, many people will program with absolutely no monetary incentive.
Programming has an irresistible fascination for some people, usually the
people who are best at it.  There is no shortage of professional musicians
who keep at it even though they have no hope of making a living that way.

But really this question, though commonly asked, is not appropriate to the
situation.  Pay for programmers will not disappear, only become less.  So
the right question is, will anyone program with a reduced monetary
incentive?  My experience shows that they will.

For more than ten years, many of the world's best programmers worked at the
Artificial Intelligence Lab for far less money than they could have had
anywhere else.  They got many kinds of non-monetary rewards: fame and
appreciation, for example.  And creativity is also fun, a reward in itself.
@page
Then most of them left when offered a chance to do the same interesting
work for a lot of money.

What the facts show is that people will program for reasons other than
riches; but if given a chance to make a lot of money as well, they will
come to expect and demand it.  Low-paying organizations do poorly in
competition with high-paying ones, but they do not have to do badly if the
high-paying ones are banned.

@quotation
``We need the programmers desperately.  If they demand that we
stop helping our neighbors, we have to obey.''
@end quotation

You're never so desperate that you have to obey this sort of demand.
Remember: millions for defense, but not a cent for tribute!

@quotation
``Programmers need to make a living somehow.''
@end quotation

In the short run, this is true.  However, there are plenty of ways that
programmers could make a living without selling the right to use a program.
This way is customary now because it brings programmers and businessmen the
most money, not because it is the only way to make a living.  It is easy to
find other ways if you want to find them.  Here are a number of examples.

A manufacturer introducing a new computer will pay for the porting of
operating systems onto the new hardware.

The sale of teaching, hand-holding and maintenance services could also
employ programmers.

People with new ideas could distribute programs as freeware, asking for
donations from satisfied users, or selling hand-holding services.  I have
met people who are already working this way successfully.

Users with related needs can form users' groups, and pay dues.  A group
would contract with programming companies to write programs that the
group's members would like to use.

All sorts of development can be funded with a Software Tax:

@quotation
Suppose everyone who buys a computer has to pay x percent of
the price as a software tax.  The government gives this to
an agency like the NSF to spend on software development.

But if the computer buyer makes a donation to software development
himself, he can take a credit against the tax.  He can donate to
the project of his own choosing---often, chosen because he hopes to
use the results when
@page 
it is done.  He can take a credit for any amount
of donation up to the total tax he had to pay.

The total tax rate could be decided by a vote of the payers of
the tax, weighted according to the amount they will be taxed on.

The consequences:

@itemize @bullet
@item
The computer-using community supports software development.
@item
This community decides what level of support is needed.
@item
Users who care which projects their share is spent on
can choose this for themselves.
@end itemize
@end quotation

In the long run, making programs free is a step toward the post-scarcity
world, where nobody will have to work very hard just to make a living.
People will be free to devote themselves to activities that are fun, such
as programming, after spending the necessary ten hours a week on required
tasks such as legislation, family counseling, robot repair and asteroid
prospecting.  There will be no need to be able to make a living from
programming.

We have already greatly reduced the amount of work that the whole society
must do for its actual productivity, but only a little of this has
translated itself into leisure for workers because much nonproductive
activity is required to accompany productive activity.  The main causes of
this are bureaucracy and isometric struggles against competition.  Free
software will greatly reduce these drains in the area of software
production.  We must do this, in order for technical gains in productivity
to translate into less work for us.

@node Glossary, Key Index, Intro, Top
@unnumbered Glossary

@table @asis
@item Abbrev
An abbrev is a text string which expands into a different text string
when present in the buffer.  For example, you might define a short
word as an abbrev for a long phrase that you want to insert
frequently.  @xref{Abbrevs}.

@item Aborting
Aborting means getting out of a recursive edit (q.v.@:).  You can use
the commands @kbd{C-]} and @kbd{M-x top-level} for this.
@xref{Quitting}.

@item Auto Fill mode
Auto Fill mode is a minor mode in which text you insert is
automatically broken into lines of fixed width.  @xref{Filling}.

@item Auto Saving
Auto saving means that Emacs automatically stores the contents of an
Emacs buffer in a specially-named file so the information will not be
lost if the buffer is lost due to a system error or user error.
@xref{Auto Save}.

@item Backup File
A backup file records the contents that a file had before the current
editing session.  Emacs creates backup files automatically to help you
track down or cancel changes you later regret.  @xref{Backup}.

@item Balance Parentheses
Emacs can balance parentheses manually or automatically.  Manual
balancing is done by the commands to move over balanced expressions
(@pxref{Lists}).  Automatic balancing is done by blinking the
parenthesis that matches one just inserted (@pxref{Matching,,Matching
Parens}).

@item Bind
To bind a key is to change its binding (q.v.@:).  @xref{Rebinding}.

@item Binding
A key gets its meaning in Emacs by having a binding which is a
command (q.v.@:), a Lisp function that is run when the key is typed.
@xref{Commands,Binding}.  Customization often involves rebinding a
character to a different command function.  The bindings of all keys
are recorded in the keymaps (q.v.@:).  @xref{Keymaps}.

@item Blank Lines
Blank lines are lines that contain only whitespace.  Emacs has several
commands for operating on the blank lines in a buffer.

@item Buffer
The buffer is the basic editing unit; one buffer corresponds to one
piece of text being edited.  You can have several buffers, but at any
time you are editing only one, the `selected' buffer, though several
buffers can be visible when you are using multiple windows.  @xref{Buffers}.

@item Buffer Selection History
Emacs keeps a buffer selection history which records how recently each
Emacs buffer was selected.  Emacs uses this list when choosing a buffer to
select.  @xref{Buffers}.

@item C-
@samp{C} in the name of a character is an abbreviation for Control.
@xref{Keystrokes,C-}.

@item C-M-
@samp{C-M-} in the name of a character is an abbreviation for
Control-Meta.  @xref{Keystrokes,C-M-}.

@item Case Conversion
Case conversion means changing text from upper case to lower case or
vice versa.  @xref{Case}, for the commands for case conversion.

@item Characters
Characters form the contents of an Emacs buffer; also, Emacs commands
are invoked by keys (q.v.@:), which are sequences of one or more
characters.  @xref{Keystrokes}.

@item Command
A command is a Lisp function specially defined to be able to serve as a
key binding in Emacs.  When you type a key (q.v.@:), Emacs looks up its
binding (q.v.@:) in the relevant keymaps (q.v.@:) to find the command to
run.  @xref{Commands}.

@item Command Name
A command name is the name of a Lisp symbol which is a command
(@pxref{Commands}).  You can invoke any command by its name using
@kbd{M-x} (@pxref{M-x}).

@item Comments
A comment is text in a program which is intended only for the people
reading the program, and is marked specially so that it will be
ignored when the program is loaded or compiled.  Emacs offers special
commands for creating, aligning and killing comments.
@xref{Comments}.

@item Compilation
Compilation is the process of creating an executable program from
source code.  Emacs has commands for compiling files of Emacs Lisp
code (@pxref{Lisp Libraries}) and programs in C and other languages
(@pxref{Compilation}).

@item Complete Key
A complete key is a character or sequence of characters which, when typed
by the user, fully specifies one action to be performed by Emacs.  For
example, @kbd{X} and @kbd{Control-f} and @kbd{Control-x m} are keys.  Keys
derive their meanings from being bound (q.v.@:) to commands (q.v.@:).
Thus, @kbd{X} is conventionally bound to a command to insert @samp{X} in
the buffer; @kbd{C-x m} is conventionally bound to a command to begin
composing a mail message. @xref{Keystrokes}.

@item Completion
When Emacs automatically fills an abbreviation for a name into the
entire name, that process is called completion.  Completion is done for
minibuffer (q.v.@:) arguments when the set of possible valid inputs is
known; for example, on command names, buffer names, and file names.
Completion occurs when you type @key{TAB}, @key{SPC} or @key{RET}.
@xref{Completion}.@refill

@item Continuation Line
When a line of text is longer than the width of the screen, it
takes up more than one screen line when displayed.  We say that the
text line is continued, and all screen lines used for it after the
first are called continuation lines.  @xref{Basic,Continuation,Basic
Editing}.

@item Control-Character
ASCII characters with octal codes 0 through 037, and also code 0177,
do not have graphic images assigned to them.  These are the control
characters.  Any control character can be typed by holding down the
@key{CTRL} key and typing some other character; some have special keys
on the keyboard.  @key{RET}, @key{TAB}, @key{ESC}, @key{LFD} and
@key{DEL} are all control characters.  @xref{Keystrokes}.@refill

@item Copyleft
A copyleft is a notice giving the public legal permission to redistribute
a program or other work of art.  Copylefts are used by leftists to enrich
the public just as copyrights are used by rightists to gain power over
the public.

@item Current Buffer
The current buffer in Emacs is the Emacs buffer on which most editing
commands operate.  You can select any Emacs buffer as the current one.
@xref{Buffers}.

@item Current Line
The line point is on (@pxref{Point}).

@item Current Paragraph
The paragraph that point is in.  If point is between paragraphs, the
current paragraph is the one that follows point.  @xref{Paragraphs}.

@item Current Defun
The defun (q.v.@:) that point is in.  If point is between defuns, the
current defun is the one that follows point.  @xref{Defuns}.

@item Cursor
The cursor is the rectangle on the screen which indicates the position
called point (q.v.@:) at which insertion and deletion takes place.
The cursor is on or under the character that follows point.  Often
people speak of `the cursor' when, strictly speaking, they mean
`point'.  @xref{Basic,Cursor,Basic Editing}.

@item Customization
Customization is making minor changes in the way Emacs works.  It is
often done by setting variables (@pxref{Variables}) or by rebinding
keys (@pxref{Keymaps}).

@item Default Argument
The default for an argument is the value that is used if you do not
specify one.  When Emacs prompts you in the minibuffer for an argument,
the default argument is used if you just type @key{RET}.
@xref{Minibuffer}.

@item Default Directory
When you specify a file name that does not start with @samp{/} or @samp{~},
it is interpreted relative to the current buffer's default directory.
@xref{Minibuffer File,Default Directory}.

@item Defun
A defun is a list at the top level of parenthesis or bracket structure
in a program.  It is so named because most such lists in Lisp programs
are calls to the Lisp function @code{defun}.  @xref{Defuns}.

@item @key{DEL}
The @key{DEL} character runs the command that deletes one character of
text.  @xref{Basic,DEL,Basic Editing}.

@item Deletion
Deleting text means erasing it without saving it.  Emacs deletes text
only when it is expected not to be worth saving (all whitespace, or
only one character).  The alternative is killing (q.v.@:).
@xref{Killing,Deletion}.

@item Deletion of Files
Deleting a file means removing it from the file system.
@xref{Misc File Ops}.

@page
@item Deletion of Messages
Deleting a message means flagging it to be eliminated from your mail
file.  Until the mail file is expunged, you can undo this by undeleting
the message.  @xref{Rmail Deletion}.

@item Deletion of Screens
When working under the multi-screen X-based version of Lucid GNU Emacs,
you can delete individual screens using the @b{Close} menu item from the
@b{File} menu.

@item Deletion of Windows
When you delete a subwindow of an Emacs screen, you eliminate it from
the screen.  Other windows expand to use up the space.  The deleted
window can never come back, but no actual text is lost.  @xref{Windows}.

@item Directory
Files in the Unix file system are grouped into file directories.
@xref{ListDir,,Directories}.

@item Dired
Dired is the Emacs facility that displays the contents of a file
directory and allows you to ``edit the directory'', performing
operations on the files in the directory.  @xref{Dired}.

@item Disabled Command
A disabled command is one that you may not run without special
confirmation.  Commands are usually disabled because they are
confusing for beginning users.  @xref{Disabling}.

@item Dribble File
A file into which Emacs writes all the characters that the user types
on the keyboard.  Dribble files are used to make a record for
debugging Emacs bugs.  Emacs does not make a dribble file unless you
tell it to.  @xref{Bugs}.

@item Echo Area
The area at the bottom of the Emacs screen which is used for echoing the
arguments to commands, for asking questions, and for printing brief
messages (including error messages).  @xref{Echo Area}.

@item Echoing
Echoing refers to acknowledging the receipt of commands by displaying them
(in the echo area).  Emacs never echoes single-character keys; longer
keys echo only if you pause while typing them.

@item Error
An error occurs when an Emacs command cannot execute in the current
circumstances.  When an error occurs, execution of the command stops
(unless the command has been programmed to do otherwise) and Emacs
reports the error by printing an error message (q.v.).  Type-ahead
is discarded.  Then Emacs is ready to read another editing command.

@item Error Messages
Error messages are single lines of output printed by Emacs when the
user asks for something impossible to do (such as, killing text
forward when point is at the end of the buffer).  They appear in the
echo area, accompanied by a beep.

@item @key{ESC}
@key{ESC} is a character used as a prefix for typing Meta characters on
keyboards lacking a @key{META} key.  Unlike the @key{META} key (which,
like the @key{SHIFT} key, is held down while another character is
typed), the @key{ESC} key is pressed and released, and applies to the
next character typed. 

@item Fill Prefix
The fill prefix is a string that is Emacs enters at the beginning
of each line when it performs filling.  It is not regarded as part of the
text to be filled.  @xref{Filling}.

@item Filling
Filling text means moving text from line to line so that all the lines
are approximately the same length.  @xref{Filling}.

@item Global
Global means `independent of the current environment; in effect
@*throughout Emacs'.  It is the opposite of local (q.v.@:).
Examples of the use of `global' appear below.

@item Global Abbrev
A global definition of an abbrev (q.v.@:) is effective in all major
modes that do not have local (q.v.@:) definitions for the same abbrev.
@xref{Abbrevs}.

@item Global Keymap
The global keymap (q.v.@:) contains key bindings that are in effect
unless local key bindings in a major mode's local
keymap (q.v.@:) override them.@xref{Keymaps}.

@item Global Substitution
Global substitution means replacing each occurrence of one string by
another string through a large amount of text.  @xref{Replace}.

@item Global Variable
The global value of a variable (q.v.@:) takes effect in all buffers
that do not have their own local (q.v.@:) values for the variable.
@xref{Variables}.

@item Graphic Character
Graphic characters are those assigned pictorial images rather than
just names.  All the non-Meta (q.v.@:) characters except for the
Control (q.v.@:) characters are graphic characters.  These include
letters, digits, punctuation, and spaces; they do not include
@key{RET} or @key{ESC}.  In Emacs, typing a graphic character inserts
that character (in ordinary editing modes).  @xref{Basic,,Basic Editing}.

@item Grinding
Grinding means adjusting the indentation in a program to fit the
nesting structure.  @xref{Indentation,Grinding}.

@item Hardcopy
Hardcopy means printed output.  Emacs has commands for making printed
listings of text in Emacs buffers.  @xref{Hardcopy}.

@item @key{HELP}
You can type @key{HELP} at any time to ask what options you have, or
to ask what any command does.  @key{HELP} is really @kbd{Control-h}.
@xref{Help}.

@item Inbox
An inbox is a file in which mail is delivered by the operating system.
Rmail transfers mail from inboxes to mail files (q.v.) in which the
mail is then stored permanently or until explicitly deleted.
@xref{Rmail Inbox}.

@item Indentation
Indentation means blank space at the beginning of a line.  Most
programming languages have conventions for using indentation to
illuminate the structure of the program, and Emacs has special
features to help you set up the correct indentation.
@xref{Indentation}.

@item Insertion
Insertion means copying text into the buffer, either from the keyboard
or from some other place in Emacs.

@item Justification
Justification means adding extra spaces to lines of text to make them
come exactly to a specified width.  @xref{Filling,Justification}.

@item Keyboard Macros
Keyboard macros are a way of defining new Emacs commands from
sequences of existing ones, with no need to write a Lisp program.
@xref{Keyboard Macros}.

@item Key
A key is a sequence of characters that, when input to Emacs, specify
or begin to specify a single action for Emacs to perform.  That is,
the sequence is considered a single unit.  If the key is enough to
specify one action, it is a complete key (q.v.); if it is less than
enough, it is a prefix key (q.v.).  @xref{Keystrokes}.

@item Keymap
The keymap is the data structure that records the bindings (q.v.@:) of
keys to the commands that they run.  For example, the keymap binds the
character @kbd{C-n} to the command function @code{next-line}.
@xref{Keymaps}.

@page
@item Kill Ring
The kill ring is the place where all text you have killed recently is saved.
You can re-insert any of the killed text still in the ring; this is
called yanking (q.v.@:).  @xref{Yanking}.

@item Killing
Killing means erasing text and saving it on the kill ring so it can be
yanked (q.v.@:) later.  Some other systems call this ``cutting''.
Most Emacs commands to erase text do killing, as opposed to deletion
(q.v.@:).  @xref{Killing}.

@item Killing Jobs
Killing a job (such as, an invocation of Emacs) means making it cease
to exist.  Any data within it, if not saved in a file, is lost.
@xref{Exiting}.

@item List
A list is, approximately, a text string beginning with an open
parenthesis and ending with the matching close parenthesis.  In C mode
and other non-Lisp modes, groupings surrounded by other kinds of matched
delimiters appropriate to the language, such as braces, are also
considered lists.  Emacs has special commands for many operations on
lists.  @xref{Lists}.

@item Local
Local means `in effect only in a particular context'; the relevant
kind of context is a particular function execution, a particular
buffer, or a particular major mode.  Local is the opposite of `global'
(q.v.@:).  Specific uses of `local' in Emacs terminology appear below.

@item Local Abbrev
A local abbrev definition is effective only if a particular major mode
is selected.  In that major mode, it overrides any global definition
for the same abbrev.  @xref{Abbrevs}.

@item Local Keymap
A local keymap is used in a particular major mode; the key bindings
(q.v.@:) in the current local keymap override global bindings of the
same keys.  @xref{Keymaps}.

@item Local Variable
A local value of a variable (q.v.@:) applies to only one buffer.
@xref{Locals}.

@item M-
@kbd{M-} in the name of a character is an abbreviation for @key{META},
one of the modifier keys that can accompany any character.
@xref{Keystrokes}.

@item M-C-
@samp{M-C-} in the name of a character is an abbreviation for
Control-Meta; it means the same thing as @samp{C-M-}.  If your
terminal lacks a real @key{META} key, you type a Control-Meta character by
typing @key{ESC} and then typing the corresponding Control character.
@xref{Keystrokes,C-M-}.

@item M-x
@kbd{M-x} is the key which is used to call an Emacs command by name.
You use it to call commands that are not bound to keys.
@xref{M-x}.

@item Mail
Mail means messages sent from one user to another through the computer
system, to be read at the recipient's convenience.  Emacs has commands for
composing and sending mail, and for reading and editing the mail you have
received.  @xref{Sending Mail}.  @xref{Rmail}, for how to read mail.

@item Mail File
A mail file is a file which is edited using Rmail and in which Rmail
stores mail.  @xref{Rmail}.

@item Major Mode
The major modes are a mutually exclusive set of options each of which
configures Emacs for editing a certain sort of text.  Ideally, each
programming language has its own major mode.  @xref{Major Modes}.

@item Mark
The mark points to a position in the text.  It specifies one end of the
region (q.v.@:), point being the other end.  Many commands operate on
the whole region, that is, all the text from point to the mark.
@xref{Mark}.

@item Mark Ring
The mark ring is used to hold several recent previous locations of the
mark, just in case you want to move back to them.  @xref{Mark Ring}.

@item Message
See `mail'.

@item Meta
Meta is the name of a modifier bit which a command character may have.
It is present in a character if the character is typed with the
@key{META} key held down.  Such characters are given names that start
with @kbd{Meta-}.  For example, @kbd{Meta-<} is typed by holding down
@key{META} and at the same time typing @kbd{<} (which itself is done,
on most terminals, by holding down @key{SHIFT} and typing @kbd{,}).
@xref{Keystrokes,Meta}.

@item Meta Character
A Meta character is one whose character code includes the Meta bit.

@item Minibuffer
The minibuffer is the window that Emacs displays inside the
echo area (q.v.@:) when it prompts you for arguments to commands.
@xref{Minibuffer}.
@page
@item Minor Mode
A minor mode is an optional feature of Emacs which can be switched on
or off independent of the features the major mode.  Each minor mode has a
command to turn it on or off.  @xref{Minor Modes}.

@item Mode Line
The mode line is the line at the bottom of each text window (q.v.@:),
which gives status information on the buffer displayed in that window.
@xref{Mode Line}.

@item Modified Buffer
A buffer (q.v.@:) is modified if its text has been changed since the
last time the buffer was saved (or since it was created, if it
has never been saved).  @xref{Saving}.

@item Moving Text
Moving text means erasing it from one place and inserting it in
another.  This is done by killing (q.v.@:) and then yanking (q.v.@:).
@xref{Killing}.

@item Named Mark
A named mark is a register (q.v.@:) in its role of recording a
location in text so that you can move point to that location.
@xref{Registers}.

@item Narrowing
Narrowing means creating a restriction (q.v.@:) that limits editing in
the current buffer to only a part of the text in the buffer.  Text
outside that part is inaccessible to the user until the boundaries are
widened again, but it is still there, and saving the file saves the
invisible text.  @xref{Narrowing}.

@item Newline
@key{LFD} characters in the buffer terminate lines of text and are
called newlines.  @xref{Keystrokes,Newline}.

@item Numeric Argument
A numeric argument is a number, specified before a command, to change
the effect of the command.  Often the numeric argument serves as a
repeat count.  @xref{Arguments}.

@item Option
An option is a variable (q.v.@:) that allows you to customize
Emacs by giving it a new value.  @xref{Variables}.

@item Overwrite Mode
Overwrite mode is a minor mode.  When it is enabled, ordinary text
characters replace the existing text after point rather than pushing
it to the right.  @xref{Minor Modes}.

@item Page
A page is a unit of text, delimited by formfeed characters (ASCII
Control-L, code 014) coming at the beginning of a line.  Some Emacs
commands are provided for moving over and operating on pages.
@xref{Pages}.

@item Paragraphs
Paragraphs are the medium-size unit of English text.  There are
special Emacs commands for moving over and operating on paragraphs.
@xref{Paragraphs}.

@item Parsing
We say that Emacs parses words or expressions in the text being
edited.  Really, all it knows how to do is find the other end of a
word or expression.  @xref{Syntax}.

@item Point
Point is the place in the buffer at which insertion and deletion
occur.  Point is considered to be between two characters, not at one
character.  The terminal's cursor (q.v.@:) indicates the location of
point.  @xref{Basic,Point}.

@item Prefix Key
A prefix key is a key (q.v.@:) whose sole function is to introduce a
set of multi-character keys.  @kbd{Control-x} is an example of a prefix
key; any two-character sequence starting with @kbd{C-x} is also
a legitimate key.  @xref{Keystrokes}.

@item Primary Mail File
Your primary mail file is the file named @samp{RMAIL} in your home
directory, where Rmail stores all mail you receive unless you
make arrangements to do otherwise.  @xref{Rmail}.

@item Prompt
A prompt is text printed to ask the user for input.  Printing a prompt
is called prompting.  Emacs prompts always appear in the echo area
(q.v.@:).  One kind of prompting happens when the minibuffer is used
to read an argument (@pxref{Minibuffer}); the echoing which happens
when you pause in the middle of typing a multi-character key is also a
kind of prompting (@pxref{Echo Area}).

@item Quitting
Quitting means cancelling a partially typed command or a running
command, using @kbd{C-g}.  @xref{Quitting}.

@item Quoting
Quoting means depriving a character of its usual special significance.
In Emacs this is usually done with @kbd{Control-q}.  What constitutes special
significance depends on the context and on convention.  For example,
an ``ordinary'' character as an Emacs command inserts itself; so in
this context, a special character is any character that does not
normally
@page
insert itself (such as @key{DEL}, for example), and quoting it makes it
insert itself as if it were not special.  Not all contexts allow
quoting.  @xref{Basic,Quoting,Basic Editing}.

@item Read-only Buffer
A read-only buffer is one whose text you are not allowed to change.
Normally Emacs makes buffers read-only when they contain text which
has a special significance to Emacs; for example, Dired buffers.
Visiting a file that is write protected also makes a read-only buffer.
@xref{Buffers}.

@item Recursive Editing Level
A recursive editing level is a state in which part of the execution of
a command involves asking the user to edit some text.  This text may
or may not be the same as the text to which the command was applied.
The mode line indicates recursive editing levels with square brackets
(@samp{[} and @samp{]}).  @xref{Recursive Edit}.

@item Redisplay
Redisplay is the process of correcting the image on the screen to
correspond to changes that have been made in the text being edited.
@xref{Screen,Redisplay}.

@item Regexp
See `regular expression'.

@item Region
The region is the text between point (q.v.@:) and the mark (q.v.@:).
Many commands operate on the text of the region.  @xref{Mark,Region}.

@item Registers
Registers are named slots in which text or buffer positions or
rectangles can be saved for later use.  @xref{Registers}.

@item Regular Expression
A regular expression is a pattern that can match various text strings;
for example, @samp{l[0-9]+} matches @samp{l} followed by one or more
digits.  @xref{Regexps}.

@item Replacement
See `global substitution'.

@item Restriction
A buffer's restriction is the amount of text, at the beginning or the
end of the buffer, that is temporarily invisible and inaccessible.
Giving a buffer a nonzero amount of restriction is called narrowing
(q.v.).  @xref{Narrowing}.

@item @key{RET}
@key{RET} is the character than runs the command to insert a
newline into the text.  It is also used to terminate most arguments
read in the minibuffer (q.v.@:).  @xref{Keystrokes,Return}.

@item Saving
Saving a buffer means copying its text into the file that was visited
(q.v.@:) in that buffer.  To actually change a file you have edited in
Emacs, you have to save it.  @xref{Saving}.

@item Scrolling
Scrolling means shifting the text in the Emacs window to make a
different part ot the buffer visible.  @xref{Display,Scrolling}.

@item Searching
Searching means moving point to the next occurrence of a specified
string.  @xref{Search}.

@item Selecting
Selecting a buffer means making it the current (q.v.@:) buffer.
@xref{Buffers,Selecting}.

@item Self-documentation
Self-documentation is the feature of Emacs which can tell you what any
command does, or give you a list of all commands related to a topic
you specify.  You ask for self-documentation with the help character,
@kbd{C-h}.  @xref{Help}.

@item Sentences
Emacs has commands for moving by or killing by sentences.
@xref{Sentences}.

@item Sexp
An sexp (short for `s-expression') is the basic syntactic unit of Lisp
in its textual form: either a list, or Lisp atom.  Many Emacs commands
operate on sexps.  The term `sexp' is generalized to languages other
than Lisp, to mean a syntactically recognizable expression.
@xref{Lists,Sexps}.

@item Simultaneous Editing
Simultaneous editing means two users modify the same file at once.
If simultaneous editing is not detected, you may lose your
work.  Emacs detects all cases of simultaneous editing and warns the
user to investigate them.  @xref{Interlocking,,Simultaneous Editing}.

@item String
A string is a kind of Lisp data object which contains a sequence of
characters.  Many Emacs variables are intended to have strings as
values.  The Lisp syntax for a string consists of the characters in
the string with a @samp{"} before and another @samp{"} after. Write a
@samp{"} that is part of the string as @samp{\"} and a
@samp{\} that is part of the string as @samp{\\}.  You can include all
other characters, including newline, just by writing
them inside the string. You can also include escape sequences as in C, such as
@samp{\n} for newline or @samp{\241} using an octal character code.

@item String Substitution
See `global substitution'.
@page
@item Syntax Table
The syntax table tells Emacs which characters are part of a word,
which characters balance each other like parentheses, etc.
@xref{Syntax}.

@item Tag Table
A tag table is a file that serves as an index to the function
definitions in one or more other files.  @xref{Tags}.

@item Termscript File
A termscript file contains a record of all characters Emacs sent to
the terminal.  It is used for tracking down bugs in Emacs redisplay.
Emacs does not make a termscript file unless explicitly instructed to do
so. 
@xref{Bugs}.

@item Text
Text has two meanings (@pxref{Text}):

@itemize @bullet
@item
Data consisting of a sequence of characters, as opposed to binary
numbers, images, graphics commands, executable programs, and the like.
The contents of an Emacs buffer are always text in this sense.
@item
Data consisting of written human language, as opposed to programs,
or following the stylistic conventions of human language.
@end itemize

@item Top Level
Top level is the normal state of Emacs, in which you are editing the
text of the file you have visited.  You are at top level whenever you
are not in a recursive editing level (q.v.@:) or the minibuffer
(q.v.@:), and not in the middle of a command.  You can get back to top
level by aborting (q.v.@:) and quitting (q.v.@:).  @xref{Quitting}.

@item Transposition
Transposing two units of text means putting each one into the place
formerly occupied by the other.  There are Emacs commands to transpose
two adjacent characters, words, sexps (q.v.@:) or lines
(@pxref{Transpose}).

@item Truncation
Truncating text lines in the display means leaving out any text on a
line that does not fit within the right margin of the window
displaying it.  See also `continuation line'.
@xref{Basic,Truncation,Basic Editing}.

@item Undoing
Undoing means making your previous editing go in reverse, bringing
back the text that existed earlier in the editing session.
@xref{Undo}.

@item Variable
A variable is Lisp object that can store an arbitrary value.  Emacs uses
some variables for internal purposes, and has others (known as `options'
(q.v.@:)) you can set to control the behavior of Emacs.  The variables
used in Emacs that you are likely to be interested in are listed in the
Variables Index of this manual.  @xref{Variables}, for information on
variables.

@item Visiting
Visiting a file means loading its contents into a buffer (q.v.@:)
where they can be edited.  @xref{Visiting}.

@item Whitespace
Whitespace is any run of consecutive formatting characters (space,
tab, newline, and backspace).

@item Widening
Widening is removing any restriction (q.v.@:) on the current buffer;
it is the opposite of narrowing (q.v.@:).  @xref{Narrowing}.

@item Window
Emacs divides the screen into one or more windows, each of which can
display the contents of one buffer (q.v.@:) at any time.
@xref{Screen}, for basic information on how Emacs uses the screen.
@xref{Windows}, for commands to control the use of windows. Note that if
you are running Emacs under X, terminology can be confusing: Each Emacs
screen occupies a separate X window and can, in turn, be divided into
different subwindows. 

@item Word Abbrev
Synonymous with `abbrev'.

@item Word Search
Word search is searching for a sequence of words, considering the
punctuation between them as insignificant.  @xref{Word Search}.

@item Yanking
Yanking means reinserting text previously killed.  It can be used to
undo a mistaken kill, or for copying or moving text.  Some other
systems call this ``pasting''.  @xref{Yanking}.
@end table

@node Key Index, Command Index, Glossary, Top
@unnumbered Key (Character) Index
@printindex ky

@node Command Index, Variable Index, Key Index, Top
@unnumbered Command and Function Index
@printindex fn

@node Variable Index, Concept Index, Command Index, Top
@unnumbered Variable Index
@printindex vr

@node Concept Index, Screen, Variable Index, Top
@unnumbered Concept Index
@printindex cp

@summarycontents
@contents
@bye


@c Remember to delete these lines before creating the info file.
@iftex
@lucidbook
@bindingoffset = 0.5in
@parindent = 0pt
@end iftex

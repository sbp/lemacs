\input texinfo  @comment -*-texinfo-*-
@comment %**start of header
@setfilename ../info/edebug.info
@settitle Edebug User Manual
@comment %**end of header

@comment ================================================================
@comment This file has the same style as the GNU Emacs Lisp Reference Manual.
@comment Run tex using version of `texinfo.tex' that comes with the elisp
@comment manual. Also, run `makeinfo' rather than `texinfo-format-buffer'.
@comment ================================================================

@comment smallbook

@comment tex
@comment \overfullrule=0pt
@comment end tex

@comment
@comment Combine indices.
@syncodeindex fn cp
@syncodeindex vr cp
@syncodeindex ky cp
@syncodeindex pg cp
@syncodeindex tp cp
@comment texinfo-format-buffer no longer ignores synindex.
@comment

@ifinfo
This file documents Edebug

This is edition 1.5 of the Edebug User Manual
for edebug Version 3.2,

Copyright (C) 1991,1992,1993 Free Software Foundation, Inc.

Permission is granted to make and distribute verbatim copies of
this manual provided the copyright notice and this permission notice
are preserved on all copies.

@ignore
Permission is granted to process this file through TeX and print the
results, provided the printed document carries copying permission
notice identical to this one except for the removal of this paragraph
(this paragraph not being relevant to the printed manual).

@end ignore
Permission is granted to copy and distribute modified versions of this
manual under the conditions for verbatim copying, provided that the entire
resulting derived work is distributed under the terms of a permission
notice identical to this one.

Permission is granted to copy and distribute translations of this manual
into another language, under the above conditions for modified versions,
except that this permission notice may be stated in a translation approved
by the Foundation.
@end ifinfo
@comment

@comment
@setchapternewpage odd

@titlepage
@title Edebug User Manual
@subtitle A Source Level Debugger for GNU Emacs Lisp
@subtitle Edition 1.5, August 1993

@author by Daniel LaLiberte,  liberte@@cs.uiuc.edu
@page
@vskip 0pt plus 1filll
Copyright @copyright{} 1991,1992,1993 Daniel LaLiberte

@sp 2
This is edition 1.5 of the @cite{Edebug User Manual}
for edebug Version 3.2, September 1993.


@sp 2

Permission is granted to make and distribute verbatim copies of
this manual provided the copyright notice and this permission notice
are preserved on all copies.

Permission is granted to copy and distribute modified versions of this
manual under the conditions for verbatim copying, provided that the entire
resulting derived work is distributed under the terms of a permission
notice identical to this one.

Permission is granted to copy and distribute translations of this manual
into another language, under the above conditions for modified versions,
except that this permission notice may be stated in a translation approved
by this author.
@end titlepage
@page


@node Top, Edebug, (dir), (dir)
@chapter Edebug

  Edebug is a source-level debugger for Emacs Lisp programs.


@menu
* Edebug::			Edebug
* Bugs and Todo List::		Bugs and Todo List
* Index::			Index
@end menu

@c from included file:
@c @node Edebug, Bugs and Todo List, Top, Top
@c @section Edebug

@include edebug.texinfo


@node Bugs and Todo List, Index, Edebug, Top
@section Bugs and Todo List

@cindex debugging edebug
@cindex bugs in edebug
If you are interested in running edebug on functions in
@file{edebug.el}, often it is easiest to first copy a reliable version
of @file{edebug.el} into another file, say @file{fdebug.el}, and replace
all strings @samp{edebug} with @samp{fdebug}, then evaluate the fdebug
buffer and run fdebug on functions in the buggy @file{edebug.el}.

@vindex debug-on-error
@vindex edebug-debugger
If there is a bug in the runtime parts of edebug, you may simply want to
set @code{debug-on-error} to @code{t} to get a backtrace when the error
occurs.  For this to work, however, before executing code that causes
the error you will also have to set the variable @code{edebug-debugger}
to @code{debug} rather than its default value of @code{edebug}.

The following is a list of things I might do in the future to edebug.
Often I do other things not on the list as I discover the need for them.
Send me your suggestions and priorities.

@itemize @bullet

@item
@cindex buffer point
Bug: I've noticed that the point of some buffers was reset to the point
of some other buffer, but I haven't been able to repeat it so perhaps
it is fixed.

@item
There may be a bug in the trace buffer display.  It should display as
much as it can of the bottom of the buffer, but I think it scrolls off
sometimes.

There is a bug in window updating when there is both a trace buffer
and an evaluation list - the source buffer doesnt get displayed.

@item 
Killing and reinserting an instrumented definition or parts of
it leaves marks in the buffer which may confuse Edebug later.

@item
Defs with complex names (e.g. defmethod) will cause problems
for remembering source code position.

@item
After some errors, with @code{debug-on-error} non-@code{nil}, continuing
execution succeeds returning @code{nil}.

@item
A source buffer displayed in multiple windows is not handled as well as
it could.  Edebug only cares about finding the first window that
displays the buffer.

@item
There are no other known bugs, so if you find any, please let me know.
There is nothing worse than a buggy debugger!

@item
"(" in the first column of doc strings messes up edebug reading.
But no more than normal reading.

@item
There could be a command to return a value from the debugger -
particularly useful for errors.

@item
@cindex side effects
Let me know if you find any side effects that could or should be avoided
or at least documented in the manual.
Also @pxref{Side Effects}.

@item
@vindex max-lisp-eval-depth
@vindex max-specpdl-size
Figure out a reliable way to set @code{max-lisp-eval-depth} and
@code{max-specpdl-size}.

@item
Remember the window configuration inside debugger between edebug calls
and remember outside configuration on the first call to edebug after an
interactive command at a lower level.

Or remember a set of buffers/windows to display inside of edebug, but we
would also need the window start of each such window.

@item
@cindex selective display
Make edebug work with selective display - don't stop in hidden lines.

@item
Debug just one or selected subexpressions of a definition - the rest is
evalled normally.

@item
Should @code{overlay-arrow-position} and @code{-string} be buffer local?

@item
Use copy of @code{current-local-map} instead of @code{emacs-lisp-mode-map}
(but only copy the first time after lower level command - to save time).

@item
Better integration with standard debug.

@item
@vindex inhibit-quit
Use @code{inhibit-quit} while edebugging?  

@item
@findex sit-for
Crawl mode would @code{sit-for} 0 or 1 in the outside window configuration
between each edebug step.
Maybe it should be a separate option that applies to trace as well.

@item
Customizable @code{sit-for} time.  Less than a second would be nice.

@item
Generalize step, trace, Trace-fast to one command with argument for
@code{sit-for} time.
Generalize go, continue, Continue-fast to another command with argument

@item
@cindex counting conditions
Counting conditions - stop after n iterations.  You can do it manually now
with conditional breakpoints.

@item
@vindex edebug-initial-mode
Provide a temporary @code{edebug-initial-mode} which is reset to the
original after it is used.

@item
@cindex minibuffer
minibuffer trace - show the current source line in the minibuffer instead
of moving point to the expression.

@item
@cindex performance monitoring
Performance monitoring - summarize trace data.

@item
Speed up edebug-defun and edebugging - always.

@item
@cindex preserve breakpoints
Preserve breakpoints across instrumenting.
You can now install calls to @code{edebug} in your code.

@item
Step into code not previously instrumented.
Maybe restore to non-instrumented code after entered.
Partially implemented with @code{edebug-step-in} command.

@item
@cindex replace with results
Optionally replace expressions with results in a separate buffer from
the source code.  This idea is based on discussions with Carl Witty
regarding his stepper debugger.  Also, unparse code into its own buffer
if source code is not available, or if user wishes to use
replace-with-results mode.

@item
@cindex local variables
@cindex variables display
Preserve previous bindings of local variables, and allow user to jump
back to previous frames, particularly binding frames (i.e. @code{let},
@code{condition-case}, function and macro calls) to view values at that
frame.  What about buffer local variables?  It would be simpler to have
access to the Lisp stack.

Variables display, like the evaluation list but automatically display
all local variables and values.

@item
@vindex debug-on-next-call
@findex backtrace-debug
Investigate minimal instrumentation that doesn't call edebug functions
but instead sets edebug index and result variables.  Stepping is done
through standard debugger features such as setting
@code{debug-on-next-call}.  Breakpoints are done by modifying code as
well as calling @code{backtrace-debug} for active frames.

@item
Edebugging of uninstrumented code.  Similar to above minimal
instrumentation but find out where we are at each edebug call by looking
in a map from each list form in the code to its position.
Problem is symbols are not unique.

@item
Investigate hiding debugger internal stack frames.  This is both to
simplify the standard debugger (which currently must be byte compiled to
work) and to better support the integration of edebug and the standard
debugger.

@item
@vindex edebug-global-break-condition
Although variables can't be tracked everywhere, watchpoints would be
nice for variables that edebug can monitor.  That is, when the value of
a specific variable changes, edebug would stop.  This can be done now
with the @code{edebug-global-break-condition}, though it is awkward.

@item
How about a command to add the previous sexp (?) to the eval-list?

@item
Highlight all instrumented code, breakpoints, and subexpressions about
to be evaluated or just evaluated.  This should be done in a way that
works with Epoch, Lemacs, and Emacs 19.

@end itemize


@page
@node Index,  , Bugs and Todo List, Top
@section Index

@printindex cp

@comment To prevent the Concept Index's last page from being numbered "i".
@page

@contents
@bye

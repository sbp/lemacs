\input texinfo  @c -*-texinfo-*-

@setfilename ../info/gdb

@settitle GDB, The GNU Debugger
@synindex ky cp

@iftex
@lucidbook
@bindingoffset = 0.25in
@parindent = 0pt
@end iftex

@ifinfo
This file documents the GNU debugger GDB.

Copyright (C) 1988, 1989 Free Software Foundation, Inc.

Permission is granted to make and distribute verbatim copies of
this manual provided the copyright notice and this permission notice
are preserved on all copies.

@ignore
Permission is granted to process this file through Tex and print the
results, provided the printed document carries copying permission
notice identical to this one except for the removal of this paragraph
(this paragraph not being relevant to the printed manual).

@end ignore
Permission is granted to copy and distribute modified versions of this
manual under the conditions for verbatim copying, provided also that the
section entitled ``GNU General Public License'' is included exactly as
in the original, and provided that the entire resulting derived work is
distributed under the terms of a permission notice identical to this
one.

Permission is granted to copy and distribute translations of this manual
into another language, under the above conditions for modified versions,
except that the section entitled ``GNU General Public License'' may be
included in a translation approved by the author instead of in the
original English.
@end ifinfo

@setchapternewpage odd
@settitle GDB Manual
@titlepage
@sp 6
@center @titlefont{GDB Manual}
@sp 1
@center The GNU Source-Level Debugger
@sp 4
@center Third Edition, GDB version 3.4
@sp 1
@center October 1989
@sp 5
@center Richard M. Stallman
@page
@vskip 0pt plus 1filll
Copyright @copyright{} 1988, 1989 Free Software Foundation, Inc.

Permission is granted to make and distribute verbatim copies of
this manual provided the copyright notice and this permission notice
are preserved on all copies.

Permission is granted to copy and distribute modified versions of this
manual under the conditions for verbatim copying, provided also that the
section entitled ``GNU General Public License'' is included exactly as
in the original, and provided that the entire resulting derived work is
distributed under the terms of a permission notice identical to this
one.

Permission is granted to copy and distribute translations of this manual
into another language, under the above conditions for modified versions,
except that the section entitled ``GNU General Public License'' may be
included in a translation approved by the author instead of in the
original English.
@end titlepage
@page

@node Top, Top, Top, (DIR)
@unnumbered Summary of GDB

The purpose of a debugger such as GDB is to allow you to execute another
program while examining what is going on inside it.  We call the other
program ``your program'' or ``the program being debugged''.

GDB can do four kinds of things (plus other things in support of these):

@enumerate
@item
Start the program, specifying anything that might affect its behavior.

@item
Make the program stop on specified conditions.

@item
Examine what has happened, when the program has stopped, so that you
can see bugs happen.

@item
Change things in the program, so you can correct the effects of one bug
and go on to learn about another without having to recompile first.
@end enumerate

GDB can be used to debug programs written in C and C++.  Pascal support
is being implemented, and Fortran support will be added when a GNU
Fortran compiler is written.

@menu
* License::    The GNU General Public License gives you permission
	       to redistribute GDB on certain terms; and also
	       explains that there is no warranty.
* User Interface::      GDB command syntax and input and output conventions.
* Files::      Specifying files for GDB to operate on.
* Options::    GDB arguments and options.
* Compilation::Compiling your program so you can debug it.
* Running::    Running your program under GDB.
* Stopping::   Making your program stop.  Why it may stop.  What to do then.
* Stack::      Examining your program's stack.
* Source::     Examining your program's source files.
* Data::       Examining data in your program.
* Symbols::    Examining the debugger's symbol table.
* Altering::   Altering things in your program.
* Sequences::  Canned command sequences for repeated use.
* Emacs::      Using GDB through GNU Emacs.
* Remote::     Remote kernel debugging across a serial line.
* Extensions:: Lucid extensions to GDB for debugging C++ code.
* Commands::   Index of GDB commands.
* Concepts::   Index of GDB concepts.
@end menu

@node License, User Interface, Top, Top
@unnumbered GNU GENERAL PUBLIC LICENSE
@center Version 1, February 1989

@display
Copyright @copyright{} 1989 Free Software Foundation, Inc.
675 Mass Ave, Cambridge, MA 02139, USA

Everyone is permitted to copy and distribute verbatim copies
of this license document, but changing it is not allowed.
@end display

@unnumberedsec Preamble

  The license agreements of most software companies try to keep users
at the mercy of those companies.  By contrast, our General Public
License is intended to guarantee your freedom to share and change free
software---to make sure the software is free for all its users.  The
General Public License applies to the Free Software Foundation's
software and to any other program whose authors commit to using it.
You can use it for your programs, too.

  When we speak of free software, we are referring to freedom, not
price.  Specifically, the General Public License is designed to make
sure that you have the freedom to give away or sell copies of free
software, that you receive source code or can get it if you want it,
that you can change the software or use pieces of it in new free
programs; and that you know you can do these things.

  To protect your rights, we need to make restrictions that forbid
anyone to deny you these rights or to ask you to surrender the rights.
These restrictions translate to certain responsibilities for you if you
distribute copies of the software, or if you modify it.

  For example, if you distribute copies of a such a program, whether
gratis or for a fee, you must give the recipients all the rights that
you have.  You must make sure that they, too, receive or can get the
source code.  And you must tell them their rights.

  We protect your rights with two steps: (1) copyright the software, and
(2) offer you this license which gives you legal permission to copy,
distribute and/or modify the software.

  Also, for each author's protection and ours, we want to make certain
that everyone understands that there is no warranty for this free
software.  If the software is modified by someone else and passed on, we
want its recipients to know that what they have is not the original, so
that any problems introduced by others will not reflect on the original
authors' reputations.

  The precise terms and conditions for copying, distribution and
modification follow.

@iftex
@unnumberedsec TERMS AND CONDITIONS
@end iftex
@ifinfo
@center TERMS AND CONDITIONS
@end ifinfo

@enumerate
@item
This License Agreement applies to any program or other work which
contains a notice placed by the copyright holder saying it may be
distributed under the terms of this General Public License.  The
``Program'', below, refers to any such program or work, and a ``work based
on the Program'' means either the Program or any work containing the
Program or a portion of it, either verbatim or with modifications.  Each
licensee is addressed as ``you''.

@item
You may copy and distribute verbatim copies of the Program's source
code as you receive it, in any medium, provided that you conspicuously and
appropriately publish on each copy an appropriate copyright notice and
disclaimer of warranty; keep intact all the notices that refer to this
General Public License and to the absence of any warranty; and give any
other recipients of the Program a copy of this General Public License
along with the Program.  You may charge a fee for the physical act of
transferring a copy.

@item
You may modify your copy or copies of the Program or any portion of
it, and copy and distribute such modifications under the terms of Paragraph
1 above, provided that you also do the following:

@itemize @bullet
@item
cause the modified files to carry prominent notices stating that
you changed the files and the date of any change; and

@item
cause the whole of any work that you distribute or publish, that
in whole or in part contains the Program or any part thereof, either
with or without modifications, to be licensed at no charge to all
third parties under the terms of this General Public License (except
that you may choose to grant warranty protection to some or all
third parties, at your option).

@item
If the modified program normally reads commands interactively when
run, you must cause it, when started running for such interactive use
in the simplest and most usual way, to print or display an
announcement including an appropriate copyright notice and a notice
that there is no warranty (or else, saying that you provide a
warranty) and that users may redistribute the program under these
conditions, and telling the user how to view a copy of this General
Public License.

@item
You may charge a fee for the physical act of transferring a
copy, and you may at your option offer warranty protection in
exchange for a fee.
@end itemize

Mere aggregation of another independent work with the Program (or its
derivative) on a volume of a storage or distribution medium does not bring
the other work under the scope of these terms.

@item
You may copy and distribute the Program (or a portion or derivative of
it, under Paragraph 2) in object code or executable form under the terms of
Paragraphs 1 and 2 above provided that you also do one of the following:

@itemize @bullet
@item
accompany it with the complete corresponding machine-readable
source code, which must be distributed under the terms of
Paragraphs 1 and 2 above; or,

@item
accompany it with a written offer, valid for at least three
years, to give any third party free (except for a nominal charge
for the cost of distribution) a complete machine-readable copy of the
corresponding source code, to be distributed under the terms of
Paragraphs 1 and 2 above; or,

@item
accompany it with the information you received as to where the
corresponding source code may be obtained.  (This alternative is
allowed only for noncommercial distribution and only if you
received the program in object code or executable form alone.)
@end itemize

Source code for a work means the preferred form of the work for making
modifications to it.  For an executable file, complete source code means
all the source code for all modules it contains; but, as a special
exception, it need not include source code for modules which are standard
libraries that accompany the operating system on which the executable
file runs, or for standard header files or definitions files that
accompany that operating system.

@item
You may not copy, modify, sublicense, distribute or transfer the
Program except as expressly provided under this General Public License.
Any attempt otherwise to copy, modify, sublicense, distribute or transfer
the Program is void, and will automatically terminate your rights to use
the Program under this License.  However, parties who have received
copies, or rights to use copies, from you under this General Public
License will not have their licenses terminated so long as such parties
remain in full compliance.

@item
By copying, distributing or modifying the Program (or any work based
on the Program) you indicate your acceptance of this license to do so,
and all its terms and conditions.

@item
Each time you redistribute the Program (or any work based on the
Program), the recipient automatically receives a license from the original
licensor to copy, distribute or modify the Program subject to these
terms and conditions.  You may not impose any further restrictions on the
recipients' exercise of the rights granted herein.

@item
The Free Software Foundation may publish revised and/or new versions
of the General Public License from time to time.  Such new versions will
be similar in spirit to the present version, but may differ in detail to
address new problems or concerns.

Each version is given a distinguishing version number.  If the Program
specifies a version number of the license which applies to it and ``any
later version'', you have the option of following the terms and conditions
either of that version or of any later version published by the Free
Software Foundation.  If the Program does not specify a version number of
the license, you may choose any version ever published by the Free Software
Foundation.

@item
If you wish to incorporate parts of the Program into other free
programs whose distribution conditions are different, write to the author
to ask for permission.  For software which is copyrighted by the Free
Software Foundation, write to the Free Software Foundation; we sometimes
make exceptions for this.  Our decision will be guided by the two goals
of preserving the free status of all derivatives of our free software and
of promoting the sharing and reuse of software generally.

@iftex
@heading NO WARRANTY
@end iftex
@ifinfo
@center NO WARRANTY
@end ifinfo

@item
BECAUSE THE PROGRAM IS LICENSED FREE OF CHARGE, THERE IS NO WARRANTY
FOR THE PROGRAM, TO THE EXTENT PERMITTED BY APPLICABLE LAW.  EXCEPT WHEN
OTHERWISE STATED IN WRITING THE COPYRIGHT HOLDERS AND/OR OTHER PARTIES
PROVIDE THE PROGRAM ``AS IS'' WITHOUT WARRANTY OF ANY KIND, EITHER EXPRESSED
OR IMPLIED, INCLUDING, BUT NOT LIMITED TO, THE IMPLIED WARRANTIES OF
MERCHANTABILITY AND FITNESS FOR A PARTICULAR PURPOSE.  THE ENTIRE RISK AS
TO THE QUALITY AND PERFORMANCE OF THE PROGRAM IS WITH YOU.  SHOULD THE
PROGRAM PROVE DEFECTIVE, YOU ASSUME THE COST OF ALL NECESSARY SERVICING,
REPAIR OR CORRECTION.

@item
IN NO EVENT UNLESS REQUIRED BY APPLICABLE LAW OR AGREED TO IN WRITING WILL
ANY COPYRIGHT HOLDER, OR ANY OTHER PARTY WHO MAY MODIFY AND/OR
REDISTRIBUTE THE PROGRAM AS PERMITTED ABOVE, BE LIABLE TO YOU FOR DAMAGES,
INCLUDING ANY GENERAL, SPECIAL, INCIDENTAL OR CONSEQUENTIAL DAMAGES
ARISING OUT OF THE USE OR INABILITY TO USE THE PROGRAM (INCLUDING BUT NOT
LIMITED TO LOSS OF DATA OR DATA BEING RENDERED INACCURATE OR LOSSES
SUSTAINED BY YOU OR THIRD PARTIES OR A FAILURE OF THE PROGRAM TO OPERATE
WITH ANY OTHER PROGRAMS), EVEN IF SUCH HOLDER OR OTHER PARTY HAS BEEN
ADVISED OF THE POSSIBILITY OF SUCH DAMAGES.
@end enumerate

@iftex
@heading END OF TERMS AND CONDITIONS
@end iftex
@ifinfo
@center END OF TERMS AND CONDITIONS
@end ifinfo

@page
@unnumberedsec Appendix: How to Apply These Terms to Your New Programs

  If you develop a new program, and you want it to be of the greatest
possible use to humanity, the best way to achieve this is to make it
free software which everyone can redistribute and change under these
terms.

  To do so, attach the following notices to the program.  It is safest to
attach them to the start of each source file to most effectively convey
the exclusion of warranty; and each file should have at least the
``copyright'' line and a pointer to where the full notice is found.

@smallexample
@var{one line to give the program's name and a brief idea of what it does.}
Copyright (C) 19@var{yy}  @var{name of author}

This program is free software; you can redistribute it and/or modify
it under the terms of the GNU General Public License as published by
the Free Software Foundation; either version 1, or (at your option)
any later version.

This program is distributed in the hope that it will be useful,
but WITHOUT ANY WARRANTY; without even the implied warranty of
MERCHANTABILITY or FITNESS FOR A PARTICULAR PURPOSE.  See the
GNU General Public License for more details.

You should have received a copy of the GNU General Public License
along with this program; if not, write to the Free Software
Foundation, Inc., 675 Mass Ave, Cambridge, MA 02139, USA.
@end smallexample

Also add information on how to contact you by electronic and paper mail.

If the program is interactive, make it output a short notice like this
when it starts in an interactive mode:

@smallexample
Gnomovision version 69, Copyright (C) 19@var{yy} @var{name of author}
Gnomovision comes with ABSOLUTELY NO WARRANTY; for details type `show w'.
This is free software, and you are welcome to redistribute it
under certain conditions; type `show c' for details.
@end smallexample

The hypothetical commands `show w' and `show c' should show the
appropriate parts of the General Public License.  Of course, the
commands you use may be called something other than `show w' and `show
c'; they could even be mouse-clicks or menu items---whatever suits your
program.

You should also get your employer (if you work as a programmer) or your
school, if any, to sign a ``copyright disclaimer'' for the program, if
necessary.  Here a sample; alter the names:

@example
Yoyodyne, Inc., hereby disclaims all copyright interest in the
program `Gnomovision' (a program to direct compilers to make passes
at assemblers) written by James Hacker.

@var{signature of Ty Coon}, 1 April 1989
Ty Coon, President of Vice
@end example

That's all there is to it!

@node User Interface, Files, License, Top
@chapter GDB Input and Output Conventions

GDB is invoked with the shell command @samp{gdb}.  Once started, it reads
commands from the terminal until you tell it to exit.

A GDB command is a single line of input.  There is no limit on how long
it can be.  It starts with a command name, which is followed by arguments
whose meaning depends on the command name.  For example, the command
@samp{step} accepts an argument which is the number of times to step,
as in @samp{step 5}.  You can also use the @samp{step} command with
no arguments.  Some command names do not allow any arguments.

@cindex abbreviation
GDB command names may always be abbreviated if the abbreviation is
unambiguous.  Sometimes even ambiguous abbreviations are allowed; for
example, @samp{s} is specially defined as equivalent to @samp{step}
even though there are other commands whose names start with @samp{s}.
Possible command abbreviations are often stated in the documentation
of the individual commands.

@cindex repeating commands
A blank line as input to GDB means to repeat the previous command verbatim.
Certain commands do not allow themselves to be repeated this way; these are
commands for which unintentional repetition might cause trouble and which
you are unlikely to want to repeat.  Certain others (@samp{list} and
@samp{x}) act differently when repeated because that is more useful.

A line of input starting with @samp{#} is a comment; it does nothing.
This is useful mainly in command files (@xref{Command Files}).

@cindex prompt
GDB indicates its readiness to read a command by printing a string
called the @dfn{prompt}.  This string is normally @samp{(gdb)}.  You can
change the prompt string with the @samp{set prompt} command.  For
instance, when debugging GDB with GDB, it is useful to change the prompt
in one of the GDBs so that you tell which one you are talking to.

@table @code
@item set prompt @var{newprompt}
@kindex set prompt
Directs GDB to use @var{newprompt} as its prompt string henceforth.
@end table

@cindex exiting GDB
@kindex quit
To exit GDB, use the @samp{quit} command (abbreviated @samp{q}).
@kbd{Ctrl-c} will not exit from GDB, but rather will terminate the action
of any GDB command that is in progress and return to GDB command level.
It is safe to type @kbd{Ctrl-c} at any time because GDB does not allow
it to take effect until a time when it is safe.

@cindex screen size
@cindex pauses in output
Certain commands to GDB may produce large amounts of information output
to the screen.  To help you read all of it, GDB pauses and asks you for
input at the end of each page of output.  Type @key{RET} when you want
to continue the output.  Normally GDB knows the size of the screen from
on the termcap data base together with the value of the @code{TERM}
environment variable; if this is not correct, you can override it with
the @samp{set screensize} command:

@table @code
@item set screensize @var{lpp}
@itemx set screensize @var{lpp} @var{cpl}
@kindex set screensize
Specify a screen height of @var{lpp} lines and (optionally) a width of
@var{cpl} characters.  If you omit @var{cpl}, the width does not change.

If you specify a height of zero lines, GDB will not pause during output
no matter how long the output is.  This is useful if output is to a file
or to an editor buffer.
@end table

Also, GDB may at times produce more information about its own workings
than is of interest to the user.  Some of these informational messages
can be turned on and off with the @samp{set verbose} command:

@table @code
@kindex set verbose
@item set verbose off
Disables GDB's output of certain informational messages.

@item set verbose on
Re-enables GDB's output of certain informational messages.
@end table

Currently, the messages controlled by @samp{set verbose} are those which
announce that the symbol table for a source file is being read
(@pxref{File Commands}, in the description of the command
@samp{symbol-file}).
@c The following is the right way to do it, but emacs 18.55 doesn't support
@c @ref, and neither the emacs lisp manual version of texinfmt or makeinfo
@c is released.  
@ignore
see @samp{symbol-file} in @ref{File Commands}).
@end ignore

@node Files, Compilation, User Interface, Top
@chapter Specifying GDB's Files

@cindex core dump file
@cindex executable file
@cindex symbol table
GDB needs to know the file name of the program to be debugged, both in
order to read its symbol table and in order to start the program.  To
debug a core dump of a previous run, GDB must be told the file name of
the core dump.

@menu
* Arguments: File Arguments.   Specifying files with arguments
                                (when you start GDB).
* Commands: File Commands.     Specifying files with GDB commands.
@end menu

@node File Arguments, File Commands, Files, Files
@section Specifying Files with Arguments

The usual way to specify the executable and core dump file names is with
two command arguments given when you start GDB.  The first argument is used
as the file for execution and symbols, and the second argument (if any) is
used as the core dump file name.  Thus,

@example
gdb progm core
@end example

@noindent
specifies @file{progm} as the executable program and @file{core} as a core
dump file to examine.  (You do not need to have a core dump file if what
you plan to do is debug the program interactively.)

@xref{Options}, for full information on options and arguments for
invoking GDB.

@node File Commands,, File Arguments, Files
@section Specifying Files with Commands

Usually you specify the files for GDB to work with by giving arguments when
you invoke GDB.  But occasionally it is necessary to change to a different
file during a GDB session.  Or you may run GDB and forget to specify the
files you want to use.  In these situations the GDB commands to specify new
files are useful.

@table @code
@item exec-file @var{filename}
@kindex exec-file
Specify that the program to be run is found in @var{filename}.  If you
do not specify a directory and the file is not found in GDB's working
directory, GDB will use the environment variable @code{PATH} as a list
of directories to search, just as the shell does when looking for a
program to run.

@item symbol-file @var{filename}
@kindex symbol-file
Read symbol table information from file @var{filename}.  @code{PATH}
is searched when necessary.  Most of the time you will use both the
@samp{exec-file} and @samp{symbol-file} commands on the same file.

@samp{symbol-file} with no argument clears out GDB's symbol table.

The @samp{symbol-file} command does not actually read the symbol table in
full right away.  Instead, it scans the symbol table quickly to find
which source files and which symbols are present.  The details are read
later, one source file at a time, when they are needed.

The purpose of this two-stage reading strategy is to make GDB start up
faster.  For the most part, it is invisible except for occasional
messages telling you that the symbol table details for a particular
source file are being read.  (The @samp{set verbose} command controls
whether these messages are printed; @pxref{User Interface}).

However, you will sometimes see in backtraces lines for functions in
source files whose data has not been read in; these lines omit some of
the information, such as argument values, which cannot be printed
without full details of the symbol table.

When the symbol table is stored in COFF format, @samp{symbol-file} does
read the symbol table data in full right away.  We haven't bothered to
implement the two-stage strategy for COFF.

@item core-file @var{filename}
@kindex core-file
Specify the whereabouts of a core dump file to be used as the
``contents of memory''.  Note that the core dump contains only the
writable parts of memory; the read-only parts must come from the
executable file.

@samp{core-file} with no argument specifies that no core file is
to be used.

Note that the core file is ignored when your program is actually running
under GDB.  So, if you have been running the program and you wish to
debug a core file instead, you must kill the subprocess in which the
program is running.  To do this, use the @samp{kill} command
(@pxref{Kill Process}).

@item add-file @var{filename} @var{address}
@kindex add-file
@cindex dynamic linking
The @samp{add-file} command reads additional symbol table information
from the file @var{filename}.  You would use this when that file has
been dynamically loaded into the program that is running.  @var{address}
should be the memory address at which the file has been loaded; GDB cannot
figure this out for itself.

The symbol table of the file @var{filename} is added to the symbol table
originally read with the @samp{symbol-file} command.  You can use the
@samp{add-file} command any number of times; the new symbol data thus
read keeps adding to the old.  The @samp{symbol-file} command forgets
all the symbol data GDB has read; that is the only time symbol data is
forgotten in GDB.

@item info files
@kindex info files
Print the names of the executable and core dump files currently in
use by GDB, and the file from which symbols were loaded.
@end table

While all three file-specifying commands allow both absolute and relative
file names as arguments, GDB always converts the file name to an absolute
one and remembers it that way.

The @samp{symbol-file} command causes GDB to forget the contents of its
convenience variables, the value history, and all breakpoints and
auto-display expressions.  This is because they may contain pointers to the
internal data recording symbols and data types, which are part of the old
symbol table data being discarded inside GDB.

@node Compilation, Running, Files, Top
@chapter Compiling Your Program for Debugging

In order to debug a program effectively, you need to ask for debugging
information when you compile it.  This information in the object file
describes the data type of each variable or function and the correspondence
between source line numbers and addresses in the executable code.

To request debugging information, specify the @samp{-g} option when you run
the compiler.

The Unix C compiler is unable to handle the @samp{-g} and @samp{-O} options
together.  This means that you cannot ask for optimization if you ask for
debugger information.

The GNU C compiler supports @samp{-g} with or without @samp{-O}, making it
possible to debug optimized code.  We recommend that you @emph{always} use
@samp{-g} whenever you compile a program.  You may think the program is
correct, but there's no sense in pushing your luck.

GDB no longer supports the debugging information produced by giving the
GNU C compiler the @samp{-gg} option, so do not use this option.

@ignore
@comment As far as I know, there are no cases in which GDB will
@comment produce strange output in this case.  (but no promises).
If your program includes archives made with the @code{ar} program, and
if the object files used as input to @code{ar} were compiled without the
@samp{-g} option and have names longer than 15 characters, GDB will get
confused reading the program's symbol table.  No error message will be
given, but GDB may behave strangely.  The reason for this problem is a
deficiency in the Unix archive file format, which cannot represent file
names longer than 15 characters.

To avoid this problem, compile the archive members with the @samp{-g}
option or use shorter file names.  Alternatively, use a version of GNU
@code{ar} dated more recently than August 1989.
@end ignore

@node Running, Stopping, Compilation, Top
@chapter Running Your Program Under GDB

@cindex running
@kindex run
To start your program under GDB, use the @samp{run} command.  The program
must already have been specified using the @samp{exec-file} command or with
an argument to GDB (@pxref{Files}); what @samp{run} does is create an
inferior process, load the program into it, and set it in motion.

The execution of a program is affected by certain information it
receives from its superior.  GDB provides ways to specify this
information, which you must do @i{before} starting the program.  (You
can change it after starting the program, but such changes do not affect
the program unless you start it over again.)  This information may be
divided into three categories:

@table @asis
@item The @i{arguments.}
You specify the arguments to give the program as the arguments of the
@samp{run} command.  

@item The @i{environment.}
The program normally inherits its environment from GDB, but you can
use the GDB commands @samp{set environment} and
@samp{unset environment} to change parts of the environment that will
be given to the program.@refill

@item The @i{working directory.}
The program inherits its working directory from GDB.  You can set GDB's
working directory with the @samp{cd} command in GDB.
@end table

After the @samp{run} command, the debugger does nothing but wait for your
program to stop.  @xref{Stopping}.

Note that once your program has been started by the @samp{run} command,
you may evaluate expressions that involve calls to functions in the
inferior.  @xref{Expressions}.  If you wish to evaluate a function
simply for its side affects, you may use the @samp{set} command.
@xref{Assignment}.

@menu
* Arguments::          Specifying the arguments for your program.
* Environment::        Specifying the environment for your program.
* Working Directory::  Specifying the working directory for giving
                       to your program when it is run.
* Input/Output::       Specifying the program's standard input and output.
* Attach::             Debugging a process started outside GDB.
* Kill Process::       Getting rid of the child process running your program.
@end menu

@node Arguments, Environment, Running, Running
@section Your Program's Arguments

@cindex arguments (to your program)
The arguments to your program are specified by the arguments of the 
@samp{run} command.  They are passed to a shell, which expands wildcard
characters and performs redirection of I/O, and thence to the program.

@samp{run} with no arguments uses the same arguments used by the previous
@samp{run}.

@kindex set args
The command @samp{set args} can be used to specify the arguments to be used
the next time the program is run.  If @samp{set args} has no arguments, it
means to use no arguments the next time the program is run.  If you have
run your program with arguments and want to run it again with no arguments,
this is the only way to do so.

@node Environment, Working Directory, Arguments, Running
@section Your Program's Environment

@cindex environment (of your program)
The @dfn{environment} consists of a set of @dfn{environment variables} and
their values.  Environment variables conventionally record such things as
your user name, your home directory, your terminal type, and your search
path for programs to run.  Usually you set up environment variables with
the shell and they are inherited by all the other programs you run.  When
debugging, it can be useful to try running the program with different
environments without having to start the debugger over again.

@table @code
@item info environment @var{varname}
@kindex info environment
Print the value of environment variable @var{varname} to be given to
your program when it is started.  This command can be abbreviated
@samp{i env @var{varname}}.

@item info environment
Print the names and values of all environment variables to be given to
your program when it is started.  This command can be abbreviated
@samp{i env}.

@item set environment @var{varname} @var{value}
@itemx set environment @var{varname} = @var{value}
@kindex set environment
Sets environment variable @var{varname} to @var{value}, for your program
only, not for GDB itself.  @var{value} may be any string; the values of
environment variables are just strings, and any interpretation is
supplied by your program itself.  The @var{value} parameter is optional;
if it is eliminated, the variable is set to a null value.  This command
can be abbreviated as short as @samp{set e}.

For example, this command:

@example
set env USER = foo
@end example

@noindent
tells the program, when subsequently run, to assume it is being run
on behalf of the user named @samp{foo}.

@item delete environment @var{varname}
@itemx unset environment @var{varname}
@kindex delete environment
@kindex unset environment
Remove variable @var{varname} from the environment to be passed to your
program.  This is different from @samp{set env @var{varname}@ =} because
@samp{delete environment} leaves the variable with no value, which is
distinguishable from an empty value.  This command can be abbreviated
@samp{d e}.
@end table

@node Working Directory, Input/Output, Environment, Running
@section Your Program's Working Directory

@cindex working directory (of your program)
Each time you start your program with @samp{run}, it inherits its
working directory from the current working directory of GDB.  GDB's
working directory is initially whatever it inherited from its parent
process (typically the shell), but you can specify a new working
directory in GDB with the @samp{cd} command.

The GDB working directory also serves as a default for the commands
that specify files for GDB to operate on.  @xref{Files}.

@table @code
@item cd @var{directory}
@kindex cd
Set GDB's working directory to @var{directory}.

@item pwd
@kindex pwd
Print GDB's working directory.
@end table

@node Input/Output, Attach, Working Directory, Running
@section Your Program's Input and Output

@cindex redirection
@cindex controlling terminal
By default, the program you run under GDB does input and output to the same
terminal that GDB uses.

You can redirect the program's input and/or output using @samp{sh}-style
redirection commands in the @samp{run} command.  For example,

@example
run > outfile
@end example

@noindent
starts the program, diverting its output to the file @file{outfile}.

@kindex tty
Another way to specify where the program should do input and output is
with the @samp{tty} command.  This command accepts a file name as
argument, and causes this file to be the default for future @samp{run}
commands.  It also resets the controlling terminal for the child
process, for future @samp{run} commands.  For example,

@example
tty /dev/ttyb
@end example

@noindent
directs that processes started with subsequent @samp{run} commands
default to do input and output on the terminal @file{/dev/ttyb} and have
that as their controlling terminal.

An explicit redirection in @samp{run} overrides the @samp{tty} command's
effect on input/output redirection, but not its effect on the
controlling terminal.

When you use the @samp{tty} command or redirect input in the @samp{run}
command, only the @emph{input for your program} is affected.  The input
for GDB still comes from your terminal.

@node Attach, Kill Process, Input/Output, Running
@section Debugging an Already-Running Process
@kindex detach
@kindex attach
@cindex attach

Some operating systems allow GDB to debug an already-running process
that was started outside of GDB.  To do this, you use the @samp{attach}
command instead of the @samp{run} command.

The @samp{attach} command requires one argument, which is the process-id
of the process you want to debug.  (The usual way to find out the
process-id of the process is with the @code{ps} utility.)

The first thing GDB does after arranging to debug the process is to stop
it.  You can examine and modify an attached process with all the GDB
commands that ordinarily available when you start processes with
@samp{run}.  You can insert breakpoints; you can step and continue; you
can modify storage.  If you would rather the process continue running,
you may use the @samp{continue} command after attaching GDB to the
process.

When you have finished debugging the attached process, you can use the
@samp{detach} command to release it from GDB's control.  Detaching
the process continues its execution.  After the @samp{detach} command,
that process and GDB become completely independent once more, and you
are ready to @samp{attach} another process or start one with @samp{run}.

If you exit GDB or use the @samp{run} command while you have an attached
process, you kill that process.  You will be asked for confirmation if you
try to do either of these things.

The @samp{attach} command is also used to debug a remote machine via a
serial connection.  @xref{Attach}, for more info.

@node Kill Process,, Attach, Running
@section Killing the Child Process

@table @code
@item kill
@kindex kill
Kill the child process in which the program being debugged is running
under GDB.

This command is useful if you wish to debug a core dump instead.  GDB
ignores any core dump file if it is actually running the program, so the
@samp{kill} command is the only sure way to make sure the core dump file
is used once again.

It is also useful if you wish to run the program outside the debugger
for once and then go back to debugging it.

The @samp{kill} command is also useful if you wish to recompile and
relink the program, since on many systems it is impossible to modify an
executable file which is running in a process.  But, in this case, it is
just as good to exit GDB, since you will need to read a new symbol table
after the program is recompiled if you wish to debug the new version,
and restarting GDB is the easiest way to do that.
@end table

@node Stopping, Stack, Running, Top
@chapter Stopping and Continuing

When you run a program normally, it runs until it terminates.  The
principal purpose of using a debugger is so that you can stop it before
that point; or so that if the program runs into trouble you can
investigate and find out why.

@menu
* Signals::      Fatal signals in your program just stop it;
                 then you can use GDB to see what is going on.
* Breakpoints::  Breakpoints let you stop your program when it
                 reaches a specified point in the code.
* Continuing::   Resuming execution until the next signal or breakpoint.
* Stepping::     Stepping runs the program a short distance and
                 then stops it wherever it has come to.
@end menu

@node Signals, Breakpoints, Stopping, Stopping
@section Signals
@cindex signals

A signal is an asynchronous event that can happen in a program.  The
operating system defines the possible kinds of signals, and gives each kind
a name and a number.  For example, @code{SIGINT} is the signal a program
gets when you type @kbd{Ctrl-c}; @code{SIGSEGV} is the signal a program
gets from referencing a place in memory far away from all the areas in use;
@code{SIGALRM} occurs when the alarm clock timer goes off (which happens
only if the program has requested an alarm).

@cindex fatal signals
Some signals, including @code{SIGALRM}, are a normal part of the
functioning of the program.  Others, such as @code{SIGSEGV}, indicate
errors; these signals are @dfn{fatal} (kill the program immediately) if the
program has not specified in advance some other way to handle the signal.
@code{SIGINT} does not indicate an error in the program, but it is normally
fatal so it can carry out the purpose of @kbd{Ctrl-c}: to kill the program.

GDB has the ability to detect any occurrence of a signal in the program
running under GDB's control.  You can tell GDB in advance what to do for
each kind of signal.

@cindex handling signals
Normally, GDB is set up to ignore non-erroneous signals like @code{SIGALRM}
(so as not to interfere with their role in the functioning of the program)
but to stop the program immediately whenever an error signal happens.
You can change these settings with the @samp{handle} command.  You must
specify which signal you are talking about with its number.

@table @code
@item info signal
@kindex info signal
Print a table of all the kinds of signals and how GDB has been told to
handle each one.  You can use this to see the signal numbers of all
the defined types of signals.

@item handle @var{signalnum} @var{keywords}@dots{}
@kindex handle
Change the way GDB handles signal @var{signalnum}.  The @var{keywords}
say what change to make.
@end table

To use the @samp{handle} command you must know the code number of the
signal you are concerned with.  To find the code number, type @samp{info
signal} which prints a table of signal names and numbers.

The keywords allowed by the handle command can be abbreviated.  Their full
names are

@table @code
@item stop
GDB should stop the program when this signal happens.  This implies
the @samp{print} keyword as well.

@item print
GDB should print a message when this signal happens.

@item nostop
GDB should not stop the program when this signal happens.  It may
still print a message telling you that the signal has come in.

@item noprint
GDB should not mention the occurrence of the signal at all.  This
implies the @samp{nostop} keyword as well.

@item pass
GDB should allow the program to see this signal; the program will be
able to handle the signal, or may be terminated if the signal is fatal
and not handled.

@item nopass
GDB should not allow the program to see this signal.
@end table

When a signal has been set to stop the program, the program cannot see the
signal until you continue.  It will see the signal then, if @samp{pass} is
in effect for the signal in question @i{at that time}.  In other words,
after GDB reports a signal, you can use the @samp{handle} command with
@samp{pass} or @samp{nopass} to control whether that signal will be seen by
the program when you later continue it.

You can also use the @samp{signal} command to prevent the program from
seeing a signal, or cause it to see a signal it normally would not see,
or to give it any signal at any time.  @xref{Signaling}.

@node Breakpoints, Continuing, Signals, Stopping
@section Breakpoints

@cindex breakpoints
A @dfn{breakpoint} makes your program stop whenever a certain point in the
program is reached.  You set breakpoints explicitly with GDB commands,
specifying the place where the program should stop by line number, function
name or exact address in the program.  You can add various other conditions
to control whether the program will stop.

Each breakpoint is assigned a number when it is created; these numbers are
successive integers starting with 1.  In many of the commands for controlling
various features of breakpoints you use the breakpoint number to say which
breakpoint you want to change.  Each breakpoint may be @dfn{enabled} or
@dfn{disabled}; if disabled, it has no effect on the program until you
enable it again.

@kindex info break
@kindex $_
The command @samp{info break} prints a list of all breakpoints set and not
deleted, showing their numbers, where in the program they are, and any
special features in use for them.  Disabled breakpoints are included in the
list, but marked as disabled.  @samp{info break} with a breakpoint number
as argument lists only that breakpoint.  The convenience variable @code{$_}
and the default examining-address for the @samp{x} command are set to the
address of the last breakpoint listed (@pxref{Memory}).

@menu
* Set Breaks::     How to establish breakpoints.
* Delete Breaks::   How to remove breakpoints no longer needed.
* Disabling::      How to disable breakpoints (turn them off temporarily).
* Conditions::     Making extra conditions on whether to stop.
* Break Commands:: Commands to be executed at a breakpoint.
* Error in Breakpoints:: "Cannot insert breakpoints" error--why, what to do.
@end menu

@node Set Breaks, Delete Breaks, Breakpoints, Breakpoints
@subsection Setting Breakpoints

@kindex break
Breakpoints are set with the @samp{break} command (abbreviated @samp{b}).
You have several ways to say where the breakpoint should go.

@table @code
@item break @var{function}
Set a breakpoint at entry to function @var{function}.

@item break @var{+offset}
@itemx break @var{-offset}
Set a breakpoint some number of lines forward or back from the position
at which execution stopped in the currently selected frame.

@item break @var{linenum}
Set a breakpoint at line @var{linenum} in the current source file.
That file is the last file whose source text was printed.  This
breakpoint will stop the program just before it executes any of the
code on that line.

@item break @var{filename}:@var{linenum}
Set a breakpoint at line @var{linenum} in source file @var{filename}.

@item break @var{filename}:@var{function}
Set a breakpoint at entry to function @var{function} found in file
@var{filename}.  Specifying a file name as well as a function name is
superfluous except when multiple files contain similarly named
functions.

@item break *@var{address}
Set a breakpoint at address @var{address}.  You can use this to set
breakpoints in parts of the program which do not have debugging
information or source files.

@item break
Set a breakpoint at the next instruction to be executed in the selected
stack frame (@pxref{Stack}).  In any selected frame but the innermost,
this will cause the program to stop as soon as control returns to that
frame.  This is equivalent to a @samp{finish} command in the frame
inside the selected frame.  If this is done in the innermost frame, GDB
will stop the next time it reaches the current location; this may be
useful inside of loops.

GDB normally ignores breakpoints when it resumes execution, until at
least one instruction has been executed.  If it did not do this, you
would be unable to proceed past a breakpoint without first disabling the
breakpoint.  This rule applies whether or not the breakpoint already
existed when the program stopped.

@item break @dots{} if @var{cond}
Set a breakpoint with condition @var{cond}; evaluate the expression
@var{cond} each time the breakpoint is reached, and stop only if the
value is nonzero.  @samp{@dots{}} stands for one of the possible
arguments described above (or no argument) specifying where to break.
@xref{Conditions}, for more information on breakpoint conditions.

@item tbreak @var{args}
@kindex tbreak
Set a breakpoint enabled only for one stop.  @var{args} are the
same as in the @samp{break} command, and the breakpoint is set in the same
way, but the breakpoint is automatically disabled the first time it
is hit.  @xref{Disabling}.
@end table

GDB allows you to set any number of breakpoints at the same place in the
program.  There is nothing silly or meaningless about this.  When the
breakpoints are conditional, this is even useful (@pxref{Conditions}).

@node Delete Breaks, Disabling, Set Breaks, Breakpoints
@subsection Deleting Breakpoints

@cindex clearing breakpoint
@cindex deleting breakpoints
It is often necessary to eliminate a breakpoint once it has done its job
and you no longer want the program to stop there.  This is called
@dfn{deleting} the breakpoint.  A breakpoint that has been deleted no
longer exists in any sense; it is forgotten.

With the @samp{clear} command you can delete breakpoints according to where
they are in the program.  With the @samp{delete} command you can delete
individual breakpoints by specifying their breakpoint numbers.

@b{It is not necessary to delete a breakpoint to proceed past it.}  GDB
automatically ignores breakpoints in the first instruction to be executed
when you continue execution without changing the execution address.

@table @code
@item clear
@kindex clear
Delete any breakpoints at the next instruction to be executed in the
selected stack frame (@pxref{Selection}).  When the innermost frame
is selected, this is a good way to delete a breakpoint that the program
just stopped at.

@item clear @var{function}
@itemx clear @var{filename}:@var{function}
Delete any breakpoints set at entry to the function @var{function}.

@item clear @var{linenum}
@itemx clear @var{filename}:@var{linenum}
Delete any breakpoints set at or within the code of the specified line.

@item delete @var{bnums}@dots{}
@kindex delete
Delete the breakpoints of the numbers specified as arguments.
@end table

@node Disabling, Conditions, Delete Breaks, Breakpoints
@subsection Disabling Breakpoints

@cindex disabled breakpoints
@cindex enabled breakpoints
Rather than deleting a breakpoint, you might prefer to @dfn{disable} it.
This makes the breakpoint inoperative as if it had been deleted, but
remembers the information on the breakpoint so that you can @dfn{enable}
it again later.

You disable and enable breakpoints with the @samp{enable} and
@samp{disable} commands, specifying one or more breakpoint numbers as
arguments.  Use @samp{info break} to print a list of breakpoints if you
don't know which breakpoint numbers to use.

A breakpoint can have any of four different states of enablement:

@itemize @bullet
@item
Enabled.  The breakpoint will stop the program.  A breakpoint made
with the @samp{break} command starts out in this state.
@item
Disabled.  The breakpoint has no effect on the program.
@item
Enabled once.  The breakpoint will stop the program, but
when it does so it will become disabled.  A breakpoint made
with the @samp{tbreak} command starts out in this state.
@item
Enabled for deletion.  The breakpoint will stop the program, but
immediately after it does so it will be deleted permanently.
@end itemize

You change the state of enablement of a breakpoint with the following
commands:

@table @code
@item disable breakpoints @var{bnums}@dots{}
@itemx disable @var{bnums}@dots{}
@kindex disable breakpoints
@kindex disable
Disable the specified breakpoints.  A disabled breakpoint has no
effect but is not forgotten.  All options such as ignore-counts,
conditions and commands are remembered in case the breakpoint is
enabled again later.

@item enable breakpoints @var{bnums}@dots{}
@itemx enable @var{bnums}@dots{}
@kindex enable breakpoints
@kindex enable
Enable the specified breakpoints.  They become effective once again in
stopping the program, until you specify otherwise.

@item enable breakpoints once @var{bnums}@dots{}
@itemx enable once @var{bnums}@dots{}
Enable the specified breakpoints temporarily.  Each will be disabled
again the next time it stops the program (unless you have used one of
these commands to specify a different state before that time comes).

@item enable breakpoints delete @var{bnums}@dots{}
@itemx enable delete @var{bnums}@dots{}
Enable the specified breakpoints to work once and then die.  Each of
the breakpoints will be deleted the next time it stops the program
(unless you have used one of these commands to specify a different
state before that time comes).
@end table

Aside from the automatic disablement or deletion of a breakpoint when it
stops the program, which happens only in certain states, the state of
enablement of a breakpoint changes only when one of the commands above
is used.

@node Conditions, Break Commands, Disabling, Breakpoints
@subsection Break Conditions
@cindex conditional breakpoints
@cindex breakpoint conditions

The simplest sort of breakpoint breaks every time the program reaches a
specified place.  You can also specify a @dfn{condition} for a
breakpoint.  A condition is just a boolean expression in your
programming language.  (@xref{Expressions}).  A breakpoint with a
condition evaluates the expression each time the program reaches it, and
the program stops only if the condition is true.

Break conditions may have side effects, and may even call functions in your
program.  These may sound like strange things to do, but their effects are
completely predictable unless there is another enabled breakpoint at the
same address.  (In that case, GDB might see the other breakpoint first and
stop the program without checking the condition of this one.)  Note that
breakpoint commands are usually more convenient and flexible for the
purpose of performing side effects when a breakpoint is reached
(@pxref{Break Commands}).

Break conditions can be specified when a breakpoint is set, by using
@samp{if} in the arguments to the @samp{break} command.  @xref{Set Breaks}.
They can also be changed at any time with the @samp{condition} command:

@table @code
@item condition @var{bnum} @var{expression}
@kindex condition
Specify @var{expression} as the break condition for breakpoint number
@var{bnum}.  From now on, this breakpoint will stop the program only if
the value of @var{expression} is true (nonzero, in C).  @var{expression}
is not evaluated at the time the @samp{condition} command is given.
@xref{Expressions}.

@item condition @var{bnum}
Remove the condition from breakpoint number @var{bnum}.  It becomes
an ordinary unconditional breakpoint.
@end table

@cindex ignore count (of breakpoint)
A special case of a breakpoint condition is to stop only when the
breakpoint has been reached a certain number of times.  This is so
useful that there is a special way to do it, using the @dfn{ignore
count} of the breakpoint.  Every breakpoint has an ignore count, which
is an integer.  Most of the time, the ignore count is zero, and
therefore has no effect.  But if the program reaches a breakpoint whose
ignore count is positive, then instead of stopping, it just decrements
the ignore count by one and continues.  As a result, if the ignore count
value is @var{n}, the breakpoint will not stop the next @var{n} times it
is reached.

@table @code
@item ignore @var{bnum} @var{count}
@kindex ignore
Set the ignore count of breakpoint number @var{bnum} to @var{count}.
The next @var{count} times the breakpoint is reached, it will not stop.

To make the breakpoint stop the next time it is reached, specify
a count of zero.

@item cont @var{count}
Continue execution of the program, setting the ignore count of the
breakpoint that the program stopped at to @var{count} minus one.
Thus, the program will not stop at this breakpoint until the
@var{count}'th time it is reached.

This command is allowed only when the program stopped due to a
breakpoint.  At other times, the argument to @samp{cont} is ignored.
@end table

If a breakpoint has a positive ignore count and a condition, the condition
is not checked.  Once the ignore count reaches zero, the condition will
start to be checked.

Note that you could achieve the effect of the ignore count with a
condition such as @w{@samp{$foo-- <= 0}} using a debugger convenience
variable that is decremented each time.  @xref{Convenience Vars}.

@node Break Commands, Error in Breakpoints, Conditions, Breakpoints
@subsection Commands Executed on Breaking

@cindex breakpoint commands
You can give any breakpoint a series of commands to execute when the
program stops due to that breakpoint.  For example, you might want to
print the values of certain expressions, or enable other breakpoints.

@table @code
@item commands @var{bnum}
Specify commands for breakpoint number @var{bnum}.  The commands
themselves appear on the following lines.  Type a line containing just
@samp{end} to terminate the commands.

\hfuzz = 10pt
To remove all commands from a breakpoint, use the command
@samp{commands} and follow it immediately by @samp{end}; that is, give
no commands.

With no arguments, @samp{commands} refers to the last breakpoint set.
@end table

It is possible for breakpoint commands to start the program up again.
Simply use the @samp{cont} command, or @samp{step}, or any other command
to resume execution.  However, any remaining breakpoint commands are
ignored.  When the program stops again, GDB will act according to the
cause of that stop.

@kindex silent
If the first command specified is @samp{silent}, the usual message about
stopping at a breakpoint is not printed.  This may be desirable for
breakpoints that are to print a specific message and then continue.
If the remaining commands too print nothing, you will see no sign that
the breakpoint was reached at all.  @samp{silent} is not really a command;
it is meaningful only at the beginning of the commands for a breakpoint.

The commands @samp{echo} and @samp{output} that allow you to print precisely
controlled output are often useful in silent breakpoints.  @xref{Output}.

For example, here is how you could use breakpoint commands to print the
value of @code{x} at entry to @code{foo} whenever it is positive.

@example
break foo if x>0
commands
silent
echo x is\040
output x
echo \n
cont
end
@end example

One application for breakpoint commands is to correct one bug so you can
test another.  Put a breakpoint just after the erroneous line of code, give
it a condition to detect the case in which something erroneous has been
done, and give it commands to assign correct values to any variables that
need them.  End with the @samp{cont} command so that the program does not
stop, and start with the @samp{silent} command so that no output is
produced.  Here is an example:

@example
break 403
commands
silent
set x = y + 4
cont
end
@end example

One deficiency in the operation of automatically continuing breakpoints
under Unix appears when your program uses raw mode for the terminal.
GDB switches back to its own terminal modes (not raw) before executing
commands, and then must switch back to raw mode when your program is
continued.  This causes any pending terminal input to be lost.

In the GNU system, this will be fixed by changing the behavior of
terminal modes.

Under Unix, when you have this problem, you might be able to get around
it by putting your actions into the breakpoint condition instead of
commands.  For example

@example
condition 5  (x = y + 4), 0
@end example

@noindent
specifies a condition expression (@xref{Expressions}) that will change
@code{x} as needed, then always have the value 0 so the program will not
stop.  Loss of input is avoided here because break conditions are
evaluated without changing the terminal modes.  When you want to have
nontrivial conditions for performing the side effects, the operators
@samp{&&}, @samp{||} and @samp{?@dots{}:} may be useful.

@node Error in Breakpoints,, Break Commands, Breakpoints
@subsection ``Cannot Insert Breakpoints'' Error

Under some operating systems, breakpoints cannot be used in a program if
any other process is running that program.  Attempting to run or
continue the program with a breakpoint in this case will cause GDB to
stop it.

When this happens, you have three ways to proceed:

@enumerate
@item
Remove or disable the breakpoints, then continue.

@item
Suspend GDB, and copy the file containing the program to a new name.
Resume GDB and use the @samp{exec-file} command to specify that GDB
should run the program under that name.  Then start the program again.

@item
Relink the program so that the text segment is nonsharable, using the
linker option @samp{-N}.  The operating system limitation may not apply
to nonsharable executables.
@end enumerate

@node Continuing, Stepping, Breakpoints, Stopping
@section Continuing

After your program stops, most likely you will want it to run some more if
the bug you are looking for has not happened yet.

@table @code
@item cont
@kindex cont
Continue running the program at the place where it stopped.
@end table

If the program stopped at a breakpoint, the place to continue running
is the address of the breakpoint.  You might expect that continuing would
just stop at the same breakpoint immediately.  In fact, @samp{cont}
takes special care to prevent that from happening.  You do not need
to delete the breakpoint to proceed through it after stopping at it.

You can, however, specify an ignore-count for the breakpoint that the
program stopped at, by means of an argument to the @samp{cont} command.
@xref{Conditions}.

If the program stopped because of a signal other than @code{SIGINT} or
@code{SIGTRAP}, continuing will cause the program to see that signal.
You may not want this to happen.  For example, if the program stopped
due to some sort of memory reference error, you might store correct
values into the erroneous variables and continue, hoping to see more
execution; but the program would probably terminate immediately as
a result of the fatal signal once it sees the signal.  To prevent this,
you can continue with @samp{signal 0}.  @xref{Signaling}.  You can
also act in advance to prevent the program from seeing certain kinds
of signals, using the @samp{handle} command (@pxref{Signals}).

@node Stepping,, Continuing, Stopping
@section Stepping

@cindex stepping
@dfn{Stepping} means setting your program in motion for a limited time, so
that control will return automatically to the debugger after one line of
code or one machine instruction.  Breakpoints are active during stepping
and the program will stop for them even if it has not gone as far as the
stepping command specifies.

@table @code
@item step
@kindex step
Continue running the program until control reaches a different line,
then stop it and return control to the debugger.  This command is
abbreviated @samp{s}.

This command may be given when control is within a function for which
there is no debugging information.  In that case, execution will proceed
until control reaches a different function, or is about to return from
this function.  An argument repeats this action.

@item step @var{count}
Continue running as in @samp{step}, but do so @var{count} times.  If a
breakpoint is reached or a signal not related to stepping occurs before
@var{count} steps, stepping stops right away.

@item next
@kindex next
Similar to @samp{step}, but any function calls appearing within the line of
code are executed without stopping.  Execution stops when control reaches a
different line of code at the stack level which was executing when the
@samp{next} command was given.  This command is abbreviated @samp{n}.

An argument is a repeat count, as in @samp{step}.

@samp{next} within a function without debugging information acts as does
@samp{step}, but any function calls appearing within the code of the
function are executed without stopping.

@item finish
@kindex finish
Continue running until just after the selected stack frame returns (or
until there is some other reason to stop, such as a fatal signal or a
breakpoint).  Print value returned by the selected stack frame (if any).

Contrast this with the @samp{return} command (@pxref{Returning}).

@item until
@kindex until
This command is used to avoid single stepping through a loop more than
once.  It is like the @samp{next} command, except that when @samp{until}
encounters a jump, it automatically continues execution until the
program counter is greater than the address of the jump.

This means that when you reach the end of a loop after single stepping
though it, @samp{until} will cause the program to continue execution
until the loop is exited.  In contrast, a @samp{next} command at the end
of a loop will simply step back to the beginning of the loop, which
would force you to step through the next iteration.

@samp{until} always stops the program if it attempts to exit the current
stack frame.

@samp{until} may produce somewhat counterintuitive results if the order
of the source lines does not match the actual order of execution.  For
example, in a typical C @code{for}-loop, the third expression in the
@code{for}-statement (the loop-step expression) is executed after the
statements in the body of the loop, but is written before them.
Therefore, the @samp{until} command would appear to step back to the
beginning of the loop when it advances to this expression.  However, it
has not really done so, not in terms of the actual machine code.

Note that @samp{until} with no argument works by means of single
instruction stepping, and hence is slower than @samp{until} with an
argument.

@item until @var{location}
Continue running the program until either the specified location is
reached, or the current (innermost) stack frame returns.  This form of
the command uses breakpoints, and hence is quicker than @samp{until}
without an argument.

@item stepi
@itemx si
@kindex stepi
@kindex si
Execute one machine instruction, then stop and return to the debugger.

It is often useful to do @samp{display/i $pc} when stepping by machine
instructions.  This will cause the next instruction to be executed to
be displayed automatically at each stop.  @xref{Auto Display}.

An argument is a repeat count, as in @samp{step}.

@item nexti
@itemx ni
@kindex nexti
@kindex ni
Execute one machine instruction, but if it is a subroutine call,
proceed until the subroutine returns.

An argument is a repeat count, as in @samp{next}.
@end table

A typical technique for using stepping is to put a breakpoint
(@pxref{Breakpoints}) at the beginning of the function or the section of
the program in which a problem is believed to lie, and then step through
the suspect area, examining the variables that are interesting, until the
problem happens.

The @samp{cont} command can be used after stepping to resume execution
until the next breakpoint or signal.

@node Stack, Source, Stopping, Top
@chapter Examining the Stack

When your program has stopped, the first thing you need to know is where it
stopped and how it got there.

@cindex call stack
Each time your program performs a function call, the information about
where in the program the call was made from is saved in a block of data
called a @dfn{stack frame}.  The frame also contains the arguments of the
call and the local variables of the function that was called.  All the
stack frames are allocated in a region of memory called the @dfn{call
stack}.

When your program stops, the GDB commands for examining the stack allow you
to see all of this information.

One of the stack frames is @dfn{selected} by GDB and many GDB commands
refer implicitly to the selected frame.  In particular, whenever you ask
GDB for the value of a variable in the program, the value is found in the
selected frame.  There are special GDB commands to select whichever frame
you are interested in.

When the program stops, GDB automatically selects the currently executing
frame and describes it briefly as the @samp{frame} command does
(@pxref{Frame Info, Info}).

@menu
* Frames::          Explanation of stack frames and terminology.
* Backtrace::       Summarizing many frames at once.
* Selection::       How to select a stack frame.
* Info: Frame Info, Commands to print information on stack frames.
@end menu

@node Frames, Backtrace, Stack, Stack
@section Stack Frames

@cindex frame
@cindex stack frame
The call stack is divided up into contiguous pieces called @dfn{stack
frames}, or @dfn{frames} for short; each frame is the data associated
with one call to one function.  The frame contains the arguments given
to the function, the function's local variables, and the address at
which the function is executing.

@cindex initial frame
@cindex outermost frame
@cindex innermost frame
When your program is started, the stack has only one frame, that of the
function @code{main}.  This is called the @dfn{initial} frame or the
@dfn{outermost} frame.  Each time a function is called, a new frame is
made.  Each time a function returns, the frame for that function invocation
is eliminated.  If a function is recursive, there can be many frames for
the same function.  The frame for the function in which execution is
actually occurring is called the @dfn{innermost} frame.  This is the most
recently created of all the stack frames that still exist.

@cindex frame pointer
Inside your program, stack frames are identified by their addresses.  A
stack frame consists of many bytes, each of which has its own address; each
kind of computer has a convention for choosing one of those bytes whose
address serves as the address of the frame.  Usually this address is kept
in a register called the @dfn{frame pointer register} while execution is
going on in that frame.

@cindex frame number
GDB assigns numbers to all existing stack frames, starting with zero for
the innermost frame, one for the frame that called it, and so on upward.
These numbers do not really exist in your program; they are to give you a
way of talking about stack frames in GDB commands.

@cindex selected frame
Many GDB commands refer implicitly to one stack frame.  GDB records a stack
frame that is called the @dfn{selected} stack frame; you can select any
frame using one set of GDB commands, and then other commands will operate
on that frame.  When your program stops, GDB automatically selects the
innermost frame.

@cindex frameless execution
Some functions can be compiled to run without a frame reserved for them
on the stack.  This is occasionally done with heavily used library
functions to save the frame setup time.  GDB has limited facilities for
dealing with these function invocations; if the innermost function
invocation has no stack frame, GDB will give it a virtual stack frame of
0 and correctly allow tracing of the function call chain.  Results are
undefined if a function invocation besides the innermost one is
frameless. 

@node Backtrace, Selection, Frames, Stack
@section Backtraces

A backtrace is a summary of how the program got where it is.  It shows one
line per frame, for many frames, starting with the currently executing
frame (frame zero), followed by its caller (frame one), and on up the
stack.

@table @code
@item backtrace
@itemx bt
@kindex backtrace
@kindex bt
Print a backtrace of the entire stack: one line per frame for all
frames in the stack.

You can stop the backtrace at any time by typing the system interrupt
character, normally @kbd{Control-C}.

@item backtrace @var{n}
@itemx bt @var{n}
Similar, but print only the innermost @var{n} frames.

@item backtrace @var{-n}
@itemx bt @var{-n}
Similar, but print only the outermost @var{n} frames.
@end table

@kindex where
@kindex info stack
The names @samp{where} and @samp{info stack} are additional aliases
for @samp{backtrace}.

Every line in the backtrace shows the frame number, the function name
and the program counter value.

If the function is in a source file whose symbol table data has been
fully read, the backtrace shows the source file name and line number, as
well as the arguments to the function.  (The program counter value is
omitted if it is at the beginning of the code for that line number.)

If the source file's symbol data has not been fully read, just scanned,
this extra information is replaced with an ellipsis.  You can force the
symbol data for that frame's source file to be read by selecting the
frame.  (@xref{Selection}).

Here is an example of a backtrace.  It was made with the command
@samp{bt 3}, so it shows the innermost three frames.

@example
#0  rtx_equal_p (x=(rtx) 0x8e58c, y=(rtx) 0x1086c4) 
(/gp/rms/cc/rtlanal.c line 337)
#1  0x246b0 in expand_call (...) (...)
#2  0x21cfc in expand_expr (...) (...)
(More stack frames follow...)
@end example

@noindent
The functions @code{expand_call} and @code{expand_expr} are in a file
whose symbol details have not been fully read.  Full detail is available
for the function @code{rtx_equal_p}, which is in the file
@file{rtlanal.c}.  Its arguments, named @code{x} and @code{y}, are shown
with their typed values.

@node Selection, Frame Info, Backtrace, Stack
@section Selecting a Frame

Most commands for examining the stack and other data in the program work on
whichever stack frame is selected at the moment.  Here are the commands for
selecting a stack frame; all of them finish by printing a brief description
of the stack frame just selected.

@table @code
@item frame @var{n}
@kindex frame
Select frame number @var{n}.  Recall that frame zero is the innermost
(currently executing) frame, frame one is the frame that called the
innermost one, and so on.  The highest-numbered frame is @code{main}'s
frame.

@item frame @var{addr}
Select the frame at address @var{addr}.  This is useful mainly if the
chaining of stack frames has been damaged by a bug, making it
impossible for GDB to assign numbers properly to all frames.  In
addition, this can be useful when the program has multiple stacks and
switches between them.

@item up @var{n}
@kindex up
Select the frame @var{n} frames up from the frame previously selected.
For positive numbers @var{n}, this advances toward the outermost
frame, to higher frame numbers, to frames that have existed longer.
@var{n} defaults to one.

@item down @var{n}
@kindex down
Select the frame @var{n} frames down from the frame previously
selected.  For positive numbers @var{n}, this advances toward the
innermost frame, to lower frame numbers, to frames that were created
more recently.  @var{n} defaults to one.
@end table

All of these commands end by printing some information on the frame that
has been selected: the frame number, the function name, the arguments, the
source file and line number of execution in that frame, and the text of
that source line.  For example:

@example
#3  main (argc=3, argv=??, env=??) at main.c, line 67
67        read_input_file (argv[i]);
@end example

After such a printout, the @samp{list} command with no arguments will print
ten lines centered on the point of execution in the frame.  @xref{List}.

@node Frame Info,, Selection, Stack
@section Information on a Frame

There are several other commands to print information about the selected
stack frame.

@table @code
@item frame
This command prints a brief description of the selected stack frame.
It can be abbreviated @samp{f}.  With an argument, this command is
used to select a stack frame; with no argument, it does not change
which frame is selected, but still prints the same information.

@item info frame
@kindex info frame
This command prints a verbose description of the selected stack frame,
including the address of the frame, the addresses of the next frame in
(called by this frame) and the next frame out (caller of this frame),
the address of the frame's arguments, the program counter saved in it
(the address of execution in the caller frame), and which registers
were saved in the frame.  The verbose description is useful when
something has gone wrong that has made the stack format fail to fit
the usual conventions.

@item info frame @var{addr}
Print a verbose description of the frame at address @var{addr},
without selecting that frame.  The selected frame remains unchanged by
this command.

@item info args
@kindex info args
Print the arguments of the selected frame, each on a separate line.

@item info locals
@kindex info locals
Print the local variables of the selected frame, each on a separate
line.  These are all variables declared static or automatic within all
program blocks that execution in this frame is currently inside of.
@end table

@node Source, Data, Stack, Top
@chapter Examining Source Files

GDB knows which source files your program was compiled from, and
can print parts of their text.  When your program stops, GDB
spontaneously prints the line it stopped in.  Likewise, when you
select a stack frame (@pxref{Selection}), GDB prints the line
which execution in that frame has stopped in.  You can also
print parts of source files by explicit command.

@menu
* List::        Using the @samp{list} command to print source files.
* Search::      Commands for searching source files.
* Source Path:: Specifying the directories to search for source files.
@end menu

@node List, Search, Source, Source
@section Printing Source Lines

@kindex list
To print lines from a source file, use the @samp{list} command
(abbreviated @samp{l}).  There are several ways to specify what part
of the file you want to print.

Here are the forms of the @samp{list} command most commonly used:

@table @code
@item list @var{linenum}
Print ten lines centered around line number @var{linenum} in the
current source file.

@item list @var{function}
Print ten lines centered around the beginning of function
@var{function}.

@item list
Print ten more lines.  If the last lines printed were printed with a
@samp{list} command, this prints ten lines following the last lines
printed; however, if the last line printed was a solitary line printed
as part of displaying a stack frame (@pxref{Stack}), this prints ten
lines centered around that line.

@item list -
Print ten lines just before the lines last printed.
@end table

Repeating a @samp{list} command with @key{RET} discards the argument,
so it is equivalent to typing just @samp{list}.  This is more useful
than listing the same lines again.  An exception is made for an
argument of @samp{-}; that argument is preserved in repetition so that
each repetition moves up in the file.

@cindex linespec
In general, the @samp{list} command expects you to supply zero, one or two
@dfn{linespecs}.  Linespecs specify source lines; there are several ways
of writing them but the effect is always to specify some source line.
Here is a complete description of the possible arguments for @samp{list}:

@table @code
@item list @var{linespec}
Print ten lines centered around the line specified by @var{linespec}.

@item list @var{first},@var{last}
Print lines from @var{first} to @var{last}.  Both arguments are
linespecs.

@item list ,@var{last}
Print ten lines ending with @var{last}.

@item list @var{first},
Print ten lines starting with @var{first}.

@item list +
Print ten lines just after the lines last printed.

@item list -
Print ten lines just before the lines last printed.

@item list
As described in the preceding table.
@end table

Here are the ways of specifying a single source line---all the
kinds of linespec.

@table @code
@item @var{linenum}
Specifies line @var{linenum} of the current source file.
When a @samp{list} command has two linespecs, this refers to
the same source file as the first linespec.

@item +@var{offset}
Specifies the line @var{offset} lines after the last line printed.
When used as the second linespec in a @samp{list} command that has
two, this specifies the line @var{offset} lines down from the
first linespec.

@item -@var{offset}
Specifies the line @var{offset} lines before the last line printed.

@item @var{filename}:@var{linenum}
Specifies line @var{linenum} in the source file @var{filename}.

@item @var{function}
Specifies the line of the open-brace that begins the body of the
function @var{function}.

@item @var{filename}:@var{function}
Specifies the line of the open-brace that begins the body of the
function @var{function} in the file @var{filename}.  The file name is
needed with a function name only for disambiguation of identically
named functions in different source files.

@item *@var{address}
Specifies the line containing the program address @var{address}.
@var{address} may be any expression.
@end table

One other command is used to map source lines to program addresses.

@table @code
@item info line @var{linenum}
@kindex info line
Print the starting and ending addresses of the compiled code for
source line @var{linenum}.

@kindex $_
The default examine address for the @samp{x} command is changed to the
starting address of the line, so that @samp{x/i} is sufficient to
begin examining the machine code (@pxref{Memory}).  Also, this address
is saved as the value of the convenience variable @code{$_}
(@pxref{Convenience Vars}).
@end table

@node Search, Source Path, List, Source
@section Searching Source Files
@cindex searching
@kindex forward-search
@kindex reverse-search

There are two commands for searching through the current source file for a
regular expression.

The command @samp{forward-search @var{regexp}} checks each line, starting
with the one following the last line listed, for a match for @var{regexp}.
It lists the line that is found.  You can abbreviate the command name
as @samp{fo}.

The command @samp{reverse-search @var{regexp}} checks each line, starting
with the one before the last line listed and going backward, for a match
for @var{regexp}.  It lists the line that is found.  You can abbreviate
this command with as little as @samp{rev}.

@node Source Path,, Search, Source
@section Specifying Source Directories

@cindex source path
@cindex directories for source files
Executable programs do not record the directories of the source files
from which they were compiled, just the names.  GDB remembers a list of
directories to search for source files; this is called the @dfn{source
path}.  Each time GDB wants a source file, it tries all the directories
in the list, in the order they are present in the list, until it finds a
file with the desired name.  @b{Note that the executable search path is
@i{not} used for this purpose.  Neither is the current working
directory, unless it happens to be in the source path.}

@kindex directory
When you start GDB, its source path contains just the current working
directory.  To add other directories, use the @samp{directory} command.

@table @code
@item directory @var{dirnames...}
Add directory @var{dirname} to the end of the source path.  Several
directory names may be given to this command, separated by whitespace or
@samp{:}.

@item directory
Reset the source path to just the current working directory of GDB.
This requires confirmation.

Since this command deletes directories from the search path, it may
change the directory in which a previously read source file will be
discovered.  To make this work correctly, this command also clears out
the tables GDB maintains about the source files it has already found.

@item info directories
@kindex info directories
Print the source path: show which directories it contains.
@end table

Because the @samp{directory} command adds to the end of the source path,
it does not affect any file that GDB has already found.  If the source
path contains directories that you do not want, and these directories
contain misleading files with names matching your source files, the
way to correct the situation is as follows:

@enumerate
@item
Choose the directory you want at the beginning of the source path.
Use the @samp{cd} command to make that the current working directory.

@item
Use @samp{directory} with no argument to reset the source path to just
that directory.

@item
Use @samp{directory} with suitable arguments to add any other
directories you want in the source path.
@end enumerate

@node Data, Symbols, Source, Top
@chapter Examining Data

@cindex printing data
@cindex examining data
@kindex print
The usual way to examine data in your program is with the @samp{print}
command (abbreviated @samp{p}).  It evaluates and prints the value of any
valid expression of the language the program is written in (for now, C).
You type

@example
print @var{exp}
@end example

@noindent
where @var{exp} is any valid expression, and the value of @var{exp}
is printed in a format appropriate to its data type.

A more low-level way of examining data is with the @samp{x} command.
It examines data in memory at a specified address and prints it in a
specified format.

@menu
* Expressions::      Expressions that can be computed and printed.
* Variables::        Using your program's variables in expressions.
* Assignment::       Setting your program's variables.
* Arrays::           Examining part of memory as an array.
* Format Options::   Controlling how structures and arrays are printed.
* Output formats::   Specifying formats for printing values.
* Memory::           Examining memory explicitly.
* Auto Display::     Printing certain expressions whenever program stops.
* Value History::    Referring to values previously printed.
* Convenience Vars:: Giving names to values for future reference.
* Registers::        Referring to and storing in machine registers.
@end menu

@node Expressions, Variables, Data, Data
@section Expressions

@cindex expressions
Many different GDB commands accept an expression and compute its value.
Any kind of constant, variable or operator defined by the programming
language you are using is legal in an expression in GDB.  This includes
conditional expressions, function calls, casts and string constants.
It unfortunately does not include symbols defined by preprocessor
@code{#define} commands.

Casts are supported in all languages, not just in C, because it is so
useful to cast a number into a pointer so as to examine a structure
at that address in memory.

GDB supports three kinds of operator in addition to those of programming
languages:

@table @code
@item @@
@samp{@@} is a binary operator for treating parts of memory as arrays.
@xref{Arrays}, for more information.

@item ::
@samp{::} allows you to specify a variable in terms of the file or
function it is defined in.  @xref{Variables}.

@item @{@var{type}@} @var{addr}
Refers to an object of type @var{type} stored at address @var{addr} in
memory.  @var{addr} may be any expression whose value is an integer or
pointer (but parentheses are required around nonunary operators, just as in
a cast).  This construct is allowed regardless of what kind of data is
officially supposed to reside at @var{addr}.@refill
@end table

@node Variables, Arrays, Expressions, Data
@section Program Variables

The most common kind of expression to use is the name of a variable
in your program.

Variables in expressions are understood in the selected stack frame
(@pxref{Selection}); they must either be global (or static) or be visible
according to the scope rules of the programming language from the point of
execution in that frame.  This means that in the function

@example
foo (a)
     int a;
@{
  bar (a);
  @{
    int b = test ();
    bar (b);
  @}
@}
@end example

@noindent
the variable @code{a} is usable whenever the program is executing
within the function @code{foo}, but the variable @code{b} is visible
only while the program is executing inside the block in which @code{b}
is declared.

As a special exception, you can refer to a variable or function whose
scope is a single source file even if the current execution point is not
in this file.  But it is possible to have more than one such variable
or function with the same name (if they are in different source files).
In such a case, it is not defined which one you will get.  If you wish,
you can specify any one of them using the colon-colon construct:

@example
@var{block}::@var{variable}
@end example

@noindent
Here @var{block} is the name of the source file whose variable you want.

@node Arrays, Format options, Variables, Data
@section Artificial Arrays

@cindex artificial array
It is often useful to print out several successive objects of the
same type in memory; a section of an array, or an array of
dynamically determined size for which only a pointer exists in the
program.

This can be done by constructing an @dfn{artificial array} with the
binary operator @samp{@@}.  The left operand of @samp{@@} should be
the first element of the desired array, as an individual object.
The right operand should be the length of the array.  The result is
an array value whose elements are all of the type of the left argument.
The first element is actually the left argument; the second element
comes from bytes of memory immediately following those that hold the
first element, and so on.  Here is an example.  If a program says

@example
int *array = (int *) malloc (len * sizeof (int));
@end example

@noindent
you can print the contents of @code{array} with

@example
p *array@@len
@end example

The left operand of @samp{@@} must reside in memory.  Array values made
with @samp{@@} in this way behave just like other arrays in terms of
subscripting, and are coerced to pointers when used in expressions.
(It would probably appear in an expression via the value history,
after you had printed it out.)

@node Format options, Output formats, Arrays, Data
@section Format options

@cindex format options
GDB provides a few ways to control how arrays and structures are
printed.  

@table @code
@item info format
@kindex info format
Display the current settings for the format options.

@item set array-max @var{number-of-elements}
@kindex set array-max
If GDB is printing a large array, it will stop printing after it has
printed the number of elements set by the @samp{set array-max} command.
This limit also applies to the display of strings.

@item set prettyprint on
@kindex set prettyprint
Cause GDB to print structures in an indented format with one member per
line, like this:

@example
$1 = @{
  next = 0x0,
  flags = @{
    sweet = 1,
    sour = 1
  @},
  meat = 0x54 "Pork"
@}
@end example

@item set prettyprint off
Cause GDB to print structures in a compact format, like this:

@example
$1 = @{next = 0x0, flags = @{sweet = 1, sour = 1@}, 
meat = 0x54 "Pork"@}
@end example

This is the default format.

@item set unionprint on
@kindex set unionprint
Tell GDB to print unions which are contained in structures.  This is the
default setting.
@item set unionprint off
Tell GDB not to print unions which are contained in structures.

For example, given the declarations

@example
typedef enum @{Tree, Bug@} Species;
typedef enum @{Big_tree, Acorn, Seedling@} Tree_forms;
typedef enum @{Caterpiller, Cocoon, Butterfly@} Bug_forms;

struct thing @{
  Species it;
  union @{
    Tree_forms tree;
    Bug_forms bug;
  @} form;
@};

struct thing foo = @{Tree, @{Acorn@}@};
@end example

@noindent
with @samp{set unionprint on} in effect @samp{p foo} would print

@example
$1 = @{it = Tree, form = @{tree = Acorn, bug = Cocoon@}@}
@end example

@noindent
and with @samp{set unionprint off} in effect it would print

@example
$1 = @{it = Tree, form = @{...@}@}
@end example
@end table

@node Output formats, Memory, Format options, Data
@section Output formats

@cindex formatted output
@cindex output formats
GDB normally prints all values according to their data types.  Sometimes
this is not what you want.  For example, you might want to print a number
in hex, or a pointer in decimal.  Or you might want to view data in memory
at a certain address as a character string or an instruction.  These things
can be done with @dfn{output formats}.

The simplest use of output formats is to say how to print a value
already computed.  This is done by starting the arguments of the
@samp{print} command with a slash and a format letter.  The format
letters supported are:

@table @samp
@item x
Regard the bits of the value as an integer, and print the integer in
hexadecimal.

@item d
Print as integer in signed decimal.

@item u
Print as integer in unsigned decimal.

@item o
Print as integer in octal.

@item a
Print as an address, both absolute in hex and then relative
to a symbol defined as an address below it.

@item c
Regard as an integer and print it as a character constant.

@item f
Regard the bits of the value as a floating point number and print
using typical floating point syntax.
@end table

For example, to print the program counter in hex (@pxref{Registers}), type

@example
p/x $pc
@end example

@noindent
Note that no space is required before the slash; this is because command
names in GDB cannot contain a slash.

To reprint the last value in the value history with a different format,
you can use the @samp{print} command with just a format and no
expression.  For example, @samp{p/x} reprints the last value in hex.

@node Memory, Auto Display, Output formats, Data
@subsection Examining Memory

@cindex examining memory
@kindex x
The command @samp{x} (for `examine') can be used to examine memory
without reference to the program's data types.  The format in which you
wish to examine memory is instead explicitly specified.  The allowable
formats are a superset of the formats described in the previous section.

@samp{x} is followed by a slash and an output format specification,
followed by an expression for an address.  The expression need not have
a pointer value (though it may); it is used as an integer, as the
address of a byte of memory.  @xref{Expressions} for more information on
expressions.  For example, @samp{x/4xw $sp} prints the four words of
memory above the stack pointer in hexadecimal.

The output format in this case specifies both how big a unit of memory
to examine and how to print the contents of that unit.  It is done
with one or two of the following letters:

These letters specify just the size of unit to examine:

@table @samp
@item b
Examine individual bytes.

@item h
Examine halfwords (two bytes each).

@item w
Examine words (four bytes each).

@cindex word
Many assemblers and cpu designers still use `word' for a 16-bit quantity,
as a holdover from specific predecessor machines of the 1970's that really
did use two-byte words.  But more generally the term `word' has always
referred to the size of quantity that a machine normally operates on and
stores in its registers.  This is 32 bits for all the machines that GDB
runs on.

@item g
Examine giant words (8 bytes).
@end table

These letters specify just the way to print the contents:

@table @samp
@item x
Print as integers in unsigned hexadecimal.

@item d
Print as integers in signed decimal.

@item u
Print as integers in unsigned decimal.

@item o
Print as integers in unsigned octal.

@item a
Print as an address, both absolute in hex and then relative
to a symbol defined as an address below it.

@item c
Print as character constants.

@item f
Print as floating point.  This works only with sizes @samp{w} and
@samp{g}.

@item s
Print a null-terminated string of characters.  The specified unit size
is ignored; instead, the unit is however many bytes it takes to reach
a null character (including the null character).

@item i
Print a machine instruction in assembler syntax (or nearly).  The
specified unit size is ignored; the number of bytes in an instruction
varies depending on the type of machine, the opcode and the addressing
modes used.
@end table

If either the manner of printing or the size of unit fails to be specified,
the default is to use the same one that was used last.  If you don't want
to use any letters after the slash, you can omit the slash as well.

You can also omit the address to examine.  Then the address used is
just after the last unit examined.  This is why string and instruction
formats actually compute a unit-size based on the data: so that the
next string or instruction examined will start in the right place.
The @samp{print} command sometimes sets the default address for
the @samp{x} command; when the value printed resides in memory, the
default is set to examine the same location.  @samp{info line} also
sets the default for @samp{x}, to the address of the start of the
machine code for the specified line and @samp{info breakpoints} sets
it to the address of the last breakpoint listed.

When you use @key{RET} to repeat an @samp{x} command, it does not repeat
exactly the same: the address specified previously (if any) is ignored, so
that the repeated command examines the successive locations in memory
rather than the same ones.

You can examine several consecutive units of memory with one command by
writing a repeat-count after the slash (before the format letters, if any).
The repeat count must be a decimal integer.  It has the same effect as
repeating the @samp{x} command that many times except that the output may
be more compact with several units per line.  For example,

@example
x/10i $pc
@end example

@noindent
prints ten instructions starting with the one to be executed next in the
selected frame.  After doing this, you could print another ten following
instructions with

@example
x/10
@end example

@noindent
in which the format and address are allowed to default.

@kindex $_
@kindex $__
The addresses and contents printed by the @samp{x} command are not put in
the value history because there is often too much of them and they would
get in the way.  Instead, GDB makes these values available for subsequent
use in expressions as values of the convenience variables @code{$_} and
@code{$__}.

After an @samp{x} command, the last address examined is available for use
in expressions in the convenience variable @code{$_}.  The contents of that
address, as examined, are available in the convenience variable @code{$__}.

If the @samp{x} command has a repeat count, the address and contents saved
are from the last memory unit printed; this is not the same as the last
address printed if several units were printed on the last line of output.

@kindex disassemble
The specialized command @samp{disassemble} is also provided to dump a
range of memory as machine instructions.  The default memory range is
the function surrounding the program counter of the selected frame.  A
single argument to this command is a program counter value; the function
surrounding this value will be dumped.  Two arguments specify a range of
addresss (first inclusive, second exclusive) to be dumped.

@node Auto Display, Value History, Memory, Data
@section Automatic Display
@cindex automatic display
@cindex display of expressions

If you find that you want to print the value of an expression frequently
(to see how it changes), you might want to add it to the @dfn{automatic
display list} so that GDB will print its value each time the program stops.
Each expression added to the list is given a number to identify it;
to remove an expression from the list, you specify that number.
The automatic display looks like this:

@example
2: foo = 38
3: bar[5] = (struct hack *) 0x3804
@end example

@noindent
showing item numbers, expressions and their current values.

If the expression refers to local variables, then it does not make sense
outside the lexical context for which it was set up.  Such an expression
is printed only when execution is inside that lexical context.  For
example, if you give the command @samp{display name} while inside a
function with an argument @code{name}, then this argument will be
displayed whenever the program stops inside that function, but not when
it stops elsewhere (since this argument doesn't exist elsewhere).

@table @code
@item display @var{exp}
@kindex display
Add the expression @var{exp} to the list of expressions to display
each time the program stops.  @xref{Expressions}.

@item display/@var{fmt} @var{exp}
For @var{fmt} specifying only a display format and not a size or
count, add the expression @var{exp} to the auto-display list but
arranges to display it each time in the specified format @var{fmt}.

@item display/@var{fmt} @var{addr}
For @var{fmt} @samp{i} or @samp{s}, or including a unit-size or a
number of units, add the expression @var{addr} as a memory address to
be examined each time the program stops.  Examining means in effect
doing @samp{x/@var{fmt} @var{addr}}.  @xref{Memory}.

@item undisplay @var{dnums}@dots{}
@itemx delete display @var{dnums}@dots{}
@kindex delete display
@kindex undisplay
Remove item numbers @var{dnums} from the list of expressions to display.

@item disable display @var{dnums}@dots{}
@kindex disable display
Disable the display of item numbers @var{dnums}.  A disabled display
item is not printed automatically, but is not forgotten.  It may be
reenabled later.

@item enable display @var{dnums}@dots{}
@kindex enable display
Enable display of item numbers @var{dnums}.  It becomes effective once
again in auto display of its expression, until you specify otherwise.

@item display
Display the current values of the expressions on the list, just as is
done when the program stops.

@item info display
@kindex info display
Print the list of expressions previously set up to display
automatically, each one with its item number, but without showing the
values.  This includes disabled expressions, which are marked as such.
It also includes expressions which would not be displayed right now
because they refer to automatic variables not currently available.
@end table

@node Value History, Convenience Vars, Auto Display, Data
@section Value History

@cindex value history
Every value printed by the @samp{print} command is saved for the entire
session in GDB's @dfn{value history} so that you can refer to it in
other expressions.

@cindex @code{$}
@cindex @code{$$}
@cindex history number
The values printed are given @dfn{history numbers} for you to refer to them
by.  These are successive integers starting with 1.  @samp{print} shows you
the history number assigned to a value by printing @samp{$@var{num} = }
before the value; here @var{num} is the history number.

To refer to any previous value, use @samp{$} followed by the value's
history number.  The output printed by @samp{print} is designed to remind
you of this.  Just @code{$} refers to the most recent value in the history,
and @code{$$} refers to the value before that.

For example, suppose you have just printed a pointer to a structure and
want to see the contents of the structure.  It suffices to type

@example
p *$
@end example

If you have a chain of structures where the component @samp{next} points
to the next one, you can print the contents of the next one with this:

@example
p *$.next
@end example

@noindent
It might be useful to repeat this command many times by typing @key{RET}.

Note that the history records values, not expressions.  If the value of
@code{x} is 4 and you type this command:

@example
print x
set x=5
@end example

@noindent
then the value recorded in the value history by the @samp{print} command
remains 4 even though the value of @code{x} has changed.

@table @code
@item info values
@kindex info values
Print the last ten values in the value history, with their item
numbers.  This is like @samp{p $$9} repeated ten times, except that
@samp{info values} does not change the history.

@item info values @var{n}
Print ten history values centered on history item number @var{n}.

@item info values +
Print ten history values just after the values last printed.
@end table

@node Convenience Vars, Registers, Value History, Data
@section Convenience Variables

@cindex convenience variables
GDB provides @dfn{convenience variables} that you can use within GDB to
hold on to a value and refer to it later.  These variables exist entirely
within GDB; they are not part of your program, and setting a convenience
variable has no effect on further execution of your program.  That's why
you can use them freely.

Convenience variables have names starting with @samp{$}.  Any name starting
with @samp{$} can be used for a convenience variable, unless it is one of
the predefined set of register names (@pxref{Registers}).

You can save a value in a convenience variable with an assignment
expression, just as you would set a variable in your program.  Example:

@example
set $foo = *object_ptr
@end example

@noindent
would save in @code{$foo} the value contained in the object pointed to by
@code{object_ptr}.

Using a convenience variable for the first time creates it; but its value
is @code{void} until you assign a new value.  You can alter the value with
another assignment at any time.

Convenience variables have no fixed types.  You can assign a convenience
variable any type of value, even if it already has a value of a different
type.  The convenience variable as an expression has whatever type its
current value has.

@table @code
@item info convenience
@kindex info convenience
Print a list of convenience variables used so far, and their values.
Abbreviated @samp{i con}.
@end table

One of the ways to use a convenience variable is as a counter to be
incremented or a pointer to be advanced.  For example:

@example
set $i = 0
print bar[$i++]->contents
@i{@dots{}repeat that command by typing @key{RET}.}
@end example

Some convenience variables are created automatically by GDB and given
values likely to be useful.

@table @code
@item $_
The variable @code{$_} is automatically set by the @samp{x} command to
the last address examined (@pxref{Memory}).  Other commands which
provide a default address for @samp{x} to examine also set @code{$_}
to that address; these commands include @samp{info line} and @samp{info
breakpoint}.

@item $__
The variable @code{$__} is automatically set by the @samp{x} command
to the value found in the last address examined.
@end table

@node Registers,, Convenience Vars, Data
@section Registers

@cindex registers
Machine register contents can be referred to in expressions as variables
with names starting with @samp{$}.  The names of registers are different
for each machine; use @samp{info registers} to see the names used on your
machine.  The names @code{$pc} and @code{$sp} are used on all machines for
the program counter register and the stack pointer.  Often @code{$fp} is
used for a register that contains a pointer to the current stack frame,
and @code{$ps} is used for a register that contains the processor
status.  These standard register names may be available on your machine
even though the @code{info registers} command displays them with a
different name.  For example, on the SPARC, @code{info registers}
displays the processor status register as @code{$psr} but you can also
refer to it as @code{$ps}.

GDB always considers the contents of an ordinary register as an integer
when the register is examined in this way.  Some machines have special
registers which can hold nothing but floating point; these registers are
considered floating point.  There is no way to refer to the contents of an
ordinary register as floating point value (although you can @emph{print}
it as a floating point value with @samp{print/f $@var{regname}}).

Some registers have distinct ``raw'' and ``virtual'' data formats.  This
means that the data format in which the register contents are saved by the
operating system is not the same one that your program normally sees.  For
example, the registers of the 68881 floating point coprocessor are always
saved in ``extended'' format, but all C programs expect to work with
``double'' format.  In such cases, GDB normally works with the virtual
format only (the format that makes sense for your program), but the
@samp{info registers} command prints the data in both formats.

Register values are relative to the selected stack frame
(@pxref{Selection}).  This means that you get the value that the register
would contain if all stack frames farther in were exited and their saved
registers restored.  In order to see the real contents of all registers,
you must select the innermost frame (with @samp{frame 0}).

Some registers are never saved (typically those numbered zero or one)
because they are used for returning function values; for these registers,
relativization makes no difference.

@table @code
@item info registers
@kindex info registers
Print the names and relativized values of all registers.

@item info registers @var{regname}
Print the relativized value of register @var{regname}.  @var{regname}
may be any register name valid on the machine you are using, with
or without the initial @samp{$}.
@end table

@subsection Examples

You could print the program counter in hex with

@example
p/x $pc
@end example

@noindent
or print the instruction to be executed next with

@example
x/i $pc
@end example

@noindent
or add four to the stack pointer with

@example
set $sp += 4
@end example

@noindent
The last is a way of removing one word from the stack, on machines where
stacks grow downward in memory (most machines, nowadays).  This assumes
that the innermost stack frame is selected.  Setting @code{$sp} is
not allowed when other stack frames are selected.

@node Symbols, Altering, Data, Top
@chapter Examining the Symbol Table

The commands described in this section allow you to make inquiries for
information about the symbols (names of variables, functions and types)
defined in your program.  This information is found by GDB in the symbol
table loaded by the @samp{symbol-file} command; it is inherent in the text
of your program and does not change as the program executes.

@table @code
@item whatis @var{exp}
@kindex whatis
Print the data type of expression @var{exp}.  @var{exp} is not
actually evaluated, and any side-effecting operations (such as
assignments or function calls) inside it do not take place.
@xref{Expressions}.

@item whatis
Print the data type of @code{$}, the last value in the value history.

@item info address @var{symbol}
@kindex info address
Describe where the data for @var{symbol} is stored.  For a register
variable, this says which register it is kept in.  For a non-register
local variable, this prints the stack-frame offset at which the variable
is always stored.

Note the contrast with @samp{print &@var{symbol}}, which does not work
at all for a register variables, and for a stack local variable prints
the exact address of the current instantiation of the variable.

@item ptype @var{typename}
@kindex ptype
Print a description of data type @var{typename}.  @var{typename} may be
the name of a type, or for C code it may have the form
@samp{struct @var{struct-tag}}, @samp{union @var{union-tag}} or
@samp{enum @var{enum-tag}}.@refill

@item info sources
@kindex info sources
Print the names of all source files in the program for which there
is debugging information.

@item info functions
@kindex info functions
Print the names and data types of all defined functions.

@item info functions @var{regexp}
Print the names and data types of all defined functions
whose names contain a match for regular expression @var{regexp}.
Thus, @samp{info fun step} finds all functions whose names
include @samp{step}; @samp{info fun ^step} finds those whose names
start with @samp{step}.

@item info variables
@kindex info variables
Print the names and data types of all variables that are declared
outside of functions (i.e., except for local variables).

@item info variables @var{regexp}
Print the names and data types of all variables (except for local
variables) whose names contain a match for regular expression
@var{regexp}.

@item info types
@kindex info types
Print all data types that are defined in the program.

@item info types @var{regexp}
Print all data types that are defined in the program whose names
contain a match for regular expression @var{regexp}.

@ignore
This was never implemented.
@item info methods
@itemx info methods @var{regexp}
@kindex info methods
The @samp{info-methods} command permits the user to examine all defined
methods within C++ program, or (with the @var{regexp} argument) a
specific set of methods found in the various C++ classes.  Many
C++ classes provide a large number of methods.  Thus, the output
from the @samp{ptype} command can be overwhelming and hard to use.  The
@samp{info-methods} command filters the methods, printing only those
which match the regular-expression @var{regexp}.
@end ignore

@item printsyms @var{filename}
@kindex printsyms
Write a complete dump of the debugger's symbol data into the
file @var{filename}.
@end table

@node Altering, Sequences, Symbols, Top
@chapter Altering Execution

Once you think you have find an error in the program, you might want to
find out for certain whether correcting the apparent error would lead to
correct results in the rest of the run.  You can find the answer by
experiment, using the GDB features for altering execution of the
program.

For example, you can store new values into variables or memory
locations, give the program a signal, restart it at a different address,
or even return prematurely from a function to its caller.

@menu
* Assignment::    Altering variable values or memory contents.
* Jumping::       Altering control flow.
* Signaling::     Making signals happen in the program.
* Returning::     Making a function return prematurely.
@end menu

@node Assignment, Jumping, Altering, Altering
@section Assignment to Variables

@cindex assignment
@cindex setting variables
To alter the value of a variable, evaluate an assignment expression.
@xref{Expressions}.  For example,

@example
print x=4
@end example

@noindent
would store the value 4 into the variable @code{x}, and then print
the value of the assignment expression (which is 4).

All the assignment operators of C are supported, including the
incrementation operators @samp{++} and @samp{--}, and combining
assignments such as @samp{+=} and @samp{<<=}.

@kindex set
@kindex set variable
If you are not interested in seeing the value of the assignment, use the
@samp{set} command instead of the @samp{print} command.  @samp{set} is
really the same as @samp{print} except that the expression's value is not
printed and is not put in the value history (@pxref{Value History}).  The
expression is evaluated only for side effects.

Note that if the beginning of the argument string of the @samp{set} command
appears identical to a @samp{set} subcommand, it may be necessary to use
the @samp{set variable} command.  This command is identical to @samp{set}
except for its lack of subcommands.

GDB allows more implicit conversions in assignments than C does; you can
freely store an integer value into a pointer variable or vice versa, and
any structure can be converted to any other structure that is the same
length or shorter.

To store values into arbitrary places in memory, use the @samp{@{@dots{}@}}
construct to generate a value of specified type at a specified address
(@pxref{Expressions}).  For example, @code{@{int@}0x83040} would refer
to memory location 0x83040 as an integer (which implies a certain size
and representation in memory), and

@example
set @{int@}0x83040 = 4
@end example

would store the value 4 into that memory location.

@node Jumping, Signaling, Assignment, Altering
@section Continuing at a Different Address

Ordinarily, when you continue the program, you do so at the place where
it stopped, with the @samp{cont} command.  You can instead continue at
an address of your own choosing, with the following commands:

@table @code
@item jump @var{linenum}
@kindex jump
Resume execution at line number @var{linenum}.  Execution may stop
immediately if there is a breakpoint there.

The @samp{jump} command does not change the current stack frame, or
the stack pointer, or the contents of any memory location or any
register other than the program counter.  If line @var{linenum} is in
a different function from the one currently executing, the results may
be bizarre if the two functions expect different patterns of arguments or
of local variables.  For this reason, the @samp{jump} command requests
confirmation if the specified line is not in the function currently
executing.  However, even bizarre results are predictable based on
careful study of the machine-language code of the program.

@item jump *@var{address}
Resume execution at the instruction at address @var{address}.
@end table

You can get much the same effect as the @code{jump} command by storing a
new value into the register @code{$pc}.  The difference is that this
does not start the program running; it only changes the address where it
@emph{will} run when it is continued.  For example,

@example
set $pc = 0x485
@end example

@noindent
causes the next @samp{cont} command or stepping command to execute at
address 0x485, rather than at the address where the program stopped.
@xref{Stepping}.

The most common occasion to use the @samp{jump} command is when you have
stepped across a function call with @code{next}, and found that the
return value is incorrect.  If all the relevant data appeared correct
before the function call, the error is probably in the function that
just returned.

In general, your next step would now be to rerun the program and execute
up to this function call, and then step into it to see where it goes
astray.  But this may be time consuming.  If the function did not have
significant side effects, you could get the same information by resuming
execution just before the function call and stepping through it.  To do this,
first put a breakpoint on that function; then, use the @samp{jump} command
to continue on the line with the function call.

@node Signaling, Returning, Jumping, Altering
@section Giving the Program a Signal

@table @code
@item signal @var{signalnum}
@kindex signal
Resume execution where the program stopped, but give it immediately the
signal number @var{signalnum}.

Alternatively, if @var{signalnum} is zero, continue execution without
giving a signal.  This is useful when the program stopped on account of
a signal and would ordinary see the signal when resumed with the
@samp{cont} command; @samp{signal 0} causes it to resume without a
signal.
@end table

@node Returning,, Signaling, Altering
@section Returning from a Function

@cindex returning from a function
@kindex return
You can cancel execution of a function call with the @samp{return}
command.  This command has the effect of discarding the selected stack
frame (and all frames within it), so that control moves to the caller of
that function.  You can think of this as making the discarded frame
return prematurely.

First select the stack frame that you wish to return from
(@pxref{Selection}).  Then type the @samp{return} command.  If you wish
to specify the value to be returned, give that as an argument.

This pops the selected stack frame (and any other frames inside of it),
leaving its caller as the innermost remaining frame.  That frame becomes
selected.  The specified value is stored in the registers used for
returning values of functions.

The @samp{return} command does not resume execution; it leaves the
program stopped in the state that would exist if the function had just
returned.  Contrast this with the @samp{finish} command
(@pxref{Stepping}), which resumes execution until the selected stack
frame returns @emph{naturally}.

@node Sequences, Options, Altering, Top
@chapter Canned Sequences of Commands

GDB provides two ways to store sequences of commands for execution as a
unit: user-defined commands and command files.

@menu
* Define::         User-defined commands.
* Command Files::  Command files.
* Output::         Controlled output commands useful in
                   user-defined commands and command files.
@end menu

@node Define, Command Files, Sequences, Sequences
@section User-Defined Commands

@cindex user-defined command
A @dfn{user-defined command} is a sequence of GDB commands to which you
assign a new name as a command.  This is done with the @samp{define}
command.

@table @code
@item define @var{commandname}
@kindex define
Define a command named @var{commandname}.  If there is already a command
by that name, you are asked to confirm that you want to redefine it.

The definition of the command is made up of other GDB command lines,
which are given following the @samp{define} command.  The end of these
commands is marked by a line containing @samp{end}.

@item document @var{commandname}
@kindex document
Give documentation to the user-defined command @var{commandname}.  The
command @var{commandname} must already be defined.  This command reads
lines of documentation just as @samp{define} reads the lines of the
command definition, ending with @samp{end}.  After the @samp{document}
command is finished, @samp{help} on command @var{commandname} will print
the documentation you have specified.

You may use the @samp{document} command again to change the
documentation of a command.  Redefining the command with @samp{define}
does not change the documentation.
@end table

User-defined commands do not take arguments.  When they are executed, the
commands of the definition are not printed.  An error in any command
stops execution of the user-defined command.

Commands that would ask for confirmation if used interactively proceed
without asking when used inside a user-defined command.  Many GDB commands
that normally print messages to say what they are doing omit the messages
when used in user-defined command.

@node Command Files, Output, Define, Sequences
@section Command Files

@cindex command files
A command file for GDB is a file of lines that are GDB commands.  Comments
(lines starting with @samp{#}) may also be included.  An empty line in a
command file does nothing; it does not mean to repeat the last command, as
it would from the terminal.

@cindex init file
@cindex @file{.gdbinit}
When GDB starts, it automatically executes its @dfn{init files}, command
files named @file{.gdbinit}.  GDB reads the init file (if any) in your home
directory and then the init file (if any) in the current working
directory.  (The init files are not executed if the @samp{-nx} option
is given.)  You can also request the execution of a command file with the
@samp{source} command:

@table @code
@item source @var{filename}
@kindex source
Execute the command file @var{filename}.
@end table

The lines in a command file are executed sequentially.  They are not
printed as they are executed.  An error in any command terminates execution
of the command file.

Commands that would ask for confirmation if used interactively proceed
without asking when used in a command file.  Many GDB commands that
normally print messages to say what they are doing omit the messages
when used in a command file.

@node Output,, Command Files, Sequences
@section Commands for Controlled Output

During the execution of a command file or a user-defined command, the only
output that appears is what is explicitly printed by the commands of the
definition.  This section describes three commands useful for generating
exactly the output you want.

@table @code
@item echo @var{text}
@kindex echo
@comment I don't consider backslash-space a standard C escape sequence
@comment because it's not in ANSI.
Print @var{text}.  Nonprinting characters can be included in @var{text}
using C escape sequences, such as @samp{\n} to print a newline.  @b{No
newline will be printed unless you specify one.} In addition to the
standard C escape sequences a backslash followed by a space stands for a
space.  This is useful for outputting a string with spaces at the
beginning or the end, since leading and trailing spaces are trimmed from
all arguments.  Thus, to print @w{`` and foo = ''}, use the command
@w{``echo \ and foo = \ ''}.
@comment AAARGGG!  How am I supposed to do @samp{ and foo = } and not
@comment have the spaces be invisible in TeX?

A backslash at the end of @var{text} can be used, as in C, to continue
the command onto subsequent lines.  For example,

@example
echo This is some text\n\
which is continued\n\
onto several lines.\n
@end example

produces the same output as

@example
echo This is some text\n
echo which is continued\n
echo onto several lines.\n
@end example

@item output @var{expression}
@kindex output
Print the value of @var{expression} and nothing but that value: no
newlines, no @samp{$@var{nn} = }.  The value is not entered in the
value history either.  @xref{Expressions} for more information on
expressions. 

@item output/@var{fmt} @var{expression}
Print the value of @var{expression} in format @var{fmt}.
@xref{Output formats}, for more information.

@item printf @var{string}, @var{expressions}@dots{}
@kindex printf
Print the values of the @var{expressions} under the control of
@var{string}.  The @var{expressions} are separated by commas and may
be either numbers or pointers.  Their values are printed as specified
by @var{string}, exactly as if the program were to execute

@example
printf (@var{string}, @var{expressions}@dots{});
@end example

For example, you can print two values in hex like this:

@example
printf "foo, bar-foo = 0x%x, 0x%x\n", foo, bar-foo
@end example

The only backslash-escape sequences that you can use in the string are
the simple ones that consist of backslash followed by a letter.
@end table

@node Options, Emacs, Sequences, Top
@chapter Options and Arguments for GDB

When you invoke GDB, you can specify arguments telling it what files to
operate on and what other things to do.

@menu
* Mode Options::     Options controlling modes of operation.
* File Options::     Options to specify files (executable, coredump, commands)
* Other Arguments::  Any other arguments without options
			also specify files.
@end menu

@node Mode Options, File Options, Options, Options
@section Mode Options

@table @samp
@item -nx
Do not execute commands from the init files @file{.gdbinit}.
Normally, the commands in these files are executed after all the
command options and arguments have been processed.  @xref{Command
Files}.

@item -q
``Quiet''.  Do not print the usual introductory messages.

@item -batch
Run in batch mode.  Exit with code 0 after processing all the command
files specified with @samp{-x} (and @file{.gdbinit}, if not inhibited).
Exit with nonzero status if an error occurs in executing the GDB
commands in the command files.

@item -fullname
This option is used when Emacs runs GDB as a subprocess.  It tells GDB
to output the full file name and line number in a standard,
recognizable fashion each time a stack frame is displayed (which
includes each time the program stops).  This recognizable format looks
like two @samp{\032} characters, followed by the file name, line number
and character position separated by colons, and a newline.  The
Emacs-to-GDB interface program uses the two @samp{\032} characters as
a signal to display the source code for the frame.
@end table

@node File Options, Other Arguments, Mode Options, Options
@section File-specifying Options

All the options and command line arguments given are processed
in sequential order.  The order makes a difference when the
@samp{-x} option is used.

@table @samp
@item -s @var{file}
Read symbol table from file @var{file}.

@item -e @var{file}
Use file @var{file} as the executable file to execute when
appropriate, and for examining pure data in conjunction with a core
dump.

@item -se @var{file}
Read symbol table from file @var{file} and use it as the executable
file.

@item -c @var{file}
Use file @var{file} as a core dump to examine.

@item -x @var{file}
Execute GDB commands from file @var{file}.

@item -d @var{directory}
Add @var{directory} to the path to search for source files.
@end table

@node Other Arguments,, File Options, Options
@section Other Arguments

If there are arguments to GDB that are not options or associated with
options, the first one specifies the symbol table and executable file name
(as if it were preceded by @samp{-se}) and the second one specifies a core
dump file name (as if it were preceded by @samp{-c}).

@node Emacs, Remote, Options, Top
@chapter Using GDB under GNU Emacs

A special interface allows you to use GNU Emacs to view (and
edit) the source files for the program you are debugging with
GDB.

To use this interface, use the command @kbd{M-x gdb} in Emacs.  Give the
executable file you want to debug as an argument.  This command starts
GDB as a subprocess of Emacs, with input and output through a newly
created Emacs buffer.

Using GDB under Emacs is just like using GDB normally except for two
things:

@itemize @bullet
@item
All ``terminal'' input and output goes through the Emacs buffer.  This
applies both to GDB commands and their output, and to the input and
output done by the program you are debugging.

This is useful because it means that you can copy the text of previous
commands and input them again; you can even use parts of the output
in this way.

All the facilities of Emacs's Shell mode are available for this purpose.

@item
GDB displays source code through Emacs.  Each time GDB displays a
stack frame, Emacs automatically finds the source file for that frame
and puts an arrow (@samp{=>}) at the left margin of the current line.

Explicit GDB @samp{list} or search commands still produce output as
usual, but you probably will have no reason to use them.
@end itemize

In the GDB I/O buffer, you can use these special Emacs commands:

@table @kbd
@item M-s
Execute to another source line, like the GDB @samp{step} command.

@item M-n
Execute to next source line in this function, skipping all function
calls, like the GDB @samp{next} command.

@item M-i
Execute one instruction, like the GDB @samp{stepi} command.

@item C-c C-f
Execute until exit from the selected stack frame, like the GDB
@samp{finish} command.

@item M-c
@comment C-c C-p in emacs 19
Continue execution of the program, like the GDB @samp{cont} command.

@item M-u
@comment C-c C-u in emacs 19
Go up the number of frames indicated by the numeric argument
(@pxref{Arguments, , Numeric Arguments, emacs, The GNU Emacs Manual}),
like the GDB @samp{up} command.@refill

@item M-d
@comment C-c C-d in emacs 19
Go down the number of frames indicated by the numeric argument, like the
GDB @samp{down} command.
@end table

In any source file, the Emacs command @kbd{C-x SPC} (@code{gdb-break})
tells GDB to set a breakpoint on the source line point is on.

The source files displayed in Emacs are in ordinary Emacs buffers
which are visiting the source files in the usual way.  You can edit
the files with these buffers if you wish; but keep in mind that GDB
communicates with Emacs in terms of line numbers.  If you add or
delete lines from the text, the line numbers that GDB knows will cease
to correspond properly to the code.

@node Remote, Extensions, Emacs, Top
@chapter Remote Kernel Debugging

If you are trying to debug a program running on a machine that can't run
GDB in the usual way, it is often useful to use remote debugging.  For
example, you might be debugging an operating system kernel, or debugging
a small system which does not have a general purpose operating system
powerful enough to run a full-featured debugger.  Currently GDB supports
remote debugging over a serial connection.

The program to be debugged on the remote machine needs to contain a
debugging device driver which talks to GDB over the serial line using the
protocol described below.  The same version of GDB that is used ordinarily
can be used for this.  Several sample remote debugging drivers are
distributed with GDB; see the @file{README} file in the GDB distribution for
more information.

@menu
* Remote Commands::       Commands used to start and finish remote debugging.
@end menu

For details of the communication protocol, see the comments in the GDB
source file @file{remote.c}.

@node Remote Commands,, Remote, Remote
@section Commands for Remote Debugging

To start remote debugging, first run GDB and specify as an executable file
the program that is running in the remote machine.  This tells GDB how
to find the program's symbols and the contents of its pure text.  Then
establish communication using the @samp{attach} command with a device
name rather than a pid as an argument.  For example:

@example
attach /dev/ttyd
@end example

@noindent
if the serial line is connected to the device named @file{/dev/ttyd}.  This
will stop the remote machine if it is not already stopped.

Now you can use all the usual commands to examine and change data and to
step and continue the remote program.

To resume the remote program and stop debugging it, use the @samp{detach}
command.

@ignore
This material will be merged in when better Readline documentation
is done.

@node GDB Readline, History Top ,Readline Top, Command Editing
@subsection GDB Readline

You may control the behavior of command line editing in GDB with the
following commands:

@table @code
@kindex set editing
@item set editing
@itemx set editing on
Enable command line editing (enabled by default).

@item set editing off
Disable command line editing.

@kindex set history file
@item set history file @var{filename}
Set the name of the GDB command history file to @var{filename}.  This is
the file from which GDB will read an initial command history
list or to which it will write this list when it exits.  This list is
accessed through history expansion or through the history
command editing characters listed below.  This file defaults to the
value of the environmental variable @code{GDBHISTFILE}, or to
@code{./.gdb_history} if this variable is not set.

@kindex set history write
@item set history write
@itemx set history write on
Enable the writing of the command history to the command history file
named above.  This is enabled by default.

@item set history write off
Disable the writing of the command history to the command history file. 

@kindex set history size
@item set history size @var{size}
Set the number of commands which GDB will keep in its history list.
This defaults to the value of the environmental variable
@code{HISTSIZE}, or to 256 if this variable is not set.

@kindex info editing
@item info editing
Display the current settings relating to command line editing, and also
display the last ten commands in the command history.

@item info editing @var{n}
Print ten commands centered on command number @var{n}.

@item info editing +
Print ten commands just after the commands last printed.
@end table

@node GDB History, , History Top, Command editing    
@comment  node-name,  next,  previous,  up
Note that because of the additional meaning of @code{!} to GDB (as the
logical not operator in C), history expansion is off by default.  If you
decide to enable history expansion with the @samp{set history expansion
on} command, you will need to follow @samp{!} with a space or a tab to
prevent it from being expanded.

The commands to control history expansion are:

@table @code

@kindex set history expansion
@item set history expansion on
@itemx set history expansion
Enable history expansion.

@item set history expansion off
Disable history expansion.  History expansion is off by default.

@end table
@end ignore

@node Extensions, Commands, Remote, Top
@chapter C++ Extensions

@cindex qualified names 
@cindex overload functions 
Lucid has added special extensions to GDB for debugging C++ code,
including support for:

@itemize @bullet 

@item 
Qualified names

@item 
Identifier demangling

@item
Overloaded and member functions

@item
Data member access inside a function member
@end itemize

Lucid has also improved symbol table dynamics.  

To effectively use these extensions, debug programs compiled with the
Lucid compiler option @samp{-g}. Consult ``Debugging, Profiling, and
Linking Options'' in Chapter 2 of the @i{Lucid C++ User's Guide} for a
description of @samp{-g}.

@menu
* Special Conditions::  Special conditions while debugging C++ code.
* Qualified Names::     Using qualified identifiers with a class name.
* Constructors::        Defining constructor and destructor functions.
* Overloading::         How GDB handles overloaded functions.
* Demangling::          How GDB demangles output.
@end menu

@node Special Conditions, Qualified Names, Extensions, Extensions
@section Special Conditions
@cindex C++ enhancements
@cindex C++, special conditions

This section describes special conditions you might encounter
debugging C++ code.

@itemize @bullet 

@item 
GDB evaluates expressions in debugging commands as C expressions,
using C type rules and coercions. For example, print @samp{a + b}
evaluates the expression @samp{a + b} as a C addition expression. If
@samp{a +b a} is an inappropriate operand for @samp{+} in C, GDB
generates an error, even if @samp{a} is an instance of a C++ class
that defines a member function @samp{operator +()}.

@item 
GDB cannot interpret @samp{operator ()} in command expressions. For
example, assume you debug the following C++ code:

@example
struct x @{ int operator() (int a) @{ return 2*a; @} @};
void main() @{ int i;   x   t;     i = t(3); @}
@end example

GDB will not correctly interpret @samp{break x::operator()}. You can
still set a break with the mangled name, @samp{break__cl__lx}.

@item
You must explicitly pass the @samp{this} argument when evaluating
member functions. For example, assume the following class and instance
are defined:

@example
class C @{
      int i
    public:
      int get_i ()@{
       return i;
      @}
   @};
C c;
@end example

To evaluate @samp{get_i} in GDB, use this command to explicitly pass 
the address of a C instance:

@example
(gdb) print C::get_i (&c) 
@end example

@item
To call a member function of a class, you must refer to it directly by
qualified name. Prefix the function's name with the name of the class
that defines the member function, followed by a double colon (::).
For example, assume @samp{class D} is derived from @samp{class C} and
refines the member function @samp{f}:

@example
class C @{ virtual int f (); @}; 
class D : public C @{ virtual int f (); @}; 
@end example

To call @samp{C::f} on the instance @samp{c} of class C and
@samp{D::f} on the instance @samp{d} of @samp{class D}, enter this
command:

@example
(gdb) print C::f(&c) (gdb) print D::f(&d)
@end example

@item
GDB cannot handle coercions between a derived class and its
non-left-most base classes. Using the preceding example, you can only
call @samp{C::f} with an instance of @samp{class D} as an argument
because class C is a left-most base class of @samp{D}:

@example
(gdb) print C::f(&d) 	
@end example

Assume class D also has a non-left-most base @samp{class B}:

@example
class B @{ public: virtual int g (); @}; 
class C @{ public: virtual int f (); @};
class D : public C, public B @{
public: 
virtual int f ();
virtual int g (); 
@}; 
@end example

In this case, you can still call @samp{C::f} with an instance of
@samp{class D}. However, you must supply address arithmetic to call
@samp{B::g} with an instance of @samp{D}.

@item
GDB might report a source line twice or skip a line when stepping
through nested @samp{if} and @samp{switch} statements. This condition
results from branch optimizations generated by the Lucid compiler.

@end itemize

@node Qualified Names, Constructors, Special Conditions, Extensions
@section Qualified Names 
@cindex qualified names

You can qualify identifiers with a class name in a GDB command.  This
method is particularly useful when you refer to static class members.
For example, assume the following class and instance are defined:

@example
class C1 @{ 	
   public:
      static int i; 				
      void set_i (int x) @{ C1::i = x; @} 		 
   @}; 		
C1 c1; 
@end example

You can use the following GDB commands to first set @samp{i} to
@samp{10} and then verify the value of @samp{i}:

@example
(gdb) print C1::set_i(&c1, 10) 	 
10 	 
(gdb) print C1::i 10
@end example

@node Constructors, Overloading, Qualified Names, Extensions
@section Constructors, Destructors, and Operators 
@cindex constructors
@cindex destructors 
@cindex operators 

In C++, a class can define constructor and destructor functions:

@itemize @bullet

@item
Use a @i{constructor} to allocate or initialize a class's instances.

@item
Use a @i{destructor} to destroy or delete a class's instance. 
@end itemize

You can also overload operators as global or member functions by
referring to them in the expression part of a GDB command. You refer
to constructors, destructors, and member operators as they are
defined:

@itemize @bullet

@item
@i{class_name::class_name} for a constructor.

@item
@i{class_name::~class_name} for a destructor. 

@item
@samp{::operator} @i{op}, @samp{::operator}@i{op}, or @i{::op}, for a
global operator, where @i{op} is any operator decribed in section 13.4
of @i{The Annotated C++ Reference Manual}.

Note that @samp{operator} is not a reserved name in C++. You must
include a double colon prefix (::) to force the C++ interpretation. It
is legal, particularly in standard C, to name functions, types, and
variables @samp{operator}.

@item
@samp{class_name::operator} @i{op},@samp{class_name::operator}@i{op}, 
@i{class_name::op} for C++ member function operators.

@item
@samp{class_name::operator} @i{type} for user-type conversion
operators.  

@end itemize

As standard with any member function, you must explicitly supply the
@samp{this} argument when you call a constructor, destructor, or
member operator.
 
Consult the @i{UNIX System V, AT&T C++ Language System, Release 2.1
Product Reference Manual}: Select Code 307-160, AT&T 1989 for more
information on constructors, destructors, and operators.

@node Overloading, Demangling, Constructors, Extensions
@section Overloaded Functions
@cindex overloaded functions

C++ allows functions to be overloaded. @i{Overloading} allows more
than one function to share the same name as long these functions are
distinct in their argument types. GDB lets you to select a desired
function from the overloaded set. For example, assume the following
functions are defined:

@example
int add (int i1, int i2);
int add (float f1, float f2);
@end example

To set a break point on @samp{int add (int i1, int i2);}, 
enter this command:

@example
(gdb) break add
@end example

GDB presents the set of overloads for @samp{add}:

@example
[0] cancel
[1] add (int, int)
[2] add (float, float)
> 
@end example

You select the desired function by entering the corresponding
index following the > prompt:

@example
[0] cancel
[1] add (int, int)
[2] add (float, float)
> 1
breakpoint set 0x3456; add (int, int)
@end example

GDB similarly handles overloaded member functions, except that it
qualifies each function name with its containing class's name.  For
example, assume the @samp{add} functions in the previous example are
member functions of @samp{class A}. In this case, the following GDB
sequence appears:

@example
(gdb) break A::add
[0] cancel
[1] A::add (int, int)
[2] A::add (float, float)
> 1
breakpoint set 0x3456; A::add (int, int)
@end example

Using the qualifier in the command @samp{break A::add} restricts the
set of possible overloaded functions to the member functions of
@samp{class A}.

GDB should only display the set of overloads, if any, associated with
the given function name.  Note, however, that GDB cannot distinguish
between valid overloads and other functions with the same name.

@node Demangling, , Overloading, Extensions
@section Output Demangling
@cindex output demangling

The GDB extensions attempt to @i{demangle} all output; that is, it
changes most of the names the compiler mangles into their original
source form. In the following cases, GDB does not perform demangling
and instead uses the mangled form as output:

@itemize @bullet

@item
If the name is an enumerated constant defined in a class. 

@item
If the name is a member of a non-left-most base class. In this case,
GDB prefixes the name with the path of the C struct generated around
such members.  Note that a reference to any non-left-most member must
include the name of the generated C struct.

@item
If a @samp{type-safe-link} name is longer than 1024 characters
@end itemize

@node Commands, Concepts, Extensions, Top
@unnumbered Command Index

@printindex ky

@node Concepts, , Commands, Top
@unnumbered Concept Index

@printindex cp

@contents
@bye




Occasionally it is useful to execute a shell command from within GDB.
This can be done with the @samp{shell} command.

@table @code
@item shell @var{shell command string}
@kindex shell
@cindex shell escape
Directs GDB to invoke an inferior shell to execute @var{shell command string}.
The environment variable @code{SHELL} is used if it exists, otherwise GDB
uses @samp{/bin/sh}.
@end table
